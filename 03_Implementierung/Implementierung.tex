\documentclass[11pt,a4paper]{scrartcl}
\usepackage{color}
\usepackage{ifthen}
\usepackage{ifpdf}
\usepackage[headings]{fullpage}
\usepackage{listings}
\lstset{language=Java,breaklines=true}
\ifpdf \usepackage[pdftex, pdfpagemode={UseOutlines},bookmarks,colorlinks,linkcolor={blue},plainpages=false,pdfpagelabels,citecolor={red},breaklinks=true]{hyperref}
  \usepackage[pdftex]{graphicx}
  \pdfcompresslevel=9
  \DeclareGraphicsRule{*}{mps}{*}{}
\else
  \usepackage[dvips]{graphicx}
\fi

\newcommand{\entityintro}[3]{%
  \hbox to \hsize{%
    \vbox{%
      \hbox to .2in{}%
    }%
    {\bf  #1}%
    \dotfill\pageref{#2}%
  }
  \makebox[\hsize]{%
    \parbox{.4in}{}%
    \parbox[l]{5in}{%
      \vspace{1mm}%
      #3%
      \vspace{1mm}%
    }%
  }%
}
\newcommand{\refdefined}[1]{
\expandafter\ifx\csname r@#1\endcsname\relax
\relax\else
{$($in \ref{#1}, page \pageref{#1}$)$}\fi}
\date{\today}
\chardef\textbackslash=`\\
\usepackage{pdfpages}
\usepackage[utf8]{inputenc}
\usepackage[T1]{fontenc}
\usepackage[german]{babel}
\usepackage{hyperref}
\hypersetup{
	pdftitle={Pflichtenheft},
	bookmarks=true,
}
\usepackage{csquotes}

\usepackage{fancyhdr}%<-------------to control headers and footers
\usepackage[a4paper,margin=1in,footskip=.25in]{geometry}
\fancyhf{}
\fancyfoot[C]{\thepage} %<----to get page number below text
\pagestyle{fancy} %<-------the page style itself

\usepackage{xcolor}
\usepackage{framed}
\definecolor{shadecolor}{RGB}{220,220,220}
\usepackage{float}


\title{Android GO! App - Implementierung}
\author{Gruppe 3}
\date{20.08.17}

% define custom lists
\usepackage{enumitem}
\usepackage{lipsum}

\begin{document}

\begin{titlepage}
	\begin{center}
	{\scshape\LARGE \bfseries Implementierungsbericht \par}
	\vspace{1cm}
	{\scshape\Large Praktikum der Softwareentwicklung \\ Sommersemester 2017\par}
	\vspace{1.5cm}
	{\huge\bfseries Android GO! App\par}
	\vspace{2cm}
	{\Large\itshape - Gruppe 3 -\par}
	\vfill
	{\bfseries erstellt von:\par}
	Arsenii Dunaev \\
	Florian Kröger \\
	Tina Maria Strößner \\
	Volodymyr Shpylka \\	
	\vfill
	% Bottom of the page
	{\large 20.08.17 \par}	
	\end{center}
\end{titlepage}

\newpage

\tableofcontents

\newpage

\section{Änderungen zum Pflichtenheft und Entwurf}
\begin{enumerate}
	\item \textbf{Hinzufügen von Service-Klassen in der Service-Layer des Servers}\\
	Der Business-Layer des Servers wurden die Klassen \textit{userService, GroupService} und \textit{GoService} hinzugefügt. Diese dienen hauptsächlich der klareren Strukturierung zwischen den Aufgaben der restController und der Dao-Klassen. Anstatt direkt auf die Persistence-Layer zuzugreifen, interagieren die RestController-Klassen nur noch mit dem für sie zuständigen Service. In den Service-Klassen wird das Anpassen von Daten für die Umwandlung zu JSON oder zur Persistierung übernommen, sowie die benachrichtigung der Observer, die vorher in den DAO-Klassen stattgefunden hat.
	
Durch das Hinzufügen dieser Klassen haben sowohl die RestController als auch die DAO-Klassen nur noch einen Verantwortlichkeitsbereich, um den sie sich kümmern müssen. Jeglicher Code der nur zur Umwandlung und Verwaltung der Daten dient, kann an die Service-Klassen ausgelagert werden.

	\item \textbf{Zusätzliche Client-Entity Klassen}\\
	Da sich die Anforderungen an die Datenmodellierung auf dem Server und auf dem Client satark unterscheiden, wurden jeweils unterschiedliche Entity-Klassen zur Modellierung der NEtitäten verwendet. Um die Anwendungen trotzdem miteinander kompatibel zu machen, verfügt der Server zusätzlich über ClientEntity-Klassen, die die Datenmodellierung auf den Clients entsprechen.
	
	Die Verwendung dieser Klassen erlaubt das einfache Parsen von Json-Objekten, die über die Client-Server Schnittstellen übertragen wurden. Die Umwandlung von ClientEntities zu ServerEntities übernehmen die entsprechenden Service-Klassen. Im Quellcode lassen sich Instanzen der ClientEntity-Klassen an dem Buchstaben c im Variablennamen erkennen (z.B. \textit{cGroup}).
	
\end{enumerate}

\newpage

\section{Ablauf}

\newpage

\section{Unittests}

\subsection{Server}

\subsubsection{Testdaten}\label{Testdaten}
Für die Unittests des Servers wurden Testdaten verwendet, die sowohl in eine Testdatenbank gespeichert wurden, als auch als Java-Objekte über die Klasse 'TestData'.

\subsubsection{Tests für Observer-Klassen}
Die Tests der Observer-Klassen sind in der Testklasse 'ObserverTest' zusammengefasst. Für jede Observer-Klasse muss die öffentliche Methode \textit{update()} getestet werden.
Um die Korrektheit der Methode zu überprüfen, wird getestet, ob der Methodenaufruf \textit{send(String data, EventArg arg, Set<UserEntity> receiver)} korrekt ausgeführt wird. Korrekt hei"st in diesem Fall, dass der Methodenaufruf erfolgt und die übergebenen Argumente den Vorgaben entsprechen.

Das Argument \textit{data} ist ein Json-String, der wie im Entwurfsdokunment beschrieben aufgebaut ist. Das Argument \textit{arg} ist ein Element des Enums \textit{EventArg} und repräsentiert, die aufgetretene Änderung. Das Set \textit{receiver} enthält alle user, an die eine Benachrichtigung geschickt werden soll.

Für die Tests der Klasse 'ObserverTest' werden Mock-Objekte der Dao-Klassen verwendet. Diese sind mit dem Framework 'Mockito' erstellt und so konfiguriert, das bei Aufruf der Methode \textit{get({Long | String} key)} (Methode des AbstractDao Interfaces) ein zuvor definiertes TestObjekt zurückgegeben wird. Die verwendeten Testobjekte stammen aus der Klasse 'TestData', die für die Bereitstellung von Testobjekten zuständig ist (siehe \ref{Testdaten}).

Das FcmClient-Attribut der Observer-Klassen wird in den test ebenfalls gemockt und so konfiguriert, das bei einem Methodenaufruf von \textit{send(...)} die übergebenen Argumente in Feldern der ObserverTest-Klasse gespeichert werden, so dass diese anschließend mit den erwarteten Werten verglichen werden können.

Die Ergebnisse, die bei Testausführung erwartet werden, sind als statische Felder in der Testklasse gespeichert. Dazu gehören die erwarteten Json-Strings und die Receiver der send()-Methode.
Die erwarteten Ergebniss wurden, teils händisch, anhand des Entwurfsdokuments aus den in TestData definierten Testdaten generiert. Sie sind nur gültig, solange keine Änderungen an diesen Daten vorgenommen werden.

\subsubsection{Tests für RestController-Klassen}
Die Tests der Klassen User-, Group- und GoRestController sind jeweils darauf zu überprüfen, ob die Calls an die Rest-Ressourcen richtig interpretiert und ausgeführt werden. Dazu gehört das korrekte Auslesen von Methodenparametern aus der URI der Rest-Ressource und JSON Strings aus dem Request-Body.

Die Rest API wird bei dem Tests mithilfe des Spring Boot Starter Test-Frameworks simuliert. Dadurch kann sowohl das Mapping von Rest-Ressourcen, als auch die Methodenaufrufe in den Controller-Klassen getestet werden.

Um die Korrektheit der Methodenaufrufe zu überprüfen, wird in jeder Testmethode der Methodenaufruf an die jeweilige Service-Klasse überprüft, die übergebenen Parameter, sowie die HTTP-Response, die zurückgegeben wird. Die dabei verwendeten Testdaten, die aus der TestData-Klasse stammen, wurden mithilfe eines Rest-Clients auf Korrektheit überprüft.

\subsubsection{Tests für Service-Klassen}
Für die Klassen User-, Go- und GroupService wurde je eine Testklasse erstellt, in denen sämtliche Methoden getestet werden. Da die meisten dieser Methoden \textit{void}-Methoden sind, kommt es vor allem darauf an, zu testen, ob der richtige Methodenaufrauf auf dem heweiligen Dao-Objekt erfolgt und ob jeweils er passende Observer benachrichtigt wird.

Für die Realisierung dieser Tests wurden das Testframework Mockito, sowie Daten der \textit{TestData}-Klasse verwendet.

\subsubsection{Tests für Entity-Klassen}
Die Entity-Klassen bestehen hauptsächlich aus Getter- und Setter-Methoden, die aufgrund ihrer Einfachheit nicht extra getestet werden. Die einzigen Methoden, die aus den Entity-Klassen getestet werden, sind jeweils die Methoden \textit{equals()} und \textit{hashCode()}. Ob der \textit{equals()}/\textit{hashCode()} Contract eingehalten wird, wird mittels des EqualsVerifier getestet.

\end{document}
\grid
