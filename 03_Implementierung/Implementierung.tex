\documentclass[11pt,a4paper]{scrartcl}
\usepackage{color}
\usepackage{ifthen}
\usepackage{ifpdf}
\usepackage[headings]{fullpage}
\usepackage{listings}
\lstset{language=Java,breaklines=true}
\ifpdf \usepackage[pdftex, pdfpagemode={UseOutlines},bookmarks,colorlinks,linkcolor={blue},plainpages=false,pdfpagelabels,citecolor={red},breaklinks=true]{hyperref}
  \usepackage[pdftex]{graphicx}
  \pdfcompresslevel=9
  \DeclareGraphicsRule{*}{mps}{*}{}
\else
  \usepackage[dvips]{graphicx}
\fi

\newcommand{\entityintro}[3]{%
  \hbox to \hsize{%
    \vbox{%
      \hbox to .2in{}%
    }%
    {\bf  #1}%
    \dotfill\pageref{#2}%
  }
  \makebox[\hsize]{%
    \parbox{.4in}{}%
    \parbox[l]{5in}{%
      \vspace{1mm}%
      #3%
      \vspace{1mm}%
    }%
  }%
}
\newcommand{\refdefined}[1]{
\expandafter\ifx\csname r@#1\endcsname\relax
\relax\else
{$($in \ref{#1}, page \pageref{#1}$)$}\fi}
\date{\today}
\chardef\textbackslash=`\\
\usepackage{pdfpages}
\usepackage[utf8]{inputenc}
\usepackage[T1]{fontenc}
\usepackage[german]{babel}
\usepackage{hyperref}
\hypersetup{
	pdftitle={Pflichtenheft},
	bookmarks=true,
}
\usepackage{csquotes}

\usepackage{fancyhdr}%<-------------to control headers and footers
\usepackage[a4paper,margin=1in,footskip=.25in]{geometry}
\fancyhf{}
\fancyfoot[C]{\thepage} %<----to get page number below text
\pagestyle{fancy} %<-------the page style itself

\usepackage{xcolor}
\usepackage{framed}
\definecolor{shadecolor}{RGB}{220,220,220}
\usepackage{float}


\title{Android GO! App - Implementierung}
\author{Gruppe 3}
\date{20.08.17}

% define custom lists
\usepackage{enumitem}
\usepackage{lipsum}

\begin{document}

\begin{titlepage}
	\begin{center}
	{\scshape\LARGE \bfseries Implementierungsbericht \par}
	\vspace{1cm}
	{\scshape\Large Praktikum der Softwareentwicklung \\ Sommersemester 2017\par}
	\vspace{1.5cm}
	{\huge\bfseries Android GO! App\par}
	\vspace{2cm}
	{\Large\itshape - Gruppe 3 -\par}
	\vfill
	{\bfseries erstellt von:\par}
	Arsenii Dunaev \\
	Florian Kröger \\
	Tina Maria Strößner \\
	Volodymyr Shpylka \\	
	\vfill
	% Bottom of the page
	{\large 20.08.17 \par}	
	\end{center}
\end{titlepage}

\newpage

\tableofcontents

\newpage

\section{Änderungen zum Pflichtenheft und Entwurf}

\newpage

\section{Ablauf}

\newpage

\section{Unittests}

\subsection{Server}

\subsubsection{Testdaten}\label{Testdaten}
Für die Unittests des Servers wurden testdaten verwendet, die sowohl in eine Testdatenbank gespeichert wurden, als auch als Java-Objekte über die Klasse 'TestData'.

\subsubsection{Tests für Observer-Klassen}
Die Tests der Observer-Klassen sind in der Testklasse 'ObserverTest' zusammengefasst. Für jede Observer-Klasse muss die öffentliche Methode \textit{update()} getestset werden.
Um die Korrektheit der Methode zu überprüfen, wird getestet, ob der Methodenaufruf \textit{send(String data, EventArg arg, Set<UserEntity> receiver)} korrekt ausgeführt wird. Korrekt hei"st in diesem Fall, dass der Methodenaufruf erfolgt und die übergebenen Argumente den Vorgaben entsprechen.

Das Argument \textit{data} ist ein Json-String, der wie im Entwurfsdokunment beschrieben aufgebaut ist. Das Argument \textit{arg} ist ein Element des Enums \textit{EventArg} und repräsentiert, die aufgetretene Änderung. Das Set \textit{receiver} enthält alle user, an die eine Benachrichtigung geschickt werden soll.\\

\textbf{Testressourcen}\\
Für die Tests der Klasse 'ObserverTest' werden Mock-Objekte der Dao-Klassen verwerndet. Diese sind mit dem Framework 'Mockito' erstellt und so konfiguriert, das bei Aufruf der Methode \textit{get({Long | String} key)} (Methode des AbstractDao Interfaces) ein zuvor definiertes TestObjekt zurückgegeben wird. Die verwendeten Testobjekte stammen aus der Klasse 'TestData', die für die Bereitstellung von Testobjekten zuständig ist (siehe \ref{Testdaten}).\\

Das FcmClient-Attribut der Observer-Klassen wird in den test ebenfalls gemockt und so konfiguriert, das bei einem Methodenaufruf von \textit{send(...)} die übergebenen Argumente in Feldern der ObserverTest-Klasse gespeichert werden, so dass diese anschließend mit den erwarteten Werten verglichen werden können.\\

\textbf{Erwartete Ergebnisse}\\
Die Ergebnisse, die bei Testausführung erwartet werden, sind als statische Felder in der Testklasse gespeichert. Dazu gehören die erwarteten Json-Strings und die Receiver der send()-Methode.
Die erwarteten Ergebniss wurden, teils händisch, anhand des Entwurfsdokuments aus den in TestData definierten Testdaten generiert. Sie sind nur gültig, solange keine Änderungen an diesen Daten vorgenommen werden.

\end{document}
\grid
