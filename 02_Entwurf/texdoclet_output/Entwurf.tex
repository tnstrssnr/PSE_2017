\documentclass[11pt,a4paper]{report}
\usepackage{color}
\usepackage{ifthen}
\usepackage{ifpdf}
\usepackage[headings]{fullpage}
\usepackage{listings}
\lstset{language=Java,breaklines=true}
\ifpdf \usepackage[pdftex, pdfpagemode={UseOutlines},bookmarks,colorlinks,linkcolor={blue},plainpages=false,pdfpagelabels,citecolor={red},breaklinks=true]{hyperref}
  \usepackage[pdftex]{graphicx}
  \pdfcompresslevel=9
  \DeclareGraphicsRule{*}{mps}{*}{}
\else
  \usepackage[dvips]{graphicx}
\fi

\newcommand{\entityintro}[3]{%
  \hbox to \hsize{%
    \vbox{%
      \hbox to .2in{}%
    }%
    {\bf  #1}%
    \dotfill\pageref{#2}%
  }
  \makebox[\hsize]{%
    \parbox{.4in}{}%
    \parbox[l]{5in}{%
      \vspace{1mm}%
      #3%
      \vspace{1mm}%
    }%
  }%
}
\newcommand{\refdefined}[1]{
\expandafter\ifx\csname r@#1\endcsname\relax
\relax\else
{$($in \ref{#1}, page \pageref{#1}$)$}\fi}
\date{\today}
\chardef\textbackslash=`\\
\usepackage{pdfpages}
\usepackage[utf8]{inputenc}
\usepackage[T1]{fontenc}
\usepackage[german]{babel}
\usepackage{hyperref}
\hypersetup{
	pdftitle={Pflichtenheft},
	bookmarks=true,
}
\usepackage{csquotes}

\usepackage{fancyhdr}%<-------------to control headers and footers
\usepackage[a4paper,margin=1in,footskip=.25in]{geometry}
\fancyhf{}
\fancyfoot[C]{\thepage} %<----to get page number below text
\pagestyle{fancy} %<-------the page style itself

\usepackage{xcolor}
\usepackage{framed}
\definecolor{shadecolor}{RGB}{220,220,220}
\usepackage{float}


\title{Android GO! App - Pflichtenheft}
\author{Gruppe 3}
\date{11.06.17}

% define custom lists
\usepackage{enumitem}
\usepackage{lipsum}

\begin{document}

\begin{titlepage}
	\begin{center}
	{\scshape\LARGE \bfseries Entwurfsdokument \par}
	\vspace{1cm}
	{\scshape\Large Praktikum der Softwareentwicklung \\ Sommersemester 2017\par}
	\vspace{1.5cm}
	{\huge\bfseries Android GO! App\par}
	\vspace{2cm}
	{\Large\itshape - Gruppe 3 -\par}
	\vfill
	{\bfseries erstellt von:\par}
	Arsenii Dunaev \\
	Florian Kröger \\
	Tina Maria Strößner \\
	Volodymyr Shpylka \\	
	\vfill
	% Bottom of the page
	{\large 09.07.17 \par}	
	\end{center}
\end{titlepage}

\begin{abstract}
Die Android App GO! ist eine mobile Applikation, die speziell zur Organisation von Treffen (z. B. gemeinsames Essen im Café oder in der Mensa) entwickelt wird. Beim erfolgreichen gemeinsamen Losgehen wird der gemittelte GPS-Standort von Mitgliedern der Gruppe angezeigt.\\

Dieses Dokument erläutert den Entwurf des Systems auf der Grundlage des Pflichtenhefts.
\end{abstract}

% ------- textdoclet_include/setup.tex end

\sloppy
\addtocontents{toc}{\protect\markboth{Contents}{Contents}}
\tableofcontents



% add something here

\subsection{Änderungen zum Pflichtenheft}

Es wurden im Entwurf folgende Änderungen gegenüber dem Pflichtenheft vorgenommen:
\begin{enumerate}
	\item \textbf{Produktdaten - Benutzer} \\
	Es werden in den Produktdaten zusätzlich eine (von Firebase automatisch generierte) InstanceID gespeichert, die es dem Server erlaubt, Daten an das Android-Gerät eines bestimmten Benutzers zu senden.
\end{enumerate}


\subsection{Paketstruktur}

\subsubsection{Client}
Der Programmteil, der auf dem Client - also auf dem Android-Gerät - ausgeführt wird, ist in folgende Pakete (die ggfs. Unterpakete enthalten) aufgeteilt:
\begin{itemize}
	\item Views
	\item Controller
	\item Model
	\item ServerCom
\end{itemize}

Der folgende Abschnitt erläutert, welche Aufgaben die einzelnen Module haben und welche Abhängigkeiten zu anderen Paketen und Klassen bestehen.

\subsubsubsection{Views}
Das Paket Views enthält alle Klassen, die am User Interface des Benutzers beteiligt sind. Die Hauptaufgabe des Pakets ist es zum Einen, dem Benutzer ein Interface zur Verfügung zu stellen, mit dem er Interagieren kann, zum Anderen werden hier Benutzereingaben entgegengenommen und soweit ausgewertet, dass die Verarbeitung der Eingabe an die dafür zuständige Stelle im Programm weitergegeben werden kann.\\

\textbf{Abhängigkeiten zu anderen Paketen:}\\
Das Paket Views kann die Informationen, die dem Benutzer angezeigt werden, nicht selbst generieren, sondern bekommt diese bereitgestellt vom Paket Model. Welche Informationen das sind, wird bestimmt vom Paket Controller. Somit besteht eine Abhängigkeit zu den Paketen Controller und Model.\\

\textbf{Unterpakete:}\\
das Paket enthält das Unterpaket 'RecyclerView'. Da in der Applikation viele (verschiedene) RecyclerViews verwendet werden, gibt es für die Erstellung derselben ein eigenes Paket, dessen Aufgabe es ist, von den Datenobjekten die das Model liefert die gewünschten Informationen zu extrahieren und diese mit dem richtigen Layout zusammenzuführen. Innerhalb des Pakets besteht eine Abhängigkeit derjenigen View-Klassen, die einen RecyclerView verwenden zu dem Unterpaket RecyclerViews. Das Unterpaket RecyclerViews selbst ist nocht von anderen Klassen und Paketen abhängig.

\subsubsubsection{Model}
Das Paket Model enthält Klassen, deren Entitäten die physischen und konzeptuellen Objekte, mit denen umgegangen werden muss, abbilden und deren Funktionen und Eigenschaften modellieren.

\textbf{Abhängigkeiten zu anderen Paketen}\\
Das Paket benötigt, um seine Aufgaben erfüllen zu können, die Dienste des Pakets ServerCom. Für die Verwaltung der Daten der modellierten Entitäten ist Kommunikation mit dem Server notwenig (für das Holen und Speichern von Daten).

\textbf{Unterpakete:}\\

\subsubsubsection{Controller}
Das Paket Controller ist dafür verantwortlich für .....

\textbf{Abhängigkeiten zu anderen Paketen:}\\


\textbf{Unterpakete:}\\
\begin{enumerate}
	\item \textit{SinInHelper}\\
	Das Unterpaket SignInHelper ist für die Koordination des SignIn Prozesses zuständig. Die Anmeldung eines benutzers erfolgt in zwei Schritten: zunächst muss die Identität des Benutzers festgestellt werden (dies geschieht über eine Schnittstelle zu Firebase), danach müssen die Daten des identifizierten Benutzers geladen werden.
\end{enumerate}

\subsubsubsection{ServerCom}
Das Paket ServerCom übernimmt die Kommunikation der App mit dem Server, also das Speichern von Daten auf dem Server bzw. das Holden von Daten von dem Server.\\

\textbf{Abhängigkeiten zu anderen Paketen}\\
Das Paket hat keine Abhängigkeiten zu anderen Paketen.


\section{verwendete Entwurfsmuster}

\subsubsection{Schablonenmethode für SignInHelper}
Die verschiedenen Anmelde-Aktivitäten aller Loginhelper-Klassen können über die signIn()-Methode angesto"sen werden. Der spezifische Ablauf der Anmelde-Aktivität wird in den Unterklassen durch die primitiven Methoden definiert. \\

\textbf{beteiligte Klassen:}
\begin{itemize}
	\item SignInHelper: besitzt die Methode signIn(), die als Schablonenmethode dient und bei der Ausführung die primitiven Methoden configureSignIn() und startSignInProcess() aufruft
	\item FirebaseSignInHelper: Unterklasse von SignInHelper, die die primitiven Methoden configureSignIn() und startSignInProcess() implementiert
	\item GoSignInHelper: Unterklasse von SignInHelper, die die primitiven Methoden configureSignIn() und startSignInProcess() implementiert
\end{itemize}

\subsubsection{Brücke}
Die verschiedenen RecyclerView-Adapter, die das Layout eines RecyclerViews definieren, können mit die dargestellten Daten aus verschiedenen ListItems beziehen.

\textbf{beteiligte Klassen:}
\begin{itemize}
	\item ListItem<T>
	\item ListAdapter
\end{itemize}

\subsubsection{DAO Pattern für Datenpersistenz auf dem Server}

\subsubsection{Observer Pattern zum Beobachten der DAO-Implementationen}

\chapter{Klassenbeschreibungen} {

% ------- textdoclet_include/intro.tex end

\section*{Class Hierarchy}{
\thispagestyle{empty}
\markboth{Class Hierarchy}{Class Hierarchy}
\addcontentsline{toc}{section}{Class Hierarchy}
\subsection*{Classes}
{\raggedright
\hspace{0.0cm} $\bullet$ java.lang.Object {\tiny \refdefined{java.lang.Object}} \\
\hspace{1.0cm} $\bullet$  {\tiny } \\
\hspace{2.0cm} $\bullet$ edu.kit.pse17.go\_app.view.recyclerView.adapter.ListAdapter {\tiny \refdefined{edu.kit.pse17.go_app.view.recyclerView.adapter.ListAdapter}} \\
\hspace{3.0cm} $\bullet$ edu.kit.pse17.go\_app.view.recyclerView.adapter.GOListAdapter {\tiny \refdefined{edu.kit.pse17.go_app.view.recyclerView.adapter.GOListAdapter}} \\
\hspace{3.0cm} $\bullet$ edu.kit.pse17.go\_app.view.recyclerView.adapter.GroupListAdapter {\tiny \refdefined{edu.kit.pse17.go_app.view.recyclerView.adapter.GroupListAdapter}} \\
\hspace{3.0cm} $\bullet$ edu.kit.pse17.go\_app.view.recyclerView.adapter.UserListAdapter {\tiny \refdefined{edu.kit.pse17.go_app.view.recyclerView.adapter.UserListAdapter}} \\
\hspace{1.0cm} $\bullet$ AppCompatActivity {\tiny } \\
\hspace{2.0cm} $\bullet$ edu.kit.pse17.go\_app.controller.login.SignInHelper {\tiny \refdefined{edu.kit.pse17.go_app.controller.login.SignInHelper}} \\
\hspace{3.0cm} $\bullet$ edu.kit.pse17.go\_app.controller.login.FirebaseSignInHelper {\tiny \refdefined{edu.kit.pse17.go_app.controller.login.FirebaseSignInHelper}} \\
\hspace{3.0cm} $\bullet$ edu.kit.pse17.go\_app.controller.login.GoSignInHelper {\tiny \refdefined{edu.kit.pse17.go_app.controller.login.GoSignInHelper}} \\
\hspace{2.0cm} $\bullet$ edu.kit.pse17.go\_app.view.BaseActivity {\tiny \refdefined{edu.kit.pse17.go_app.view.BaseActivity}} \\
\hspace{3.0cm} $\bullet$ edu.kit.pse17.go\_app.view.GoDetailActivity {\tiny \refdefined{edu.kit.pse17.go_app.view.GoDetailActivity}} \\
\hspace{4.0cm} $\bullet$ edu.kit.pse17.go\_app.view.GoDetailActivityOwner {\tiny \refdefined{edu.kit.pse17.go_app.view.GoDetailActivityOwner}} \\
\hspace{3.0cm} $\bullet$ edu.kit.pse17.go\_app.view.GroupDetailActivity {\tiny \refdefined{edu.kit.pse17.go_app.view.GroupDetailActivity}} \\
\hspace{4.0cm} $\bullet$ edu.kit.pse17.go\_app.view.GroupDetailActivityAdmin {\tiny \refdefined{edu.kit.pse17.go_app.view.GroupDetailActivityAdmin}} \\
\hspace{3.0cm} $\bullet$ edu.kit.pse17.go\_app.view.GroupListActivity {\tiny \refdefined{edu.kit.pse17.go_app.view.GroupListActivity}} \\
\hspace{3.0cm} $\bullet$ edu.kit.pse17.go\_app.view.InformationActivity {\tiny \refdefined{edu.kit.pse17.go_app.view.InformationActivity}} \\
\hspace{3.0cm} $\bullet$ edu.kit.pse17.go\_app.view.SettingsActivity {\tiny \refdefined{edu.kit.pse17.go_app.view.SettingsActivity}} \\
\hspace{3.0cm} $\bullet$ edu.kit.pse17.go\_app.view.SignInActivity {\tiny \refdefined{edu.kit.pse17.go_app.view.SignInActivity}} \\
\hspace{1.0cm} $\bullet$ FirebaseInstanceIdService {\tiny } \\
\hspace{2.0cm} $\bullet$ edu.kit.pse17.go\_app.serverCommunication.downstream.TokenService {\tiny \refdefined{edu.kit.pse17.go_app.serverCommunication.downstream.TokenService}} \\
\hspace{1.0cm} $\bullet$ FirebaseMessagingService {\tiny } \\
\hspace{2.0cm} $\bullet$ edu.kit.pse17.go\_app.serverCommunication.downstream.MessagingService {\tiny \refdefined{edu.kit.pse17.go_app.serverCommunication.downstream.MessagingService}} \\
\hspace{1.0cm} $\bullet$ ViewHolder {\tiny } \\
\hspace{2.0cm} $\bullet$ edu.kit.pse17.go\_app.view.recyclerView.ListViewHolder {\tiny \refdefined{edu.kit.pse17.go_app.view.recyclerView.ListViewHolder}} \\
\hspace{1.0cm} $\bullet$ edu.kit.pse17.go\_app.ClientCommunication.Downstream.FcmClient {\tiny \refdefined{edu.kit.pse17.go_app.ClientCommunication.Downstream.FcmClient}} \\
\hspace{1.0cm} $\bullet$ edu.kit.pse17.go\_app.ClientCommunication.Upstream.GroupRestController {\tiny \refdefined{edu.kit.pse17.go_app.ClientCommunication.Upstream.GroupRestController}} \\
\hspace{1.0cm} $\bullet$ edu.kit.pse17.go\_app.Main {\tiny \refdefined{edu.kit.pse17.go_app.Main}} \\
\hspace{1.0cm} $\bullet$ edu.kit.pse17.go\_app.PersistenceLayer.GoEntity {\tiny \refdefined{edu.kit.pse17.go_app.PersistenceLayer.GoEntity}} \\
\hspace{1.0cm} $\bullet$ edu.kit.pse17.go\_app.PersistenceLayer.GroupEntity {\tiny \refdefined{edu.kit.pse17.go_app.PersistenceLayer.GroupEntity}} \\
\hspace{1.0cm} $\bullet$ edu.kit.pse17.go\_app.PersistenceLayer.UserEntity {\tiny \refdefined{edu.kit.pse17.go_app.PersistenceLayer.UserEntity}} \\
\hspace{1.0cm} $\bullet$ edu.kit.pse17.go\_app.PersistenceLayer.daos.AbstractDao {\tiny \refdefined{edu.kit.pse17.go_app.PersistenceLayer.daos.AbstractDao}} \\
\hspace{2.0cm} $\bullet$ edu.kit.pse17.go\_app.PersistenceLayer.daos.GoDaoImp {\tiny \refdefined{edu.kit.pse17.go_app.PersistenceLayer.daos.GoDaoImp}} \\
\hspace{2.0cm} $\bullet$ edu.kit.pse17.go\_app.PersistenceLayer.daos.GroupDaoImp {\tiny \refdefined{edu.kit.pse17.go_app.PersistenceLayer.daos.GroupDaoImp}} \\
\hspace{2.0cm} $\bullet$ edu.kit.pse17.go\_app.PersistenceLayer.daos.UserDaoImp {\tiny \refdefined{edu.kit.pse17.go_app.PersistenceLayer.daos.UserDaoImp}} \\
\hspace{1.0cm} $\bullet$ edu.kit.pse17.go\_app.ServiceLayer.GoService {\tiny \refdefined{edu.kit.pse17.go_app.ServiceLayer.GoService}} \\
\hspace{1.0cm} $\bullet$ edu.kit.pse17.go\_app.ServiceLayer.GroupService {\tiny \refdefined{edu.kit.pse17.go_app.ServiceLayer.GroupService}} \\
\hspace{1.0cm} $\bullet$ edu.kit.pse17.go\_app.ServiceLayer.LocationService {\tiny \refdefined{edu.kit.pse17.go_app.ServiceLayer.LocationService}} \\
\hspace{1.0cm} $\bullet$ edu.kit.pse17.go\_app.ServiceLayer.Observer {\tiny \refdefined{edu.kit.pse17.go_app.ServiceLayer.Observer}} \\
\hspace{2.0cm} $\bullet$ edu.kit.pse17.go\_app.ServiceLayer.GoObserver {\tiny \refdefined{edu.kit.pse17.go_app.ServiceLayer.GoObserver}} \\
\hspace{2.0cm} $\bullet$ edu.kit.pse17.go\_app.ServiceLayer.GroupObserver {\tiny \refdefined{edu.kit.pse17.go_app.ServiceLayer.GroupObserver}} \\
\hspace{2.0cm} $\bullet$ edu.kit.pse17.go\_app.ServiceLayer.UserObserver {\tiny \refdefined{edu.kit.pse17.go_app.ServiceLayer.UserObserver}} \\
\hspace{1.0cm} $\bullet$ edu.kit.pse17.go\_app.ServiceLayer.UserService {\tiny \refdefined{edu.kit.pse17.go_app.ServiceLayer.UserService}} \\
\hspace{1.0cm} $\bullet$ edu.kit.pse17.go\_app.controller.ServerMessageReceivedEvent {\tiny \refdefined{edu.kit.pse17.go_app.controller.ServerMessageReceivedEvent}} \\
\hspace{1.0cm} $\bullet$ edu.kit.pse17.go\_app.model.GO {\tiny \refdefined{edu.kit.pse17.go_app.model.GO}} \\
\hspace{1.0cm} $\bullet$ edu.kit.pse17.go\_app.model.Group {\tiny \refdefined{edu.kit.pse17.go_app.model.Group}} \\
\hspace{1.0cm} $\bullet$ edu.kit.pse17.go\_app.model.GroupLocation {\tiny \refdefined{edu.kit.pse17.go_app.model.GroupLocation}} \\
\hspace{1.0cm} $\bullet$ edu.kit.pse17.go\_app.model.User {\tiny \refdefined{edu.kit.pse17.go_app.model.User}} \\
\hspace{1.0cm} $\bullet$ edu.kit.pse17.go\_app.serverCommunication.upstream.RestMessagingService {\tiny \refdefined{edu.kit.pse17.go_app.serverCommunication.upstream.RestMessagingService}} \\
\hspace{1.0cm} $\bullet$ edu.kit.pse17.go\_app.view.recyclerView.listItems.GOListItem {\tiny \refdefined{edu.kit.pse17.go_app.view.recyclerView.listItems.GOListItem}} \\
\hspace{1.0cm} $\bullet$ edu.kit.pse17.go\_app.view.recyclerView.listItems.GroupListItem {\tiny \refdefined{edu.kit.pse17.go_app.view.recyclerView.listItems.GroupListItem}} \\
\hspace{1.0cm} $\bullet$ edu.kit.pse17.go\_app.view.recyclerView.listItems.UserMailListItem {\tiny \refdefined{edu.kit.pse17.go_app.view.recyclerView.listItems.UserMailListItem}} \\
\hspace{1.0cm} $\bullet$ edu.kit.pse17.go\_app.view.recyclerView.listItems.UserStatusListItem {\tiny \refdefined{edu.kit.pse17.go_app.view.recyclerView.listItems.UserStatusListItem}} \\
\hspace{1.0cm} $\bullet$ java.lang.Enum {\tiny \refdefined{java.lang.Enum}} \\
\hspace{2.0cm} $\bullet$ edu.kit.pse17.go\_app.model.Status {\tiny \refdefined{edu.kit.pse17.go_app.model.Status}} \\
}
\subsection*{Interfaces}
\hspace{0.0cm} $\bullet$ edu.kit.pse17.go\_app.ClientCommunication.Downstream.FcmApi {\tiny \refdefined{edu.kit.pse17.go_app.ClientCommunication.Downstream.FcmApi}} \\
\hspace{0.0cm} $\bullet$ edu.kit.pse17.go\_app.PersistenceLayer.daos.GoDao {\tiny \refdefined{edu.kit.pse17.go_app.PersistenceLayer.daos.GoDao}} \\
\hspace{0.0cm} $\bullet$ edu.kit.pse17.go\_app.PersistenceLayer.daos.GroupDao {\tiny \refdefined{edu.kit.pse17.go_app.PersistenceLayer.daos.GroupDao}} \\
\hspace{0.0cm} $\bullet$ edu.kit.pse17.go\_app.PersistenceLayer.daos.UserDao {\tiny \refdefined{edu.kit.pse17.go_app.PersistenceLayer.daos.UserDao}} \\
\hspace{0.0cm} $\bullet$ edu.kit.pse17.go\_app.ServiceLayer.Observable {\tiny \refdefined{edu.kit.pse17.go_app.ServiceLayer.Observable}} \\
\hspace{0.0cm} $\bullet$ edu.kit.pse17.go\_app.serverCommunication.upstream.TomcatRestApi {\tiny \refdefined{edu.kit.pse17.go_app.serverCommunication.upstream.TomcatRestApi}} \\
\hspace{0.0cm} $\bullet$ edu.kit.pse17.go\_app.view.recyclerView.OnListItemClicked {\tiny \refdefined{edu.kit.pse17.go_app.view.recyclerView.OnListItemClicked}} \\
\hspace{0.0cm} $\bullet$ edu.kit.pse17.go\_app.view.recyclerView.listItems.ListItem {\tiny \refdefined{edu.kit.pse17.go_app.view.recyclerView.listItems.ListItem}} \\
}
\section{Package edu.kit.pse17.go\_app.PersistenceLayer}{
\label{edu.kit.pse17.go_app.PersistenceLayer}\hypertarget{edu.kit.pse17.go_app.PersistenceLayer}{}
\hskip -.05in
\hbox to \hsize{\textit{ Package Contents\hfil Page}}
\vskip .13in
\hbox{{\bf  Classes}}
\entityintro{GoEntity}{edu.kit.pse17.go_app.PersistenceLayer.GoEntity}{Created by tina on 30.06.17.}
\entityintro{GroupEntity}{edu.kit.pse17.go_app.PersistenceLayer.GroupEntity}{Created by tina on 30.06.17.}
\entityintro{UserEntity}{edu.kit.pse17.go_app.PersistenceLayer.UserEntity}{Created by tina on 30.06.17.}
\vskip .1in
\vskip .1in
\subsection{\label{edu.kit.pse17.go_app.PersistenceLayer.GoEntity}Class GoEntity}{
\hypertarget{edu.kit.pse17.go_app.PersistenceLayer.GoEntity}{}\vskip .1in 
Created by tina on 30.06.17.\vskip .1in 
\subsubsection{Declaration}{
\begin{lstlisting}[frame=none]
public class GoEntity
 extends java.lang.Object\end{lstlisting}
\subsubsection{Constructor summary}{
\begin{verse}
\hyperlink{edu.kit.pse17.go_app.PersistenceLayer.GoEntity()}{{\bf GoEntity()}} \\
\end{verse}
}
\subsubsection{Method summary}{
\begin{verse}
\hyperlink{edu.kit.pse17.go_app.PersistenceLayer.GoEntity.equals(java.lang.Object)}{{\bf equals(Object)}} \\
\hyperlink{edu.kit.pse17.go_app.PersistenceLayer.GoEntity.getDescription()}{{\bf getDescription()}} \\
\hyperlink{edu.kit.pse17.go_app.PersistenceLayer.GoEntity.getEnd()}{{\bf getEnd()}} \\
\hyperlink{edu.kit.pse17.go_app.PersistenceLayer.GoEntity.getGoingUsers()}{{\bf getGoingUsers()}} \\
\hyperlink{edu.kit.pse17.go_app.PersistenceLayer.GoEntity.getGoneUsers()}{{\bf getGoneUsers()}} \\
\hyperlink{edu.kit.pse17.go_app.PersistenceLayer.GoEntity.getID()}{{\bf getID()}} \\
\hyperlink{edu.kit.pse17.go_app.PersistenceLayer.GoEntity.getLat()}{{\bf getLat()}} \\
\hyperlink{edu.kit.pse17.go_app.PersistenceLayer.GoEntity.getLon()}{{\bf getLon()}} \\
\hyperlink{edu.kit.pse17.go_app.PersistenceLayer.GoEntity.getName()}{{\bf getName()}} \\
\hyperlink{edu.kit.pse17.go_app.PersistenceLayer.GoEntity.getNotGoingUsers()}{{\bf getNotGoingUsers()}} \\
\hyperlink{edu.kit.pse17.go_app.PersistenceLayer.GoEntity.getStart()}{{\bf getStart()}} \\
\hyperlink{edu.kit.pse17.go_app.PersistenceLayer.GoEntity.hashCode()}{{\bf hashCode()}} \\
\hyperlink{edu.kit.pse17.go_app.PersistenceLayer.GoEntity.setDescription(java.lang.String)}{{\bf setDescription(String)}} \\
\hyperlink{edu.kit.pse17.go_app.PersistenceLayer.GoEntity.setEnd(java.util.Date)}{{\bf setEnd(Date)}} \\
\hyperlink{edu.kit.pse17.go_app.PersistenceLayer.GoEntity.setGoingUsers(java.util.List)}{{\bf setGoingUsers(List)}} \\
\hyperlink{edu.kit.pse17.go_app.PersistenceLayer.GoEntity.setGoneUsers(java.util.List)}{{\bf setGoneUsers(List)}} \\
\hyperlink{edu.kit.pse17.go_app.PersistenceLayer.GoEntity.setID(long)}{{\bf setID(long)}} \\
\hyperlink{edu.kit.pse17.go_app.PersistenceLayer.GoEntity.setLat(long)}{{\bf setLat(long)}} \\
\hyperlink{edu.kit.pse17.go_app.PersistenceLayer.GoEntity.setLon(long)}{{\bf setLon(long)}} \\
\hyperlink{edu.kit.pse17.go_app.PersistenceLayer.GoEntity.setName(java.lang.String)}{{\bf setName(String)}} \\
\hyperlink{edu.kit.pse17.go_app.PersistenceLayer.GoEntity.setNotGoingUsers(java.util.List)}{{\bf setNotGoingUsers(List)}} \\
\hyperlink{edu.kit.pse17.go_app.PersistenceLayer.GoEntity.setStart(java.util.Date)}{{\bf setStart(Date)}} \\
\end{verse}
}
\subsubsection{Constructors}{
\vskip -2em
\begin{itemize}
\item{ 
\index{GoEntity()}
\hypertarget{edu.kit.pse17.go_app.PersistenceLayer.GoEntity()}{{\bf  GoEntity}\\}
\begin{lstlisting}[frame=none]
public GoEntity()\end{lstlisting} %end signature
}%end item
\end{itemize}
}
\subsubsection{Methods}{
\vskip -2em
\begin{itemize}
\item{ 
\index{equals(Object)}
\hypertarget{edu.kit.pse17.go_app.PersistenceLayer.GoEntity.equals(java.lang.Object)}{{\bf  equals}\\}
\begin{lstlisting}[frame=none]
public boolean equals(java.lang.Object arg0)\end{lstlisting} %end signature
}%end item
\item{ 
\index{getDescription()}
\hypertarget{edu.kit.pse17.go_app.PersistenceLayer.GoEntity.getDescription()}{{\bf  getDescription}\\}
\begin{lstlisting}[frame=none]
public java.lang.String getDescription()\end{lstlisting} %end signature
}%end item
\item{ 
\index{getEnd()}
\hypertarget{edu.kit.pse17.go_app.PersistenceLayer.GoEntity.getEnd()}{{\bf  getEnd}\\}
\begin{lstlisting}[frame=none]
public java.util.Date getEnd()\end{lstlisting} %end signature
}%end item
\item{ 
\index{getGoingUsers()}
\hypertarget{edu.kit.pse17.go_app.PersistenceLayer.GoEntity.getGoingUsers()}{{\bf  getGoingUsers}\\}
\begin{lstlisting}[frame=none]
public java.util.List getGoingUsers()\end{lstlisting} %end signature
}%end item
\item{ 
\index{getGoneUsers()}
\hypertarget{edu.kit.pse17.go_app.PersistenceLayer.GoEntity.getGoneUsers()}{{\bf  getGoneUsers}\\}
\begin{lstlisting}[frame=none]
public java.util.List getGoneUsers()\end{lstlisting} %end signature
}%end item
\item{ 
\index{getID()}
\hypertarget{edu.kit.pse17.go_app.PersistenceLayer.GoEntity.getID()}{{\bf  getID}\\}
\begin{lstlisting}[frame=none]
public long getID()\end{lstlisting} %end signature
}%end item
\item{ 
\index{getLat()}
\hypertarget{edu.kit.pse17.go_app.PersistenceLayer.GoEntity.getLat()}{{\bf  getLat}\\}
\begin{lstlisting}[frame=none]
public long getLat()\end{lstlisting} %end signature
}%end item
\item{ 
\index{getLon()}
\hypertarget{edu.kit.pse17.go_app.PersistenceLayer.GoEntity.getLon()}{{\bf  getLon}\\}
\begin{lstlisting}[frame=none]
public long getLon()\end{lstlisting} %end signature
}%end item
\item{ 
\index{getName()}
\hypertarget{edu.kit.pse17.go_app.PersistenceLayer.GoEntity.getName()}{{\bf  getName}\\}
\begin{lstlisting}[frame=none]
public java.lang.String getName()\end{lstlisting} %end signature
}%end item
\item{ 
\index{getNotGoingUsers()}
\hypertarget{edu.kit.pse17.go_app.PersistenceLayer.GoEntity.getNotGoingUsers()}{{\bf  getNotGoingUsers}\\}
\begin{lstlisting}[frame=none]
public java.util.List getNotGoingUsers()\end{lstlisting} %end signature
}%end item
\item{ 
\index{getStart()}
\hypertarget{edu.kit.pse17.go_app.PersistenceLayer.GoEntity.getStart()}{{\bf  getStart}\\}
\begin{lstlisting}[frame=none]
public java.util.Date getStart()\end{lstlisting} %end signature
}%end item
\item{ 
\index{hashCode()}
\hypertarget{edu.kit.pse17.go_app.PersistenceLayer.GoEntity.hashCode()}{{\bf  hashCode}\\}
\begin{lstlisting}[frame=none]
public native int hashCode()\end{lstlisting} %end signature
}%end item
\item{ 
\index{setDescription(String)}
\hypertarget{edu.kit.pse17.go_app.PersistenceLayer.GoEntity.setDescription(java.lang.String)}{{\bf  setDescription}\\}
\begin{lstlisting}[frame=none]
public void setDescription(java.lang.String description)\end{lstlisting} %end signature
}%end item
\item{ 
\index{setEnd(Date)}
\hypertarget{edu.kit.pse17.go_app.PersistenceLayer.GoEntity.setEnd(java.util.Date)}{{\bf  setEnd}\\}
\begin{lstlisting}[frame=none]
public void setEnd(java.util.Date end)\end{lstlisting} %end signature
}%end item
\item{ 
\index{setGoingUsers(List)}
\hypertarget{edu.kit.pse17.go_app.PersistenceLayer.GoEntity.setGoingUsers(java.util.List)}{{\bf  setGoingUsers}\\}
\begin{lstlisting}[frame=none]
public void setGoingUsers(java.util.List goingUsers)\end{lstlisting} %end signature
}%end item
\item{ 
\index{setGoneUsers(List)}
\hypertarget{edu.kit.pse17.go_app.PersistenceLayer.GoEntity.setGoneUsers(java.util.List)}{{\bf  setGoneUsers}\\}
\begin{lstlisting}[frame=none]
public void setGoneUsers(java.util.List goneUsers)\end{lstlisting} %end signature
}%end item
\item{ 
\index{setID(long)}
\hypertarget{edu.kit.pse17.go_app.PersistenceLayer.GoEntity.setID(long)}{{\bf  setID}\\}
\begin{lstlisting}[frame=none]
public void setID(long ID)\end{lstlisting} %end signature
}%end item
\item{ 
\index{setLat(long)}
\hypertarget{edu.kit.pse17.go_app.PersistenceLayer.GoEntity.setLat(long)}{{\bf  setLat}\\}
\begin{lstlisting}[frame=none]
public void setLat(long lat)\end{lstlisting} %end signature
}%end item
\item{ 
\index{setLon(long)}
\hypertarget{edu.kit.pse17.go_app.PersistenceLayer.GoEntity.setLon(long)}{{\bf  setLon}\\}
\begin{lstlisting}[frame=none]
public void setLon(long lon)\end{lstlisting} %end signature
}%end item
\item{ 
\index{setName(String)}
\hypertarget{edu.kit.pse17.go_app.PersistenceLayer.GoEntity.setName(java.lang.String)}{{\bf  setName}\\}
\begin{lstlisting}[frame=none]
public void setName(java.lang.String name)\end{lstlisting} %end signature
}%end item
\item{ 
\index{setNotGoingUsers(List)}
\hypertarget{edu.kit.pse17.go_app.PersistenceLayer.GoEntity.setNotGoingUsers(java.util.List)}{{\bf  setNotGoingUsers}\\}
\begin{lstlisting}[frame=none]
public void setNotGoingUsers(java.util.List notGoingUsers)\end{lstlisting} %end signature
}%end item
\item{ 
\index{setStart(Date)}
\hypertarget{edu.kit.pse17.go_app.PersistenceLayer.GoEntity.setStart(java.util.Date)}{{\bf  setStart}\\}
\begin{lstlisting}[frame=none]
public void setStart(java.util.Date start)\end{lstlisting} %end signature
}%end item
\end{itemize}
}
}
\subsection{\label{edu.kit.pse17.go_app.PersistenceLayer.GroupEntity}Class GroupEntity}{
\hypertarget{edu.kit.pse17.go_app.PersistenceLayer.GroupEntity}{}\vskip .1in 
Created by tina on 30.06.17.\vskip .1in 
\subsubsection{Declaration}{
\begin{lstlisting}[frame=none]
public class GroupEntity
 extends java.lang.Object\end{lstlisting}
\subsubsection{Constructor summary}{
\begin{verse}
\hyperlink{edu.kit.pse17.go_app.PersistenceLayer.GroupEntity()}{{\bf GroupEntity()}} \\
\end{verse}
}
\subsubsection{Method summary}{
\begin{verse}
\hyperlink{edu.kit.pse17.go_app.PersistenceLayer.GroupEntity.equals(java.lang.Object)}{{\bf equals(Object)}} \\
\hyperlink{edu.kit.pse17.go_app.PersistenceLayer.GroupEntity.getAdmins()}{{\bf getAdmins()}} \\
\hyperlink{edu.kit.pse17.go_app.PersistenceLayer.GroupEntity.getDescription()}{{\bf getDescription()}} \\
\hyperlink{edu.kit.pse17.go_app.PersistenceLayer.GroupEntity.getID()}{{\bf getID()}} \\
\hyperlink{edu.kit.pse17.go_app.PersistenceLayer.GroupEntity.getImagePath()}{{\bf getImagePath()}} \\
\hyperlink{edu.kit.pse17.go_app.PersistenceLayer.GroupEntity.getMemberCount()}{{\bf getMemberCount()}} \\
\hyperlink{edu.kit.pse17.go_app.PersistenceLayer.GroupEntity.getMembers()}{{\bf getMembers()}} \\
\hyperlink{edu.kit.pse17.go_app.PersistenceLayer.GroupEntity.getName()}{{\bf getName()}} \\
\hyperlink{edu.kit.pse17.go_app.PersistenceLayer.GroupEntity.getRequests()}{{\bf getRequests()}} \\
\hyperlink{edu.kit.pse17.go_app.PersistenceLayer.GroupEntity.hashCode()}{{\bf hashCode()}} \\
\hyperlink{edu.kit.pse17.go_app.PersistenceLayer.GroupEntity.setAdmins(java.util.List)}{{\bf setAdmins(List)}} \\
\hyperlink{edu.kit.pse17.go_app.PersistenceLayer.GroupEntity.setDescription(java.lang.String)}{{\bf setDescription(String)}} \\
\hyperlink{edu.kit.pse17.go_app.PersistenceLayer.GroupEntity.setID(int)}{{\bf setID(int)}} \\
\hyperlink{edu.kit.pse17.go_app.PersistenceLayer.GroupEntity.setImagePath(java.lang.String)}{{\bf setImagePath(String)}} \\
\hyperlink{edu.kit.pse17.go_app.PersistenceLayer.GroupEntity.setMemberCount(int)}{{\bf setMemberCount(int)}} \\
\hyperlink{edu.kit.pse17.go_app.PersistenceLayer.GroupEntity.setMembers(java.util.List)}{{\bf setMembers(List)}} \\
\hyperlink{edu.kit.pse17.go_app.PersistenceLayer.GroupEntity.setName(java.lang.String)}{{\bf setName(String)}} \\
\hyperlink{edu.kit.pse17.go_app.PersistenceLayer.GroupEntity.setRequests(java.util.List)}{{\bf setRequests(List)}} \\
\end{verse}
}
\subsubsection{Constructors}{
\vskip -2em
\begin{itemize}
\item{ 
\index{GroupEntity()}
\hypertarget{edu.kit.pse17.go_app.PersistenceLayer.GroupEntity()}{{\bf  GroupEntity}\\}
\begin{lstlisting}[frame=none]
public GroupEntity()\end{lstlisting} %end signature
}%end item
\end{itemize}
}
\subsubsection{Methods}{
\vskip -2em
\begin{itemize}
\item{ 
\index{equals(Object)}
\hypertarget{edu.kit.pse17.go_app.PersistenceLayer.GroupEntity.equals(java.lang.Object)}{{\bf  equals}\\}
\begin{lstlisting}[frame=none]
public boolean equals(java.lang.Object arg0)\end{lstlisting} %end signature
}%end item
\item{ 
\index{getAdmins()}
\hypertarget{edu.kit.pse17.go_app.PersistenceLayer.GroupEntity.getAdmins()}{{\bf  getAdmins}\\}
\begin{lstlisting}[frame=none]
public java.util.List getAdmins()\end{lstlisting} %end signature
}%end item
\item{ 
\index{getDescription()}
\hypertarget{edu.kit.pse17.go_app.PersistenceLayer.GroupEntity.getDescription()}{{\bf  getDescription}\\}
\begin{lstlisting}[frame=none]
public java.lang.String getDescription()\end{lstlisting} %end signature
}%end item
\item{ 
\index{getID()}
\hypertarget{edu.kit.pse17.go_app.PersistenceLayer.GroupEntity.getID()}{{\bf  getID}\\}
\begin{lstlisting}[frame=none]
public int getID()\end{lstlisting} %end signature
}%end item
\item{ 
\index{getImagePath()}
\hypertarget{edu.kit.pse17.go_app.PersistenceLayer.GroupEntity.getImagePath()}{{\bf  getImagePath}\\}
\begin{lstlisting}[frame=none]
public java.lang.String getImagePath()\end{lstlisting} %end signature
}%end item
\item{ 
\index{getMemberCount()}
\hypertarget{edu.kit.pse17.go_app.PersistenceLayer.GroupEntity.getMemberCount()}{{\bf  getMemberCount}\\}
\begin{lstlisting}[frame=none]
public int getMemberCount()\end{lstlisting} %end signature
}%end item
\item{ 
\index{getMembers()}
\hypertarget{edu.kit.pse17.go_app.PersistenceLayer.GroupEntity.getMembers()}{{\bf  getMembers}\\}
\begin{lstlisting}[frame=none]
public java.util.List getMembers()\end{lstlisting} %end signature
}%end item
\item{ 
\index{getName()}
\hypertarget{edu.kit.pse17.go_app.PersistenceLayer.GroupEntity.getName()}{{\bf  getName}\\}
\begin{lstlisting}[frame=none]
public java.lang.String getName()\end{lstlisting} %end signature
}%end item
\item{ 
\index{getRequests()}
\hypertarget{edu.kit.pse17.go_app.PersistenceLayer.GroupEntity.getRequests()}{{\bf  getRequests}\\}
\begin{lstlisting}[frame=none]
public java.util.List getRequests()\end{lstlisting} %end signature
}%end item
\item{ 
\index{hashCode()}
\hypertarget{edu.kit.pse17.go_app.PersistenceLayer.GroupEntity.hashCode()}{{\bf  hashCode}\\}
\begin{lstlisting}[frame=none]
public native int hashCode()\end{lstlisting} %end signature
}%end item
\item{ 
\index{setAdmins(List)}
\hypertarget{edu.kit.pse17.go_app.PersistenceLayer.GroupEntity.setAdmins(java.util.List)}{{\bf  setAdmins}\\}
\begin{lstlisting}[frame=none]
public void setAdmins(java.util.List admins)\end{lstlisting} %end signature
}%end item
\item{ 
\index{setDescription(String)}
\hypertarget{edu.kit.pse17.go_app.PersistenceLayer.GroupEntity.setDescription(java.lang.String)}{{\bf  setDescription}\\}
\begin{lstlisting}[frame=none]
public void setDescription(java.lang.String description)\end{lstlisting} %end signature
}%end item
\item{ 
\index{setID(int)}
\hypertarget{edu.kit.pse17.go_app.PersistenceLayer.GroupEntity.setID(int)}{{\bf  setID}\\}
\begin{lstlisting}[frame=none]
public void setID(int ID)\end{lstlisting} %end signature
}%end item
\item{ 
\index{setImagePath(String)}
\hypertarget{edu.kit.pse17.go_app.PersistenceLayer.GroupEntity.setImagePath(java.lang.String)}{{\bf  setImagePath}\\}
\begin{lstlisting}[frame=none]
public void setImagePath(java.lang.String imagePath)\end{lstlisting} %end signature
}%end item
\item{ 
\index{setMemberCount(int)}
\hypertarget{edu.kit.pse17.go_app.PersistenceLayer.GroupEntity.setMemberCount(int)}{{\bf  setMemberCount}\\}
\begin{lstlisting}[frame=none]
public void setMemberCount(int memberCount)\end{lstlisting} %end signature
}%end item
\item{ 
\index{setMembers(List)}
\hypertarget{edu.kit.pse17.go_app.PersistenceLayer.GroupEntity.setMembers(java.util.List)}{{\bf  setMembers}\\}
\begin{lstlisting}[frame=none]
public void setMembers(java.util.List members)\end{lstlisting} %end signature
}%end item
\item{ 
\index{setName(String)}
\hypertarget{edu.kit.pse17.go_app.PersistenceLayer.GroupEntity.setName(java.lang.String)}{{\bf  setName}\\}
\begin{lstlisting}[frame=none]
public void setName(java.lang.String name)\end{lstlisting} %end signature
}%end item
\item{ 
\index{setRequests(List)}
\hypertarget{edu.kit.pse17.go_app.PersistenceLayer.GroupEntity.setRequests(java.util.List)}{{\bf  setRequests}\\}
\begin{lstlisting}[frame=none]
public void setRequests(java.util.List requests)\end{lstlisting} %end signature
}%end item
\end{itemize}
}
}
\subsection{\label{edu.kit.pse17.go_app.PersistenceLayer.UserEntity}Class UserEntity}{
\hypertarget{edu.kit.pse17.go_app.PersistenceLayer.UserEntity}{}\vskip .1in 
Created by tina on 30.06.17.\vskip .1in 
\subsubsection{Declaration}{
\begin{lstlisting}[frame=none]
public class UserEntity
 extends java.lang.Object\end{lstlisting}
\subsubsection{Constructor summary}{
\begin{verse}
\hyperlink{edu.kit.pse17.go_app.PersistenceLayer.UserEntity()}{{\bf UserEntity()}} \\
\end{verse}
}
\subsubsection{Method summary}{
\begin{verse}
\hyperlink{edu.kit.pse17.go_app.PersistenceLayer.UserEntity.equals(java.lang.Object)}{{\bf equals(Object)}} \\
\hyperlink{edu.kit.pse17.go_app.PersistenceLayer.UserEntity.getEmail()}{{\bf getEmail()}} \\
\hyperlink{edu.kit.pse17.go_app.PersistenceLayer.UserEntity.getImagePath()}{{\bf getImagePath()}} \\
\hyperlink{edu.kit.pse17.go_app.PersistenceLayer.UserEntity.getInstanceId()}{{\bf getInstanceId()}} \\
\hyperlink{edu.kit.pse17.go_app.PersistenceLayer.UserEntity.getName()}{{\bf getName()}} \\
\hyperlink{edu.kit.pse17.go_app.PersistenceLayer.UserEntity.getUid()}{{\bf getUid()}} \\
\hyperlink{edu.kit.pse17.go_app.PersistenceLayer.UserEntity.hashCode()}{{\bf hashCode()}} \\
\hyperlink{edu.kit.pse17.go_app.PersistenceLayer.UserEntity.setEmail(java.lang.String)}{{\bf setEmail(String)}} \\
\hyperlink{edu.kit.pse17.go_app.PersistenceLayer.UserEntity.setImagePath(java.lang.String)}{{\bf setImagePath(String)}} \\
\hyperlink{edu.kit.pse17.go_app.PersistenceLayer.UserEntity.setInstanceId(java.lang.String)}{{\bf setInstanceId(String)}} \\
\hyperlink{edu.kit.pse17.go_app.PersistenceLayer.UserEntity.setName(java.lang.String)}{{\bf setName(String)}} \\
\hyperlink{edu.kit.pse17.go_app.PersistenceLayer.UserEntity.setUid(java.lang.String)}{{\bf setUid(String)}} \\
\end{verse}
}
\subsubsection{Constructors}{
\vskip -2em
\begin{itemize}
\item{ 
\index{UserEntity()}
\hypertarget{edu.kit.pse17.go_app.PersistenceLayer.UserEntity()}{{\bf  UserEntity}\\}
\begin{lstlisting}[frame=none]
public UserEntity()\end{lstlisting} %end signature
}%end item
\end{itemize}
}
\subsubsection{Methods}{
\vskip -2em
\begin{itemize}
\item{ 
\index{equals(Object)}
\hypertarget{edu.kit.pse17.go_app.PersistenceLayer.UserEntity.equals(java.lang.Object)}{{\bf  equals}\\}
\begin{lstlisting}[frame=none]
public boolean equals(java.lang.Object arg0)\end{lstlisting} %end signature
}%end item
\item{ 
\index{getEmail()}
\hypertarget{edu.kit.pse17.go_app.PersistenceLayer.UserEntity.getEmail()}{{\bf  getEmail}\\}
\begin{lstlisting}[frame=none]
public java.lang.String getEmail()\end{lstlisting} %end signature
}%end item
\item{ 
\index{getImagePath()}
\hypertarget{edu.kit.pse17.go_app.PersistenceLayer.UserEntity.getImagePath()}{{\bf  getImagePath}\\}
\begin{lstlisting}[frame=none]
public java.lang.String getImagePath()\end{lstlisting} %end signature
}%end item
\item{ 
\index{getInstanceId()}
\hypertarget{edu.kit.pse17.go_app.PersistenceLayer.UserEntity.getInstanceId()}{{\bf  getInstanceId}\\}
\begin{lstlisting}[frame=none]
public java.lang.String getInstanceId()\end{lstlisting} %end signature
}%end item
\item{ 
\index{getName()}
\hypertarget{edu.kit.pse17.go_app.PersistenceLayer.UserEntity.getName()}{{\bf  getName}\\}
\begin{lstlisting}[frame=none]
public java.lang.String getName()\end{lstlisting} %end signature
}%end item
\item{ 
\index{getUid()}
\hypertarget{edu.kit.pse17.go_app.PersistenceLayer.UserEntity.getUid()}{{\bf  getUid}\\}
\begin{lstlisting}[frame=none]
public java.lang.String getUid()\end{lstlisting} %end signature
}%end item
\item{ 
\index{hashCode()}
\hypertarget{edu.kit.pse17.go_app.PersistenceLayer.UserEntity.hashCode()}{{\bf  hashCode}\\}
\begin{lstlisting}[frame=none]
public native int hashCode()\end{lstlisting} %end signature
}%end item
\item{ 
\index{setEmail(String)}
\hypertarget{edu.kit.pse17.go_app.PersistenceLayer.UserEntity.setEmail(java.lang.String)}{{\bf  setEmail}\\}
\begin{lstlisting}[frame=none]
public void setEmail(java.lang.String email)\end{lstlisting} %end signature
}%end item
\item{ 
\index{setImagePath(String)}
\hypertarget{edu.kit.pse17.go_app.PersistenceLayer.UserEntity.setImagePath(java.lang.String)}{{\bf  setImagePath}\\}
\begin{lstlisting}[frame=none]
public void setImagePath(java.lang.String imagePath)\end{lstlisting} %end signature
}%end item
\item{ 
\index{setInstanceId(String)}
\hypertarget{edu.kit.pse17.go_app.PersistenceLayer.UserEntity.setInstanceId(java.lang.String)}{{\bf  setInstanceId}\\}
\begin{lstlisting}[frame=none]
public void setInstanceId(java.lang.String instanceId)\end{lstlisting} %end signature
}%end item
\item{ 
\index{setName(String)}
\hypertarget{edu.kit.pse17.go_app.PersistenceLayer.UserEntity.setName(java.lang.String)}{{\bf  setName}\\}
\begin{lstlisting}[frame=none]
public void setName(java.lang.String name)\end{lstlisting} %end signature
}%end item
\item{ 
\index{setUid(String)}
\hypertarget{edu.kit.pse17.go_app.PersistenceLayer.UserEntity.setUid(java.lang.String)}{{\bf  setUid}\\}
\begin{lstlisting}[frame=none]
public void setUid(java.lang.String uid)\end{lstlisting} %end signature
}%end item
\end{itemize}
}
}
}
\section{Package edu.kit.pse17.go\_app.PersistenceLayer.daos}{
\label{edu.kit.pse17.go_app.PersistenceLayer.daos}\hypertarget{edu.kit.pse17.go_app.PersistenceLayer.daos}{}
\hskip -.05in
\hbox to \hsize{\textit{ Package Contents\hfil Page}}
\vskip .13in
\hbox{{\bf  Interfaces}}
\entityintro{GoDao}{edu.kit.pse17.go_app.PersistenceLayer.daos.GoDao}{Created by tina on 30.06.17.}
\entityintro{GroupDao}{edu.kit.pse17.go_app.PersistenceLayer.daos.GroupDao}{Created by tina on 30.06.17.}
\entityintro{UserDao}{edu.kit.pse17.go_app.PersistenceLayer.daos.UserDao}{Created by tina on 30.06.17.}
\vskip .13in
\hbox{{\bf  Classes}}
\entityintro{AbstractDao}{edu.kit.pse17.go_app.PersistenceLayer.daos.AbstractDao}{Abstrakte DAO-Klasse, die einfache CRUD-Methoden besitzt Created by tina on 30.06.17.}
\entityintro{GoDaoImp}{edu.kit.pse17.go_app.PersistenceLayer.daos.GoDaoImp}{Created by tina on 30.06.17.}
\entityintro{GroupDaoImp}{edu.kit.pse17.go_app.PersistenceLayer.daos.GroupDaoImp}{Created by tina on 30.06.17.}
\entityintro{UserDaoImp}{edu.kit.pse17.go_app.PersistenceLayer.daos.UserDaoImp}{Created by tina on 30.06.17.}
\vskip .1in
\vskip .1in
\subsection{\label{edu.kit.pse17.go_app.PersistenceLayer.daos.GoDao}Interface GoDao}{
\hypertarget{edu.kit.pse17.go_app.PersistenceLayer.daos.GoDao}{}\vskip .1in 
Created by tina on 30.06.17.\vskip .1in 
\subsubsection{Declaration}{
\begin{lstlisting}[frame=none]
public interface GoDao
\end{lstlisting}
\subsubsection{All known subinterfaces}{GoDaoImp\small{\refdefined{edu.kit.pse17.go_app.PersistenceLayer.daos.GoDaoImp}}, GroupDaoImp\small{\refdefined{edu.kit.pse17.go_app.PersistenceLayer.daos.GroupDaoImp}}}
\subsubsection{All classes known to implement interface}{GoDaoImp\small{\refdefined{edu.kit.pse17.go_app.PersistenceLayer.daos.GoDaoImp}}, GroupDaoImp\small{\refdefined{edu.kit.pse17.go_app.PersistenceLayer.daos.GroupDaoImp}}}
\subsubsection{Method summary}{
\begin{verse}
\hyperlink{edu.kit.pse17.go_app.PersistenceLayer.daos.GoDao.getActiveGosByGroup(long)}{{\bf getActiveGosByGroup(long)}} \\
\hyperlink{edu.kit.pse17.go_app.PersistenceLayer.daos.GoDao.getActiveGosByUser(java.lang.String)}{{\bf getActiveGosByUser(String)}} \\
\hyperlink{edu.kit.pse17.go_app.PersistenceLayer.daos.GoDao.getActiveUsers(long)}{{\bf getActiveUsers(long)}} \\
\hyperlink{edu.kit.pse17.go_app.PersistenceLayer.daos.GoDao.getAllGosByGroup(long)}{{\bf getAllGosByGroup(long)}} \\
\hyperlink{edu.kit.pse17.go_app.PersistenceLayer.daos.GoDao.getAllGosByUser(java.lang.String)}{{\bf getAllGosByUser(String)}} \\
\hyperlink{edu.kit.pse17.go_app.PersistenceLayer.daos.GoDao.getDeclinedusers(long)}{{\bf getDeclinedusers(long)}} \\
\hyperlink{edu.kit.pse17.go_app.PersistenceLayer.daos.GoDao.getGoingUsers(long)}{{\bf getGoingUsers(long)}} \\
\end{verse}
}
\subsubsection{Methods}{
\vskip -2em
\begin{itemize}
\item{ 
\index{getActiveGosByGroup(long)}
\hypertarget{edu.kit.pse17.go_app.PersistenceLayer.daos.GoDao.getActiveGosByGroup(long)}{{\bf  getActiveGosByGroup}\\}
\begin{lstlisting}[frame=none]
java.util.List getActiveGosByGroup(long id)\end{lstlisting} %end signature
}%end item
\item{ 
\index{getActiveGosByUser(String)}
\hypertarget{edu.kit.pse17.go_app.PersistenceLayer.daos.GoDao.getActiveGosByUser(java.lang.String)}{{\bf  getActiveGosByUser}\\}
\begin{lstlisting}[frame=none]
java.util.List getActiveGosByUser(java.lang.String uid)\end{lstlisting} %end signature
}%end item
\item{ 
\index{getActiveUsers(long)}
\hypertarget{edu.kit.pse17.go_app.PersistenceLayer.daos.GoDao.getActiveUsers(long)}{{\bf  getActiveUsers}\\}
\begin{lstlisting}[frame=none]
java.util.List getActiveUsers(long id)\end{lstlisting} %end signature
}%end item
\item{ 
\index{getAllGosByGroup(long)}
\hypertarget{edu.kit.pse17.go_app.PersistenceLayer.daos.GoDao.getAllGosByGroup(long)}{{\bf  getAllGosByGroup}\\}
\begin{lstlisting}[frame=none]
java.util.List getAllGosByGroup(long id)\end{lstlisting} %end signature
}%end item
\item{ 
\index{getAllGosByUser(String)}
\hypertarget{edu.kit.pse17.go_app.PersistenceLayer.daos.GoDao.getAllGosByUser(java.lang.String)}{{\bf  getAllGosByUser}\\}
\begin{lstlisting}[frame=none]
java.util.List getAllGosByUser(java.lang.String uid)\end{lstlisting} %end signature
}%end item
\item{ 
\index{getDeclinedusers(long)}
\hypertarget{edu.kit.pse17.go_app.PersistenceLayer.daos.GoDao.getDeclinedusers(long)}{{\bf  getDeclinedusers}\\}
\begin{lstlisting}[frame=none]
java.util.List getDeclinedusers(long id)\end{lstlisting} %end signature
}%end item
\item{ 
\index{getGoingUsers(long)}
\hypertarget{edu.kit.pse17.go_app.PersistenceLayer.daos.GoDao.getGoingUsers(long)}{{\bf  getGoingUsers}\\}
\begin{lstlisting}[frame=none]
java.util.List getGoingUsers(long id)\end{lstlisting} %end signature
}%end item
\end{itemize}
}
}
\subsection{\label{edu.kit.pse17.go_app.PersistenceLayer.daos.GroupDao}Interface GroupDao}{
\hypertarget{edu.kit.pse17.go_app.PersistenceLayer.daos.GroupDao}{}\vskip .1in 
Created by tina on 30.06.17.\vskip .1in 
\subsubsection{Declaration}{
\begin{lstlisting}[frame=none]
public interface GroupDao
\end{lstlisting}
\subsubsection{Method summary}{
\begin{verse}
\hyperlink{edu.kit.pse17.go_app.PersistenceLayer.daos.GroupDao.getAdmins(long)}{{\bf getAdmins(long)}} \\
\hyperlink{edu.kit.pse17.go_app.PersistenceLayer.daos.GroupDao.getGroupsByUser(java.lang.String)}{{\bf getGroupsByUser(String)}} \\
\hyperlink{edu.kit.pse17.go_app.PersistenceLayer.daos.GroupDao.getRequests(long)}{{\bf getRequests(long)}} \\
\hyperlink{edu.kit.pse17.go_app.PersistenceLayer.daos.GroupDao.getRequestsbyUser(java.lang.String)}{{\bf getRequestsbyUser(String)}} \\
\end{verse}
}
\subsubsection{Methods}{
\vskip -2em
\begin{itemize}
\item{ 
\index{getAdmins(long)}
\hypertarget{edu.kit.pse17.go_app.PersistenceLayer.daos.GroupDao.getAdmins(long)}{{\bf  getAdmins}\\}
\begin{lstlisting}[frame=none]
java.util.List getAdmins(long id)\end{lstlisting} %end signature
}%end item
\item{ 
\index{getGroupsByUser(String)}
\hypertarget{edu.kit.pse17.go_app.PersistenceLayer.daos.GroupDao.getGroupsByUser(java.lang.String)}{{\bf  getGroupsByUser}\\}
\begin{lstlisting}[frame=none]
java.util.List getGroupsByUser(java.lang.String uid)\end{lstlisting} %end signature
}%end item
\item{ 
\index{getRequests(long)}
\hypertarget{edu.kit.pse17.go_app.PersistenceLayer.daos.GroupDao.getRequests(long)}{{\bf  getRequests}\\}
\begin{lstlisting}[frame=none]
java.util.List getRequests(long id)\end{lstlisting} %end signature
}%end item
\item{ 
\index{getRequestsbyUser(String)}
\hypertarget{edu.kit.pse17.go_app.PersistenceLayer.daos.GroupDao.getRequestsbyUser(java.lang.String)}{{\bf  getRequestsbyUser}\\}
\begin{lstlisting}[frame=none]
java.util.List getRequestsbyUser(java.lang.String uid)\end{lstlisting} %end signature
}%end item
\end{itemize}
}
}
\subsection{\label{edu.kit.pse17.go_app.PersistenceLayer.daos.UserDao}Interface UserDao}{
\hypertarget{edu.kit.pse17.go_app.PersistenceLayer.daos.UserDao}{}\vskip .1in 
Created by tina on 30.06.17.\vskip .1in 
\subsubsection{Declaration}{
\begin{lstlisting}[frame=none]
public interface UserDao
\end{lstlisting}
\subsubsection{All known subinterfaces}{UserDaoImp\small{\refdefined{edu.kit.pse17.go_app.PersistenceLayer.daos.UserDaoImp}}}
\subsubsection{All classes known to implement interface}{UserDaoImp\small{\refdefined{edu.kit.pse17.go_app.PersistenceLayer.daos.UserDaoImp}}}
\subsubsection{Method summary}{
\begin{verse}
\hyperlink{edu.kit.pse17.go_app.PersistenceLayer.daos.UserDao.getUserByEmail(java.lang.String)}{{\bf getUserByEmail(String)}} \\
\end{verse}
}
\subsubsection{Methods}{
\vskip -2em
\begin{itemize}
\item{ 
\index{getUserByEmail(String)}
\hypertarget{edu.kit.pse17.go_app.PersistenceLayer.daos.UserDao.getUserByEmail(java.lang.String)}{{\bf  getUserByEmail}\\}
\begin{lstlisting}[frame=none]
edu.kit.pse17.go_app.PersistenceLayer.UserEntity getUserByEmail(java.lang.String mail)\end{lstlisting} %end signature
}%end item
\end{itemize}
}
}
\subsection{\label{edu.kit.pse17.go_app.PersistenceLayer.daos.AbstractDao}Class AbstractDao}{
\hypertarget{edu.kit.pse17.go_app.PersistenceLayer.daos.AbstractDao}{}\vskip .1in 
Abstrakte DAO-Klasse, die einfache CRUD-Methoden besitzt Created by tina on 30.06.17.\vskip .1in 
\subsubsection{Declaration}{
\begin{lstlisting}[frame=none]
public abstract class AbstractDao
 extends java.lang.Object\end{lstlisting}
\subsubsection{All known subclasses}{GoDaoImp\small{\refdefined{edu.kit.pse17.go_app.PersistenceLayer.daos.GoDaoImp}}, UserDaoImp\small{\refdefined{edu.kit.pse17.go_app.PersistenceLayer.daos.UserDaoImp}}, GroupDaoImp\small{\refdefined{edu.kit.pse17.go_app.PersistenceLayer.daos.GroupDaoImp}}}
\subsubsection{Constructor summary}{
\begin{verse}
\hyperlink{edu.kit.pse17.go_app.PersistenceLayer.daos.AbstractDao()}{{\bf AbstractDao()}} \\
\end{verse}
}
\subsubsection{Method summary}{
\begin{verse}
\hyperlink{edu.kit.pse17.go_app.PersistenceLayer.daos.AbstractDao.create(T)}{{\bf create(T)}} \\
\hyperlink{edu.kit.pse17.go_app.PersistenceLayer.daos.AbstractDao.delete(T)}{{\bf delete(T)}} \\
\hyperlink{edu.kit.pse17.go_app.PersistenceLayer.daos.AbstractDao.getAll()}{{\bf getAll()}} \\
\hyperlink{edu.kit.pse17.go_app.PersistenceLayer.daos.AbstractDao.getById(PK)}{{\bf getById(PK)}} \\
\hyperlink{edu.kit.pse17.go_app.PersistenceLayer.daos.AbstractDao.update(T)}{{\bf update(T)}} \\
\end{verse}
}
\subsubsection{Constructors}{
\vskip -2em
\begin{itemize}
\item{ 
\index{AbstractDao()}
\hypertarget{edu.kit.pse17.go_app.PersistenceLayer.daos.AbstractDao()}{{\bf  AbstractDao}\\}
\begin{lstlisting}[frame=none]
public AbstractDao()\end{lstlisting} %end signature
}%end item
\end{itemize}
}
\subsubsection{Methods}{
\vskip -2em
\begin{itemize}
\item{ 
\index{create(T)}
\hypertarget{edu.kit.pse17.go_app.PersistenceLayer.daos.AbstractDao.create(T)}{{\bf  create}\\}
\begin{lstlisting}[frame=none]
public abstract java.lang.Object create(java.lang.Object t)\end{lstlisting} %end signature
}%end item
\item{ 
\index{delete(T)}
\hypertarget{edu.kit.pse17.go_app.PersistenceLayer.daos.AbstractDao.delete(T)}{{\bf  delete}\\}
\begin{lstlisting}[frame=none]
public abstract java.lang.Object delete(java.lang.Object t)\end{lstlisting} %end signature
}%end item
\item{ 
\index{getAll()}
\hypertarget{edu.kit.pse17.go_app.PersistenceLayer.daos.AbstractDao.getAll()}{{\bf  getAll}\\}
\begin{lstlisting}[frame=none]
public abstract java.util.List getAll()\end{lstlisting} %end signature
}%end item
\item{ 
\index{getById(PK)}
\hypertarget{edu.kit.pse17.go_app.PersistenceLayer.daos.AbstractDao.getById(PK)}{{\bf  getById}\\}
\begin{lstlisting}[frame=none]
public abstract java.lang.Object getById(java.io.Serializable id)\end{lstlisting} %end signature
}%end item
\item{ 
\index{update(T)}
\hypertarget{edu.kit.pse17.go_app.PersistenceLayer.daos.AbstractDao.update(T)}{{\bf  update}\\}
\begin{lstlisting}[frame=none]
public abstract java.lang.Object update(java.lang.Object t)\end{lstlisting} %end signature
}%end item
\end{itemize}
}
}
\subsection{\label{edu.kit.pse17.go_app.PersistenceLayer.daos.GoDaoImp}Class GoDaoImp}{
\hypertarget{edu.kit.pse17.go_app.PersistenceLayer.daos.GoDaoImp}{}\vskip .1in 
Created by tina on 30.06.17.\vskip .1in 
\subsubsection{Declaration}{
\begin{lstlisting}[frame=none]
public class GoDaoImp
 extends edu.kit.pse17.go_app.PersistenceLayer.daos.AbstractDao implements GoDao, java.io.Serializable, edu.kit.pse17.go_app.ServiceLayer.Observable\end{lstlisting}
\subsubsection{Constructor summary}{
\begin{verse}
\hyperlink{edu.kit.pse17.go_app.PersistenceLayer.daos.GoDaoImp()}{{\bf GoDaoImp()}} \\
\end{verse}
}
\subsubsection{Method summary}{
\begin{verse}
\hyperlink{edu.kit.pse17.go_app.PersistenceLayer.daos.GoDaoImp.create(edu.kit.pse17.go_app.PersistenceLayer.GoEntity)}{{\bf create(GoEntity)}} \\
\hyperlink{edu.kit.pse17.go_app.PersistenceLayer.daos.GoDaoImp.delete(edu.kit.pse17.go_app.PersistenceLayer.GoEntity)}{{\bf delete(GoEntity)}} \\
\hyperlink{edu.kit.pse17.go_app.PersistenceLayer.daos.GoDaoImp.getActiveGosByGroup(long)}{{\bf getActiveGosByGroup(long)}} \\
\hyperlink{edu.kit.pse17.go_app.PersistenceLayer.daos.GoDaoImp.getActiveGosByUser(java.lang.String)}{{\bf getActiveGosByUser(String)}} \\
\hyperlink{edu.kit.pse17.go_app.PersistenceLayer.daos.GoDaoImp.getActiveUsers(long)}{{\bf getActiveUsers(long)}} \\
\hyperlink{edu.kit.pse17.go_app.PersistenceLayer.daos.GoDaoImp.getAll()}{{\bf getAll()}} \\
\hyperlink{edu.kit.pse17.go_app.PersistenceLayer.daos.GoDaoImp.getAllGosByGroup(long)}{{\bf getAllGosByGroup(long)}} \\
\hyperlink{edu.kit.pse17.go_app.PersistenceLayer.daos.GoDaoImp.getAllGosByUser(java.lang.String)}{{\bf getAllGosByUser(String)}} \\
\hyperlink{edu.kit.pse17.go_app.PersistenceLayer.daos.GoDaoImp.getById(java.lang.Long)}{{\bf getById(Long)}} \\
\hyperlink{edu.kit.pse17.go_app.PersistenceLayer.daos.GoDaoImp.getDeclinedusers(long)}{{\bf getDeclinedusers(long)}} \\
\hyperlink{edu.kit.pse17.go_app.PersistenceLayer.daos.GoDaoImp.getGoingUsers(long)}{{\bf getGoingUsers(long)}} \\
\hyperlink{edu.kit.pse17.go_app.PersistenceLayer.daos.GoDaoImp.notify(edu.kit.pse17.go_app.PersistenceLayer.GoEntity)}{{\bf notify(GoEntity)}} \\
\hyperlink{edu.kit.pse17.go_app.PersistenceLayer.daos.GoDaoImp.register(edu.kit.pse17.go_app.ServiceLayer.Observer)}{{\bf register(Observer)}} \\
\hyperlink{edu.kit.pse17.go_app.PersistenceLayer.daos.GoDaoImp.unregister(edu.kit.pse17.go_app.ServiceLayer.Observer)}{{\bf unregister(Observer)}} \\
\hyperlink{edu.kit.pse17.go_app.PersistenceLayer.daos.GoDaoImp.update(edu.kit.pse17.go_app.PersistenceLayer.GoEntity)}{{\bf update(GoEntity)}} \\
\end{verse}
}
\subsubsection{Constructors}{
\vskip -2em
\begin{itemize}
\item{ 
\index{GoDaoImp()}
\hypertarget{edu.kit.pse17.go_app.PersistenceLayer.daos.GoDaoImp()}{{\bf  GoDaoImp}\\}
\begin{lstlisting}[frame=none]
public GoDaoImp()\end{lstlisting} %end signature
}%end item
\end{itemize}
}
\subsubsection{Methods}{
\vskip -2em
\begin{itemize}
\item{ 
\index{create(GoEntity)}
\hypertarget{edu.kit.pse17.go_app.PersistenceLayer.daos.GoDaoImp.create(edu.kit.pse17.go_app.PersistenceLayer.GoEntity)}{{\bf  create}\\}
\begin{lstlisting}[frame=none]
public edu.kit.pse17.go_app.PersistenceLayer.GoEntity create(edu.kit.pse17.go_app.PersistenceLayer.GoEntity goEntity)\end{lstlisting} %end signature
}%end item
\item{ 
\index{delete(GoEntity)}
\hypertarget{edu.kit.pse17.go_app.PersistenceLayer.daos.GoDaoImp.delete(edu.kit.pse17.go_app.PersistenceLayer.GoEntity)}{{\bf  delete}\\}
\begin{lstlisting}[frame=none]
public edu.kit.pse17.go_app.PersistenceLayer.GoEntity delete(edu.kit.pse17.go_app.PersistenceLayer.GoEntity goEntity)\end{lstlisting} %end signature
}%end item
\item{ 
\index{getActiveGosByGroup(long)}
\hypertarget{edu.kit.pse17.go_app.PersistenceLayer.daos.GoDaoImp.getActiveGosByGroup(long)}{{\bf  getActiveGosByGroup}\\}
\begin{lstlisting}[frame=none]
java.util.List getActiveGosByGroup(long id)\end{lstlisting} %end signature
}%end item
\item{ 
\index{getActiveGosByUser(String)}
\hypertarget{edu.kit.pse17.go_app.PersistenceLayer.daos.GoDaoImp.getActiveGosByUser(java.lang.String)}{{\bf  getActiveGosByUser}\\}
\begin{lstlisting}[frame=none]
java.util.List getActiveGosByUser(java.lang.String uid)\end{lstlisting} %end signature
}%end item
\item{ 
\index{getActiveUsers(long)}
\hypertarget{edu.kit.pse17.go_app.PersistenceLayer.daos.GoDaoImp.getActiveUsers(long)}{{\bf  getActiveUsers}\\}
\begin{lstlisting}[frame=none]
java.util.List getActiveUsers(long id)\end{lstlisting} %end signature
}%end item
\item{ 
\index{getAll()}
\hypertarget{edu.kit.pse17.go_app.PersistenceLayer.daos.GoDaoImp.getAll()}{{\bf  getAll}\\}
\begin{lstlisting}[frame=none]
public abstract java.util.List getAll()\end{lstlisting} %end signature
}%end item
\item{ 
\index{getAllGosByGroup(long)}
\hypertarget{edu.kit.pse17.go_app.PersistenceLayer.daos.GoDaoImp.getAllGosByGroup(long)}{{\bf  getAllGosByGroup}\\}
\begin{lstlisting}[frame=none]
java.util.List getAllGosByGroup(long id)\end{lstlisting} %end signature
}%end item
\item{ 
\index{getAllGosByUser(String)}
\hypertarget{edu.kit.pse17.go_app.PersistenceLayer.daos.GoDaoImp.getAllGosByUser(java.lang.String)}{{\bf  getAllGosByUser}\\}
\begin{lstlisting}[frame=none]
java.util.List getAllGosByUser(java.lang.String uid)\end{lstlisting} %end signature
}%end item
\item{ 
\index{getById(Long)}
\hypertarget{edu.kit.pse17.go_app.PersistenceLayer.daos.GoDaoImp.getById(java.lang.Long)}{{\bf  getById}\\}
\begin{lstlisting}[frame=none]
public edu.kit.pse17.go_app.PersistenceLayer.GoEntity getById(java.lang.Long id)\end{lstlisting} %end signature
}%end item
\item{ 
\index{getDeclinedusers(long)}
\hypertarget{edu.kit.pse17.go_app.PersistenceLayer.daos.GoDaoImp.getDeclinedusers(long)}{{\bf  getDeclinedusers}\\}
\begin{lstlisting}[frame=none]
java.util.List getDeclinedusers(long id)\end{lstlisting} %end signature
}%end item
\item{ 
\index{getGoingUsers(long)}
\hypertarget{edu.kit.pse17.go_app.PersistenceLayer.daos.GoDaoImp.getGoingUsers(long)}{{\bf  getGoingUsers}\\}
\begin{lstlisting}[frame=none]
java.util.List getGoingUsers(long id)\end{lstlisting} %end signature
}%end item
\item{ 
\index{notify(GoEntity)}
\hypertarget{edu.kit.pse17.go_app.PersistenceLayer.daos.GoDaoImp.notify(edu.kit.pse17.go_app.PersistenceLayer.GoEntity)}{{\bf  notify}\\}
\begin{lstlisting}[frame=none]
public void notify(edu.kit.pse17.go_app.PersistenceLayer.GoEntity goEntity)\end{lstlisting} %end signature
}%end item
\item{ 
\index{register(Observer)}
\hypertarget{edu.kit.pse17.go_app.PersistenceLayer.daos.GoDaoImp.register(edu.kit.pse17.go_app.ServiceLayer.Observer)}{{\bf  register}\\}
\begin{lstlisting}[frame=none]
void register(edu.kit.pse17.go_app.ServiceLayer.Observer observer)\end{lstlisting} %end signature
}%end item
\item{ 
\index{unregister(Observer)}
\hypertarget{edu.kit.pse17.go_app.PersistenceLayer.daos.GoDaoImp.unregister(edu.kit.pse17.go_app.ServiceLayer.Observer)}{{\bf  unregister}\\}
\begin{lstlisting}[frame=none]
void unregister(edu.kit.pse17.go_app.ServiceLayer.Observer observer)\end{lstlisting} %end signature
}%end item
\item{ 
\index{update(GoEntity)}
\hypertarget{edu.kit.pse17.go_app.PersistenceLayer.daos.GoDaoImp.update(edu.kit.pse17.go_app.PersistenceLayer.GoEntity)}{{\bf  update}\\}
\begin{lstlisting}[frame=none]
public edu.kit.pse17.go_app.PersistenceLayer.GoEntity update(edu.kit.pse17.go_app.PersistenceLayer.GoEntity goEntity)\end{lstlisting} %end signature
}%end item
\end{itemize}
}
\subsubsection{Members inherited from class AbstractDao }{
\texttt{edu.kit.pse17.go_app.PersistenceLayer.daos.AbstractDao} {\small 
\refdefined{edu.kit.pse17.go_app.PersistenceLayer.daos.AbstractDao}}
{\small 

\vskip -2em
\begin{itemize}
\item{\vskip -1.5ex 
\texttt{public abstract Object {\bf  create}(\texttt{java.lang.Object} {\bf  t})
}%end signature
}%end item
\item{\vskip -1.5ex 
\texttt{public abstract Object {\bf  delete}(\texttt{java.lang.Object} {\bf  t})
}%end signature
}%end item
\item{\vskip -1.5ex 
\texttt{public abstract List {\bf  getAll}()
}%end signature
}%end item
\item{\vskip -1.5ex 
\texttt{public abstract Object {\bf  getById}(\texttt{java.io.Serializable} {\bf  id})
}%end signature
}%end item
\item{\vskip -1.5ex 
\texttt{public abstract Object {\bf  update}(\texttt{java.lang.Object} {\bf  t})
}%end signature
}%end item
\end{itemize}
}
}
\subsection{\label{edu.kit.pse17.go_app.PersistenceLayer.daos.GroupDaoImp}Class GroupDaoImp}{
\hypertarget{edu.kit.pse17.go_app.PersistenceLayer.daos.GroupDaoImp}{}\vskip .1in 
Created by tina on 30.06.17.\vskip .1in 
\subsubsection{Declaration}{
\begin{lstlisting}[frame=none]
public class GroupDaoImp
 extends edu.kit.pse17.go_app.PersistenceLayer.daos.AbstractDao implements GoDao, java.io.Serializable, edu.kit.pse17.go_app.ServiceLayer.Observable\end{lstlisting}
\subsubsection{Constructor summary}{
\begin{verse}
\hyperlink{edu.kit.pse17.go_app.PersistenceLayer.daos.GroupDaoImp()}{{\bf GroupDaoImp()}} \\
\end{verse}
}
\subsubsection{Method summary}{
\begin{verse}
\hyperlink{edu.kit.pse17.go_app.PersistenceLayer.daos.GroupDaoImp.create(edu.kit.pse17.go_app.PersistenceLayer.GoEntity)}{{\bf create(GoEntity)}} \\
\hyperlink{edu.kit.pse17.go_app.PersistenceLayer.daos.GroupDaoImp.delete(edu.kit.pse17.go_app.PersistenceLayer.GoEntity)}{{\bf delete(GoEntity)}} \\
\hyperlink{edu.kit.pse17.go_app.PersistenceLayer.daos.GroupDaoImp.getActiveGosByGroup(long)}{{\bf getActiveGosByGroup(long)}} \\
\hyperlink{edu.kit.pse17.go_app.PersistenceLayer.daos.GroupDaoImp.getActiveGosByUser(java.lang.String)}{{\bf getActiveGosByUser(String)}} \\
\hyperlink{edu.kit.pse17.go_app.PersistenceLayer.daos.GroupDaoImp.getActiveUsers(long)}{{\bf getActiveUsers(long)}} \\
\hyperlink{edu.kit.pse17.go_app.PersistenceLayer.daos.GroupDaoImp.getAll()}{{\bf getAll()}} \\
\hyperlink{edu.kit.pse17.go_app.PersistenceLayer.daos.GroupDaoImp.getAllGosByGroup(long)}{{\bf getAllGosByGroup(long)}} \\
\hyperlink{edu.kit.pse17.go_app.PersistenceLayer.daos.GroupDaoImp.getAllGosByUser(java.lang.String)}{{\bf getAllGosByUser(String)}} \\
\hyperlink{edu.kit.pse17.go_app.PersistenceLayer.daos.GroupDaoImp.getById(java.lang.Long)}{{\bf getById(Long)}} \\
\hyperlink{edu.kit.pse17.go_app.PersistenceLayer.daos.GroupDaoImp.getDeclinedusers(long)}{{\bf getDeclinedusers(long)}} \\
\hyperlink{edu.kit.pse17.go_app.PersistenceLayer.daos.GroupDaoImp.getGoingUsers(long)}{{\bf getGoingUsers(long)}} \\
\hyperlink{edu.kit.pse17.go_app.PersistenceLayer.daos.GroupDaoImp.notify(edu.kit.pse17.go_app.PersistenceLayer.GroupEntity)}{{\bf notify(GroupEntity)}} \\
\hyperlink{edu.kit.pse17.go_app.PersistenceLayer.daos.GroupDaoImp.register(edu.kit.pse17.go_app.ServiceLayer.Observer)}{{\bf register(Observer)}} \\
\hyperlink{edu.kit.pse17.go_app.PersistenceLayer.daos.GroupDaoImp.unregister(edu.kit.pse17.go_app.ServiceLayer.Observer)}{{\bf unregister(Observer)}} \\
\hyperlink{edu.kit.pse17.go_app.PersistenceLayer.daos.GroupDaoImp.update(edu.kit.pse17.go_app.PersistenceLayer.GoEntity)}{{\bf update(GoEntity)}} \\
\end{verse}
}
\subsubsection{Constructors}{
\vskip -2em
\begin{itemize}
\item{ 
\index{GroupDaoImp()}
\hypertarget{edu.kit.pse17.go_app.PersistenceLayer.daos.GroupDaoImp()}{{\bf  GroupDaoImp}\\}
\begin{lstlisting}[frame=none]
public GroupDaoImp()\end{lstlisting} %end signature
}%end item
\end{itemize}
}
\subsubsection{Methods}{
\vskip -2em
\begin{itemize}
\item{ 
\index{create(GoEntity)}
\hypertarget{edu.kit.pse17.go_app.PersistenceLayer.daos.GroupDaoImp.create(edu.kit.pse17.go_app.PersistenceLayer.GoEntity)}{{\bf  create}\\}
\begin{lstlisting}[frame=none]
public edu.kit.pse17.go_app.PersistenceLayer.GoEntity create(edu.kit.pse17.go_app.PersistenceLayer.GoEntity goEntity)\end{lstlisting} %end signature
}%end item
\item{ 
\index{delete(GoEntity)}
\hypertarget{edu.kit.pse17.go_app.PersistenceLayer.daos.GroupDaoImp.delete(edu.kit.pse17.go_app.PersistenceLayer.GoEntity)}{{\bf  delete}\\}
\begin{lstlisting}[frame=none]
public edu.kit.pse17.go_app.PersistenceLayer.GoEntity delete(edu.kit.pse17.go_app.PersistenceLayer.GoEntity goEntity)\end{lstlisting} %end signature
}%end item
\item{ 
\index{getActiveGosByGroup(long)}
\hypertarget{edu.kit.pse17.go_app.PersistenceLayer.daos.GroupDaoImp.getActiveGosByGroup(long)}{{\bf  getActiveGosByGroup}\\}
\begin{lstlisting}[frame=none]
java.util.List getActiveGosByGroup(long id)\end{lstlisting} %end signature
}%end item
\item{ 
\index{getActiveGosByUser(String)}
\hypertarget{edu.kit.pse17.go_app.PersistenceLayer.daos.GroupDaoImp.getActiveGosByUser(java.lang.String)}{{\bf  getActiveGosByUser}\\}
\begin{lstlisting}[frame=none]
java.util.List getActiveGosByUser(java.lang.String uid)\end{lstlisting} %end signature
}%end item
\item{ 
\index{getActiveUsers(long)}
\hypertarget{edu.kit.pse17.go_app.PersistenceLayer.daos.GroupDaoImp.getActiveUsers(long)}{{\bf  getActiveUsers}\\}
\begin{lstlisting}[frame=none]
java.util.List getActiveUsers(long id)\end{lstlisting} %end signature
}%end item
\item{ 
\index{getAll()}
\hypertarget{edu.kit.pse17.go_app.PersistenceLayer.daos.GroupDaoImp.getAll()}{{\bf  getAll}\\}
\begin{lstlisting}[frame=none]
public abstract java.util.List getAll()\end{lstlisting} %end signature
}%end item
\item{ 
\index{getAllGosByGroup(long)}
\hypertarget{edu.kit.pse17.go_app.PersistenceLayer.daos.GroupDaoImp.getAllGosByGroup(long)}{{\bf  getAllGosByGroup}\\}
\begin{lstlisting}[frame=none]
java.util.List getAllGosByGroup(long id)\end{lstlisting} %end signature
}%end item
\item{ 
\index{getAllGosByUser(String)}
\hypertarget{edu.kit.pse17.go_app.PersistenceLayer.daos.GroupDaoImp.getAllGosByUser(java.lang.String)}{{\bf  getAllGosByUser}\\}
\begin{lstlisting}[frame=none]
java.util.List getAllGosByUser(java.lang.String uid)\end{lstlisting} %end signature
}%end item
\item{ 
\index{getById(Long)}
\hypertarget{edu.kit.pse17.go_app.PersistenceLayer.daos.GroupDaoImp.getById(java.lang.Long)}{{\bf  getById}\\}
\begin{lstlisting}[frame=none]
public edu.kit.pse17.go_app.PersistenceLayer.GoEntity getById(java.lang.Long id)\end{lstlisting} %end signature
}%end item
\item{ 
\index{getDeclinedusers(long)}
\hypertarget{edu.kit.pse17.go_app.PersistenceLayer.daos.GroupDaoImp.getDeclinedusers(long)}{{\bf  getDeclinedusers}\\}
\begin{lstlisting}[frame=none]
java.util.List getDeclinedusers(long id)\end{lstlisting} %end signature
}%end item
\item{ 
\index{getGoingUsers(long)}
\hypertarget{edu.kit.pse17.go_app.PersistenceLayer.daos.GroupDaoImp.getGoingUsers(long)}{{\bf  getGoingUsers}\\}
\begin{lstlisting}[frame=none]
java.util.List getGoingUsers(long id)\end{lstlisting} %end signature
}%end item
\item{ 
\index{notify(GroupEntity)}
\hypertarget{edu.kit.pse17.go_app.PersistenceLayer.daos.GroupDaoImp.notify(edu.kit.pse17.go_app.PersistenceLayer.GroupEntity)}{{\bf  notify}\\}
\begin{lstlisting}[frame=none]
public void notify(edu.kit.pse17.go_app.PersistenceLayer.GroupEntity groupEntity)\end{lstlisting} %end signature
}%end item
\item{ 
\index{register(Observer)}
\hypertarget{edu.kit.pse17.go_app.PersistenceLayer.daos.GroupDaoImp.register(edu.kit.pse17.go_app.ServiceLayer.Observer)}{{\bf  register}\\}
\begin{lstlisting}[frame=none]
void register(edu.kit.pse17.go_app.ServiceLayer.Observer observer)\end{lstlisting} %end signature
}%end item
\item{ 
\index{unregister(Observer)}
\hypertarget{edu.kit.pse17.go_app.PersistenceLayer.daos.GroupDaoImp.unregister(edu.kit.pse17.go_app.ServiceLayer.Observer)}{{\bf  unregister}\\}
\begin{lstlisting}[frame=none]
void unregister(edu.kit.pse17.go_app.ServiceLayer.Observer observer)\end{lstlisting} %end signature
}%end item
\item{ 
\index{update(GoEntity)}
\hypertarget{edu.kit.pse17.go_app.PersistenceLayer.daos.GroupDaoImp.update(edu.kit.pse17.go_app.PersistenceLayer.GoEntity)}{{\bf  update}\\}
\begin{lstlisting}[frame=none]
public edu.kit.pse17.go_app.PersistenceLayer.GoEntity update(edu.kit.pse17.go_app.PersistenceLayer.GoEntity goEntity)\end{lstlisting} %end signature
}%end item
\end{itemize}
}
\subsubsection{Members inherited from class AbstractDao }{
\texttt{edu.kit.pse17.go_app.PersistenceLayer.daos.AbstractDao} {\small 
\refdefined{edu.kit.pse17.go_app.PersistenceLayer.daos.AbstractDao}}
{\small 

\vskip -2em
\begin{itemize}
\item{\vskip -1.5ex 
\texttt{public abstract Object {\bf  create}(\texttt{java.lang.Object} {\bf  t})
}%end signature
}%end item
\item{\vskip -1.5ex 
\texttt{public abstract Object {\bf  delete}(\texttt{java.lang.Object} {\bf  t})
}%end signature
}%end item
\item{\vskip -1.5ex 
\texttt{public abstract List {\bf  getAll}()
}%end signature
}%end item
\item{\vskip -1.5ex 
\texttt{public abstract Object {\bf  getById}(\texttt{java.io.Serializable} {\bf  id})
}%end signature
}%end item
\item{\vskip -1.5ex 
\texttt{public abstract Object {\bf  update}(\texttt{java.lang.Object} {\bf  t})
}%end signature
}%end item
\end{itemize}
}
}
\subsection{\label{edu.kit.pse17.go_app.PersistenceLayer.daos.UserDaoImp}Class UserDaoImp}{
\hypertarget{edu.kit.pse17.go_app.PersistenceLayer.daos.UserDaoImp}{}\vskip .1in 
Created by tina on 30.06.17.\vskip .1in 
\subsubsection{Declaration}{
\begin{lstlisting}[frame=none]
public class UserDaoImp
 extends edu.kit.pse17.go_app.PersistenceLayer.daos.AbstractDao implements UserDao, java.io.Serializable, edu.kit.pse17.go_app.ServiceLayer.Observable\end{lstlisting}
\subsubsection{Constructor summary}{
\begin{verse}
\hyperlink{edu.kit.pse17.go_app.PersistenceLayer.daos.UserDaoImp()}{{\bf UserDaoImp()}} \\
\end{verse}
}
\subsubsection{Method summary}{
\begin{verse}
\hyperlink{edu.kit.pse17.go_app.PersistenceLayer.daos.UserDaoImp.create(edu.kit.pse17.go_app.PersistenceLayer.UserEntity)}{{\bf create(UserEntity)}} \\
\hyperlink{edu.kit.pse17.go_app.PersistenceLayer.daos.UserDaoImp.delete(edu.kit.pse17.go_app.PersistenceLayer.UserEntity)}{{\bf delete(UserEntity)}} \\
\hyperlink{edu.kit.pse17.go_app.PersistenceLayer.daos.UserDaoImp.getAll()}{{\bf getAll()}} \\
\hyperlink{edu.kit.pse17.go_app.PersistenceLayer.daos.UserDaoImp.getById(java.lang.String)}{{\bf getById(String)}} \\
\hyperlink{edu.kit.pse17.go_app.PersistenceLayer.daos.UserDaoImp.getUserByEmail(java.lang.String)}{{\bf getUserByEmail(String)}} \\
\hyperlink{edu.kit.pse17.go_app.PersistenceLayer.daos.UserDaoImp.notify(edu.kit.pse17.go_app.PersistenceLayer.UserEntity)}{{\bf notify(UserEntity)}} \\
\hyperlink{edu.kit.pse17.go_app.PersistenceLayer.daos.UserDaoImp.register(edu.kit.pse17.go_app.ServiceLayer.Observer)}{{\bf register(Observer)}} \\
\hyperlink{edu.kit.pse17.go_app.PersistenceLayer.daos.UserDaoImp.unregister(edu.kit.pse17.go_app.ServiceLayer.Observer)}{{\bf unregister(Observer)}} \\
\hyperlink{edu.kit.pse17.go_app.PersistenceLayer.daos.UserDaoImp.update(edu.kit.pse17.go_app.PersistenceLayer.UserEntity)}{{\bf update(UserEntity)}} \\
\end{verse}
}
\subsubsection{Constructors}{
\vskip -2em
\begin{itemize}
\item{ 
\index{UserDaoImp()}
\hypertarget{edu.kit.pse17.go_app.PersistenceLayer.daos.UserDaoImp()}{{\bf  UserDaoImp}\\}
\begin{lstlisting}[frame=none]
public UserDaoImp()\end{lstlisting} %end signature
}%end item
\end{itemize}
}
\subsubsection{Methods}{
\vskip -2em
\begin{itemize}
\item{ 
\index{create(UserEntity)}
\hypertarget{edu.kit.pse17.go_app.PersistenceLayer.daos.UserDaoImp.create(edu.kit.pse17.go_app.PersistenceLayer.UserEntity)}{{\bf  create}\\}
\begin{lstlisting}[frame=none]
public edu.kit.pse17.go_app.PersistenceLayer.UserEntity create(edu.kit.pse17.go_app.PersistenceLayer.UserEntity userEntity)\end{lstlisting} %end signature
}%end item
\item{ 
\index{delete(UserEntity)}
\hypertarget{edu.kit.pse17.go_app.PersistenceLayer.daos.UserDaoImp.delete(edu.kit.pse17.go_app.PersistenceLayer.UserEntity)}{{\bf  delete}\\}
\begin{lstlisting}[frame=none]
public edu.kit.pse17.go_app.PersistenceLayer.UserEntity delete(edu.kit.pse17.go_app.PersistenceLayer.UserEntity userEntity)\end{lstlisting} %end signature
}%end item
\item{ 
\index{getAll()}
\hypertarget{edu.kit.pse17.go_app.PersistenceLayer.daos.UserDaoImp.getAll()}{{\bf  getAll}\\}
\begin{lstlisting}[frame=none]
public abstract java.util.List getAll()\end{lstlisting} %end signature
}%end item
\item{ 
\index{getById(String)}
\hypertarget{edu.kit.pse17.go_app.PersistenceLayer.daos.UserDaoImp.getById(java.lang.String)}{{\bf  getById}\\}
\begin{lstlisting}[frame=none]
public edu.kit.pse17.go_app.PersistenceLayer.UserEntity getById(java.lang.String id)\end{lstlisting} %end signature
}%end item
\item{ 
\index{getUserByEmail(String)}
\hypertarget{edu.kit.pse17.go_app.PersistenceLayer.daos.UserDaoImp.getUserByEmail(java.lang.String)}{{\bf  getUserByEmail}\\}
\begin{lstlisting}[frame=none]
edu.kit.pse17.go_app.PersistenceLayer.UserEntity getUserByEmail(java.lang.String mail)\end{lstlisting} %end signature
}%end item
\item{ 
\index{notify(UserEntity)}
\hypertarget{edu.kit.pse17.go_app.PersistenceLayer.daos.UserDaoImp.notify(edu.kit.pse17.go_app.PersistenceLayer.UserEntity)}{{\bf  notify}\\}
\begin{lstlisting}[frame=none]
public void notify(edu.kit.pse17.go_app.PersistenceLayer.UserEntity userEntity)\end{lstlisting} %end signature
}%end item
\item{ 
\index{register(Observer)}
\hypertarget{edu.kit.pse17.go_app.PersistenceLayer.daos.UserDaoImp.register(edu.kit.pse17.go_app.ServiceLayer.Observer)}{{\bf  register}\\}
\begin{lstlisting}[frame=none]
void register(edu.kit.pse17.go_app.ServiceLayer.Observer observer)\end{lstlisting} %end signature
}%end item
\item{ 
\index{unregister(Observer)}
\hypertarget{edu.kit.pse17.go_app.PersistenceLayer.daos.UserDaoImp.unregister(edu.kit.pse17.go_app.ServiceLayer.Observer)}{{\bf  unregister}\\}
\begin{lstlisting}[frame=none]
void unregister(edu.kit.pse17.go_app.ServiceLayer.Observer observer)\end{lstlisting} %end signature
}%end item
\item{ 
\index{update(UserEntity)}
\hypertarget{edu.kit.pse17.go_app.PersistenceLayer.daos.UserDaoImp.update(edu.kit.pse17.go_app.PersistenceLayer.UserEntity)}{{\bf  update}\\}
\begin{lstlisting}[frame=none]
public edu.kit.pse17.go_app.PersistenceLayer.UserEntity update(edu.kit.pse17.go_app.PersistenceLayer.UserEntity userEntity)\end{lstlisting} %end signature
}%end item
\end{itemize}
}
\subsubsection{Members inherited from class AbstractDao }{
\texttt{edu.kit.pse17.go_app.PersistenceLayer.daos.AbstractDao} {\small 
\refdefined{edu.kit.pse17.go_app.PersistenceLayer.daos.AbstractDao}}
{\small 

\vskip -2em
\begin{itemize}
\item{\vskip -1.5ex 
\texttt{public abstract Object {\bf  create}(\texttt{java.lang.Object} {\bf  t})
}%end signature
}%end item
\item{\vskip -1.5ex 
\texttt{public abstract Object {\bf  delete}(\texttt{java.lang.Object} {\bf  t})
}%end signature
}%end item
\item{\vskip -1.5ex 
\texttt{public abstract List {\bf  getAll}()
}%end signature
}%end item
\item{\vskip -1.5ex 
\texttt{public abstract Object {\bf  getById}(\texttt{java.io.Serializable} {\bf  id})
}%end signature
}%end item
\item{\vskip -1.5ex 
\texttt{public abstract Object {\bf  update}(\texttt{java.lang.Object} {\bf  t})
}%end signature
}%end item
\end{itemize}
}
}
}
\section{Package edu.kit.pse17.go\_app.ClientCommunication.Downstream}{
\label{edu.kit.pse17.go_app.ClientCommunication.Downstream}\hypertarget{edu.kit.pse17.go_app.ClientCommunication.Downstream}{}
\hskip -.05in
\hbox to \hsize{\textit{ Package Contents\hfil Page}}
\vskip .13in
\hbox{{\bf  Interfaces}}
\entityintro{FcmApi}{edu.kit.pse17.go_app.ClientCommunication.Downstream.FcmApi}{API des FCM-Servers.}
\vskip .13in
\hbox{{\bf  Classes}}
\entityintro{FcmClient}{edu.kit.pse17.go_app.ClientCommunication.Downstream.FcmClient}{Client-Klasse, die ein HTTP POST\_Request an den FCM-Server schickt, wo die Nachricht wiederum an das User-Endgerät weitergeleitet wird.}
\vskip .1in
\vskip .1in
\subsection{\label{edu.kit.pse17.go_app.ClientCommunication.Downstream.FcmApi}Interface FcmApi}{
\hypertarget{edu.kit.pse17.go_app.ClientCommunication.Downstream.FcmApi}{}\vskip .1in 
API des FCM-Servers. Kann von Retrofit-Objekten verwendet werden, um POST-Requests an den FCM Server zu schicken. Created by tina on 29.06.17.\vskip .1in 
\subsubsection{Declaration}{
\begin{lstlisting}[frame=none]
public interface FcmApi
\end{lstlisting}
\subsubsection{Method summary}{
\begin{verse}
\hyperlink{edu.kit.pse17.go_app.ClientCommunication.Downstream.FcmApi.sendDownStreamMessage()}{{\bf sendDownStreamMessage()}} \\
\end{verse}
}
\subsubsection{Methods}{
\vskip -2em
\begin{itemize}
\item{ 
\index{sendDownStreamMessage()}
\hypertarget{edu.kit.pse17.go_app.ClientCommunication.Downstream.FcmApi.sendDownStreamMessage()}{{\bf  sendDownStreamMessage}\\}
\begin{lstlisting}[frame=none]
void sendDownStreamMessage()\end{lstlisting} %end signature
}%end item
\end{itemize}
}
}
\subsection{\label{edu.kit.pse17.go_app.ClientCommunication.Downstream.FcmClient}Class FcmClient}{
\hypertarget{edu.kit.pse17.go_app.ClientCommunication.Downstream.FcmClient}{}\vskip .1in 
Client-Klasse, die ein HTTP POST\_Request an den FCM-Server schickt, wo die Nachricht wiederum an das User-Endgerät weitergeleitet wird. Created by tina on 29.06.17.\vskip .1in 
\subsubsection{Declaration}{
\begin{lstlisting}[frame=none]
public class FcmClient
 extends java.lang.Object\end{lstlisting}
\subsubsection{Constructor summary}{
\begin{verse}
\hyperlink{edu.kit.pse17.go_app.ClientCommunication.Downstream.FcmClient()}{{\bf FcmClient()}} Konstruktor\\
\end{verse}
}
\subsubsection{Method summary}{
\begin{verse}
\hyperlink{edu.kit.pse17.go_app.ClientCommunication.Downstream.FcmClient.send()}{{\bf send()}} sendet eine Message an den FCM-Server, der diese an das User-Endgerät weiterleitet\\
\end{verse}
}
\subsubsection{Constructors}{
\vskip -2em
\begin{itemize}
\item{ 
\index{FcmClient()}
\hypertarget{edu.kit.pse17.go_app.ClientCommunication.Downstream.FcmClient()}{{\bf  FcmClient}\\}
\begin{lstlisting}[frame=none]
public FcmClient()\end{lstlisting} %end signature
\begin{itemize}
\item{
{\bf  Description}

Konstruktor
}
\end{itemize}
}%end item
\end{itemize}
}
\subsubsection{Methods}{
\vskip -2em
\begin{itemize}
\item{ 
\index{send()}
\hypertarget{edu.kit.pse17.go_app.ClientCommunication.Downstream.FcmClient.send()}{{\bf  send}\\}
\begin{lstlisting}[frame=none]
public void send()\end{lstlisting} %end signature
\begin{itemize}
\item{
{\bf  Description}

sendet eine Message an den FCM-Server, der diese an das User-Endgerät weiterleitet
}
\end{itemize}
}%end item
\end{itemize}
}
}
}
\section{Package edu.kit.pse17.go\_app.ClientCommunication.Upstream}{
\label{edu.kit.pse17.go_app.ClientCommunication.Upstream}\hypertarget{edu.kit.pse17.go_app.ClientCommunication.Upstream}{}
\hskip -.05in
\hbox to \hsize{\textit{ Package Contents\hfil Page}}
\vskip .13in
\hbox{{\bf  Classes}}
\entityintro{GroupRestController}{edu.kit.pse17.go_app.ClientCommunication.Upstream.GroupRestController}{Created by tina on 29.06.17.}
\vskip .1in
\vskip .1in
\subsection{\label{edu.kit.pse17.go_app.ClientCommunication.Upstream.GroupRestController}Class GroupRestController}{
\hypertarget{edu.kit.pse17.go_app.ClientCommunication.Upstream.GroupRestController}{}\vskip .1in 
Created by tina on 29.06.17.\vskip .1in 
\subsubsection{Declaration}{
\begin{lstlisting}[frame=none]
public class GroupRestController
 extends java.lang.Object\end{lstlisting}
\subsubsection{Constructor summary}{
\begin{verse}
\hyperlink{edu.kit.pse17.go_app.ClientCommunication.Upstream.GroupRestController()}{{\bf GroupRestController()}} \\
\end{verse}
}
\subsubsection{Method summary}{
\begin{verse}
\hyperlink{edu.kit.pse17.go_app.ClientCommunication.Upstream.GroupRestController.addGoup(edu.kit.pse17.go_app.PersistenceLayer.GroupEntity)}{{\bf addGoup(GroupEntity)}} \\
\hyperlink{edu.kit.pse17.go_app.ClientCommunication.Upstream.GroupRestController.addMember(java.lang.Long, java.lang.String)}{{\bf addMember(Long, String)}} \\
\hyperlink{edu.kit.pse17.go_app.ClientCommunication.Upstream.GroupRestController.alterGroup(java.lang.Long)}{{\bf alterGroup(Long)}} \\
\hyperlink{edu.kit.pse17.go_app.ClientCommunication.Upstream.GroupRestController.deleteGroup(java.lang.Long)}{{\bf deleteGroup(Long)}} \\
\hyperlink{edu.kit.pse17.go_app.ClientCommunication.Upstream.GroupRestController.getgroupInfo(java.lang.Long)}{{\bf getgroupInfo(Long)}} \\
\hyperlink{edu.kit.pse17.go_app.ClientCommunication.Upstream.GroupRestController.getGroupsById(java.lang.String)}{{\bf getGroupsById(String)}} \\
\hyperlink{edu.kit.pse17.go_app.ClientCommunication.Upstream.GroupRestController.inviteMember(java.lang.Long, java.lang.String)}{{\bf inviteMember(Long, String)}} \\
\hyperlink{edu.kit.pse17.go_app.ClientCommunication.Upstream.GroupRestController.removeMember(java.lang.String)}{{\bf removeMember(String)}} \\
\end{verse}
}
\subsubsection{Constructors}{
\vskip -2em
\begin{itemize}
\item{ 
\index{GroupRestController()}
\hypertarget{edu.kit.pse17.go_app.ClientCommunication.Upstream.GroupRestController()}{{\bf  GroupRestController}\\}
\begin{lstlisting}[frame=none]
public GroupRestController()\end{lstlisting} %end signature
}%end item
\end{itemize}
}
\subsubsection{Methods}{
\vskip -2em
\begin{itemize}
\item{ 
\index{addGoup(GroupEntity)}
\hypertarget{edu.kit.pse17.go_app.ClientCommunication.Upstream.GroupRestController.addGoup(edu.kit.pse17.go_app.PersistenceLayer.GroupEntity)}{{\bf  addGoup}\\}
\begin{lstlisting}[frame=none]
public void addGoup(edu.kit.pse17.go_app.PersistenceLayer.GroupEntity groupEntity)\end{lstlisting} %end signature
}%end item
\item{ 
\index{addMember(Long, String)}
\hypertarget{edu.kit.pse17.go_app.ClientCommunication.Upstream.GroupRestController.addMember(java.lang.Long, java.lang.String)}{{\bf  addMember}\\}
\begin{lstlisting}[frame=none]
public void addMember(java.lang.Long groupId,java.lang.String userId)\end{lstlisting} %end signature
}%end item
\item{ 
\index{alterGroup(Long)}
\hypertarget{edu.kit.pse17.go_app.ClientCommunication.Upstream.GroupRestController.alterGroup(java.lang.Long)}{{\bf  alterGroup}\\}
\begin{lstlisting}[frame=none]
public void alterGroup(java.lang.Long groupId)\end{lstlisting} %end signature
}%end item
\item{ 
\index{deleteGroup(Long)}
\hypertarget{edu.kit.pse17.go_app.ClientCommunication.Upstream.GroupRestController.deleteGroup(java.lang.Long)}{{\bf  deleteGroup}\\}
\begin{lstlisting}[frame=none]
public void deleteGroup(java.lang.Long groupId)\end{lstlisting} %end signature
}%end item
\item{ 
\index{getgroupInfo(Long)}
\hypertarget{edu.kit.pse17.go_app.ClientCommunication.Upstream.GroupRestController.getgroupInfo(java.lang.Long)}{{\bf  getgroupInfo}\\}
\begin{lstlisting}[frame=none]
public edu.kit.pse17.go_app.PersistenceLayer.GroupEntity getgroupInfo(java.lang.Long groupId)\end{lstlisting} %end signature
}%end item
\item{ 
\index{getGroupsById(String)}
\hypertarget{edu.kit.pse17.go_app.ClientCommunication.Upstream.GroupRestController.getGroupsById(java.lang.String)}{{\bf  getGroupsById}\\}
\begin{lstlisting}[frame=none]
public java.util.Collection getGroupsById(java.lang.String userId)\end{lstlisting} %end signature
}%end item
\item{ 
\index{inviteMember(Long, String)}
\hypertarget{edu.kit.pse17.go_app.ClientCommunication.Upstream.GroupRestController.inviteMember(java.lang.Long, java.lang.String)}{{\bf  inviteMember}\\}
\begin{lstlisting}[frame=none]
public void inviteMember(java.lang.Long groupId,java.lang.String userId)\end{lstlisting} %end signature
}%end item
\item{ 
\index{removeMember(String)}
\hypertarget{edu.kit.pse17.go_app.ClientCommunication.Upstream.GroupRestController.removeMember(java.lang.String)}{{\bf  removeMember}\\}
\begin{lstlisting}[frame=none]
public void removeMember(java.lang.String userId)\end{lstlisting} %end signature
}%end item
\end{itemize}
}
}
}
\section{Package edu.kit.pse17.go\_app}{
\label{edu.kit.pse17.go_app}\hypertarget{edu.kit.pse17.go_app}{}
\hskip -.05in
\hbox to \hsize{\textit{ Package Contents\hfil Page}}
\vskip .13in
\hbox{{\bf  Classes}}
\entityintro{Main}{edu.kit.pse17.go_app.Main}{Created by tina on 29.06.17.}
\vskip .1in
\vskip .1in
\subsection{\label{edu.kit.pse17.go_app.Main}Class Main}{
\hypertarget{edu.kit.pse17.go_app.Main}{}\vskip .1in 
Created by tina on 29.06.17.\vskip .1in 
\subsubsection{Declaration}{
\begin{lstlisting}[frame=none]
public class Main
 extends java.lang.Object\end{lstlisting}
\subsubsection{Constructor summary}{
\begin{verse}
\hyperlink{edu.kit.pse17.go_app.Main()}{{\bf Main()}} \\
\end{verse}
}
\subsubsection{Method summary}{
\begin{verse}
\hyperlink{edu.kit.pse17.go_app.Main.main(java.lang.String[])}{{\bf main(String\lbrack \rbrack )}} \\
\end{verse}
}
\subsubsection{Constructors}{
\vskip -2em
\begin{itemize}
\item{ 
\index{Main()}
\hypertarget{edu.kit.pse17.go_app.Main()}{{\bf  Main}\\}
\begin{lstlisting}[frame=none]
public Main()\end{lstlisting} %end signature
}%end item
\end{itemize}
}
\subsubsection{Methods}{
\vskip -2em
\begin{itemize}
\item{ 
\index{main(String\lbrack \rbrack )}
\hypertarget{edu.kit.pse17.go_app.Main.main(java.lang.String[])}{{\bf  main}\\}
\begin{lstlisting}[frame=none]
public static void main(java.lang.String[] args)\end{lstlisting} %end signature
}%end item
\end{itemize}
}
}
}
\section{Package edu.kit.pse17.go\_app.ServiceLayer}{
\label{edu.kit.pse17.go_app.ServiceLayer}\hypertarget{edu.kit.pse17.go_app.ServiceLayer}{}
\hskip -.05in
\hbox to \hsize{\textit{ Package Contents\hfil Page}}
\vskip .13in
\hbox{{\bf  Interfaces}}
\entityintro{Observable}{edu.kit.pse17.go_app.ServiceLayer.Observable}{Created by tina on 30.06.17.}
\vskip .13in
\hbox{{\bf  Classes}}
\entityintro{GoObserver}{edu.kit.pse17.go_app.ServiceLayer.GoObserver}{Created by tina on 30.06.17.}
\entityintro{GoService}{edu.kit.pse17.go_app.ServiceLayer.GoService}{Created by tina on 30.06.17.}
\entityintro{GroupObserver}{edu.kit.pse17.go_app.ServiceLayer.GroupObserver}{Created by tina on 30.06.17.}
\entityintro{GroupService}{edu.kit.pse17.go_app.ServiceLayer.GroupService}{Created by tina on 30.06.17.}
\entityintro{LocationService}{edu.kit.pse17.go_app.ServiceLayer.LocationService}{Created by tina on 30.06.17.}
\entityintro{Observer}{edu.kit.pse17.go_app.ServiceLayer.Observer}{Created by tina on 30.06.17.}
\entityintro{UserObserver}{edu.kit.pse17.go_app.ServiceLayer.UserObserver}{Created by tina on 30.06.17.}
\entityintro{UserService}{edu.kit.pse17.go_app.ServiceLayer.UserService}{Created by tina on 30.06.17.}
\vskip .1in
\vskip .1in
\subsection{\label{edu.kit.pse17.go_app.ServiceLayer.Observable}Interface Observable}{
\hypertarget{edu.kit.pse17.go_app.ServiceLayer.Observable}{}\vskip .1in 
Created by tina on 30.06.17.\vskip .1in 
\subsubsection{Declaration}{
\begin{lstlisting}[frame=none]
public interface Observable
\end{lstlisting}
\subsubsection{All known subinterfaces}{GoDaoImp\small{\refdefined{edu.kit.pse17.go_app.PersistenceLayer.daos.GoDaoImp}}, UserDaoImp\small{\refdefined{edu.kit.pse17.go_app.PersistenceLayer.daos.UserDaoImp}}, GroupDaoImp\small{\refdefined{edu.kit.pse17.go_app.PersistenceLayer.daos.GroupDaoImp}}}
\subsubsection{All classes known to implement interface}{GoDaoImp\small{\refdefined{edu.kit.pse17.go_app.PersistenceLayer.daos.GoDaoImp}}, UserDaoImp\small{\refdefined{edu.kit.pse17.go_app.PersistenceLayer.daos.UserDaoImp}}, GroupDaoImp\small{\refdefined{edu.kit.pse17.go_app.PersistenceLayer.daos.GroupDaoImp}}}
\subsubsection{Method summary}{
\begin{verse}
\hyperlink{edu.kit.pse17.go_app.ServiceLayer.Observable.notify(T)}{{\bf notify(T)}} \\
\hyperlink{edu.kit.pse17.go_app.ServiceLayer.Observable.register(edu.kit.pse17.go_app.ServiceLayer.Observer)}{{\bf register(Observer)}} \\
\hyperlink{edu.kit.pse17.go_app.ServiceLayer.Observable.unregister(edu.kit.pse17.go_app.ServiceLayer.Observer)}{{\bf unregister(Observer)}} \\
\end{verse}
}
\subsubsection{Methods}{
\vskip -2em
\begin{itemize}
\item{ 
\index{notify(T)}
\hypertarget{edu.kit.pse17.go_app.ServiceLayer.Observable.notify(T)}{{\bf  notify}\\}
\begin{lstlisting}[frame=none]
void notify(java.lang.Object t)\end{lstlisting} %end signature
}%end item
\item{ 
\index{register(Observer)}
\hypertarget{edu.kit.pse17.go_app.ServiceLayer.Observable.register(edu.kit.pse17.go_app.ServiceLayer.Observer)}{{\bf  register}\\}
\begin{lstlisting}[frame=none]
void register(Observer observer)\end{lstlisting} %end signature
}%end item
\item{ 
\index{unregister(Observer)}
\hypertarget{edu.kit.pse17.go_app.ServiceLayer.Observable.unregister(edu.kit.pse17.go_app.ServiceLayer.Observer)}{{\bf  unregister}\\}
\begin{lstlisting}[frame=none]
void unregister(Observer observer)\end{lstlisting} %end signature
}%end item
\end{itemize}
}
}
\subsection{\label{edu.kit.pse17.go_app.ServiceLayer.GoObserver}Class GoObserver}{
\hypertarget{edu.kit.pse17.go_app.ServiceLayer.GoObserver}{}\vskip .1in 
Created by tina on 30.06.17.\vskip .1in 
\subsubsection{Declaration}{
\begin{lstlisting}[frame=none]
public class GoObserver
 extends edu.kit.pse17.go_app.ServiceLayer.Observer\end{lstlisting}
\subsubsection{Constructor summary}{
\begin{verse}
\hyperlink{edu.kit.pse17.go_app.ServiceLayer.GoObserver()}{{\bf GoObserver()}} \\
\end{verse}
}
\subsubsection{Method summary}{
\begin{verse}
\hyperlink{edu.kit.pse17.go_app.ServiceLayer.GoObserver.update()}{{\bf update()}} \\
\end{verse}
}
\subsubsection{Constructors}{
\vskip -2em
\begin{itemize}
\item{ 
\index{GoObserver()}
\hypertarget{edu.kit.pse17.go_app.ServiceLayer.GoObserver()}{{\bf  GoObserver}\\}
\begin{lstlisting}[frame=none]
public GoObserver()\end{lstlisting} %end signature
}%end item
\end{itemize}
}
\subsubsection{Methods}{
\vskip -2em
\begin{itemize}
\item{ 
\index{update()}
\hypertarget{edu.kit.pse17.go_app.ServiceLayer.GoObserver.update()}{{\bf  update}\\}
\begin{lstlisting}[frame=none]
public abstract void update()\end{lstlisting} %end signature
}%end item
\end{itemize}
}
\subsubsection{Members inherited from class Observer }{
\texttt{edu.kit.pse17.go_app.ServiceLayer.Observer} {\small 
\refdefined{edu.kit.pse17.go_app.ServiceLayer.Observer}}
{\small 

\vskip -2em
\begin{itemize}
\item{\vskip -1.5ex 
\texttt{public abstract void {\bf  update}()
}%end signature
}%end item
\end{itemize}
}
}
\subsection{\label{edu.kit.pse17.go_app.ServiceLayer.GoService}Class GoService}{
\hypertarget{edu.kit.pse17.go_app.ServiceLayer.GoService}{}\vskip .1in 
Created by tina on 30.06.17.\vskip .1in 
\subsubsection{Declaration}{
\begin{lstlisting}[frame=none]
public class GoService
 extends java.lang.Object\end{lstlisting}
\subsubsection{Constructor summary}{
\begin{verse}
\hyperlink{edu.kit.pse17.go_app.ServiceLayer.GoService()}{{\bf GoService()}} \\
\end{verse}
}
\subsubsection{Constructors}{
\vskip -2em
\begin{itemize}
\item{ 
\index{GoService()}
\hypertarget{edu.kit.pse17.go_app.ServiceLayer.GoService()}{{\bf  GoService}\\}
\begin{lstlisting}[frame=none]
public GoService()\end{lstlisting} %end signature
}%end item
\end{itemize}
}
}
\subsection{\label{edu.kit.pse17.go_app.ServiceLayer.GroupObserver}Class GroupObserver}{
\hypertarget{edu.kit.pse17.go_app.ServiceLayer.GroupObserver}{}\vskip .1in 
Created by tina on 30.06.17.\vskip .1in 
\subsubsection{Declaration}{
\begin{lstlisting}[frame=none]
public class GroupObserver
 extends edu.kit.pse17.go_app.ServiceLayer.Observer\end{lstlisting}
\subsubsection{Constructor summary}{
\begin{verse}
\hyperlink{edu.kit.pse17.go_app.ServiceLayer.GroupObserver()}{{\bf GroupObserver()}} \\
\end{verse}
}
\subsubsection{Method summary}{
\begin{verse}
\hyperlink{edu.kit.pse17.go_app.ServiceLayer.GroupObserver.update()}{{\bf update()}} \\
\end{verse}
}
\subsubsection{Constructors}{
\vskip -2em
\begin{itemize}
\item{ 
\index{GroupObserver()}
\hypertarget{edu.kit.pse17.go_app.ServiceLayer.GroupObserver()}{{\bf  GroupObserver}\\}
\begin{lstlisting}[frame=none]
public GroupObserver()\end{lstlisting} %end signature
}%end item
\end{itemize}
}
\subsubsection{Methods}{
\vskip -2em
\begin{itemize}
\item{ 
\index{update()}
\hypertarget{edu.kit.pse17.go_app.ServiceLayer.GroupObserver.update()}{{\bf  update}\\}
\begin{lstlisting}[frame=none]
public abstract void update()\end{lstlisting} %end signature
}%end item
\end{itemize}
}
\subsubsection{Members inherited from class Observer }{
\texttt{edu.kit.pse17.go_app.ServiceLayer.Observer} {\small 
\refdefined{edu.kit.pse17.go_app.ServiceLayer.Observer}}
{\small 

\vskip -2em
\begin{itemize}
\item{\vskip -1.5ex 
\texttt{public abstract void {\bf  update}()
}%end signature
}%end item
\end{itemize}
}
}
\subsection{\label{edu.kit.pse17.go_app.ServiceLayer.GroupService}Class GroupService}{
\hypertarget{edu.kit.pse17.go_app.ServiceLayer.GroupService}{}\vskip .1in 
Created by tina on 30.06.17.\vskip .1in 
\subsubsection{Declaration}{
\begin{lstlisting}[frame=none]
public class GroupService
 extends java.lang.Object\end{lstlisting}
\subsubsection{Constructor summary}{
\begin{verse}
\hyperlink{edu.kit.pse17.go_app.ServiceLayer.GroupService()}{{\bf GroupService()}} \\
\end{verse}
}
\subsubsection{Constructors}{
\vskip -2em
\begin{itemize}
\item{ 
\index{GroupService()}
\hypertarget{edu.kit.pse17.go_app.ServiceLayer.GroupService()}{{\bf  GroupService}\\}
\begin{lstlisting}[frame=none]
public GroupService()\end{lstlisting} %end signature
}%end item
\end{itemize}
}
}
\subsection{\label{edu.kit.pse17.go_app.ServiceLayer.LocationService}Class LocationService}{
\hypertarget{edu.kit.pse17.go_app.ServiceLayer.LocationService}{}\vskip .1in 
Created by tina on 30.06.17.\vskip .1in 
\subsubsection{Declaration}{
\begin{lstlisting}[frame=none]
public class LocationService
 extends java.lang.Object\end{lstlisting}
\subsubsection{Constructor summary}{
\begin{verse}
\hyperlink{edu.kit.pse17.go_app.ServiceLayer.LocationService()}{{\bf LocationService()}} \\
\end{verse}
}
\subsubsection{Constructors}{
\vskip -2em
\begin{itemize}
\item{ 
\index{LocationService()}
\hypertarget{edu.kit.pse17.go_app.ServiceLayer.LocationService()}{{\bf  LocationService}\\}
\begin{lstlisting}[frame=none]
public LocationService()\end{lstlisting} %end signature
}%end item
\end{itemize}
}
}
\subsection{\label{edu.kit.pse17.go_app.ServiceLayer.Observer}Class Observer}{
\hypertarget{edu.kit.pse17.go_app.ServiceLayer.Observer}{}\vskip .1in 
Created by tina on 30.06.17.\vskip .1in 
\subsubsection{Declaration}{
\begin{lstlisting}[frame=none]
public abstract class Observer
 extends java.lang.Object\end{lstlisting}
\subsubsection{All known subclasses}{GroupObserver\small{\refdefined{edu.kit.pse17.go_app.ServiceLayer.GroupObserver}}, GoObserver\small{\refdefined{edu.kit.pse17.go_app.ServiceLayer.GoObserver}}, UserObserver\small{\refdefined{edu.kit.pse17.go_app.ServiceLayer.UserObserver}}}
\subsubsection{Constructor summary}{
\begin{verse}
\hyperlink{edu.kit.pse17.go_app.ServiceLayer.Observer()}{{\bf Observer()}} \\
\end{verse}
}
\subsubsection{Method summary}{
\begin{verse}
\hyperlink{edu.kit.pse17.go_app.ServiceLayer.Observer.update()}{{\bf update()}} \\
\end{verse}
}
\subsubsection{Constructors}{
\vskip -2em
\begin{itemize}
\item{ 
\index{Observer()}
\hypertarget{edu.kit.pse17.go_app.ServiceLayer.Observer()}{{\bf  Observer}\\}
\begin{lstlisting}[frame=none]
public Observer()\end{lstlisting} %end signature
}%end item
\end{itemize}
}
\subsubsection{Methods}{
\vskip -2em
\begin{itemize}
\item{ 
\index{update()}
\hypertarget{edu.kit.pse17.go_app.ServiceLayer.Observer.update()}{{\bf  update}\\}
\begin{lstlisting}[frame=none]
public abstract void update()\end{lstlisting} %end signature
}%end item
\end{itemize}
}
}
\subsection{\label{edu.kit.pse17.go_app.ServiceLayer.UserObserver}Class UserObserver}{
\hypertarget{edu.kit.pse17.go_app.ServiceLayer.UserObserver}{}\vskip .1in 
Created by tina on 30.06.17.\vskip .1in 
\subsubsection{Declaration}{
\begin{lstlisting}[frame=none]
public class UserObserver
 extends edu.kit.pse17.go_app.ServiceLayer.Observer\end{lstlisting}
\subsubsection{Constructor summary}{
\begin{verse}
\hyperlink{edu.kit.pse17.go_app.ServiceLayer.UserObserver()}{{\bf UserObserver()}} \\
\end{verse}
}
\subsubsection{Method summary}{
\begin{verse}
\hyperlink{edu.kit.pse17.go_app.ServiceLayer.UserObserver.update()}{{\bf update()}} \\
\end{verse}
}
\subsubsection{Constructors}{
\vskip -2em
\begin{itemize}
\item{ 
\index{UserObserver()}
\hypertarget{edu.kit.pse17.go_app.ServiceLayer.UserObserver()}{{\bf  UserObserver}\\}
\begin{lstlisting}[frame=none]
public UserObserver()\end{lstlisting} %end signature
}%end item
\end{itemize}
}
\subsubsection{Methods}{
\vskip -2em
\begin{itemize}
\item{ 
\index{update()}
\hypertarget{edu.kit.pse17.go_app.ServiceLayer.UserObserver.update()}{{\bf  update}\\}
\begin{lstlisting}[frame=none]
public abstract void update()\end{lstlisting} %end signature
}%end item
\end{itemize}
}
\subsubsection{Members inherited from class Observer }{
\texttt{edu.kit.pse17.go_app.ServiceLayer.Observer} {\small 
\refdefined{edu.kit.pse17.go_app.ServiceLayer.Observer}}
{\small 

\vskip -2em
\begin{itemize}
\item{\vskip -1.5ex 
\texttt{public abstract void {\bf  update}()
}%end signature
}%end item
\end{itemize}
}
}
\subsection{\label{edu.kit.pse17.go_app.ServiceLayer.UserService}Class UserService}{
\hypertarget{edu.kit.pse17.go_app.ServiceLayer.UserService}{}\vskip .1in 
Created by tina on 30.06.17.\vskip .1in 
\subsubsection{Declaration}{
\begin{lstlisting}[frame=none]
public class UserService
 extends java.lang.Object\end{lstlisting}
\subsubsection{Constructor summary}{
\begin{verse}
\hyperlink{edu.kit.pse17.go_app.ServiceLayer.UserService()}{{\bf UserService()}} \\
\end{verse}
}
\subsubsection{Constructors}{
\vskip -2em
\begin{itemize}
\item{ 
\index{UserService()}
\hypertarget{edu.kit.pse17.go_app.ServiceLayer.UserService()}{{\bf  UserService}\\}
\begin{lstlisting}[frame=none]
public UserService()\end{lstlisting} %end signature
}%end item
\end{itemize}
}
}
}
\section{Package edu.kit.pse17.go\_app.model}{
\label{edu.kit.pse17.go_app.model}\hypertarget{edu.kit.pse17.go_app.model}{}
\hskip -.05in
\hbox to \hsize{\textit{ Package Contents\hfil Page}}
\vskip .13in
\hbox{{\bf  Classes}}
\entityintro{GO}{edu.kit.pse17.go_app.model.GO}{Diese Klasse verwaltet GO Objekte Created by tina on 17.06.17.}
\entityintro{Group}{edu.kit.pse17.go_app.model.Group}{Diese Klasse verwaltet Gruppen Objekte Created by tina on 17.06.17.}
\entityintro{GroupLocation}{edu.kit.pse17.go_app.model.GroupLocation}{Die Objekte der Klasse kapseln die geclusterten GPS-Daten der GO-Teilnehmer Created by tina on 20.06.17.}
\entityintro{Status}{edu.kit.pse17.go_app.model.Status}{möglicher Teilnehmerstatus für GOs Created by tina on 19.06.17.}
\entityintro{User}{edu.kit.pse17.go_app.model.User}{Diese Klasse verwaltet User Objekte Created by tina on 17.06.17.}
\vskip .1in
\vskip .1in
\subsection{\label{edu.kit.pse17.go_app.model.GO}Class GO}{
\hypertarget{edu.kit.pse17.go_app.model.GO}{}\vskip .1in 
Diese Klasse verwaltet GO Objekte Created by tina on 17.06.17.\vskip .1in 
\subsubsection{Declaration}{
\begin{lstlisting}[frame=none]
public class GO
 extends java.lang.Object\end{lstlisting}
\subsubsection{Constructor summary}{
\begin{verse}
\hyperlink{edu.kit.pse17.go_app.model.GO(long, java.lang.String, java.lang.String, java.util.Date, java.util.Date, Location, long, long, edu.kit.pse17.go_app.model.User, edu.kit.pse17.go_app.model.GroupLocation, java.util.List, java.util.List, java.util.List)}{{\bf GO(long, String, String, Date, Date, Location, long, long, User, GroupLocation, List, List, List)}} Konstruktor\\
\end{verse}
}
\subsubsection{Method summary}{
\begin{verse}
\hyperlink{edu.kit.pse17.go_app.model.GO.createGO()}{{\bf createGO()}} erzeugt ein neues GO-Objekt und speichert die GO-Daten in der Datenbank auf dem Tomcat Server\\
\hyperlink{edu.kit.pse17.go_app.model.GO.getDescription()}{{\bf getDescription()}} \\
\hyperlink{edu.kit.pse17.go_app.model.GO.getEnd()}{{\bf getEnd()}} \\
\hyperlink{edu.kit.pse17.go_app.model.GO.getGoingUsers()}{{\bf getGoingUsers()}} \\
\hyperlink{edu.kit.pse17.go_app.model.GO.getGoneUsers()}{{\bf getGoneUsers()}} \\
\hyperlink{edu.kit.pse17.go_app.model.GO.getID()}{{\bf getID()}} \\
\hyperlink{edu.kit.pse17.go_app.model.GO.getLat()}{{\bf getLat()}} \\
\hyperlink{edu.kit.pse17.go_app.model.GO.getLocationData()}{{\bf getLocationData()}} \\
\hyperlink{edu.kit.pse17.go_app.model.GO.getLon()}{{\bf getLon()}} \\
\hyperlink{edu.kit.pse17.go_app.model.GO.getName()}{{\bf getName()}} \\
\hyperlink{edu.kit.pse17.go_app.model.GO.getNotGoingUsers()}{{\bf getNotGoingUsers()}} \\
\hyperlink{edu.kit.pse17.go_app.model.GO.getOwner()}{{\bf getOwner()}} \\
\hyperlink{edu.kit.pse17.go_app.model.GO.getStart()}{{\bf getStart()}} \\
\hyperlink{edu.kit.pse17.go_app.model.GO.getUserStatus(edu.kit.pse17.go_app.model.User)}{{\bf getUserStatus(User)}} \\
\hyperlink{edu.kit.pse17.go_app.model.GO.isOwner(edu.kit.pse17.go_app.model.User)}{{\bf isOwner(User)}} \\
\hyperlink{edu.kit.pse17.go_app.model.GO.setDescription(java.lang.String)}{{\bf setDescription(String)}} \\
\hyperlink{edu.kit.pse17.go_app.model.GO.setEnd(java.util.Date)}{{\bf setEnd(Date)}} \\
\hyperlink{edu.kit.pse17.go_app.model.GO.setGoingUsers(java.util.List)}{{\bf setGoingUsers(List)}} \\
\hyperlink{edu.kit.pse17.go_app.model.GO.setGoneUsers(java.util.List)}{{\bf setGoneUsers(List)}} \\
\hyperlink{edu.kit.pse17.go_app.model.GO.setID(long)}{{\bf setID(long)}} \\
\hyperlink{edu.kit.pse17.go_app.model.GO.setLat(long)}{{\bf setLat(long)}} \\
\hyperlink{edu.kit.pse17.go_app.model.GO.setLocationData(edu.kit.pse17.go_app.model.GroupLocation)}{{\bf setLocationData(GroupLocation)}} \\
\hyperlink{edu.kit.pse17.go_app.model.GO.setLon(long)}{{\bf setLon(long)}} \\
\hyperlink{edu.kit.pse17.go_app.model.GO.setName(java.lang.String)}{{\bf setName(String)}} \\
\hyperlink{edu.kit.pse17.go_app.model.GO.setNotGoingUsers(java.util.List)}{{\bf setNotGoingUsers(List)}} \\
\hyperlink{edu.kit.pse17.go_app.model.GO.setOwner(edu.kit.pse17.go_app.model.User)}{{\bf setOwner(User)}} \\
\hyperlink{edu.kit.pse17.go_app.model.GO.setStart(java.util.Date)}{{\bf setStart(Date)}} \\
\end{verse}
}
\subsubsection{Constructors}{
\vskip -2em
\begin{itemize}
\item{ 
\index{GO(long, String, String, Date, Date, Location, long, long, User, GroupLocation, List, List, List)}
\hypertarget{edu.kit.pse17.go_app.model.GO(long, java.lang.String, java.lang.String, java.util.Date, java.util.Date, Location, long, long, edu.kit.pse17.go_app.model.User, edu.kit.pse17.go_app.model.GroupLocation, java.util.List, java.util.List, java.util.List)}{{\bf  GO}\\}
\begin{lstlisting}[frame=none]
public GO(long ID,java.lang.String name,java.lang.String description,java.util.Date start,java.util.Date end,Location location,long lat,long lon,User owner,GroupLocation locationData,java.util.List notGoingUsers,java.util.List goingUsers,java.util.List goneUsers)\end{lstlisting} %end signature
\begin{itemize}
\item{
{\bf  Description}

Konstruktor
}
\item{
{\bf  Parameters}
  \begin{itemize}
   \item{
\texttt{ID} -- eindeutige Nummer, mit der ein GO identifiziert werden kann}
   \item{
\texttt{name} -- GO Bezeichnung}
   \item{
\texttt{description} -- Go Beschreibung}
   \item{
\texttt{start} -- Startzeitpunkt}
   \item{
\texttt{end} -- Endzeitpunkt}
   \item{
\texttt{location} -- Treffpunkt (kann null sein)}
   \item{
\texttt{lat} -- }
   \item{
\texttt{lon} -- }
   \item{
\texttt{owner} -- GO-Verantwortlicher}
   \item{
\texttt{locationData} -- Die GPS-Daten der Go-Teilnehmer}
   \item{
\texttt{notGoingUsers} -- }
   \item{
\texttt{goingUsers} -- }
   \item{
\texttt{goneUsers} -- }
  \end{itemize}
}%end item
\end{itemize}
}%end item
\end{itemize}
}
\subsubsection{Methods}{
\vskip -2em
\begin{itemize}
\item{ 
\index{createGO()}
\hypertarget{edu.kit.pse17.go_app.model.GO.createGO()}{{\bf  createGO}\\}
\begin{lstlisting}[frame=none]
public static GO createGO()\end{lstlisting} %end signature
\begin{itemize}
\item{
{\bf  Description}

erzeugt ein neues GO-Objekt und speichert die GO-Daten in der Datenbank auf dem Tomcat Server
}
\item{{\bf  Returns} -- 
das neue GO 
}%end item
\end{itemize}
}%end item
\item{ 
\index{getDescription()}
\hypertarget{edu.kit.pse17.go_app.model.GO.getDescription()}{{\bf  getDescription}\\}
\begin{lstlisting}[frame=none]
public java.lang.String getDescription()\end{lstlisting} %end signature
}%end item
\item{ 
\index{getEnd()}
\hypertarget{edu.kit.pse17.go_app.model.GO.getEnd()}{{\bf  getEnd}\\}
\begin{lstlisting}[frame=none]
public java.util.Date getEnd()\end{lstlisting} %end signature
}%end item
\item{ 
\index{getGoingUsers()}
\hypertarget{edu.kit.pse17.go_app.model.GO.getGoingUsers()}{{\bf  getGoingUsers}\\}
\begin{lstlisting}[frame=none]
public java.util.List getGoingUsers()\end{lstlisting} %end signature
}%end item
\item{ 
\index{getGoneUsers()}
\hypertarget{edu.kit.pse17.go_app.model.GO.getGoneUsers()}{{\bf  getGoneUsers}\\}
\begin{lstlisting}[frame=none]
public java.util.List getGoneUsers()\end{lstlisting} %end signature
}%end item
\item{ 
\index{getID()}
\hypertarget{edu.kit.pse17.go_app.model.GO.getID()}{{\bf  getID}\\}
\begin{lstlisting}[frame=none]
public long getID()\end{lstlisting} %end signature
}%end item
\item{ 
\index{getLat()}
\hypertarget{edu.kit.pse17.go_app.model.GO.getLat()}{{\bf  getLat}\\}
\begin{lstlisting}[frame=none]
public long getLat()\end{lstlisting} %end signature
}%end item
\item{ 
\index{getLocationData()}
\hypertarget{edu.kit.pse17.go_app.model.GO.getLocationData()}{{\bf  getLocationData}\\}
\begin{lstlisting}[frame=none]
public GroupLocation getLocationData()\end{lstlisting} %end signature
}%end item
\item{ 
\index{getLon()}
\hypertarget{edu.kit.pse17.go_app.model.GO.getLon()}{{\bf  getLon}\\}
\begin{lstlisting}[frame=none]
public long getLon()\end{lstlisting} %end signature
}%end item
\item{ 
\index{getName()}
\hypertarget{edu.kit.pse17.go_app.model.GO.getName()}{{\bf  getName}\\}
\begin{lstlisting}[frame=none]
public java.lang.String getName()\end{lstlisting} %end signature
}%end item
\item{ 
\index{getNotGoingUsers()}
\hypertarget{edu.kit.pse17.go_app.model.GO.getNotGoingUsers()}{{\bf  getNotGoingUsers}\\}
\begin{lstlisting}[frame=none]
public java.util.List getNotGoingUsers()\end{lstlisting} %end signature
}%end item
\item{ 
\index{getOwner()}
\hypertarget{edu.kit.pse17.go_app.model.GO.getOwner()}{{\bf  getOwner}\\}
\begin{lstlisting}[frame=none]
public User getOwner()\end{lstlisting} %end signature
}%end item
\item{ 
\index{getStart()}
\hypertarget{edu.kit.pse17.go_app.model.GO.getStart()}{{\bf  getStart}\\}
\begin{lstlisting}[frame=none]
public java.util.Date getStart()\end{lstlisting} %end signature
}%end item
\item{ 
\index{getUserStatus(User)}
\hypertarget{edu.kit.pse17.go_app.model.GO.getUserStatus(edu.kit.pse17.go_app.model.User)}{{\bf  getUserStatus}\\}
\begin{lstlisting}[frame=none]
public Status getUserStatus(User user)\end{lstlisting} %end signature
}%end item
\item{ 
\index{isOwner(User)}
\hypertarget{edu.kit.pse17.go_app.model.GO.isOwner(edu.kit.pse17.go_app.model.User)}{{\bf  isOwner}\\}
\begin{lstlisting}[frame=none]
public boolean isOwner(User user)\end{lstlisting} %end signature
}%end item
\item{ 
\index{setDescription(String)}
\hypertarget{edu.kit.pse17.go_app.model.GO.setDescription(java.lang.String)}{{\bf  setDescription}\\}
\begin{lstlisting}[frame=none]
public void setDescription(java.lang.String description)\end{lstlisting} %end signature
}%end item
\item{ 
\index{setEnd(Date)}
\hypertarget{edu.kit.pse17.go_app.model.GO.setEnd(java.util.Date)}{{\bf  setEnd}\\}
\begin{lstlisting}[frame=none]
public void setEnd(java.util.Date end)\end{lstlisting} %end signature
}%end item
\item{ 
\index{setGoingUsers(List)}
\hypertarget{edu.kit.pse17.go_app.model.GO.setGoingUsers(java.util.List)}{{\bf  setGoingUsers}\\}
\begin{lstlisting}[frame=none]
public void setGoingUsers(java.util.List goingUsers)\end{lstlisting} %end signature
}%end item
\item{ 
\index{setGoneUsers(List)}
\hypertarget{edu.kit.pse17.go_app.model.GO.setGoneUsers(java.util.List)}{{\bf  setGoneUsers}\\}
\begin{lstlisting}[frame=none]
public void setGoneUsers(java.util.List goneUsers)\end{lstlisting} %end signature
}%end item
\item{ 
\index{setID(long)}
\hypertarget{edu.kit.pse17.go_app.model.GO.setID(long)}{{\bf  setID}\\}
\begin{lstlisting}[frame=none]
public void setID(long ID)\end{lstlisting} %end signature
}%end item
\item{ 
\index{setLat(long)}
\hypertarget{edu.kit.pse17.go_app.model.GO.setLat(long)}{{\bf  setLat}\\}
\begin{lstlisting}[frame=none]
public void setLat(long lat)\end{lstlisting} %end signature
}%end item
\item{ 
\index{setLocationData(GroupLocation)}
\hypertarget{edu.kit.pse17.go_app.model.GO.setLocationData(edu.kit.pse17.go_app.model.GroupLocation)}{{\bf  setLocationData}\\}
\begin{lstlisting}[frame=none]
public void setLocationData(GroupLocation locationData)\end{lstlisting} %end signature
}%end item
\item{ 
\index{setLon(long)}
\hypertarget{edu.kit.pse17.go_app.model.GO.setLon(long)}{{\bf  setLon}\\}
\begin{lstlisting}[frame=none]
public void setLon(long lon)\end{lstlisting} %end signature
}%end item
\item{ 
\index{setName(String)}
\hypertarget{edu.kit.pse17.go_app.model.GO.setName(java.lang.String)}{{\bf  setName}\\}
\begin{lstlisting}[frame=none]
public void setName(java.lang.String name)\end{lstlisting} %end signature
}%end item
\item{ 
\index{setNotGoingUsers(List)}
\hypertarget{edu.kit.pse17.go_app.model.GO.setNotGoingUsers(java.util.List)}{{\bf  setNotGoingUsers}\\}
\begin{lstlisting}[frame=none]
public void setNotGoingUsers(java.util.List notGoingUsers)\end{lstlisting} %end signature
}%end item
\item{ 
\index{setOwner(User)}
\hypertarget{edu.kit.pse17.go_app.model.GO.setOwner(edu.kit.pse17.go_app.model.User)}{{\bf  setOwner}\\}
\begin{lstlisting}[frame=none]
public void setOwner(User owner)\end{lstlisting} %end signature
}%end item
\item{ 
\index{setStart(Date)}
\hypertarget{edu.kit.pse17.go_app.model.GO.setStart(java.util.Date)}{{\bf  setStart}\\}
\begin{lstlisting}[frame=none]
public void setStart(java.util.Date start)\end{lstlisting} %end signature
}%end item
\end{itemize}
}
}
\subsection{\label{edu.kit.pse17.go_app.model.Group}Class Group}{
\hypertarget{edu.kit.pse17.go_app.model.Group}{}\vskip .1in 
Diese Klasse verwaltet Gruppen Objekte Created by tina on 17.06.17.\vskip .1in 
\subsubsection{Declaration}{
\begin{lstlisting}[frame=none]
public class Group
 extends java.lang.Object\end{lstlisting}
\subsubsection{Constructor summary}{
\begin{verse}
\hyperlink{edu.kit.pse17.go_app.model.Group(int, java.lang.String, java.lang.String, Icon, java.util.ArrayList, int)}{{\bf Group(int, String, String, Icon, ArrayList, int)}} Konstruktor\\
\end{verse}
}
\subsubsection{Method summary}{
\begin{verse}
\hyperlink{edu.kit.pse17.go_app.model.Group.createGroup()}{{\bf createGroup()}} erzeugt ein neues Group-Objekt und speichert die Group-Daten in der Datenbank auf dem Tomcat Server\\
\hyperlink{edu.kit.pse17.go_app.model.Group.getAllGroupRequests(java.lang.String)}{{\bf getAllGroupRequests(String)}} Gibt eine Liste mit allen Gruppen zurück, zu denen der Benutzer mit der ID *uid* eingeladen wurde (und auf die Anfrage noch nicht geantwortet hat)\\
\hyperlink{edu.kit.pse17.go_app.model.Group.getAllGroups(java.lang.String)}{{\bf getAllGroups(String)}} Gibt eine Liste mit allen Gruppen des Benutzer mit der ID *uid* zurück\\
\hyperlink{edu.kit.pse17.go_app.model.Group.getDescription()}{{\bf getDescription()}} \\
\hyperlink{edu.kit.pse17.go_app.model.Group.getIcon()}{{\bf getIcon()}} \\
\hyperlink{edu.kit.pse17.go_app.model.Group.getMemberCount()}{{\bf getMemberCount()}} \\
\hyperlink{edu.kit.pse17.go_app.model.Group.getMembers()}{{\bf getMembers()}} \\
\hyperlink{edu.kit.pse17.go_app.model.Group.getName()}{{\bf getName()}} \\
\hyperlink{edu.kit.pse17.go_app.model.Group.isAdmin(edu.kit.pse17.go_app.model.User)}{{\bf isAdmin(User)}} \\
\hyperlink{edu.kit.pse17.go_app.model.Group.setDescription(java.lang.String)}{{\bf setDescription(String)}} \\
\hyperlink{edu.kit.pse17.go_app.model.Group.setIcon(Icon)}{{\bf setIcon(Icon)}} \\
\hyperlink{edu.kit.pse17.go_app.model.Group.setMemberCount(int)}{{\bf setMemberCount(int)}} \\
\hyperlink{edu.kit.pse17.go_app.model.Group.setMembers(java.util.ArrayList)}{{\bf setMembers(ArrayList)}} \\
\hyperlink{edu.kit.pse17.go_app.model.Group.setName(java.lang.String)}{{\bf setName(String)}} \\
\end{verse}
}
\subsubsection{Constructors}{
\vskip -2em
\begin{itemize}
\item{ 
\index{Group(int, String, String, Icon, ArrayList, int)}
\hypertarget{edu.kit.pse17.go_app.model.Group(int, java.lang.String, java.lang.String, Icon, java.util.ArrayList, int)}{{\bf  Group}\\}
\begin{lstlisting}[frame=none]
public Group(int ID,java.lang.String name,java.lang.String description,Icon icon,java.util.ArrayList members,int memberCount)\end{lstlisting} %end signature
\begin{itemize}
\item{
{\bf  Description}

Konstruktor
}
\item{
{\bf  Parameters}
  \begin{itemize}
   \item{
\texttt{ID} -- eindeutige Nummer, mit der eine Gruppe identifiziert werden kann}
   \item{
\texttt{name} -- Gruppenname}
   \item{
\texttt{description} -- Gruppenbeschreibung}
   \item{
\texttt{icon} -- Gruppenicon}
   \item{
\texttt{members} -- Liste aller Gruppenmitglieder}
   \item{
\texttt{memberCount} -- Anzahl der Gruppenmitglieder}
  \end{itemize}
}%end item
\end{itemize}
}%end item
\end{itemize}
}
\subsubsection{Methods}{
\vskip -2em
\begin{itemize}
\item{ 
\index{createGroup()}
\hypertarget{edu.kit.pse17.go_app.model.Group.createGroup()}{{\bf  createGroup}\\}
\begin{lstlisting}[frame=none]
public static Group createGroup()\end{lstlisting} %end signature
\begin{itemize}
\item{
{\bf  Description}

erzeugt ein neues Group-Objekt und speichert die Group-Daten in der Datenbank auf dem Tomcat Server
}
\item{{\bf  Returns} -- 
die neue Gruppe 
}%end item
\end{itemize}
}%end item
\item{ 
\index{getAllGroupRequests(String)}
\hypertarget{edu.kit.pse17.go_app.model.Group.getAllGroupRequests(java.lang.String)}{{\bf  getAllGroupRequests}\\}
\begin{lstlisting}[frame=none]
public static java.util.List getAllGroupRequests(java.lang.String uid)\end{lstlisting} %end signature
\begin{itemize}
\item{
{\bf  Description}

Gibt eine Liste mit allen Gruppen zurück, zu denen der Benutzer mit der ID *uid* eingeladen wurde (und auf die Anfrage noch nicht geantwortet hat)
}
\item{
{\bf  Parameters}
  \begin{itemize}
   \item{
\texttt{uid} -- Die User-ID des Benutzers}
  \end{itemize}
}%end item
\item{{\bf  Returns} -- 
Liste mit allen Gruppenanfragen des Benutzers 
}%end item
\end{itemize}
}%end item
\item{ 
\index{getAllGroups(String)}
\hypertarget{edu.kit.pse17.go_app.model.Group.getAllGroups(java.lang.String)}{{\bf  getAllGroups}\\}
\begin{lstlisting}[frame=none]
public static java.util.List getAllGroups(java.lang.String uid)\end{lstlisting} %end signature
\begin{itemize}
\item{
{\bf  Description}

Gibt eine Liste mit allen Gruppen des Benutzer mit der ID *uid* zurück
}
\item{
{\bf  Parameters}
  \begin{itemize}
   \item{
\texttt{uid} -- Die User-ID des Benutzer}
  \end{itemize}
}%end item
\item{{\bf  Returns} -- 
Liste mit Gruppen des Benutzers 
}%end item
\end{itemize}
}%end item
\item{ 
\index{getDescription()}
\hypertarget{edu.kit.pse17.go_app.model.Group.getDescription()}{{\bf  getDescription}\\}
\begin{lstlisting}[frame=none]
public java.lang.String getDescription()\end{lstlisting} %end signature
}%end item
\item{ 
\index{getIcon()}
\hypertarget{edu.kit.pse17.go_app.model.Group.getIcon()}{{\bf  getIcon}\\}
\begin{lstlisting}[frame=none]
public Icon getIcon()\end{lstlisting} %end signature
}%end item
\item{ 
\index{getMemberCount()}
\hypertarget{edu.kit.pse17.go_app.model.Group.getMemberCount()}{{\bf  getMemberCount}\\}
\begin{lstlisting}[frame=none]
public int getMemberCount()\end{lstlisting} %end signature
}%end item
\item{ 
\index{getMembers()}
\hypertarget{edu.kit.pse17.go_app.model.Group.getMembers()}{{\bf  getMembers}\\}
\begin{lstlisting}[frame=none]
public java.util.List getMembers()\end{lstlisting} %end signature
}%end item
\item{ 
\index{getName()}
\hypertarget{edu.kit.pse17.go_app.model.Group.getName()}{{\bf  getName}\\}
\begin{lstlisting}[frame=none]
public java.lang.String getName()\end{lstlisting} %end signature
}%end item
\item{ 
\index{isAdmin(User)}
\hypertarget{edu.kit.pse17.go_app.model.Group.isAdmin(edu.kit.pse17.go_app.model.User)}{{\bf  isAdmin}\\}
\begin{lstlisting}[frame=none]
public boolean isAdmin(User user)\end{lstlisting} %end signature
}%end item
\item{ 
\index{setDescription(String)}
\hypertarget{edu.kit.pse17.go_app.model.Group.setDescription(java.lang.String)}{{\bf  setDescription}\\}
\begin{lstlisting}[frame=none]
public void setDescription(java.lang.String description)\end{lstlisting} %end signature
}%end item
\item{ 
\index{setIcon(Icon)}
\hypertarget{edu.kit.pse17.go_app.model.Group.setIcon(Icon)}{{\bf  setIcon}\\}
\begin{lstlisting}[frame=none]
public void setIcon(Icon icon)\end{lstlisting} %end signature
}%end item
\item{ 
\index{setMemberCount(int)}
\hypertarget{edu.kit.pse17.go_app.model.Group.setMemberCount(int)}{{\bf  setMemberCount}\\}
\begin{lstlisting}[frame=none]
public void setMemberCount(int memberCount)\end{lstlisting} %end signature
}%end item
\item{ 
\index{setMembers(ArrayList)}
\hypertarget{edu.kit.pse17.go_app.model.Group.setMembers(java.util.ArrayList)}{{\bf  setMembers}\\}
\begin{lstlisting}[frame=none]
public void setMembers(java.util.ArrayList members)\end{lstlisting} %end signature
}%end item
\item{ 
\index{setName(String)}
\hypertarget{edu.kit.pse17.go_app.model.Group.setName(java.lang.String)}{{\bf  setName}\\}
\begin{lstlisting}[frame=none]
public void setName(java.lang.String name)\end{lstlisting} %end signature
}%end item
\end{itemize}
}
}
\subsection{\label{edu.kit.pse17.go_app.model.GroupLocation}Class GroupLocation}{
\hypertarget{edu.kit.pse17.go_app.model.GroupLocation}{}\vskip .1in 
Die Objekte der Klasse kapseln die geclusterten GPS-Daten der GO-Teilnehmer Created by tina on 20.06.17.\vskip .1in 
\subsubsection{Declaration}{
\begin{lstlisting}[frame=none]
public class GroupLocation
 extends java.lang.Object\end{lstlisting}
\subsubsection{Constructor summary}{
\begin{verse}
\hyperlink{edu.kit.pse17.go_app.model.GroupLocation()}{{\bf GroupLocation()}} \\
\end{verse}
}
\subsubsection{Method summary}{
\begin{verse}
\hyperlink{edu.kit.pse17.go_app.model.GroupLocation.getCluster()}{{\bf getCluster()}} \\
\hyperlink{edu.kit.pse17.go_app.model.GroupLocation.setCluster(java.util.List)}{{\bf setCluster(List)}} \\
\end{verse}
}
\subsubsection{Constructors}{
\vskip -2em
\begin{itemize}
\item{ 
\index{GroupLocation()}
\hypertarget{edu.kit.pse17.go_app.model.GroupLocation()}{{\bf  GroupLocation}\\}
\begin{lstlisting}[frame=none]
public GroupLocation()\end{lstlisting} %end signature
}%end item
\end{itemize}
}
\subsubsection{Methods}{
\vskip -2em
\begin{itemize}
\item{ 
\index{getCluster()}
\hypertarget{edu.kit.pse17.go_app.model.GroupLocation.getCluster()}{{\bf  getCluster}\\}
\begin{lstlisting}[frame=none]
public java.util.List getCluster()\end{lstlisting} %end signature
}%end item
\item{ 
\index{setCluster(List)}
\hypertarget{edu.kit.pse17.go_app.model.GroupLocation.setCluster(java.util.List)}{{\bf  setCluster}\\}
\begin{lstlisting}[frame=none]
public void setCluster(java.util.List cluster)\end{lstlisting} %end signature
}%end item
\end{itemize}
}
}
\subsection{\label{edu.kit.pse17.go_app.model.Status}Class Status}{
\hypertarget{edu.kit.pse17.go_app.model.Status}{}\vskip .1in 
möglicher Teilnehmerstatus für GOs Created by tina on 19.06.17.\vskip .1in 
\subsubsection{Declaration}{
\begin{lstlisting}[frame=none]
public final class Status
 extends java.lang.Enum\end{lstlisting}
\subsubsection{Field summary}{
\begin{verse}
\hyperlink{edu.kit.pse17.go_app.model.Status.GOING}{{\bf GOING}} \\
\hyperlink{edu.kit.pse17.go_app.model.Status.GONE}{{\bf GONE}} \\
\hyperlink{edu.kit.pse17.go_app.model.Status.NOT_GOING}{{\bf NOT\_GOING}} \\
\end{verse}
}
\subsubsection{Method summary}{
\begin{verse}
\hyperlink{edu.kit.pse17.go_app.model.Status.valueOf(java.lang.String)}{{\bf valueOf(String)}} \\
\hyperlink{edu.kit.pse17.go_app.model.Status.values()}{{\bf values()}} \\
\end{verse}
}
\subsubsection{Fields}{
\begin{itemize}
\item{
\index{NOT\_GOING}
\label{edu.kit.pse17.go_app.model.Status.NOT_GOING}\hypertarget{edu.kit.pse17.go_app.model.Status.NOT_GOING}{\texttt{public static final Status\ {\bf  NOT\_GOING}}
}
}
\item{
\index{GOING}
\label{edu.kit.pse17.go_app.model.Status.GOING}\hypertarget{edu.kit.pse17.go_app.model.Status.GOING}{\texttt{public static final Status\ {\bf  GOING}}
}
}
\item{
\index{GONE}
\label{edu.kit.pse17.go_app.model.Status.GONE}\hypertarget{edu.kit.pse17.go_app.model.Status.GONE}{\texttt{public static final Status\ {\bf  GONE}}
}
}
\end{itemize}
}
\subsubsection{Methods}{
\vskip -2em
\begin{itemize}
\item{ 
\index{valueOf(String)}
\hypertarget{edu.kit.pse17.go_app.model.Status.valueOf(java.lang.String)}{{\bf  valueOf}\\}
\begin{lstlisting}[frame=none]
public static Status valueOf(java.lang.String name)\end{lstlisting} %end signature
}%end item
\item{ 
\index{values()}
\hypertarget{edu.kit.pse17.go_app.model.Status.values()}{{\bf  values}\\}
\begin{lstlisting}[frame=none]
public static Status[] values()\end{lstlisting} %end signature
}%end item
\end{itemize}
}
\subsubsection{Members inherited from class Enum }{
\texttt{java.lang.Enum} {\small 
\refdefined{java.lang.Enum}}
{\small 

\vskip -2em
\begin{itemize}
\item{\vskip -1.5ex 
\texttt{protected final Object {\bf  clone}() throws CloneNotSupportedException
}%end signature
}%end item
\item{\vskip -1.5ex 
\texttt{public final int {\bf  compareTo}(\texttt{Enum} {\bf  arg0})
}%end signature
}%end item
\item{\vskip -1.5ex 
\texttt{public final boolean {\bf  equals}(\texttt{Object} {\bf  arg0})
}%end signature
}%end item
\item{\vskip -1.5ex 
\texttt{protected final void {\bf  finalize}()
}%end signature
}%end item
\item{\vskip -1.5ex 
\texttt{public final Class {\bf  getDeclaringClass}()
}%end signature
}%end item
\item{\vskip -1.5ex 
\texttt{public final int {\bf  hashCode}()
}%end signature
}%end item
\item{\vskip -1.5ex 
\texttt{public final String {\bf  name}()
}%end signature
}%end item
\item{\vskip -1.5ex 
\texttt{public final int {\bf  ordinal}()
}%end signature
}%end item
\item{\vskip -1.5ex 
\texttt{public String {\bf  toString}()
}%end signature
}%end item
\item{\vskip -1.5ex 
\texttt{public static Enum {\bf  valueOf}(\texttt{Class} {\bf  arg0},
\texttt{String} {\bf  arg1})
}%end signature
}%end item
\end{itemize}
}
}
\subsection{\label{edu.kit.pse17.go_app.model.User}Class User}{
\hypertarget{edu.kit.pse17.go_app.model.User}{}\vskip .1in 
Diese Klasse verwaltet User Objekte Created by tina on 17.06.17.\vskip .1in 
\subsubsection{Declaration}{
\begin{lstlisting}[frame=none]
public class User
 extends java.lang.Object implements java.io.Serializable\end{lstlisting}
\subsubsection{Constructor summary}{
\begin{verse}
\hyperlink{edu.kit.pse17.go_app.model.User(java.lang.String, java.lang.String, java.lang.String, Icon)}{{\bf User(String, String, String, Icon)}} Konstruktor\\
\end{verse}
}
\subsubsection{Method summary}{
\begin{verse}
\hyperlink{edu.kit.pse17.go_app.model.User.createUser()}{{\bf createUser()}} erzeugt ein neues User-Objekt und speichert die User-Daten in der Datenbank auf dem Tomcat Server\\
\hyperlink{edu.kit.pse17.go_app.model.User.getEmail()}{{\bf getEmail()}} \\
\hyperlink{edu.kit.pse17.go_app.model.User.getGroupRequests()}{{\bf getGroupRequests()}} \\
\hyperlink{edu.kit.pse17.go_app.model.User.getGroups()}{{\bf getGroups()}} \\
\hyperlink{edu.kit.pse17.go_app.model.User.getIcon()}{{\bf getIcon()}} \\
\hyperlink{edu.kit.pse17.go_app.model.User.getMyself()}{{\bf getMyself()}} gibt das User-Objekt, mit den Daten des momentan angemeldeten Benutzers zurück\\
\hyperlink{edu.kit.pse17.go_app.model.User.getName()}{{\bf getName()}} \\
\hyperlink{edu.kit.pse17.go_app.model.User.getStatus(edu.kit.pse17.go_app.model.GO)}{{\bf getStatus(GO)}} \\
\hyperlink{edu.kit.pse17.go_app.model.User.getUid()}{{\bf getUid()}} \\
\hyperlink{edu.kit.pse17.go_app.model.User.onChangeData()}{{\bf onChangeData()}} \\
\hyperlink{edu.kit.pse17.go_app.model.User.setEmail(java.lang.String)}{{\bf setEmail(String)}} \\
\hyperlink{edu.kit.pse17.go_app.model.User.setIcon(Icon)}{{\bf setIcon(Icon)}} \\
\hyperlink{edu.kit.pse17.go_app.model.User.setName(java.lang.String)}{{\bf setName(String)}} \\
\hyperlink{edu.kit.pse17.go_app.model.User.setUid(java.lang.String)}{{\bf setUid(String)}} \\
\end{verse}
}
\subsubsection{Constructors}{
\vskip -2em
\begin{itemize}
\item{ 
\index{User(String, String, String, Icon)}
\hypertarget{edu.kit.pse17.go_app.model.User(java.lang.String, java.lang.String, java.lang.String, Icon)}{{\bf  User}\\}
\begin{lstlisting}[frame=none]
public User(java.lang.String uid,java.lang.String name,java.lang.String email,Icon icon)\end{lstlisting} %end signature
\begin{itemize}
\item{
{\bf  Description}

Konstruktor
}
\item{
{\bf  Parameters}
  \begin{itemize}
   \item{
\texttt{uid} -- User-ID (--\textgreater  übernommen von FirebaseUser-Objekt aus der FirebaseAPI (eindeutig)}
   \item{
\texttt{name} -- Benutzername}
   \item{
\texttt{email} -- E-Mailadresse, die bei der Anmeldung verwendet wurde. Wird verwendet, um User nach anderen Usern suchen zu lassen}
   \item{
\texttt{icon} -- Profilbild}
  \end{itemize}
}%end item
\end{itemize}
}%end item
\end{itemize}
}
\subsubsection{Methods}{
\vskip -2em
\begin{itemize}
\item{ 
\index{createUser()}
\hypertarget{edu.kit.pse17.go_app.model.User.createUser()}{{\bf  createUser}\\}
\begin{lstlisting}[frame=none]
public static User createUser()\end{lstlisting} %end signature
\begin{itemize}
\item{
{\bf  Description}

erzeugt ein neues User-Objekt und speichert die User-Daten in der Datenbank auf dem Tomcat Server
}
\item{{\bf  Returns} -- 
der neue user 
}%end item
\end{itemize}
}%end item
\item{ 
\index{getEmail()}
\hypertarget{edu.kit.pse17.go_app.model.User.getEmail()}{{\bf  getEmail}\\}
\begin{lstlisting}[frame=none]
public java.lang.String getEmail()\end{lstlisting} %end signature
}%end item
\item{ 
\index{getGroupRequests()}
\hypertarget{edu.kit.pse17.go_app.model.User.getGroupRequests()}{{\bf  getGroupRequests}\\}
\begin{lstlisting}[frame=none]
public java.util.List getGroupRequests()\end{lstlisting} %end signature
}%end item
\item{ 
\index{getGroups()}
\hypertarget{edu.kit.pse17.go_app.model.User.getGroups()}{{\bf  getGroups}\\}
\begin{lstlisting}[frame=none]
public java.util.List getGroups()\end{lstlisting} %end signature
}%end item
\item{ 
\index{getIcon()}
\hypertarget{edu.kit.pse17.go_app.model.User.getIcon()}{{\bf  getIcon}\\}
\begin{lstlisting}[frame=none]
public Icon getIcon()\end{lstlisting} %end signature
}%end item
\item{ 
\index{getMyself()}
\hypertarget{edu.kit.pse17.go_app.model.User.getMyself()}{{\bf  getMyself}\\}
\begin{lstlisting}[frame=none]
public static User getMyself()\end{lstlisting} %end signature
\begin{itemize}
\item{
{\bf  Description}

gibt das User-Objekt, mit den Daten des momentan angemeldeten Benutzers zurück
}
\item{{\bf  Returns} -- 
der angemeldete Benutzer 
}%end item
\end{itemize}
}%end item
\item{ 
\index{getName()}
\hypertarget{edu.kit.pse17.go_app.model.User.getName()}{{\bf  getName}\\}
\begin{lstlisting}[frame=none]
public java.lang.String getName()\end{lstlisting} %end signature
}%end item
\item{ 
\index{getStatus(GO)}
\hypertarget{edu.kit.pse17.go_app.model.User.getStatus(edu.kit.pse17.go_app.model.GO)}{{\bf  getStatus}\\}
\begin{lstlisting}[frame=none]
public Status getStatus(GO go)\end{lstlisting} %end signature
}%end item
\item{ 
\index{getUid()}
\hypertarget{edu.kit.pse17.go_app.model.User.getUid()}{{\bf  getUid}\\}
\begin{lstlisting}[frame=none]
public java.lang.String getUid()\end{lstlisting} %end signature
}%end item
\item{ 
\index{onChangeData()}
\hypertarget{edu.kit.pse17.go_app.model.User.onChangeData()}{{\bf  onChangeData}\\}
\begin{lstlisting}[frame=none]
public void onChangeData()\end{lstlisting} %end signature
}%end item
\item{ 
\index{setEmail(String)}
\hypertarget{edu.kit.pse17.go_app.model.User.setEmail(java.lang.String)}{{\bf  setEmail}\\}
\begin{lstlisting}[frame=none]
public void setEmail(java.lang.String email)\end{lstlisting} %end signature
}%end item
\item{ 
\index{setIcon(Icon)}
\hypertarget{edu.kit.pse17.go_app.model.User.setIcon(Icon)}{{\bf  setIcon}\\}
\begin{lstlisting}[frame=none]
public void setIcon(Icon icon)\end{lstlisting} %end signature
}%end item
\item{ 
\index{setName(String)}
\hypertarget{edu.kit.pse17.go_app.model.User.setName(java.lang.String)}{{\bf  setName}\\}
\begin{lstlisting}[frame=none]
public void setName(java.lang.String name)\end{lstlisting} %end signature
}%end item
\item{ 
\index{setUid(String)}
\hypertarget{edu.kit.pse17.go_app.model.User.setUid(java.lang.String)}{{\bf  setUid}\\}
\begin{lstlisting}[frame=none]
public void setUid(java.lang.String uid)\end{lstlisting} %end signature
}%end item
\end{itemize}
}
}
}
\section{Package edu.kit.pse17.go\_app.serverCommunication.downstream}{
\label{edu.kit.pse17.go_app.serverCommunication.downstream}\hypertarget{edu.kit.pse17.go_app.serverCommunication.downstream}{}
\hskip -.05in
\hbox to \hsize{\textit{ Package Contents\hfil Page}}
\vskip .13in
\hbox{{\bf  Classes}}
\entityintro{MessagingService}{edu.kit.pse17.go_app.serverCommunication.downstream.MessagingService}{Die Klasse implementiert einen Service, der auf den GO TOmcat-Server hört.}
\entityintro{TokenService}{edu.kit.pse17.go_app.serverCommunication.downstream.TokenService}{Die Klasse erzeugt ein InstanceID Token, welches an den Server übergeben wird, um von Server-Seite aus Nachrichten an ein einzelnes Gerät schicken zu können.}
\vskip .1in
\vskip .1in
\subsection{\label{edu.kit.pse17.go_app.serverCommunication.downstream.MessagingService}Class MessagingService}{
\hypertarget{edu.kit.pse17.go_app.serverCommunication.downstream.MessagingService}{}\vskip .1in 
Die Klasse implementiert einen Service, der auf den GO TOmcat-Server hört. Bei Ankunft einer Nachricht des Servers, wird die onMessageReceived-methode aufgerufen (sofern die App im Vordergrund läuft). Läuft die App im Hintergrund, ... Created by tina on 28.06.17.\vskip .1in 
\subsubsection{Declaration}{
\begin{lstlisting}[frame=none]
public class MessagingService
 extends FirebaseMessagingService\end{lstlisting}
\subsubsection{Constructor summary}{
\begin{verse}
\hyperlink{edu.kit.pse17.go_app.serverCommunication.downstream.MessagingService()}{{\bf MessagingService()}} \\
\end{verse}
}
\subsubsection{Method summary}{
\begin{verse}
\hyperlink{edu.kit.pse17.go_app.serverCommunication.downstream.MessagingService.onMessageReceived(RemoteMessage)}{{\bf onMessageReceived(RemoteMessage)}} wird aufgerufen, soblad die App eine Nachricht des Go Tomcat-Servers erhält.\\
\end{verse}
}
\subsubsection{Constructors}{
\vskip -2em
\begin{itemize}
\item{ 
\index{MessagingService()}
\hypertarget{edu.kit.pse17.go_app.serverCommunication.downstream.MessagingService()}{{\bf  MessagingService}\\}
\begin{lstlisting}[frame=none]
public MessagingService()\end{lstlisting} %end signature
}%end item
\end{itemize}
}
\subsubsection{Methods}{
\vskip -2em
\begin{itemize}
\item{ 
\index{onMessageReceived(RemoteMessage)}
\hypertarget{edu.kit.pse17.go_app.serverCommunication.downstream.MessagingService.onMessageReceived(RemoteMessage)}{{\bf  onMessageReceived}\\}
\begin{lstlisting}[frame=none]
public void onMessageReceived(RemoteMessage remoteMessage)\end{lstlisting} %end signature
\begin{itemize}
\item{
{\bf  Description}

wird aufgerufen, soblad die App eine Nachricht des Go Tomcat-Servers erhält.
}
\item{
{\bf  Parameters}
  \begin{itemize}
   \item{
\texttt{remoteMessage} -- die erhaltene Nachricht}
  \end{itemize}
}%end item
\end{itemize}
}%end item
\end{itemize}
}
}
\subsection{\label{edu.kit.pse17.go_app.serverCommunication.downstream.TokenService}Class TokenService}{
\hypertarget{edu.kit.pse17.go_app.serverCommunication.downstream.TokenService}{}\vskip .1in 
Die Klasse erzeugt ein InstanceID Token, welches an den Server übergeben wird, um von Server-Seite aus Nachrichten an ein einzelnes Gerät schicken zu können. Created by tina on 28.06.17.\vskip .1in 
\subsubsection{Declaration}{
\begin{lstlisting}[frame=none]
public class TokenService
 extends FirebaseInstanceIdService\end{lstlisting}
\subsubsection{Constructor summary}{
\begin{verse}
\hyperlink{edu.kit.pse17.go_app.serverCommunication.downstream.TokenService()}{{\bf TokenService()}} \\
\end{verse}
}
\subsubsection{Method summary}{
\begin{verse}
\hyperlink{edu.kit.pse17.go_app.serverCommunication.downstream.TokenService.onTokenRefresh()}{{\bf onTokenRefresh()}} holt das neue Token und veranlasst das Programm, das neue Token an den Server zu übergeben.\\
\end{verse}
}
\subsubsection{Constructors}{
\vskip -2em
\begin{itemize}
\item{ 
\index{TokenService()}
\hypertarget{edu.kit.pse17.go_app.serverCommunication.downstream.TokenService()}{{\bf  TokenService}\\}
\begin{lstlisting}[frame=none]
public TokenService()\end{lstlisting} %end signature
}%end item
\end{itemize}
}
\subsubsection{Methods}{
\vskip -2em
\begin{itemize}
\item{ 
\index{onTokenRefresh()}
\hypertarget{edu.kit.pse17.go_app.serverCommunication.downstream.TokenService.onTokenRefresh()}{{\bf  onTokenRefresh}\\}
\begin{lstlisting}[frame=none]
public void onTokenRefresh()\end{lstlisting} %end signature
\begin{itemize}
\item{
{\bf  Description}

holt das neue Token und veranlasst das Programm, das neue Token an den Server zu übergeben.
}
\end{itemize}
}%end item
\end{itemize}
}
}
}
\section{Package edu.kit.pse17.go\_app.serverCommunication.upstream}{
\label{edu.kit.pse17.go_app.serverCommunication.upstream}\hypertarget{edu.kit.pse17.go_app.serverCommunication.upstream}{}
\hskip -.05in
\hbox to \hsize{\textit{ Package Contents\hfil Page}}
\vskip .13in
\hbox{{\bf  Interfaces}}
\entityintro{TomcatRestApi}{edu.kit.pse17.go_app.serverCommunication.upstream.TomcatRestApi}{das Interface ist die Schnittstelle des Clients zur REST-API des Tomcat-Servers.}
\vskip .13in
\hbox{{\bf  Classes}}
\entityintro{RestMessagingService}{edu.kit.pse17.go_app.serverCommunication.upstream.RestMessagingService}{Created by tina on 30.06.17.}
\vskip .1in
\vskip .1in
\subsection{\label{edu.kit.pse17.go_app.serverCommunication.upstream.TomcatRestApi}Interface TomcatRestApi}{
\hypertarget{edu.kit.pse17.go_app.serverCommunication.upstream.TomcatRestApi}{}\vskip .1in 
das Interface ist die Schnittstelle des Clients zur REST-API des Tomcat-Servers. Created by tina on 29.06.17.\vskip .1in 
\subsubsection{Declaration}{
\begin{lstlisting}[frame=none]
public interface TomcatRestApi
\end{lstlisting}
}
\subsection{\label{edu.kit.pse17.go_app.serverCommunication.upstream.RestMessagingService}Class RestMessagingService}{
\hypertarget{edu.kit.pse17.go_app.serverCommunication.upstream.RestMessagingService}{}\vskip .1in 
Created by tina on 30.06.17.\vskip .1in 
\subsubsection{Declaration}{
\begin{lstlisting}[frame=none]
public abstract class RestMessagingService
 extends java.lang.Object\end{lstlisting}
\subsubsection{Constructor summary}{
\begin{verse}
\hyperlink{edu.kit.pse17.go_app.serverCommunication.upstream.RestMessagingService()}{{\bf RestMessagingService()}} \\
\end{verse}
}
\subsubsection{Constructors}{
\vskip -2em
\begin{itemize}
\item{ 
\index{RestMessagingService()}
\hypertarget{edu.kit.pse17.go_app.serverCommunication.upstream.RestMessagingService()}{{\bf  RestMessagingService}\\}
\begin{lstlisting}[frame=none]
public RestMessagingService()\end{lstlisting} %end signature
}%end item
\end{itemize}
}
}
}
\section{Package edu.kit.pse17.go\_app.controller}{
\label{edu.kit.pse17.go_app.controller}\hypertarget{edu.kit.pse17.go_app.controller}{}
\hskip -.05in
\hbox to \hsize{\textit{ Package Contents\hfil Page}}
\vskip .13in
\hbox{{\bf  Classes}}
\entityintro{ServerMessageReceivedEvent}{edu.kit.pse17.go_app.controller.ServerMessageReceivedEvent}{Created by tina on 28.06.17.}
\vskip .1in
\vskip .1in
\subsection{\label{edu.kit.pse17.go_app.controller.ServerMessageReceivedEvent}Class ServerMessageReceivedEvent}{
\hypertarget{edu.kit.pse17.go_app.controller.ServerMessageReceivedEvent}{}\vskip .1in 
Created by tina on 28.06.17.\vskip .1in 
\subsubsection{Declaration}{
\begin{lstlisting}[frame=none]
public class ServerMessageReceivedEvent
 extends java.lang.Object\end{lstlisting}
\subsubsection{Constructor summary}{
\begin{verse}
\hyperlink{edu.kit.pse17.go_app.controller.ServerMessageReceivedEvent()}{{\bf ServerMessageReceivedEvent()}} \\
\end{verse}
}
\subsubsection{Method summary}{
\begin{verse}
\hyperlink{edu.kit.pse17.go_app.controller.ServerMessageReceivedEvent.onGoCreated(edu.kit.pse17.go_app.model.GO)}{{\bf onGoCreated(GO)}} Methode wird aufgerufen, wenn in einer Gruppe des Benutzers ein neues GO erstellt wird.\\
\hyperlink{edu.kit.pse17.go_app.controller.ServerMessageReceivedEvent.onGoInfoChanged(edu.kit.pse17.go_app.model.GO)}{{\bf onGoInfoChanged(GO)}} Methode wird aufgerufen, wenn der Server der App meldet, dass sich Daten innerhalb des GOs go verändert haben.\\
\hyperlink{edu.kit.pse17.go_app.controller.ServerMessageReceivedEvent.onGroupInfoChanged(edu.kit.pse17.go_app.model.Group)}{{\bf onGroupInfoChanged(Group)}} Methode wird aufgerufen, wenn der Server der App meldet, dass sich Daten in der Gruppe group verändert haben.\\
\hyperlink{edu.kit.pse17.go_app.controller.ServerMessageReceivedEvent.onGroupRequestReceived(edu.kit.pse17.go_app.model.Group)}{{\bf onGroupRequestReceived(Group)}} Methode wird aufgerufen, wenn der Benutzer der App eine neue Gruppenanfrage erhält, d.h. seinem Account eine neue Gruppe hinzugefügt wird\\
\end{verse}
}
\subsubsection{Constructors}{
\vskip -2em
\begin{itemize}
\item{ 
\index{ServerMessageReceivedEvent()}
\hypertarget{edu.kit.pse17.go_app.controller.ServerMessageReceivedEvent()}{{\bf  ServerMessageReceivedEvent}\\}
\begin{lstlisting}[frame=none]
public ServerMessageReceivedEvent()\end{lstlisting} %end signature
}%end item
\end{itemize}
}
\subsubsection{Methods}{
\vskip -2em
\begin{itemize}
\item{ 
\index{onGoCreated(GO)}
\hypertarget{edu.kit.pse17.go_app.controller.ServerMessageReceivedEvent.onGoCreated(edu.kit.pse17.go_app.model.GO)}{{\bf  onGoCreated}\\}
\begin{lstlisting}[frame=none]
public void onGoCreated(edu.kit.pse17.go_app.model.GO go)\end{lstlisting} %end signature
\begin{itemize}
\item{
{\bf  Description}

Methode wird aufgerufen, wenn in einer Gruppe des Benutzers ein neues GO erstellt wird.
}
\item{
{\bf  Parameters}
  \begin{itemize}
   \item{
\texttt{go} -- Das neu erstellte GO}
  \end{itemize}
}%end item
\end{itemize}
}%end item
\item{ 
\index{onGoInfoChanged(GO)}
\hypertarget{edu.kit.pse17.go_app.controller.ServerMessageReceivedEvent.onGoInfoChanged(edu.kit.pse17.go_app.model.GO)}{{\bf  onGoInfoChanged}\\}
\begin{lstlisting}[frame=none]
public void onGoInfoChanged(edu.kit.pse17.go_app.model.GO go)\end{lstlisting} %end signature
\begin{itemize}
\item{
{\bf  Description}

Methode wird aufgerufen, wenn der Server der App meldet, dass sich Daten innerhalb des GOs go verändert haben. Dazu gehören nicht die Standorte der Mitglieder. Diese werden separat von der App in der Klasse *** von Server abgefragt
}
\item{
{\bf  Parameters}
  \begin{itemize}
   \item{
\texttt{go} -- Das veränderte GO}
  \end{itemize}
}%end item
\end{itemize}
}%end item
\item{ 
\index{onGroupInfoChanged(Group)}
\hypertarget{edu.kit.pse17.go_app.controller.ServerMessageReceivedEvent.onGroupInfoChanged(edu.kit.pse17.go_app.model.Group)}{{\bf  onGroupInfoChanged}\\}
\begin{lstlisting}[frame=none]
public void onGroupInfoChanged(edu.kit.pse17.go_app.model.Group group)\end{lstlisting} %end signature
\begin{itemize}
\item{
{\bf  Description}

Methode wird aufgerufen, wenn der Server der App meldet, dass sich Daten in der Gruppe group verändert haben. Dazu gehören Änderungen in den Gruppendaten selbst, sowie hinzugefügte und entfernte Mitglieder.
}
\item{
{\bf  Parameters}
  \begin{itemize}
   \item{
\texttt{group} -- Die veränderte Gruppe}
  \end{itemize}
}%end item
\end{itemize}
}%end item
\item{ 
\index{onGroupRequestReceived(Group)}
\hypertarget{edu.kit.pse17.go_app.controller.ServerMessageReceivedEvent.onGroupRequestReceived(edu.kit.pse17.go_app.model.Group)}{{\bf  onGroupRequestReceived}\\}
\begin{lstlisting}[frame=none]
public void onGroupRequestReceived(edu.kit.pse17.go_app.model.Group group)\end{lstlisting} %end signature
\begin{itemize}
\item{
{\bf  Description}

Methode wird aufgerufen, wenn der Benutzer der App eine neue Gruppenanfrage erhält, d.h. seinem Account eine neue Gruppe hinzugefügt wird
}
\item{
{\bf  Parameters}
  \begin{itemize}
   \item{
\texttt{group} -- Die Gruppe, zu der der Benutzer eingeladen wurde}
  \end{itemize}
}%end item
\end{itemize}
}%end item
\end{itemize}
}
}
}
\section{Package edu.kit.pse17.go\_app.controller.login}{
\label{edu.kit.pse17.go_app.controller.login}\hypertarget{edu.kit.pse17.go_app.controller.login}{}
\hskip -.05in
\hbox to \hsize{\textit{ Package Contents\hfil Page}}
\vskip .13in
\hbox{{\bf  Classes}}
\entityintro{FirebaseSignInHelper}{edu.kit.pse17.go_app.controller.login.FirebaseSignInHelper}{Diese Klasse ist für die Kommunikation mit Firebase und Google API zuständig während des Login-Prozesses zuständig.}
\entityintro{GoSignInHelper}{edu.kit.pse17.go_app.controller.login.GoSignInHelper}{Die Klasse ist für die Anmeldung eines Beutzers am GO-Server zuständig.}
\entityintro{SignInHelper}{edu.kit.pse17.go_app.controller.login.SignInHelper}{Abstrakte Klasse, die als Schablone für den Anmelde-Prozess ihrer Unterklassen dient Created by tina on 18.06.17.}
\vskip .1in
\vskip .1in
\subsection{\label{edu.kit.pse17.go_app.controller.login.FirebaseSignInHelper}Class FirebaseSignInHelper}{
\hypertarget{edu.kit.pse17.go_app.controller.login.FirebaseSignInHelper}{}\vskip .1in 
Diese Klasse ist für die Kommunikation mit Firebase und Google API zuständig während des Login-Prozesses zuständig. Sie implementiert die Methoden configureSignIn() und startSignInProcess() zur Schablonenmethode signIn() der Oberklasse SignInHelper. Created by tina on 17.06.17.\vskip .1in 
\subsubsection{Declaration}{
\begin{lstlisting}[frame=none]
public class FirebaseSignInHelper
 extends edu.kit.pse17.go_app.controller.login.SignInHelper\end{lstlisting}
\subsubsection{Constructor summary}{
\begin{verse}
\hyperlink{edu.kit.pse17.go_app.controller.login.FirebaseSignInHelper()}{{\bf FirebaseSignInHelper()}} \\
\end{verse}
}
\subsubsection{Method summary}{
\begin{verse}
\hyperlink{edu.kit.pse17.go_app.controller.login.FirebaseSignInHelper.configureSignIn()}{{\bf configureSignIn()}} Implementierung gehört zur Schablonenmethode signIn()\\
\hyperlink{edu.kit.pse17.go_app.controller.login.FirebaseSignInHelper.onActivityResult(int, int, Intent)}{{\bf onActivityResult(int, int, Intent)}} Methode erwartet das Resultat der signInAktivität der GoogleSignInApi.\\
\hyperlink{edu.kit.pse17.go_app.controller.login.FirebaseSignInHelper.onConnectionFailed(ConnectionResult)}{{\bf onConnectionFailed(ConnectionResult)}} wird aufgerufen, falls Verbindung zu google Play Services fehlschlägt\\
\hyperlink{edu.kit.pse17.go_app.controller.login.FirebaseSignInHelper.startSignInProcess()}{{\bf startSignInProcess()}} Implementierung gehört zur Schablonenmethode signIn() Die Methode startet die signIn Aktivität der GoogleSignInApi\\
\end{verse}
}
\subsubsection{Constructors}{
\vskip -2em
\begin{itemize}
\item{ 
\index{FirebaseSignInHelper()}
\hypertarget{edu.kit.pse17.go_app.controller.login.FirebaseSignInHelper()}{{\bf  FirebaseSignInHelper}\\}
\begin{lstlisting}[frame=none]
public FirebaseSignInHelper()\end{lstlisting} %end signature
}%end item
\end{itemize}
}
\subsubsection{Methods}{
\vskip -2em
\begin{itemize}
\item{ 
\index{configureSignIn()}
\hypertarget{edu.kit.pse17.go_app.controller.login.FirebaseSignInHelper.configureSignIn()}{{\bf  configureSignIn}\\}
\begin{lstlisting}[frame=none]
protected void configureSignIn()\end{lstlisting} %end signature
\begin{itemize}
\item{
{\bf  Description}

Implementierung gehört zur Schablonenmethode signIn()
}
\end{itemize}
}%end item
\item{ 
\index{onActivityResult(int, int, Intent)}
\hypertarget{edu.kit.pse17.go_app.controller.login.FirebaseSignInHelper.onActivityResult(int, int, Intent)}{{\bf  onActivityResult}\\}
\begin{lstlisting}[frame=none]
protected void onActivityResult(int requestCode,int resultCode,Intent data)\end{lstlisting} %end signature
\begin{itemize}
\item{
{\bf  Description}

Methode erwartet das Resultat der signInAktivität der GoogleSignInApi. War die Aktivität erfolgreich, wird die Authentifizierung mit Firebase gestartet.
}
\item{
{\bf  Parameters}
  \begin{itemize}
   \item{
\texttt{requestCode} -- Request Code, mit dem Aktivität gestartet wurde}
   \item{
\texttt{resultCode} -- Result Code der Aktivität}
   \item{
\texttt{data} -- Intent, den die Aktivität übergibt}
  \end{itemize}
}%end item
\end{itemize}
}%end item
\item{ 
\index{onConnectionFailed(ConnectionResult)}
\hypertarget{edu.kit.pse17.go_app.controller.login.FirebaseSignInHelper.onConnectionFailed(ConnectionResult)}{{\bf  onConnectionFailed}\\}
\begin{lstlisting}[frame=none]
public void onConnectionFailed(ConnectionResult connectionResult)\end{lstlisting} %end signature
\begin{itemize}
\item{
{\bf  Description}

wird aufgerufen, falls Verbindung zu google Play Services fehlschlägt
}
\item{
{\bf  Parameters}
  \begin{itemize}
   \item{
\texttt{connectionResult} -- Ergebnis der fehlgeschlagenen Verbindung}
  \end{itemize}
}%end item
\end{itemize}
}%end item
\item{ 
\index{startSignInProcess()}
\hypertarget{edu.kit.pse17.go_app.controller.login.FirebaseSignInHelper.startSignInProcess()}{{\bf  startSignInProcess}\\}
\begin{lstlisting}[frame=none]
protected void startSignInProcess()\end{lstlisting} %end signature
\begin{itemize}
\item{
{\bf  Description}

Implementierung gehört zur Schablonenmethode signIn() Die Methode startet die signIn Aktivität der GoogleSignInApi
}
\end{itemize}
}%end item
\end{itemize}
}
\subsubsection{Members inherited from class SignInHelper }{
\texttt{edu.kit.pse17.go_app.controller.login.SignInHelper} {\small 
\refdefined{edu.kit.pse17.go_app.controller.login.SignInHelper}}
{\small 

\vskip -2em
\begin{itemize}
\item{\vskip -1.5ex 
\texttt{public static final {\bf  ACCOUNT\_DATA\_CODE}}%end signature
}%end item
\item{\vskip -1.5ex 
\texttt{protected abstract void {\bf  configureSignIn}()
}%end signature
}%end item
\item{\vskip -1.5ex 
\texttt{protected void {\bf  onCreate}(\texttt{Bundle} {\bf  savedInstanceState})
}%end signature
}%end item
\item{\vskip -1.5ex 
\texttt{protected void {\bf  onStart}()
}%end signature
}%end item
\item{\vskip -1.5ex 
\texttt{protected void {\bf  returnActivityResult}(\texttt{java.io.Serializable} {\bf  accountData})
}%end signature
}%end item
\item{\vskip -1.5ex 
\texttt{public static final {\bf  SIGN\_IN\_DATA\_CODE}}%end signature
}%end item
\item{\vskip -1.5ex 
\texttt{public static void {\bf  signIn}(\texttt{Activity} {\bf  activity},
\texttt{int} {\bf  requestCode},
\texttt{java.io.Serializable} {\bf  signInData},
\texttt{java.lang.Class} {\bf  signinHelper})
}%end signature
}%end item
\item{\vskip -1.5ex 
\texttt{protected abstract void {\bf  startSignInProcess}()
}%end signature
}%end item
\end{itemize}
}
}
\subsection{\label{edu.kit.pse17.go_app.controller.login.GoSignInHelper}Class GoSignInHelper}{
\hypertarget{edu.kit.pse17.go_app.controller.login.GoSignInHelper}{}\vskip .1in 
Die Klasse ist für die Anmeldung eines Beutzers am GO-Server zuständig. Sie implemetiert die Methoden configureSignIn() und startSignInProcess() zur Schablonenmethode signIn() der Oberklasse SignInHelper. Created by tina on 17.06.17.\vskip .1in 
\subsubsection{Declaration}{
\begin{lstlisting}[frame=none]
public class GoSignInHelper
 extends edu.kit.pse17.go_app.controller.login.SignInHelper\end{lstlisting}
\subsubsection{Constructor summary}{
\begin{verse}
\hyperlink{edu.kit.pse17.go_app.controller.login.GoSignInHelper()}{{\bf GoSignInHelper()}} \\
\end{verse}
}
\subsubsection{Method summary}{
\begin{verse}
\hyperlink{edu.kit.pse17.go_app.controller.login.GoSignInHelper.configureSignIn()}{{\bf configureSignIn()}} \\
\hyperlink{edu.kit.pse17.go_app.controller.login.GoSignInHelper.startSignInProcess()}{{\bf startSignInProcess()}} \\
\end{verse}
}
\subsubsection{Constructors}{
\vskip -2em
\begin{itemize}
\item{ 
\index{GoSignInHelper()}
\hypertarget{edu.kit.pse17.go_app.controller.login.GoSignInHelper()}{{\bf  GoSignInHelper}\\}
\begin{lstlisting}[frame=none]
public GoSignInHelper()\end{lstlisting} %end signature
}%end item
\end{itemize}
}
\subsubsection{Methods}{
\vskip -2em
\begin{itemize}
\item{ 
\index{configureSignIn()}
\hypertarget{edu.kit.pse17.go_app.controller.login.GoSignInHelper.configureSignIn()}{{\bf  configureSignIn}\\}
\begin{lstlisting}[frame=none]
protected abstract void configureSignIn()\end{lstlisting} %end signature
\begin{itemize}
\item{
{\bf  Description copied from \hyperlink{edu.kit.pse17.go_app.controller.login.SignInHelper}{SignInHelper}{\small \refdefined{edu.kit.pse17.go_app.controller.login.SignInHelper}} }

wird von Unterklassen implementiert und in Schablonenmethode aufgerufen
}
\end{itemize}
}%end item
\item{ 
\index{startSignInProcess()}
\hypertarget{edu.kit.pse17.go_app.controller.login.GoSignInHelper.startSignInProcess()}{{\bf  startSignInProcess}\\}
\begin{lstlisting}[frame=none]
protected abstract void startSignInProcess()\end{lstlisting} %end signature
\begin{itemize}
\item{
{\bf  Description copied from \hyperlink{edu.kit.pse17.go_app.controller.login.SignInHelper}{SignInHelper}{\small \refdefined{edu.kit.pse17.go_app.controller.login.SignInHelper}} }

wird von Unterklassen implementiert und in Schablonenmethode aufgerufen
}
\end{itemize}
}%end item
\end{itemize}
}
\subsubsection{Members inherited from class SignInHelper }{
\texttt{edu.kit.pse17.go_app.controller.login.SignInHelper} {\small 
\refdefined{edu.kit.pse17.go_app.controller.login.SignInHelper}}
{\small 

\vskip -2em
\begin{itemize}
\item{\vskip -1.5ex 
\texttt{public static final {\bf  ACCOUNT\_DATA\_CODE}}%end signature
}%end item
\item{\vskip -1.5ex 
\texttt{protected abstract void {\bf  configureSignIn}()
}%end signature
}%end item
\item{\vskip -1.5ex 
\texttt{protected void {\bf  onCreate}(\texttt{Bundle} {\bf  savedInstanceState})
}%end signature
}%end item
\item{\vskip -1.5ex 
\texttt{protected void {\bf  onStart}()
}%end signature
}%end item
\item{\vskip -1.5ex 
\texttt{protected void {\bf  returnActivityResult}(\texttt{java.io.Serializable} {\bf  accountData})
}%end signature
}%end item
\item{\vskip -1.5ex 
\texttt{public static final {\bf  SIGN\_IN\_DATA\_CODE}}%end signature
}%end item
\item{\vskip -1.5ex 
\texttt{public static void {\bf  signIn}(\texttt{Activity} {\bf  activity},
\texttt{int} {\bf  requestCode},
\texttt{java.io.Serializable} {\bf  signInData},
\texttt{java.lang.Class} {\bf  signinHelper})
}%end signature
}%end item
\item{\vskip -1.5ex 
\texttt{protected abstract void {\bf  startSignInProcess}()
}%end signature
}%end item
\end{itemize}
}
}
\subsection{\label{edu.kit.pse17.go_app.controller.login.SignInHelper}Class SignInHelper}{
\hypertarget{edu.kit.pse17.go_app.controller.login.SignInHelper}{}\vskip .1in 
Abstrakte Klasse, die als Schablone für den Anmelde-Prozess ihrer Unterklassen dient Created by tina on 18.06.17.\vskip .1in 
\subsubsection{Declaration}{
\begin{lstlisting}[frame=none]
public abstract class SignInHelper
 extends AppCompatActivity\end{lstlisting}
\subsubsection{All known subclasses}{GoSignInHelper\small{\refdefined{edu.kit.pse17.go_app.controller.login.GoSignInHelper}}, FirebaseSignInHelper\small{\refdefined{edu.kit.pse17.go_app.controller.login.FirebaseSignInHelper}}}
\subsubsection{Field summary}{
\begin{verse}
\hyperlink{edu.kit.pse17.go_app.controller.login.SignInHelper.ACCOUNT_DATA_CODE}{{\bf ACCOUNT\_DATA\_CODE}} Name des Intent-Extra, das als Ergebnis der Anmelde-Aktivität zurückgegeben wird\\
\hyperlink{edu.kit.pse17.go_app.controller.login.SignInHelper.SIGN_IN_DATA_CODE}{{\bf SIGN\_IN\_DATA\_CODE}} Name des Intent-Extra, das Anmeldedaten an die SignIn-Aktivität übergibt\\
\end{verse}
}
\subsubsection{Constructor summary}{
\begin{verse}
\hyperlink{edu.kit.pse17.go_app.controller.login.SignInHelper()}{{\bf SignInHelper()}} \\
\end{verse}
}
\subsubsection{Method summary}{
\begin{verse}
\hyperlink{edu.kit.pse17.go_app.controller.login.SignInHelper.configureSignIn()}{{\bf configureSignIn()}} wird von Unterklassen implementiert und in Schablonenmethode aufgerufen\\
\hyperlink{edu.kit.pse17.go_app.controller.login.SignInHelper.onCreate(Bundle)}{{\bf onCreate(Bundle)}} \\
\hyperlink{edu.kit.pse17.go_app.controller.login.SignInHelper.onStart()}{{\bf onStart()}} \\
\hyperlink{edu.kit.pse17.go_app.controller.login.SignInHelper.returnActivityResult(java.io.Serializable)}{{\bf returnActivityResult(Serializable)}} gibt das Ergebnis der Anmelde-Aktivität an das aufrufende Objekt zurück\\
\hyperlink{edu.kit.pse17.go_app.controller.login.SignInHelper.signIn(Activity, int, java.io.Serializable, java.lang.Class)}{{\bf signIn(Activity, int, Serializable, Class)}} Schablonenmethode für den Anmelde-Prozess der konkreten SignInHelper\\
\hyperlink{edu.kit.pse17.go_app.controller.login.SignInHelper.startSignInProcess()}{{\bf startSignInProcess()}} wird von Unterklassen implementiert und in Schablonenmethode aufgerufen\\
\end{verse}
}
\subsubsection{Fields}{
\begin{itemize}
\item{
\index{SIGN\_IN\_DATA\_CODE}
\label{edu.kit.pse17.go_app.controller.login.SignInHelper.SIGN_IN_DATA_CODE}\hypertarget{edu.kit.pse17.go_app.controller.login.SignInHelper.SIGN_IN_DATA_CODE}{\texttt{public static final java.lang.String\ {\bf  SIGN\_IN\_DATA\_CODE}}
}
\begin{itemize}
\item{\vskip -.9ex 
Name des Intent-Extra, das Anmeldedaten an die SignIn-Aktivität übergibt}
\end{itemize}
}
\item{
\index{ACCOUNT\_DATA\_CODE}
\label{edu.kit.pse17.go_app.controller.login.SignInHelper.ACCOUNT_DATA_CODE}\hypertarget{edu.kit.pse17.go_app.controller.login.SignInHelper.ACCOUNT_DATA_CODE}{\texttt{public static final java.lang.String\ {\bf  ACCOUNT\_DATA\_CODE}}
}
\begin{itemize}
\item{\vskip -.9ex 
Name des Intent-Extra, das als Ergebnis der Anmelde-Aktivität zurückgegeben wird}
\end{itemize}
}
\end{itemize}
}
\subsubsection{Constructors}{
\vskip -2em
\begin{itemize}
\item{ 
\index{SignInHelper()}
\hypertarget{edu.kit.pse17.go_app.controller.login.SignInHelper()}{{\bf  SignInHelper}\\}
\begin{lstlisting}[frame=none]
public SignInHelper()\end{lstlisting} %end signature
}%end item
\end{itemize}
}
\subsubsection{Methods}{
\vskip -2em
\begin{itemize}
\item{ 
\index{configureSignIn()}
\hypertarget{edu.kit.pse17.go_app.controller.login.SignInHelper.configureSignIn()}{{\bf  configureSignIn}\\}
\begin{lstlisting}[frame=none]
protected abstract void configureSignIn()\end{lstlisting} %end signature
\begin{itemize}
\item{
{\bf  Description}

wird von Unterklassen implementiert und in Schablonenmethode aufgerufen
}
\end{itemize}
}%end item
\item{ 
\index{onCreate(Bundle)}
\hypertarget{edu.kit.pse17.go_app.controller.login.SignInHelper.onCreate(Bundle)}{{\bf  onCreate}\\}
\begin{lstlisting}[frame=none]
protected void onCreate(Bundle savedInstanceState)\end{lstlisting} %end signature
}%end item
\item{ 
\index{onStart()}
\hypertarget{edu.kit.pse17.go_app.controller.login.SignInHelper.onStart()}{{\bf  onStart}\\}
\begin{lstlisting}[frame=none]
protected void onStart()\end{lstlisting} %end signature
}%end item
\item{ 
\index{returnActivityResult(Serializable)}
\hypertarget{edu.kit.pse17.go_app.controller.login.SignInHelper.returnActivityResult(java.io.Serializable)}{{\bf  returnActivityResult}\\}
\begin{lstlisting}[frame=none]
protected void returnActivityResult(java.io.Serializable accountData)\end{lstlisting} %end signature
\begin{itemize}
\item{
{\bf  Description}

gibt das Ergebnis der Anmelde-Aktivität an das aufrufende Objekt zurück
}
\item{
{\bf  Parameters}
  \begin{itemize}
   \item{
\texttt{accountData} -- Ergebnis der Anmelde-Aktivität}
  \end{itemize}
}%end item
\end{itemize}
}%end item
\item{ 
\index{signIn(Activity, int, Serializable, Class)}
\hypertarget{edu.kit.pse17.go_app.controller.login.SignInHelper.signIn(Activity, int, java.io.Serializable, java.lang.Class)}{{\bf  signIn}\\}
\begin{lstlisting}[frame=none]
public static void signIn(Activity activity,int requestCode,java.io.Serializable signInData,java.lang.Class signinHelper)\end{lstlisting} %end signature
\begin{itemize}
\item{
{\bf  Description}

Schablonenmethode für den Anmelde-Prozess der konkreten SignInHelper
}
\item{
{\bf  Parameters}
  \begin{itemize}
   \item{
\texttt{activity} -- Aktivity, die die anmeldung aufruft}
   \item{
\texttt{requestCode} -- Request-Code des Aktivitäts-Aufrufs}
   \item{
\texttt{signInData} -- AnmeldeDaten die ggfs an Anmelde-Aktivität übergeben werden müssen}
   \item{
\texttt{signinHelper} -- Referenz auf die Unterklasse, die Methode ausführt}
  \end{itemize}
}%end item
\end{itemize}
}%end item
\item{ 
\index{startSignInProcess()}
\hypertarget{edu.kit.pse17.go_app.controller.login.SignInHelper.startSignInProcess()}{{\bf  startSignInProcess}\\}
\begin{lstlisting}[frame=none]
protected abstract void startSignInProcess()\end{lstlisting} %end signature
\begin{itemize}
\item{
{\bf  Description}

wird von Unterklassen implementiert und in Schablonenmethode aufgerufen
}
\end{itemize}
}%end item
\end{itemize}
}
}
}
\section{Package edu.kit.pse17.go\_app.view}{
\label{edu.kit.pse17.go_app.view}\hypertarget{edu.kit.pse17.go_app.view}{}
\hskip -.05in
\hbox to \hsize{\textit{ Package Contents\hfil Page}}
\vskip .13in
\hbox{{\bf  Classes}}
\entityintro{BaseActivity}{edu.kit.pse17.go_app.view.BaseActivity}{Created by tina on 20.06.17.}
\entityintro{GoDetailActivity}{edu.kit.pse17.go_app.view.GoDetailActivity}{Created by tina on 20.06.17.}
\entityintro{GoDetailActivityOwner}{edu.kit.pse17.go_app.view.GoDetailActivityOwner}{Created by tina on 20.06.17.}
\entityintro{GroupDetailActivity}{edu.kit.pse17.go_app.view.GroupDetailActivity}{Created by tina on 19.06.17.}
\entityintro{GroupDetailActivityAdmin}{edu.kit.pse17.go_app.view.GroupDetailActivityAdmin}{Klasse dekoriert die GroupDetailActivity und fügt ihr die Admin-Funktionalitäten hinzu Created by tina on 19.06.17.}
\entityintro{GroupListActivity}{edu.kit.pse17.go_app.view.GroupListActivity}{Hauptansicht der App.}
\entityintro{InformationActivity}{edu.kit.pse17.go_app.view.InformationActivity}{Created by tina on 20.06.17.}
\entityintro{SettingsActivity}{edu.kit.pse17.go_app.view.SettingsActivity}{Created by tina on 20.06.17.}
\entityintro{SignInActivity}{edu.kit.pse17.go_app.view.SignInActivity}{Die Klasse zeigt dem User den Login-Screen an und koordiniert den Login Prozess Created by tina on 17.06.17.}
\vskip .1in
\vskip .1in
\subsection{\label{edu.kit.pse17.go_app.view.BaseActivity}Class BaseActivity}{
\hypertarget{edu.kit.pse17.go_app.view.BaseActivity}{}\vskip .1in 
Created by tina on 20.06.17.\vskip .1in 
\subsubsection{Declaration}{
\begin{lstlisting}[frame=none]
public class BaseActivity
 extends AppCompatActivity\end{lstlisting}
\subsubsection{All known subclasses}{SettingsActivity\small{\refdefined{edu.kit.pse17.go_app.view.SettingsActivity}}, GoDetailActivityOwner\small{\refdefined{edu.kit.pse17.go_app.view.GoDetailActivityOwner}}, SignInActivity\small{\refdefined{edu.kit.pse17.go_app.view.SignInActivity}}, GoDetailActivity\small{\refdefined{edu.kit.pse17.go_app.view.GoDetailActivity}}, InformationActivity\small{\refdefined{edu.kit.pse17.go_app.view.InformationActivity}}, GroupDetailActivity\small{\refdefined{edu.kit.pse17.go_app.view.GroupDetailActivity}}, GroupDetailActivityAdmin\small{\refdefined{edu.kit.pse17.go_app.view.GroupDetailActivityAdmin}}, GroupListActivity\small{\refdefined{edu.kit.pse17.go_app.view.GroupListActivity}}}
\subsubsection{Constructor summary}{
\begin{verse}
\hyperlink{edu.kit.pse17.go_app.view.BaseActivity()}{{\bf BaseActivity()}} \\
\end{verse}
}
\subsubsection{Constructors}{
\vskip -2em
\begin{itemize}
\item{ 
\index{BaseActivity()}
\hypertarget{edu.kit.pse17.go_app.view.BaseActivity()}{{\bf  BaseActivity}\\}
\begin{lstlisting}[frame=none]
public BaseActivity()\end{lstlisting} %end signature
}%end item
\end{itemize}
}
}
\subsection{\label{edu.kit.pse17.go_app.view.GoDetailActivity}Class GoDetailActivity}{
\hypertarget{edu.kit.pse17.go_app.view.GoDetailActivity}{}\vskip .1in 
Created by tina on 20.06.17.\vskip .1in 
\subsubsection{Declaration}{
\begin{lstlisting}[frame=none]
public class GoDetailActivity
 extends edu.kit.pse17.go_app.view.BaseActivity\end{lstlisting}
\subsubsection{All known subclasses}{GoDetailActivityOwner\small{\refdefined{edu.kit.pse17.go_app.view.GoDetailActivityOwner}}}
\subsubsection{Constructor summary}{
\begin{verse}
\hyperlink{edu.kit.pse17.go_app.view.GoDetailActivity()}{{\bf GoDetailActivity()}} \\
\end{verse}
}
\subsubsection{Method summary}{
\begin{verse}
\hyperlink{edu.kit.pse17.go_app.view.GoDetailActivity.onCreate(Bundle)}{{\bf onCreate(Bundle)}} \\
\end{verse}
}
\subsubsection{Constructors}{
\vskip -2em
\begin{itemize}
\item{ 
\index{GoDetailActivity()}
\hypertarget{edu.kit.pse17.go_app.view.GoDetailActivity()}{{\bf  GoDetailActivity}\\}
\begin{lstlisting}[frame=none]
public GoDetailActivity()\end{lstlisting} %end signature
}%end item
\end{itemize}
}
\subsubsection{Methods}{
\vskip -2em
\begin{itemize}
\item{ 
\index{onCreate(Bundle)}
\hypertarget{edu.kit.pse17.go_app.view.GoDetailActivity.onCreate(Bundle)}{{\bf  onCreate}\\}
\begin{lstlisting}[frame=none]
protected void onCreate(Bundle savedInstanceState)\end{lstlisting} %end signature
}%end item
\end{itemize}
}
}
\subsection{\label{edu.kit.pse17.go_app.view.GoDetailActivityOwner}Class GoDetailActivityOwner}{
\hypertarget{edu.kit.pse17.go_app.view.GoDetailActivityOwner}{}\vskip .1in 
Created by tina on 20.06.17.\vskip .1in 
\subsubsection{Declaration}{
\begin{lstlisting}[frame=none]
public class GoDetailActivityOwner
 extends edu.kit.pse17.go_app.view.GoDetailActivity\end{lstlisting}
\subsubsection{Constructor summary}{
\begin{verse}
\hyperlink{edu.kit.pse17.go_app.view.GoDetailActivityOwner()}{{\bf GoDetailActivityOwner()}} \\
\end{verse}
}
\subsubsection{Method summary}{
\begin{verse}
\hyperlink{edu.kit.pse17.go_app.view.GoDetailActivityOwner.onCreate(Bundle)}{{\bf onCreate(Bundle)}} \\
\end{verse}
}
\subsubsection{Constructors}{
\vskip -2em
\begin{itemize}
\item{ 
\index{GoDetailActivityOwner()}
\hypertarget{edu.kit.pse17.go_app.view.GoDetailActivityOwner()}{{\bf  GoDetailActivityOwner}\\}
\begin{lstlisting}[frame=none]
public GoDetailActivityOwner()\end{lstlisting} %end signature
}%end item
\end{itemize}
}
\subsubsection{Methods}{
\vskip -2em
\begin{itemize}
\item{ 
\index{onCreate(Bundle)}
\hypertarget{edu.kit.pse17.go_app.view.GoDetailActivityOwner.onCreate(Bundle)}{{\bf  onCreate}\\}
\begin{lstlisting}[frame=none]
protected void onCreate(Bundle savedInstanceState)\end{lstlisting} %end signature
}%end item
\end{itemize}
}
\subsubsection{Members inherited from class GoDetailActivity }{
\texttt{edu.kit.pse17.go_app.view.GoDetailActivity} {\small 
\refdefined{edu.kit.pse17.go_app.view.GoDetailActivity}}
{\small 

\vskip -2em
\begin{itemize}
\item{\vskip -1.5ex 
\texttt{protected void {\bf  onCreate}(\texttt{Bundle} {\bf  savedInstanceState})
}%end signature
}%end item
\end{itemize}
}
}
\subsection{\label{edu.kit.pse17.go_app.view.GroupDetailActivity}Class GroupDetailActivity}{
\hypertarget{edu.kit.pse17.go_app.view.GroupDetailActivity}{}\vskip .1in 
Created by tina on 19.06.17.\vskip .1in 
\subsubsection{Declaration}{
\begin{lstlisting}[frame=none]
public class GroupDetailActivity
 extends edu.kit.pse17.go_app.view.BaseActivity implements edu.kit.pse17.go_app.view.recyclerView.OnListItemClicked\end{lstlisting}
\subsubsection{All known subclasses}{GroupDetailActivityAdmin\small{\refdefined{edu.kit.pse17.go_app.view.GroupDetailActivityAdmin}}}
\subsubsection{Constructor summary}{
\begin{verse}
\hyperlink{edu.kit.pse17.go_app.view.GroupDetailActivity()}{{\bf GroupDetailActivity()}} \\
\end{verse}
}
\subsubsection{Method summary}{
\begin{verse}
\hyperlink{edu.kit.pse17.go_app.view.GroupDetailActivity.onCreate(Bundle)}{{\bf onCreate(Bundle)}} \\
\hyperlink{edu.kit.pse17.go_app.view.GroupDetailActivity.onItemClicked(int)}{{\bf onItemClicked(int)}} \\
\end{verse}
}
\subsubsection{Constructors}{
\vskip -2em
\begin{itemize}
\item{ 
\index{GroupDetailActivity()}
\hypertarget{edu.kit.pse17.go_app.view.GroupDetailActivity()}{{\bf  GroupDetailActivity}\\}
\begin{lstlisting}[frame=none]
public GroupDetailActivity()\end{lstlisting} %end signature
}%end item
\end{itemize}
}
\subsubsection{Methods}{
\vskip -2em
\begin{itemize}
\item{ 
\index{onCreate(Bundle)}
\hypertarget{edu.kit.pse17.go_app.view.GroupDetailActivity.onCreate(Bundle)}{{\bf  onCreate}\\}
\begin{lstlisting}[frame=none]
public void onCreate(Bundle savedInstanceState)\end{lstlisting} %end signature
}%end item
\item{ 
\index{onItemClicked(int)}
\hypertarget{edu.kit.pse17.go_app.view.GroupDetailActivity.onItemClicked(int)}{{\bf  onItemClicked}\\}
\begin{lstlisting}[frame=none]
void onItemClicked(int position)\end{lstlisting} %end signature
\begin{itemize}
\item{
{\bf  Description copied from \hyperlink{edu.kit.pse17.go_app.view.recyclerView.OnListItemClicked}{recyclerView.OnListItemClicked}{\small \refdefined{edu.kit.pse17.go_app.view.recyclerView.OnListItemClicked}} }

führt gewünschte Aktion der implemetierenden Klasse aus, falls auf das ListItem an Position position geklickt wird
}
\item{
{\bf  Parameters}
  \begin{itemize}
   \item{
\texttt{position} -- Position des ListItems, auf das geklickt wurde}
  \end{itemize}
}%end item
\end{itemize}
}%end item
\end{itemize}
}
}
\subsection{\label{edu.kit.pse17.go_app.view.GroupDetailActivityAdmin}Class GroupDetailActivityAdmin}{
\hypertarget{edu.kit.pse17.go_app.view.GroupDetailActivityAdmin}{}\vskip .1in 
Klasse dekoriert die GroupDetailActivity und fügt ihr die Admin-Funktionalitäten hinzu Created by tina on 19.06.17.\vskip .1in 
\subsubsection{Declaration}{
\begin{lstlisting}[frame=none]
public class GroupDetailActivityAdmin
 extends edu.kit.pse17.go_app.view.GroupDetailActivity\end{lstlisting}
\subsubsection{Constructor summary}{
\begin{verse}
\hyperlink{edu.kit.pse17.go_app.view.GroupDetailActivityAdmin()}{{\bf GroupDetailActivityAdmin()}} \\
\end{verse}
}
\subsubsection{Method summary}{
\begin{verse}
\hyperlink{edu.kit.pse17.go_app.view.GroupDetailActivityAdmin.onCreate(Bundle)}{{\bf onCreate(Bundle)}} \\
\end{verse}
}
\subsubsection{Constructors}{
\vskip -2em
\begin{itemize}
\item{ 
\index{GroupDetailActivityAdmin()}
\hypertarget{edu.kit.pse17.go_app.view.GroupDetailActivityAdmin()}{{\bf  GroupDetailActivityAdmin}\\}
\begin{lstlisting}[frame=none]
public GroupDetailActivityAdmin()\end{lstlisting} %end signature
}%end item
\end{itemize}
}
\subsubsection{Methods}{
\vskip -2em
\begin{itemize}
\item{ 
\index{onCreate(Bundle)}
\hypertarget{edu.kit.pse17.go_app.view.GroupDetailActivityAdmin.onCreate(Bundle)}{{\bf  onCreate}\\}
\begin{lstlisting}[frame=none]
public void onCreate(Bundle savedInstanceState)\end{lstlisting} %end signature
}%end item
\end{itemize}
}
\subsubsection{Members inherited from class GroupDetailActivity }{
\texttt{edu.kit.pse17.go_app.view.GroupDetailActivity} {\small 
\refdefined{edu.kit.pse17.go_app.view.GroupDetailActivity}}
{\small 

\vskip -2em
\begin{itemize}
\item{\vskip -1.5ex 
\texttt{public void {\bf  onCreate}(\texttt{Bundle} {\bf  savedInstanceState})
}%end signature
}%end item
\item{\vskip -1.5ex 
\texttt{public void {\bf  onItemClicked}(\texttt{int} {\bf  position})
}%end signature
}%end item
\end{itemize}
}
}
\subsection{\label{edu.kit.pse17.go_app.view.GroupListActivity}Class GroupListActivity}{
\hypertarget{edu.kit.pse17.go_app.view.GroupListActivity}{}\vskip .1in 
Hauptansicht der App. Zeigt alle Gruppen eines Benutzers\vskip .1in 
\subsubsection{Declaration}{
\begin{lstlisting}[frame=none]
public class GroupListActivity
 extends edu.kit.pse17.go_app.view.BaseActivity implements edu.kit.pse17.go_app.view.recyclerView.OnListItemClicked\end{lstlisting}
\subsubsection{Constructor summary}{
\begin{verse}
\hyperlink{edu.kit.pse17.go_app.view.GroupListActivity()}{{\bf GroupListActivity()}} \\
\end{verse}
}
\subsubsection{Method summary}{
\begin{verse}
\hyperlink{edu.kit.pse17.go_app.view.GroupListActivity.onClick(View)}{{\bf onClick(View)}} ClickListener für addGroupButton\\
\hyperlink{edu.kit.pse17.go_app.view.GroupListActivity.onCreate(Bundle)}{{\bf onCreate(Bundle)}} RecyclerView und passender Listadapter werden erzeugt\\
\hyperlink{edu.kit.pse17.go_app.view.GroupListActivity.onItemClicked(int)}{{\bf onItemClicked(int)}} ClickListener für RecyclerView-Elemente\\
\hyperlink{edu.kit.pse17.go_app.view.GroupListActivity.start(Activity, edu.kit.pse17.go_app.model.User)}{{\bf start(Activity, User)}} \\
\end{verse}
}
\subsubsection{Constructors}{
\vskip -2em
\begin{itemize}
\item{ 
\index{GroupListActivity()}
\hypertarget{edu.kit.pse17.go_app.view.GroupListActivity()}{{\bf  GroupListActivity}\\}
\begin{lstlisting}[frame=none]
public GroupListActivity()\end{lstlisting} %end signature
}%end item
\end{itemize}
}
\subsubsection{Methods}{
\vskip -2em
\begin{itemize}
\item{ 
\index{onClick(View)}
\hypertarget{edu.kit.pse17.go_app.view.GroupListActivity.onClick(View)}{{\bf  onClick}\\}
\begin{lstlisting}[frame=none]
public void onClick(View v)\end{lstlisting} %end signature
\begin{itemize}
\item{
{\bf  Description}

ClickListener für addGroupButton
}
\item{
{\bf  Parameters}
  \begin{itemize}
   \item{
\texttt{v} -- }
  \end{itemize}
}%end item
\end{itemize}
}%end item
\item{ 
\index{onCreate(Bundle)}
\hypertarget{edu.kit.pse17.go_app.view.GroupListActivity.onCreate(Bundle)}{{\bf  onCreate}\\}
\begin{lstlisting}[frame=none]
protected void onCreate(Bundle savedInstanceState)\end{lstlisting} %end signature
\begin{itemize}
\item{
{\bf  Description}

RecyclerView und passender Listadapter werden erzeugt
}
\item{
{\bf  Parameters}
  \begin{itemize}
   \item{
\texttt{savedInstanceState} -- }
  \end{itemize}
}%end item
\end{itemize}
}%end item
\item{ 
\index{onItemClicked(int)}
\hypertarget{edu.kit.pse17.go_app.view.GroupListActivity.onItemClicked(int)}{{\bf  onItemClicked}\\}
\begin{lstlisting}[frame=none]
public void onItemClicked(int position)\end{lstlisting} %end signature
\begin{itemize}
\item{
{\bf  Description}

ClickListener für RecyclerView-Elemente
}
\item{
{\bf  Parameters}
  \begin{itemize}
   \item{
\texttt{position} -- Position des ListItems, auf das geklickt wurde}
  \end{itemize}
}%end item
\end{itemize}
}%end item
\item{ 
\index{start(Activity, User)}
\hypertarget{edu.kit.pse17.go_app.view.GroupListActivity.start(Activity, edu.kit.pse17.go_app.model.User)}{{\bf  start}\\}
\begin{lstlisting}[frame=none]
public static void start(Activity activity,edu.kit.pse17.go_app.model.User user)\end{lstlisting} %end signature
}%end item
\end{itemize}
}
}
\subsection{\label{edu.kit.pse17.go_app.view.InformationActivity}Class InformationActivity}{
\hypertarget{edu.kit.pse17.go_app.view.InformationActivity}{}\vskip .1in 
Created by tina on 20.06.17.\vskip .1in 
\subsubsection{Declaration}{
\begin{lstlisting}[frame=none]
public class InformationActivity
 extends edu.kit.pse17.go_app.view.BaseActivity\end{lstlisting}
\subsubsection{Constructor summary}{
\begin{verse}
\hyperlink{edu.kit.pse17.go_app.view.InformationActivity()}{{\bf InformationActivity()}} \\
\end{verse}
}
\subsubsection{Constructors}{
\vskip -2em
\begin{itemize}
\item{ 
\index{InformationActivity()}
\hypertarget{edu.kit.pse17.go_app.view.InformationActivity()}{{\bf  InformationActivity}\\}
\begin{lstlisting}[frame=none]
public InformationActivity()\end{lstlisting} %end signature
}%end item
\end{itemize}
}
}
\subsection{\label{edu.kit.pse17.go_app.view.SettingsActivity}Class SettingsActivity}{
\hypertarget{edu.kit.pse17.go_app.view.SettingsActivity}{}\vskip .1in 
Created by tina on 20.06.17.\vskip .1in 
\subsubsection{Declaration}{
\begin{lstlisting}[frame=none]
public class SettingsActivity
 extends edu.kit.pse17.go_app.view.BaseActivity\end{lstlisting}
\subsubsection{Constructor summary}{
\begin{verse}
\hyperlink{edu.kit.pse17.go_app.view.SettingsActivity()}{{\bf SettingsActivity()}} \\
\end{verse}
}
\subsubsection{Method summary}{
\begin{verse}
\hyperlink{edu.kit.pse17.go_app.view.SettingsActivity.onCreate(Bundle)}{{\bf onCreate(Bundle)}} \\
\end{verse}
}
\subsubsection{Constructors}{
\vskip -2em
\begin{itemize}
\item{ 
\index{SettingsActivity()}
\hypertarget{edu.kit.pse17.go_app.view.SettingsActivity()}{{\bf  SettingsActivity}\\}
\begin{lstlisting}[frame=none]
public SettingsActivity()\end{lstlisting} %end signature
}%end item
\end{itemize}
}
\subsubsection{Methods}{
\vskip -2em
\begin{itemize}
\item{ 
\index{onCreate(Bundle)}
\hypertarget{edu.kit.pse17.go_app.view.SettingsActivity.onCreate(Bundle)}{{\bf  onCreate}\\}
\begin{lstlisting}[frame=none]
protected void onCreate(Bundle savedInstanceState)\end{lstlisting} %end signature
}%end item
\end{itemize}
}
}
\subsection{\label{edu.kit.pse17.go_app.view.SignInActivity}Class SignInActivity}{
\hypertarget{edu.kit.pse17.go_app.view.SignInActivity}{}\vskip .1in 
Die Klasse zeigt dem User den Login-Screen an und koordiniert den Login Prozess Created by tina on 17.06.17.\vskip .1in 
\subsubsection{Declaration}{
\begin{lstlisting}[frame=none]
public class SignInActivity
 extends edu.kit.pse17.go_app.view.BaseActivity\end{lstlisting}
\subsubsection{Constructor summary}{
\begin{verse}
\hyperlink{edu.kit.pse17.go_app.view.SignInActivity()}{{\bf SignInActivity()}} \\
\end{verse}
}
\subsubsection{Method summary}{
\begin{verse}
\hyperlink{edu.kit.pse17.go_app.view.SignInActivity.onActivityResult(int, int, Intent)}{{\bf onActivityResult(int, int, Intent)}} \\
\hyperlink{edu.kit.pse17.go_app.view.SignInActivity.onClick(View)}{{\bf onClick(View)}} Click-Listener, der auf Klicken des Signin Buttons wartet --\textgreater  SignIn wird gestartet\\
\hyperlink{edu.kit.pse17.go_app.view.SignInActivity.onCreate(Bundle)}{{\bf onCreate(Bundle)}} \\
\hyperlink{edu.kit.pse17.go_app.view.SignInActivity.onResume()}{{\bf onResume()}} \\
\end{verse}
}
\subsubsection{Constructors}{
\vskip -2em
\begin{itemize}
\item{ 
\index{SignInActivity()}
\hypertarget{edu.kit.pse17.go_app.view.SignInActivity()}{{\bf  SignInActivity}\\}
\begin{lstlisting}[frame=none]
public SignInActivity()\end{lstlisting} %end signature
}%end item
\end{itemize}
}
\subsubsection{Methods}{
\vskip -2em
\begin{itemize}
\item{ 
\index{onActivityResult(int, int, Intent)}
\hypertarget{edu.kit.pse17.go_app.view.SignInActivity.onActivityResult(int, int, Intent)}{{\bf  onActivityResult}\\}
\begin{lstlisting}[frame=none]
protected void onActivityResult(int requestCode,int resultCode,Intent data)\end{lstlisting} %end signature
}%end item
\item{ 
\index{onClick(View)}
\hypertarget{edu.kit.pse17.go_app.view.SignInActivity.onClick(View)}{{\bf  onClick}\\}
\begin{lstlisting}[frame=none]
public void onClick(View v)\end{lstlisting} %end signature
\begin{itemize}
\item{
{\bf  Description}

Click-Listener, der auf Klicken des Signin Buttons wartet --\textgreater  SignIn wird gestartet
}
\item{
{\bf  Parameters}
  \begin{itemize}
   \item{
\texttt{v} -- geklickter View}
  \end{itemize}
}%end item
\end{itemize}
}%end item
\item{ 
\index{onCreate(Bundle)}
\hypertarget{edu.kit.pse17.go_app.view.SignInActivity.onCreate(Bundle)}{{\bf  onCreate}\\}
\begin{lstlisting}[frame=none]
protected void onCreate(Bundle savedInstanceState)\end{lstlisting} %end signature
}%end item
\item{ 
\index{onResume()}
\hypertarget{edu.kit.pse17.go_app.view.SignInActivity.onResume()}{{\bf  onResume}\\}
\begin{lstlisting}[frame=none]
protected void onResume()\end{lstlisting} %end signature
}%end item
\end{itemize}
}
}
}
\section{Package edu.kit.pse17.go\_app.view.recyclerView.listItems}{
\label{edu.kit.pse17.go_app.view.recyclerView.listItems}\hypertarget{edu.kit.pse17.go_app.view.recyclerView.listItems}{}
\hskip -.05in
\hbox to \hsize{\textit{ Package Contents\hfil Page}}
\vskip .13in
\hbox{{\bf  Interfaces}}
\entityintro{ListItem}{edu.kit.pse17.go_app.view.recyclerView.listItems.ListItem}{Interface für ListItems, die die Datenobjekt in den verschiedenen RecyclerViews der App sind Created by tina on 18.06.17.}
\vskip .13in
\hbox{{\bf  Classes}}
\entityintro{GOListItem}{edu.kit.pse17.go_app.view.recyclerView.listItems.GOListItem}{Diese Klasse repräsentiert ListItems, die Informationen über ein GO in einem RecyclerView darstellen sollen Created by tina on 17.06.17.}
\entityintro{GroupListItem}{edu.kit.pse17.go_app.view.recyclerView.listItems.GroupListItem}{Diese Klasse repräsentiert ListItems, die Informationen über eine Gruppe in einem RecyclerView darstellen sollen Created by tina on 17.06.17.}
\entityintro{UserMailListItem}{edu.kit.pse17.go_app.view.recyclerView.listItems.UserMailListItem}{Diese Klasse repräsentiert ListItems, die Informationen über einen User in einem RecyclerView darstellen sollen Created by tina on 19.06.17.}
\entityintro{UserStatusListItem}{edu.kit.pse17.go_app.view.recyclerView.listItems.UserStatusListItem}{Diese Klasse repräsentiert ListItems, die Informationen über einen User in einem RecyclerView darstellen sollen Created by tina on 19.06.17.}
\vskip .1in
\vskip .1in
\subsection{\label{edu.kit.pse17.go_app.view.recyclerView.listItems.ListItem}Interface ListItem}{
\hypertarget{edu.kit.pse17.go_app.view.recyclerView.listItems.ListItem}{}\vskip .1in 
Interface für ListItems, die die Datenobjekt in den verschiedenen RecyclerViews der App sind Created by tina on 18.06.17.\vskip .1in 
\subsubsection{Declaration}{
\begin{lstlisting}[frame=none]
public interface ListItem
\end{lstlisting}
\subsubsection{All known subinterfaces}{GOListItem\small{\refdefined{edu.kit.pse17.go_app.view.recyclerView.listItems.GOListItem}}, UserMailListItem\small{\refdefined{edu.kit.pse17.go_app.view.recyclerView.listItems.UserMailListItem}}, UserStatusListItem\small{\refdefined{edu.kit.pse17.go_app.view.recyclerView.listItems.UserStatusListItem}}, GroupListItem\small{\refdefined{edu.kit.pse17.go_app.view.recyclerView.listItems.GroupListItem}}}
\subsubsection{All classes known to implement interface}{GOListItem\small{\refdefined{edu.kit.pse17.go_app.view.recyclerView.listItems.GOListItem}}, UserMailListItem\small{\refdefined{edu.kit.pse17.go_app.view.recyclerView.listItems.UserMailListItem}}, UserStatusListItem\small{\refdefined{edu.kit.pse17.go_app.view.recyclerView.listItems.UserStatusListItem}}, GroupListItem\small{\refdefined{edu.kit.pse17.go_app.view.recyclerView.listItems.GroupListItem}}}
\subsubsection{Method summary}{
\begin{verse}
\hyperlink{edu.kit.pse17.go_app.view.recyclerView.listItems.ListItem.getIcon()}{{\bf getIcon()}} getter-Methode für Icon des ListItems\\
\hyperlink{edu.kit.pse17.go_app.view.recyclerView.listItems.ListItem.getSubtitle()}{{\bf getSubtitle()}} getter-Methode für Untertitel des ListItems.\\
\hyperlink{edu.kit.pse17.go_app.view.recyclerView.listItems.ListItem.getTitle()}{{\bf getTitle()}} getter-Methode für Überschrift des ListItems\\
\hyperlink{edu.kit.pse17.go_app.view.recyclerView.listItems.ListItem.setIcon(Icon)}{{\bf setIcon(Icon)}} setter-Methode für icon des ListItems\\
\hyperlink{edu.kit.pse17.go_app.view.recyclerView.listItems.ListItem.setSubtitle(T)}{{\bf setSubtitle(T)}} setter-Methode für Untertitel.\\
\hyperlink{edu.kit.pse17.go_app.view.recyclerView.listItems.ListItem.setTitle(java.lang.String)}{{\bf setTitle(String)}} setter-Methode für Überschrift des ListItems\\
\end{verse}
}
\subsubsection{Methods}{
\vskip -2em
\begin{itemize}
\item{ 
\index{getIcon()}
\hypertarget{edu.kit.pse17.go_app.view.recyclerView.listItems.ListItem.getIcon()}{{\bf  getIcon}\\}
\begin{lstlisting}[frame=none]
Icon getIcon()\end{lstlisting} %end signature
\begin{itemize}
\item{
{\bf  Description}

getter-Methode für Icon des ListItems
}
\item{{\bf  Returns} -- 
Icon des Datenobjekts 
}%end item
\end{itemize}
}%end item
\item{ 
\index{getSubtitle()}
\hypertarget{edu.kit.pse17.go_app.view.recyclerView.listItems.ListItem.getSubtitle()}{{\bf  getSubtitle}\\}
\begin{lstlisting}[frame=none]
java.lang.String getSubtitle()\end{lstlisting} %end signature
\begin{itemize}
\item{
{\bf  Description}

getter-Methode für Untertitel des ListItems. Muss ggfs. erst generiert werden, die Information wird als Datentyp T im Objekt gespeichert
}
\item{{\bf  Returns} -- 
Untertitel des Datenobjekts 
}%end item
\end{itemize}
}%end item
\item{ 
\index{getTitle()}
\hypertarget{edu.kit.pse17.go_app.view.recyclerView.listItems.ListItem.getTitle()}{{\bf  getTitle}\\}
\begin{lstlisting}[frame=none]
java.lang.String getTitle()\end{lstlisting} %end signature
\begin{itemize}
\item{
{\bf  Description}

getter-Methode für Überschrift des ListItems
}
\item{{\bf  Returns} -- 
Titel des Datenobjekts 
}%end item
\end{itemize}
}%end item
\item{ 
\index{setIcon(Icon)}
\hypertarget{edu.kit.pse17.go_app.view.recyclerView.listItems.ListItem.setIcon(Icon)}{{\bf  setIcon}\\}
\begin{lstlisting}[frame=none]
void setIcon(Icon icon)\end{lstlisting} %end signature
\begin{itemize}
\item{
{\bf  Description}

setter-Methode für icon des ListItems
}
\item{
{\bf  Parameters}
  \begin{itemize}
   \item{
\texttt{icon} -- das neue Icon}
  \end{itemize}
}%end item
\end{itemize}
}%end item
\item{ 
\index{setSubtitle(T)}
\hypertarget{edu.kit.pse17.go_app.view.recyclerView.listItems.ListItem.setSubtitle(T)}{{\bf  setSubtitle}\\}
\begin{lstlisting}[frame=none]
void setSubtitle(java.lang.Object t)\end{lstlisting} %end signature
\begin{itemize}
\item{
{\bf  Description}

setter-Methode für Untertitel. Methode erwartet Datentyp T, der Untertitel wird dann innerhalb der Klasse als String-Objekt erzeugt
}
\item{
{\bf  Parameters}
  \begin{itemize}
   \item{
\texttt{t} -- Objekt/Datentyp, aus dem Untertitel erzeugt wird}
  \end{itemize}
}%end item
\end{itemize}
}%end item
\item{ 
\index{setTitle(String)}
\hypertarget{edu.kit.pse17.go_app.view.recyclerView.listItems.ListItem.setTitle(java.lang.String)}{{\bf  setTitle}\\}
\begin{lstlisting}[frame=none]
void setTitle(java.lang.String title)\end{lstlisting} %end signature
\begin{itemize}
\item{
{\bf  Description}

setter-Methode für Überschrift des ListItems
}
\item{
{\bf  Parameters}
  \begin{itemize}
   \item{
\texttt{title} -- der neue Titel}
  \end{itemize}
}%end item
\end{itemize}
}%end item
\end{itemize}
}
}
\subsection{\label{edu.kit.pse17.go_app.view.recyclerView.listItems.GOListItem}Class GOListItem}{
\hypertarget{edu.kit.pse17.go_app.view.recyclerView.listItems.GOListItem}{}\vskip .1in 
Diese Klasse repräsentiert ListItems, die Informationen über ein GO in einem RecyclerView darstellen sollen Created by tina on 17.06.17.\vskip .1in 
\subsubsection{Declaration}{
\begin{lstlisting}[frame=none]
public class GOListItem
 extends java.lang.Object implements ListItem\end{lstlisting}
\subsubsection{Constructor summary}{
\begin{verse}
\hyperlink{edu.kit.pse17.go_app.view.recyclerView.listItems.GOListItem(edu.kit.pse17.go_app.model.GO)}{{\bf GOListItem(GO)}} Konstruktor\\
\hyperlink{edu.kit.pse17.go_app.view.recyclerView.listItems.GOListItem(java.lang.String, java.util.Date, Icon)}{{\bf GOListItem(String, Date, Icon)}} Konstruktor\\
\end{verse}
}
\subsubsection{Method summary}{
\begin{verse}
\hyperlink{edu.kit.pse17.go_app.view.recyclerView.listItems.GOListItem.getIcon()}{{\bf getIcon()}} \\
\hyperlink{edu.kit.pse17.go_app.view.recyclerView.listItems.GOListItem.getSubtitle()}{{\bf getSubtitle()}} \\
\hyperlink{edu.kit.pse17.go_app.view.recyclerView.listItems.GOListItem.getTitle()}{{\bf getTitle()}} \\
\hyperlink{edu.kit.pse17.go_app.view.recyclerView.listItems.GOListItem.setIcon(Icon)}{{\bf setIcon(Icon)}} \\
\hyperlink{edu.kit.pse17.go_app.view.recyclerView.listItems.GOListItem.setSubtitle(java.util.Date)}{{\bf setSubtitle(Date)}} \\
\hyperlink{edu.kit.pse17.go_app.view.recyclerView.listItems.GOListItem.setTitle(java.lang.String)}{{\bf setTitle(String)}} \\
\end{verse}
}
\subsubsection{Constructors}{
\vskip -2em
\begin{itemize}
\item{ 
\index{GOListItem(GO)}
\hypertarget{edu.kit.pse17.go_app.view.recyclerView.listItems.GOListItem(edu.kit.pse17.go_app.model.GO)}{{\bf  GOListItem}\\}
\begin{lstlisting}[frame=none]
public GOListItem(edu.kit.pse17.go_app.model.GO go)\end{lstlisting} %end signature
\begin{itemize}
\item{
{\bf  Description}

Konstruktor
}
\item{
{\bf  Parameters}
  \begin{itemize}
   \item{
\texttt{go} -- Go-Objekt, das von dem ListItem repräsentiert werden soll}
  \end{itemize}
}%end item
\end{itemize}
}%end item
\item{ 
\index{GOListItem(String, Date, Icon)}
\hypertarget{edu.kit.pse17.go_app.view.recyclerView.listItems.GOListItem(java.lang.String, java.util.Date, Icon)}{{\bf  GOListItem}\\}
\begin{lstlisting}[frame=none]
public GOListItem(java.lang.String name,java.util.Date start,Icon icon)\end{lstlisting} %end signature
\begin{itemize}
\item{
{\bf  Description}

Konstruktor
}
\item{
{\bf  Parameters}
  \begin{itemize}
   \item{
\texttt{name} -- GO-Bezeichnung}
   \item{
\texttt{start} -- Startzeitpunkt des GOs}
   \item{
\texttt{icon} -- GO-Icon}
  \end{itemize}
}%end item
\end{itemize}
}%end item
\end{itemize}
}
\subsubsection{Methods}{
\vskip -2em
\begin{itemize}
\item{ 
\index{getIcon()}
\hypertarget{edu.kit.pse17.go_app.view.recyclerView.listItems.GOListItem.getIcon()}{{\bf  getIcon}\\}
\begin{lstlisting}[frame=none]
Icon getIcon()\end{lstlisting} %end signature
\begin{itemize}
\item{
{\bf  Description copied from \hyperlink{edu.kit.pse17.go_app.view.recyclerView.listItems.ListItem}{ListItem}{\small \refdefined{edu.kit.pse17.go_app.view.recyclerView.listItems.ListItem}} }

getter-Methode für Icon des ListItems
}
\item{{\bf  Returns} -- 
Icon des Datenobjekts 
}%end item
\end{itemize}
}%end item
\item{ 
\index{getSubtitle()}
\hypertarget{edu.kit.pse17.go_app.view.recyclerView.listItems.GOListItem.getSubtitle()}{{\bf  getSubtitle}\\}
\begin{lstlisting}[frame=none]
java.lang.String getSubtitle()\end{lstlisting} %end signature
\begin{itemize}
\item{
{\bf  Description copied from \hyperlink{edu.kit.pse17.go_app.view.recyclerView.listItems.ListItem}{ListItem}{\small \refdefined{edu.kit.pse17.go_app.view.recyclerView.listItems.ListItem}} }

getter-Methode für Untertitel des ListItems. Muss ggfs. erst generiert werden, die Information wird als Datentyp T im Objekt gespeichert
}
\item{{\bf  Returns} -- 
Untertitel des Datenobjekts 
}%end item
\end{itemize}
}%end item
\item{ 
\index{getTitle()}
\hypertarget{edu.kit.pse17.go_app.view.recyclerView.listItems.GOListItem.getTitle()}{{\bf  getTitle}\\}
\begin{lstlisting}[frame=none]
java.lang.String getTitle()\end{lstlisting} %end signature
\begin{itemize}
\item{
{\bf  Description copied from \hyperlink{edu.kit.pse17.go_app.view.recyclerView.listItems.ListItem}{ListItem}{\small \refdefined{edu.kit.pse17.go_app.view.recyclerView.listItems.ListItem}} }

getter-Methode für Überschrift des ListItems
}
\item{{\bf  Returns} -- 
Titel des Datenobjekts 
}%end item
\end{itemize}
}%end item
\item{ 
\index{setIcon(Icon)}
\hypertarget{edu.kit.pse17.go_app.view.recyclerView.listItems.GOListItem.setIcon(Icon)}{{\bf  setIcon}\\}
\begin{lstlisting}[frame=none]
void setIcon(Icon icon)\end{lstlisting} %end signature
\begin{itemize}
\item{
{\bf  Description copied from \hyperlink{edu.kit.pse17.go_app.view.recyclerView.listItems.ListItem}{ListItem}{\small \refdefined{edu.kit.pse17.go_app.view.recyclerView.listItems.ListItem}} }

setter-Methode für icon des ListItems
}
\item{
{\bf  Parameters}
  \begin{itemize}
   \item{
\texttt{icon} -- das neue Icon}
  \end{itemize}
}%end item
\end{itemize}
}%end item
\item{ 
\index{setSubtitle(Date)}
\hypertarget{edu.kit.pse17.go_app.view.recyclerView.listItems.GOListItem.setSubtitle(java.util.Date)}{{\bf  setSubtitle}\\}
\begin{lstlisting}[frame=none]
public void setSubtitle(java.util.Date date)\end{lstlisting} %end signature
}%end item
\item{ 
\index{setTitle(String)}
\hypertarget{edu.kit.pse17.go_app.view.recyclerView.listItems.GOListItem.setTitle(java.lang.String)}{{\bf  setTitle}\\}
\begin{lstlisting}[frame=none]
void setTitle(java.lang.String title)\end{lstlisting} %end signature
\begin{itemize}
\item{
{\bf  Description copied from \hyperlink{edu.kit.pse17.go_app.view.recyclerView.listItems.ListItem}{ListItem}{\small \refdefined{edu.kit.pse17.go_app.view.recyclerView.listItems.ListItem}} }

setter-Methode für Überschrift des ListItems
}
\item{
{\bf  Parameters}
  \begin{itemize}
   \item{
\texttt{title} -- der neue Titel}
  \end{itemize}
}%end item
\end{itemize}
}%end item
\end{itemize}
}
}
\subsection{\label{edu.kit.pse17.go_app.view.recyclerView.listItems.GroupListItem}Class GroupListItem}{
\hypertarget{edu.kit.pse17.go_app.view.recyclerView.listItems.GroupListItem}{}\vskip .1in 
Diese Klasse repräsentiert ListItems, die Informationen über eine Gruppe in einem RecyclerView darstellen sollen Created by tina on 17.06.17.\vskip .1in 
\subsubsection{Declaration}{
\begin{lstlisting}[frame=none]
public class GroupListItem
 extends java.lang.Object implements ListItem\end{lstlisting}
\subsubsection{Constructor summary}{
\begin{verse}
\hyperlink{edu.kit.pse17.go_app.view.recyclerView.listItems.GroupListItem(edu.kit.pse17.go_app.model.Group)}{{\bf GroupListItem(Group)}} Konstruktor\\
\hyperlink{edu.kit.pse17.go_app.view.recyclerView.listItems.GroupListItem(java.lang.String, int, Icon)}{{\bf GroupListItem(String, int, Icon)}} Konstruktor\\
\end{verse}
}
\subsubsection{Method summary}{
\begin{verse}
\hyperlink{edu.kit.pse17.go_app.view.recyclerView.listItems.GroupListItem.getIcon()}{{\bf getIcon()}} \\
\hyperlink{edu.kit.pse17.go_app.view.recyclerView.listItems.GroupListItem.getSubtitle()}{{\bf getSubtitle()}} \\
\hyperlink{edu.kit.pse17.go_app.view.recyclerView.listItems.GroupListItem.getTitle()}{{\bf getTitle()}} \\
\hyperlink{edu.kit.pse17.go_app.view.recyclerView.listItems.GroupListItem.setIcon(Icon)}{{\bf setIcon(Icon)}} \\
\hyperlink{edu.kit.pse17.go_app.view.recyclerView.listItems.GroupListItem.setSubtitle(java.lang.Integer)}{{\bf setSubtitle(Integer)}} \\
\hyperlink{edu.kit.pse17.go_app.view.recyclerView.listItems.GroupListItem.setTitle(java.lang.String)}{{\bf setTitle(String)}} \\
\end{verse}
}
\subsubsection{Constructors}{
\vskip -2em
\begin{itemize}
\item{ 
\index{GroupListItem(Group)}
\hypertarget{edu.kit.pse17.go_app.view.recyclerView.listItems.GroupListItem(edu.kit.pse17.go_app.model.Group)}{{\bf  GroupListItem}\\}
\begin{lstlisting}[frame=none]
public GroupListItem(edu.kit.pse17.go_app.model.Group group)\end{lstlisting} %end signature
\begin{itemize}
\item{
{\bf  Description}

Konstruktor
}
\item{
{\bf  Parameters}
  \begin{itemize}
   \item{
\texttt{group} -- gruppen-Objekt, das von dem ListItem repräsentiert werden soll}
  \end{itemize}
}%end item
\end{itemize}
}%end item
\item{ 
\index{GroupListItem(String, int, Icon)}
\hypertarget{edu.kit.pse17.go_app.view.recyclerView.listItems.GroupListItem(java.lang.String, int, Icon)}{{\bf  GroupListItem}\\}
\begin{lstlisting}[frame=none]
public GroupListItem(java.lang.String title,int memberCount,Icon icon)\end{lstlisting} %end signature
\begin{itemize}
\item{
{\bf  Description}

Konstruktor
}
\item{
{\bf  Parameters}
  \begin{itemize}
   \item{
\texttt{title} -- Gruppenname}
   \item{
\texttt{memberCount} -- Anzahl der Gruppenmitglieder}
   \item{
\texttt{icon} -- Gruppenbild}
  \end{itemize}
}%end item
\end{itemize}
}%end item
\end{itemize}
}
\subsubsection{Methods}{
\vskip -2em
\begin{itemize}
\item{ 
\index{getIcon()}
\hypertarget{edu.kit.pse17.go_app.view.recyclerView.listItems.GroupListItem.getIcon()}{{\bf  getIcon}\\}
\begin{lstlisting}[frame=none]
Icon getIcon()\end{lstlisting} %end signature
\begin{itemize}
\item{
{\bf  Description copied from \hyperlink{edu.kit.pse17.go_app.view.recyclerView.listItems.ListItem}{ListItem}{\small \refdefined{edu.kit.pse17.go_app.view.recyclerView.listItems.ListItem}} }

getter-Methode für Icon des ListItems
}
\item{{\bf  Returns} -- 
Icon des Datenobjekts 
}%end item
\end{itemize}
}%end item
\item{ 
\index{getSubtitle()}
\hypertarget{edu.kit.pse17.go_app.view.recyclerView.listItems.GroupListItem.getSubtitle()}{{\bf  getSubtitle}\\}
\begin{lstlisting}[frame=none]
java.lang.String getSubtitle()\end{lstlisting} %end signature
\begin{itemize}
\item{
{\bf  Description copied from \hyperlink{edu.kit.pse17.go_app.view.recyclerView.listItems.ListItem}{ListItem}{\small \refdefined{edu.kit.pse17.go_app.view.recyclerView.listItems.ListItem}} }

getter-Methode für Untertitel des ListItems. Muss ggfs. erst generiert werden, die Information wird als Datentyp T im Objekt gespeichert
}
\item{{\bf  Returns} -- 
Untertitel des Datenobjekts 
}%end item
\end{itemize}
}%end item
\item{ 
\index{getTitle()}
\hypertarget{edu.kit.pse17.go_app.view.recyclerView.listItems.GroupListItem.getTitle()}{{\bf  getTitle}\\}
\begin{lstlisting}[frame=none]
java.lang.String getTitle()\end{lstlisting} %end signature
\begin{itemize}
\item{
{\bf  Description copied from \hyperlink{edu.kit.pse17.go_app.view.recyclerView.listItems.ListItem}{ListItem}{\small \refdefined{edu.kit.pse17.go_app.view.recyclerView.listItems.ListItem}} }

getter-Methode für Überschrift des ListItems
}
\item{{\bf  Returns} -- 
Titel des Datenobjekts 
}%end item
\end{itemize}
}%end item
\item{ 
\index{setIcon(Icon)}
\hypertarget{edu.kit.pse17.go_app.view.recyclerView.listItems.GroupListItem.setIcon(Icon)}{{\bf  setIcon}\\}
\begin{lstlisting}[frame=none]
void setIcon(Icon icon)\end{lstlisting} %end signature
\begin{itemize}
\item{
{\bf  Description copied from \hyperlink{edu.kit.pse17.go_app.view.recyclerView.listItems.ListItem}{ListItem}{\small \refdefined{edu.kit.pse17.go_app.view.recyclerView.listItems.ListItem}} }

setter-Methode für icon des ListItems
}
\item{
{\bf  Parameters}
  \begin{itemize}
   \item{
\texttt{icon} -- das neue Icon}
  \end{itemize}
}%end item
\end{itemize}
}%end item
\item{ 
\index{setSubtitle(Integer)}
\hypertarget{edu.kit.pse17.go_app.view.recyclerView.listItems.GroupListItem.setSubtitle(java.lang.Integer)}{{\bf  setSubtitle}\\}
\begin{lstlisting}[frame=none]
public void setSubtitle(java.lang.Integer memberCount)\end{lstlisting} %end signature
}%end item
\item{ 
\index{setTitle(String)}
\hypertarget{edu.kit.pse17.go_app.view.recyclerView.listItems.GroupListItem.setTitle(java.lang.String)}{{\bf  setTitle}\\}
\begin{lstlisting}[frame=none]
void setTitle(java.lang.String title)\end{lstlisting} %end signature
\begin{itemize}
\item{
{\bf  Description copied from \hyperlink{edu.kit.pse17.go_app.view.recyclerView.listItems.ListItem}{ListItem}{\small \refdefined{edu.kit.pse17.go_app.view.recyclerView.listItems.ListItem}} }

setter-Methode für Überschrift des ListItems
}
\item{
{\bf  Parameters}
  \begin{itemize}
   \item{
\texttt{title} -- der neue Titel}
  \end{itemize}
}%end item
\end{itemize}
}%end item
\end{itemize}
}
}
\subsection{\label{edu.kit.pse17.go_app.view.recyclerView.listItems.UserMailListItem}Class UserMailListItem}{
\hypertarget{edu.kit.pse17.go_app.view.recyclerView.listItems.UserMailListItem}{}\vskip .1in 
Diese Klasse repräsentiert ListItems, die Informationen über einen User in einem RecyclerView darstellen sollen Created by tina on 19.06.17.\vskip .1in 
\subsubsection{Declaration}{
\begin{lstlisting}[frame=none]
public class UserMailListItem
 extends java.lang.Object implements ListItem\end{lstlisting}
\subsubsection{Constructor summary}{
\begin{verse}
\hyperlink{edu.kit.pse17.go_app.view.recyclerView.listItems.UserMailListItem(java.lang.String, java.lang.String, Icon)}{{\bf UserMailListItem(String, String, Icon)}} Konstruktor\\
\hyperlink{edu.kit.pse17.go_app.view.recyclerView.listItems.UserMailListItem(edu.kit.pse17.go_app.model.User)}{{\bf UserMailListItem(User)}} Konstruktor\\
\end{verse}
}
\subsubsection{Method summary}{
\begin{verse}
\hyperlink{edu.kit.pse17.go_app.view.recyclerView.listItems.UserMailListItem.getIcon()}{{\bf getIcon()}} \\
\hyperlink{edu.kit.pse17.go_app.view.recyclerView.listItems.UserMailListItem.getSubtitle()}{{\bf getSubtitle()}} \\
\hyperlink{edu.kit.pse17.go_app.view.recyclerView.listItems.UserMailListItem.getTitle()}{{\bf getTitle()}} \\
\hyperlink{edu.kit.pse17.go_app.view.recyclerView.listItems.UserMailListItem.setIcon(Icon)}{{\bf setIcon(Icon)}} \\
\hyperlink{edu.kit.pse17.go_app.view.recyclerView.listItems.UserMailListItem.setSubtitle(java.lang.String)}{{\bf setSubtitle(String)}} \\
\hyperlink{edu.kit.pse17.go_app.view.recyclerView.listItems.UserMailListItem.setTitle(java.lang.String)}{{\bf setTitle(String)}} \\
\end{verse}
}
\subsubsection{Constructors}{
\vskip -2em
\begin{itemize}
\item{ 
\index{UserMailListItem(String, String, Icon)}
\hypertarget{edu.kit.pse17.go_app.view.recyclerView.listItems.UserMailListItem(java.lang.String, java.lang.String, Icon)}{{\bf  UserMailListItem}\\}
\begin{lstlisting}[frame=none]
public UserMailListItem(java.lang.String title,java.lang.String email,Icon icon)\end{lstlisting} %end signature
\begin{itemize}
\item{
{\bf  Description}

Konstruktor
}
\item{
{\bf  Parameters}
  \begin{itemize}
   \item{
\texttt{title} -- Benutzername}
   \item{
\texttt{email} -- EMail-Adresse, die zur Anmeldung verwendet wurde}
   \item{
\texttt{icon} -- Profilbild}
  \end{itemize}
}%end item
\end{itemize}
}%end item
\item{ 
\index{UserMailListItem(User)}
\hypertarget{edu.kit.pse17.go_app.view.recyclerView.listItems.UserMailListItem(edu.kit.pse17.go_app.model.User)}{{\bf  UserMailListItem}\\}
\begin{lstlisting}[frame=none]
public UserMailListItem(edu.kit.pse17.go_app.model.User user)\end{lstlisting} %end signature
\begin{itemize}
\item{
{\bf  Description}

Konstruktor
}
\item{
{\bf  Parameters}
  \begin{itemize}
   \item{
\texttt{user} -- Das User-Objekt, das von dem ListItem repräsentiert werden soll}
  \end{itemize}
}%end item
\end{itemize}
}%end item
\end{itemize}
}
\subsubsection{Methods}{
\vskip -2em
\begin{itemize}
\item{ 
\index{getIcon()}
\hypertarget{edu.kit.pse17.go_app.view.recyclerView.listItems.UserMailListItem.getIcon()}{{\bf  getIcon}\\}
\begin{lstlisting}[frame=none]
Icon getIcon()\end{lstlisting} %end signature
\begin{itemize}
\item{
{\bf  Description copied from \hyperlink{edu.kit.pse17.go_app.view.recyclerView.listItems.ListItem}{ListItem}{\small \refdefined{edu.kit.pse17.go_app.view.recyclerView.listItems.ListItem}} }

getter-Methode für Icon des ListItems
}
\item{{\bf  Returns} -- 
Icon des Datenobjekts 
}%end item
\end{itemize}
}%end item
\item{ 
\index{getSubtitle()}
\hypertarget{edu.kit.pse17.go_app.view.recyclerView.listItems.UserMailListItem.getSubtitle()}{{\bf  getSubtitle}\\}
\begin{lstlisting}[frame=none]
java.lang.String getSubtitle()\end{lstlisting} %end signature
\begin{itemize}
\item{
{\bf  Description copied from \hyperlink{edu.kit.pse17.go_app.view.recyclerView.listItems.ListItem}{ListItem}{\small \refdefined{edu.kit.pse17.go_app.view.recyclerView.listItems.ListItem}} }

getter-Methode für Untertitel des ListItems. Muss ggfs. erst generiert werden, die Information wird als Datentyp T im Objekt gespeichert
}
\item{{\bf  Returns} -- 
Untertitel des Datenobjekts 
}%end item
\end{itemize}
}%end item
\item{ 
\index{getTitle()}
\hypertarget{edu.kit.pse17.go_app.view.recyclerView.listItems.UserMailListItem.getTitle()}{{\bf  getTitle}\\}
\begin{lstlisting}[frame=none]
java.lang.String getTitle()\end{lstlisting} %end signature
\begin{itemize}
\item{
{\bf  Description copied from \hyperlink{edu.kit.pse17.go_app.view.recyclerView.listItems.ListItem}{ListItem}{\small \refdefined{edu.kit.pse17.go_app.view.recyclerView.listItems.ListItem}} }

getter-Methode für Überschrift des ListItems
}
\item{{\bf  Returns} -- 
Titel des Datenobjekts 
}%end item
\end{itemize}
}%end item
\item{ 
\index{setIcon(Icon)}
\hypertarget{edu.kit.pse17.go_app.view.recyclerView.listItems.UserMailListItem.setIcon(Icon)}{{\bf  setIcon}\\}
\begin{lstlisting}[frame=none]
void setIcon(Icon icon)\end{lstlisting} %end signature
\begin{itemize}
\item{
{\bf  Description copied from \hyperlink{edu.kit.pse17.go_app.view.recyclerView.listItems.ListItem}{ListItem}{\small \refdefined{edu.kit.pse17.go_app.view.recyclerView.listItems.ListItem}} }

setter-Methode für icon des ListItems
}
\item{
{\bf  Parameters}
  \begin{itemize}
   \item{
\texttt{icon} -- das neue Icon}
  \end{itemize}
}%end item
\end{itemize}
}%end item
\item{ 
\index{setSubtitle(String)}
\hypertarget{edu.kit.pse17.go_app.view.recyclerView.listItems.UserMailListItem.setSubtitle(java.lang.String)}{{\bf  setSubtitle}\\}
\begin{lstlisting}[frame=none]
public void setSubtitle(java.lang.String s)\end{lstlisting} %end signature
}%end item
\item{ 
\index{setTitle(String)}
\hypertarget{edu.kit.pse17.go_app.view.recyclerView.listItems.UserMailListItem.setTitle(java.lang.String)}{{\bf  setTitle}\\}
\begin{lstlisting}[frame=none]
void setTitle(java.lang.String title)\end{lstlisting} %end signature
\begin{itemize}
\item{
{\bf  Description copied from \hyperlink{edu.kit.pse17.go_app.view.recyclerView.listItems.ListItem}{ListItem}{\small \refdefined{edu.kit.pse17.go_app.view.recyclerView.listItems.ListItem}} }

setter-Methode für Überschrift des ListItems
}
\item{
{\bf  Parameters}
  \begin{itemize}
   \item{
\texttt{title} -- der neue Titel}
  \end{itemize}
}%end item
\end{itemize}
}%end item
\end{itemize}
}
}
\subsection{\label{edu.kit.pse17.go_app.view.recyclerView.listItems.UserStatusListItem}Class UserStatusListItem}{
\hypertarget{edu.kit.pse17.go_app.view.recyclerView.listItems.UserStatusListItem}{}\vskip .1in 
Diese Klasse repräsentiert ListItems, die Informationen über einen User in einem RecyclerView darstellen sollen Created by tina on 19.06.17.\vskip .1in 
\subsubsection{Declaration}{
\begin{lstlisting}[frame=none]
public class UserStatusListItem
 extends java.lang.Object implements ListItem\end{lstlisting}
\subsubsection{Constructor summary}{
\begin{verse}
\hyperlink{edu.kit.pse17.go_app.view.recyclerView.listItems.UserStatusListItem(java.lang.String, edu.kit.pse17.go_app.model.Status, Icon)}{{\bf UserStatusListItem(String, Status, Icon)}} Konstruktor\\
\hyperlink{edu.kit.pse17.go_app.view.recyclerView.listItems.UserStatusListItem(edu.kit.pse17.go_app.model.User, edu.kit.pse17.go_app.model.GO)}{{\bf UserStatusListItem(User, GO)}} \\
\end{verse}
}
\subsubsection{Method summary}{
\begin{verse}
\hyperlink{edu.kit.pse17.go_app.view.recyclerView.listItems.UserStatusListItem.getIcon()}{{\bf getIcon()}} \\
\hyperlink{edu.kit.pse17.go_app.view.recyclerView.listItems.UserStatusListItem.getSubtitle()}{{\bf getSubtitle()}} \\
\hyperlink{edu.kit.pse17.go_app.view.recyclerView.listItems.UserStatusListItem.getTitle()}{{\bf getTitle()}} \\
\hyperlink{edu.kit.pse17.go_app.view.recyclerView.listItems.UserStatusListItem.setIcon(Icon)}{{\bf setIcon(Icon)}} \\
\hyperlink{edu.kit.pse17.go_app.view.recyclerView.listItems.UserStatusListItem.setSubtitle(edu.kit.pse17.go_app.model.Status)}{{\bf setSubtitle(Status)}} \\
\hyperlink{edu.kit.pse17.go_app.view.recyclerView.listItems.UserStatusListItem.setTitle(java.lang.String)}{{\bf setTitle(String)}} \\
\end{verse}
}
\subsubsection{Constructors}{
\vskip -2em
\begin{itemize}
\item{ 
\index{UserStatusListItem(String, Status, Icon)}
\hypertarget{edu.kit.pse17.go_app.view.recyclerView.listItems.UserStatusListItem(java.lang.String, edu.kit.pse17.go_app.model.Status, Icon)}{{\bf  UserStatusListItem}\\}
\begin{lstlisting}[frame=none]
public UserStatusListItem(java.lang.String title,edu.kit.pse17.go_app.model.Status status,Icon icon)\end{lstlisting} %end signature
\begin{itemize}
\item{
{\bf  Description}

Konstruktor
}
\item{
{\bf  Parameters}
  \begin{itemize}
   \item{
\texttt{title} -- Benutzername}
   \item{
\texttt{status} -- Status des Users}
   \item{
\texttt{icon} -- Profilbild}
  \end{itemize}
}%end item
\end{itemize}
}%end item
\item{ 
\index{UserStatusListItem(User, GO)}
\hypertarget{edu.kit.pse17.go_app.view.recyclerView.listItems.UserStatusListItem(edu.kit.pse17.go_app.model.User, edu.kit.pse17.go_app.model.GO)}{{\bf  UserStatusListItem}\\}
\begin{lstlisting}[frame=none]
public UserStatusListItem(edu.kit.pse17.go_app.model.User user,edu.kit.pse17.go_app.model.GO go)\end{lstlisting} %end signature
}%end item
\end{itemize}
}
\subsubsection{Methods}{
\vskip -2em
\begin{itemize}
\item{ 
\index{getIcon()}
\hypertarget{edu.kit.pse17.go_app.view.recyclerView.listItems.UserStatusListItem.getIcon()}{{\bf  getIcon}\\}
\begin{lstlisting}[frame=none]
Icon getIcon()\end{lstlisting} %end signature
\begin{itemize}
\item{
{\bf  Description copied from \hyperlink{edu.kit.pse17.go_app.view.recyclerView.listItems.ListItem}{ListItem}{\small \refdefined{edu.kit.pse17.go_app.view.recyclerView.listItems.ListItem}} }

getter-Methode für Icon des ListItems
}
\item{{\bf  Returns} -- 
Icon des Datenobjekts 
}%end item
\end{itemize}
}%end item
\item{ 
\index{getSubtitle()}
\hypertarget{edu.kit.pse17.go_app.view.recyclerView.listItems.UserStatusListItem.getSubtitle()}{{\bf  getSubtitle}\\}
\begin{lstlisting}[frame=none]
java.lang.String getSubtitle()\end{lstlisting} %end signature
\begin{itemize}
\item{
{\bf  Description copied from \hyperlink{edu.kit.pse17.go_app.view.recyclerView.listItems.ListItem}{ListItem}{\small \refdefined{edu.kit.pse17.go_app.view.recyclerView.listItems.ListItem}} }

getter-Methode für Untertitel des ListItems. Muss ggfs. erst generiert werden, die Information wird als Datentyp T im Objekt gespeichert
}
\item{{\bf  Returns} -- 
Untertitel des Datenobjekts 
}%end item
\end{itemize}
}%end item
\item{ 
\index{getTitle()}
\hypertarget{edu.kit.pse17.go_app.view.recyclerView.listItems.UserStatusListItem.getTitle()}{{\bf  getTitle}\\}
\begin{lstlisting}[frame=none]
java.lang.String getTitle()\end{lstlisting} %end signature
\begin{itemize}
\item{
{\bf  Description copied from \hyperlink{edu.kit.pse17.go_app.view.recyclerView.listItems.ListItem}{ListItem}{\small \refdefined{edu.kit.pse17.go_app.view.recyclerView.listItems.ListItem}} }

getter-Methode für Überschrift des ListItems
}
\item{{\bf  Returns} -- 
Titel des Datenobjekts 
}%end item
\end{itemize}
}%end item
\item{ 
\index{setIcon(Icon)}
\hypertarget{edu.kit.pse17.go_app.view.recyclerView.listItems.UserStatusListItem.setIcon(Icon)}{{\bf  setIcon}\\}
\begin{lstlisting}[frame=none]
void setIcon(Icon icon)\end{lstlisting} %end signature
\begin{itemize}
\item{
{\bf  Description copied from \hyperlink{edu.kit.pse17.go_app.view.recyclerView.listItems.ListItem}{ListItem}{\small \refdefined{edu.kit.pse17.go_app.view.recyclerView.listItems.ListItem}} }

setter-Methode für icon des ListItems
}
\item{
{\bf  Parameters}
  \begin{itemize}
   \item{
\texttt{icon} -- das neue Icon}
  \end{itemize}
}%end item
\end{itemize}
}%end item
\item{ 
\index{setSubtitle(Status)}
\hypertarget{edu.kit.pse17.go_app.view.recyclerView.listItems.UserStatusListItem.setSubtitle(edu.kit.pse17.go_app.model.Status)}{{\bf  setSubtitle}\\}
\begin{lstlisting}[frame=none]
public void setSubtitle(edu.kit.pse17.go_app.model.Status s)\end{lstlisting} %end signature
}%end item
\item{ 
\index{setTitle(String)}
\hypertarget{edu.kit.pse17.go_app.view.recyclerView.listItems.UserStatusListItem.setTitle(java.lang.String)}{{\bf  setTitle}\\}
\begin{lstlisting}[frame=none]
void setTitle(java.lang.String title)\end{lstlisting} %end signature
\begin{itemize}
\item{
{\bf  Description copied from \hyperlink{edu.kit.pse17.go_app.view.recyclerView.listItems.ListItem}{ListItem}{\small \refdefined{edu.kit.pse17.go_app.view.recyclerView.listItems.ListItem}} }

setter-Methode für Überschrift des ListItems
}
\item{
{\bf  Parameters}
  \begin{itemize}
   \item{
\texttt{title} -- der neue Titel}
  \end{itemize}
}%end item
\end{itemize}
}%end item
\end{itemize}
}
}
}
\section{Package edu.kit.pse17.go\_app.view.recyclerView.adapter}{
\label{edu.kit.pse17.go_app.view.recyclerView.adapter}\hypertarget{edu.kit.pse17.go_app.view.recyclerView.adapter}{}
\hskip -.05in
\hbox to \hsize{\textit{ Package Contents\hfil Page}}
\vskip .13in
\hbox{{\bf  Classes}}
\entityintro{GOListAdapter}{edu.kit.pse17.go_app.view.recyclerView.adapter.GOListAdapter}{Konkreter ViewHolder, der die ListItems an das go\_list\_item.xml Layout bindet Created by tina on 19.06.17.}
\entityintro{GroupListAdapter}{edu.kit.pse17.go_app.view.recyclerView.adapter.GroupListAdapter}{Konkreter ViewHolder, der die ListItems an das group\_list\_item.xml Layout bindet Created by tina on 19.06.17.}
\entityintro{ListAdapter}{edu.kit.pse17.go_app.view.recyclerView.adapter.ListAdapter}{Abstrakte Klasse, die Schablone für konkrete Adapter-Klassen bietet.}
\entityintro{UserListAdapter}{edu.kit.pse17.go_app.view.recyclerView.adapter.UserListAdapter}{* Konkreter ViewHolder, der die ListItems an das user\_list\_item.xml Layout bindet Created by tina on 19.06.17.}
\vskip .1in
\vskip .1in
\subsection{\label{edu.kit.pse17.go_app.view.recyclerView.adapter.GOListAdapter}Class GOListAdapter}{
\hypertarget{edu.kit.pse17.go_app.view.recyclerView.adapter.GOListAdapter}{}\vskip .1in 
Konkreter ViewHolder, der die ListItems an das go\_list\_item.xml Layout bindet Created by tina on 19.06.17.\vskip .1in 
\subsubsection{Declaration}{
\begin{lstlisting}[frame=none]
public class GOListAdapter
 extends edu.kit.pse17.go_app.view.recyclerView.adapter.ListAdapter\end{lstlisting}
\subsubsection{Constructor summary}{
\begin{verse}
\hyperlink{edu.kit.pse17.go_app.view.recyclerView.adapter.GOListAdapter(java.util.List, edu.kit.pse17.go_app.view.recyclerView.OnListItemClicked)}{{\bf GOListAdapter(List, OnListItemClicked)}} \\
\end{verse}
}
\subsubsection{Method summary}{
\begin{verse}
\hyperlink{edu.kit.pse17.go_app.view.recyclerView.adapter.GOListAdapter.setLayout()}{{\bf setLayout()}} \\
\end{verse}
}
\subsubsection{Constructors}{
\vskip -2em
\begin{itemize}
\item{ 
\index{GOListAdapter(List, OnListItemClicked)}
\hypertarget{edu.kit.pse17.go_app.view.recyclerView.adapter.GOListAdapter(java.util.List, edu.kit.pse17.go_app.view.recyclerView.OnListItemClicked)}{{\bf  GOListAdapter}\\}
\begin{lstlisting}[frame=none]
public GOListAdapter(java.util.List data,edu.kit.pse17.go_app.view.recyclerView.OnListItemClicked onListItemClicked)\end{lstlisting} %end signature
}%end item
\end{itemize}
}
\subsubsection{Methods}{
\vskip -2em
\begin{itemize}
\item{ 
\index{setLayout()}
\hypertarget{edu.kit.pse17.go_app.view.recyclerView.adapter.GOListAdapter.setLayout()}{{\bf  setLayout}\\}
\begin{lstlisting}[frame=none]
protected abstract int setLayout()\end{lstlisting} %end signature
\begin{itemize}
\item{
{\bf  Description copied from \hyperlink{edu.kit.pse17.go_app.view.recyclerView.adapter.ListAdapter}{ListAdapter}{\small \refdefined{edu.kit.pse17.go_app.view.recyclerView.adapter.ListAdapter}} }

Methode wird von Unterklassen implementiert, um einem konkreten Viewholder das richtige Layout zuweisen zu könne
}
\item{{\bf  Returns} -- 
ID des gewünschten XML Layouts aus R.layout 
}%end item
\end{itemize}
}%end item
\end{itemize}
}
\subsubsection{Members inherited from class ListAdapter }{
\texttt{edu.kit.pse17.go_app.view.recyclerView.adapter.ListAdapter} {\small 
\refdefined{edu.kit.pse17.go_app.view.recyclerView.adapter.ListAdapter}}
{\small 

\vskip -2em
\begin{itemize}
\item{\vskip -1.5ex 
\texttt{public void {\bf  addItem}(\texttt{edu.kit.pse17.go\_app.view.recyclerView.listItems.ListItem} {\bf  item})
}%end signature
}%end item
\item{\vskip -1.5ex 
\texttt{protected {\bf  data}}%end signature
}%end item
\item{\vskip -1.5ex 
\texttt{public ListItem {\bf  getItem}(\texttt{int} {\bf  position})
}%end signature
}%end item
\item{\vskip -1.5ex 
\texttt{public int {\bf  getItemCount}()
}%end signature
}%end item
\item{\vskip -1.5ex 
\texttt{public void {\bf  onBindViewHolder}(\texttt{edu.kit.pse17.go\_app.view.recyclerView.ListViewHolder} {\bf  holder},
\texttt{int} {\bf  position})
}%end signature
}%end item
\item{\vskip -1.5ex 
\texttt{public ListViewHolder {\bf  onCreateViewHolder}(\texttt{ViewGroup} {\bf  parent},
\texttt{int} {\bf  viewType})
}%end signature
}%end item
\item{\vskip -1.5ex 
\texttt{protected final {\bf  onListItemClicked}}%end signature
}%end item
\item{\vskip -1.5ex 
\texttt{protected abstract int {\bf  setLayout}()
}%end signature
}%end item
\end{itemize}
}
}
\subsection{\label{edu.kit.pse17.go_app.view.recyclerView.adapter.GroupListAdapter}Class GroupListAdapter}{
\hypertarget{edu.kit.pse17.go_app.view.recyclerView.adapter.GroupListAdapter}{}\vskip .1in 
Konkreter ViewHolder, der die ListItems an das group\_list\_item.xml Layout bindet Created by tina on 19.06.17.\vskip .1in 
\subsubsection{Declaration}{
\begin{lstlisting}[frame=none]
public class GroupListAdapter
 extends edu.kit.pse17.go_app.view.recyclerView.adapter.ListAdapter\end{lstlisting}
\subsubsection{Constructor summary}{
\begin{verse}
\hyperlink{edu.kit.pse17.go_app.view.recyclerView.adapter.GroupListAdapter(java.util.List, edu.kit.pse17.go_app.view.recyclerView.OnListItemClicked)}{{\bf GroupListAdapter(List, OnListItemClicked)}} \\
\end{verse}
}
\subsubsection{Method summary}{
\begin{verse}
\hyperlink{edu.kit.pse17.go_app.view.recyclerView.adapter.GroupListAdapter.setLayout()}{{\bf setLayout()}} \\
\end{verse}
}
\subsubsection{Constructors}{
\vskip -2em
\begin{itemize}
\item{ 
\index{GroupListAdapter(List, OnListItemClicked)}
\hypertarget{edu.kit.pse17.go_app.view.recyclerView.adapter.GroupListAdapter(java.util.List, edu.kit.pse17.go_app.view.recyclerView.OnListItemClicked)}{{\bf  GroupListAdapter}\\}
\begin{lstlisting}[frame=none]
public GroupListAdapter(java.util.List data,edu.kit.pse17.go_app.view.recyclerView.OnListItemClicked onListItemClicked)\end{lstlisting} %end signature
}%end item
\end{itemize}
}
\subsubsection{Methods}{
\vskip -2em
\begin{itemize}
\item{ 
\index{setLayout()}
\hypertarget{edu.kit.pse17.go_app.view.recyclerView.adapter.GroupListAdapter.setLayout()}{{\bf  setLayout}\\}
\begin{lstlisting}[frame=none]
protected abstract int setLayout()\end{lstlisting} %end signature
\begin{itemize}
\item{
{\bf  Description copied from \hyperlink{edu.kit.pse17.go_app.view.recyclerView.adapter.ListAdapter}{ListAdapter}{\small \refdefined{edu.kit.pse17.go_app.view.recyclerView.adapter.ListAdapter}} }

Methode wird von Unterklassen implementiert, um einem konkreten Viewholder das richtige Layout zuweisen zu könne
}
\item{{\bf  Returns} -- 
ID des gewünschten XML Layouts aus R.layout 
}%end item
\end{itemize}
}%end item
\end{itemize}
}
\subsubsection{Members inherited from class ListAdapter }{
\texttt{edu.kit.pse17.go_app.view.recyclerView.adapter.ListAdapter} {\small 
\refdefined{edu.kit.pse17.go_app.view.recyclerView.adapter.ListAdapter}}
{\small 

\vskip -2em
\begin{itemize}
\item{\vskip -1.5ex 
\texttt{public void {\bf  addItem}(\texttt{edu.kit.pse17.go\_app.view.recyclerView.listItems.ListItem} {\bf  item})
}%end signature
}%end item
\item{\vskip -1.5ex 
\texttt{protected {\bf  data}}%end signature
}%end item
\item{\vskip -1.5ex 
\texttt{public ListItem {\bf  getItem}(\texttt{int} {\bf  position})
}%end signature
}%end item
\item{\vskip -1.5ex 
\texttt{public int {\bf  getItemCount}()
}%end signature
}%end item
\item{\vskip -1.5ex 
\texttt{public void {\bf  onBindViewHolder}(\texttt{edu.kit.pse17.go\_app.view.recyclerView.ListViewHolder} {\bf  holder},
\texttt{int} {\bf  position})
}%end signature
}%end item
\item{\vskip -1.5ex 
\texttt{public ListViewHolder {\bf  onCreateViewHolder}(\texttt{ViewGroup} {\bf  parent},
\texttt{int} {\bf  viewType})
}%end signature
}%end item
\item{\vskip -1.5ex 
\texttt{protected final {\bf  onListItemClicked}}%end signature
}%end item
\item{\vskip -1.5ex 
\texttt{protected abstract int {\bf  setLayout}()
}%end signature
}%end item
\end{itemize}
}
}
\subsection{\label{edu.kit.pse17.go_app.view.recyclerView.adapter.ListAdapter}Class ListAdapter}{
\hypertarget{edu.kit.pse17.go_app.view.recyclerView.adapter.ListAdapter}{}\vskip .1in 
Abstrakte Klasse, die Schablone für konkrete Adapter-Klassen bietet. Unterklassen müssen die Methode setLayout() implementieren, um dem Adapter ein passendes XML-Layout zuzuweisen Created by tina on 17.06.17.\vskip .1in 
\subsubsection{Declaration}{
\begin{lstlisting}[frame=none]
public abstract class ListAdapter
 extends <any>\end{lstlisting}
\subsubsection{All known subclasses}{UserListAdapter\small{\refdefined{edu.kit.pse17.go_app.view.recyclerView.adapter.UserListAdapter}}, GOListAdapter\small{\refdefined{edu.kit.pse17.go_app.view.recyclerView.adapter.GOListAdapter}}, GroupListAdapter\small{\refdefined{edu.kit.pse17.go_app.view.recyclerView.adapter.GroupListAdapter}}}
\subsubsection{Field summary}{
\begin{verse}
\hyperlink{edu.kit.pse17.go_app.view.recyclerView.adapter.ListAdapter.data}{{\bf data}} ListItems, die in dem RecyclerView angezeigt werden sollen\\
\hyperlink{edu.kit.pse17.go_app.view.recyclerView.adapter.ListAdapter.onListItemClicked}{{\bf onListItemClicked}} ClickListener für die Listenelemente\\
\end{verse}
}
\subsubsection{Constructor summary}{
\begin{verse}
\hyperlink{edu.kit.pse17.go_app.view.recyclerView.adapter.ListAdapter(java.util.List, edu.kit.pse17.go_app.view.recyclerView.OnListItemClicked)}{{\bf ListAdapter(List, OnListItemClicked)}} Konstruktor\\
\end{verse}
}
\subsubsection{Method summary}{
\begin{verse}
\hyperlink{edu.kit.pse17.go_app.view.recyclerView.adapter.ListAdapter.addItem(edu.kit.pse17.go_app.view.recyclerView.listItems.ListItem)}{{\bf addItem(ListItem)}} \\
\hyperlink{edu.kit.pse17.go_app.view.recyclerView.adapter.ListAdapter.getItem(int)}{{\bf getItem(int)}} gibt das ListItem an der angegebenen Position zurück\\
\hyperlink{edu.kit.pse17.go_app.view.recyclerView.adapter.ListAdapter.getItemCount()}{{\bf getItemCount()}} \\
\hyperlink{edu.kit.pse17.go_app.view.recyclerView.adapter.ListAdapter.onBindViewHolder(edu.kit.pse17.go_app.view.recyclerView.ListViewHolder, int)}{{\bf onBindViewHolder(ListViewHolder, int)}} \\
\hyperlink{edu.kit.pse17.go_app.view.recyclerView.adapter.ListAdapter.onCreateViewHolder(ViewGroup, int)}{{\bf onCreateViewHolder(ViewGroup, int)}} Schablonenmethode: erzeugt ListViewHolder, dem das passende XML layout zugewiesen wird wird aufgerufen, wenn ein RecyclerView einen neuen ViewHolder braucht, um ein ListItem zu repräsentieren\\
\hyperlink{edu.kit.pse17.go_app.view.recyclerView.adapter.ListAdapter.setLayout()}{{\bf setLayout()}} Methode wird von Unterklassen implementiert, um einem konkreten Viewholder das richtige Layout zuweisen zu könne\\
\end{verse}
}
\subsubsection{Fields}{
\begin{itemize}
\item{
\index{data}
\label{edu.kit.pse17.go_app.view.recyclerView.adapter.ListAdapter.data}\hypertarget{edu.kit.pse17.go_app.view.recyclerView.adapter.ListAdapter.data}{\texttt{protected java.util.List\ {\bf  data}}
}
\begin{itemize}
\item{\vskip -.9ex 
ListItems, die in dem RecyclerView angezeigt werden sollen}
\end{itemize}
}
\item{
\index{onListItemClicked}
\label{edu.kit.pse17.go_app.view.recyclerView.adapter.ListAdapter.onListItemClicked}\hypertarget{edu.kit.pse17.go_app.view.recyclerView.adapter.ListAdapter.onListItemClicked}{\texttt{protected final edu.kit.pse17.go\_app.view.recyclerView.OnListItemClicked\ {\bf  onListItemClicked}}
}
\begin{itemize}
\item{\vskip -.9ex 
ClickListener für die Listenelemente}
\end{itemize}
}
\end{itemize}
}
\subsubsection{Constructors}{
\vskip -2em
\begin{itemize}
\item{ 
\index{ListAdapter(List, OnListItemClicked)}
\hypertarget{edu.kit.pse17.go_app.view.recyclerView.adapter.ListAdapter(java.util.List, edu.kit.pse17.go_app.view.recyclerView.OnListItemClicked)}{{\bf  ListAdapter}\\}
\begin{lstlisting}[frame=none]
public ListAdapter(java.util.List data,edu.kit.pse17.go_app.view.recyclerView.OnListItemClicked onListItemClicked)\end{lstlisting} %end signature
\begin{itemize}
\item{
{\bf  Description}

Konstruktor
}
\item{
{\bf  Parameters}
  \begin{itemize}
   \item{
\texttt{data} -- ListItems, die in dem RecyclerView angezeigt werden sollen}
   \item{
\texttt{onListItemClicked} -- ClickListener für die Listenelemente}
  \end{itemize}
}%end item
\end{itemize}
}%end item
\end{itemize}
}
\subsubsection{Methods}{
\vskip -2em
\begin{itemize}
\item{ 
\index{addItem(ListItem)}
\hypertarget{edu.kit.pse17.go_app.view.recyclerView.adapter.ListAdapter.addItem(edu.kit.pse17.go_app.view.recyclerView.listItems.ListItem)}{{\bf  addItem}\\}
\begin{lstlisting}[frame=none]
public void addItem(edu.kit.pse17.go_app.view.recyclerView.listItems.ListItem item)\end{lstlisting} %end signature
}%end item
\item{ 
\index{getItem(int)}
\hypertarget{edu.kit.pse17.go_app.view.recyclerView.adapter.ListAdapter.getItem(int)}{{\bf  getItem}\\}
\begin{lstlisting}[frame=none]
public edu.kit.pse17.go_app.view.recyclerView.listItems.ListItem getItem(int position)\end{lstlisting} %end signature
\begin{itemize}
\item{
{\bf  Description}

gibt das ListItem an der angegebenen Position zurück
}
\item{
{\bf  Parameters}
  \begin{itemize}
   \item{
\texttt{position} -- Listenposition des gewünschten ListItems}
  \end{itemize}
}%end item
\item{{\bf  Returns} -- 
ListItem, an der angegebenen Position aus der Liste data 
}%end item
\end{itemize}
}%end item
\item{ 
\index{getItemCount()}
\hypertarget{edu.kit.pse17.go_app.view.recyclerView.adapter.ListAdapter.getItemCount()}{{\bf  getItemCount}\\}
\begin{lstlisting}[frame=none]
public int getItemCount()\end{lstlisting} %end signature
}%end item
\item{ 
\index{onBindViewHolder(ListViewHolder, int)}
\hypertarget{edu.kit.pse17.go_app.view.recyclerView.adapter.ListAdapter.onBindViewHolder(edu.kit.pse17.go_app.view.recyclerView.ListViewHolder, int)}{{\bf  onBindViewHolder}\\}
\begin{lstlisting}[frame=none]
public void onBindViewHolder(edu.kit.pse17.go_app.view.recyclerView.ListViewHolder holder,int position)\end{lstlisting} %end signature
}%end item
\item{ 
\index{onCreateViewHolder(ViewGroup, int)}
\hypertarget{edu.kit.pse17.go_app.view.recyclerView.adapter.ListAdapter.onCreateViewHolder(ViewGroup, int)}{{\bf  onCreateViewHolder}\\}
\begin{lstlisting}[frame=none]
public edu.kit.pse17.go_app.view.recyclerView.ListViewHolder onCreateViewHolder(ViewGroup parent,int viewType)\end{lstlisting} %end signature
\begin{itemize}
\item{
{\bf  Description}

Schablonenmethode: erzeugt ListViewHolder, dem das passende XML layout zugewiesen wird wird aufgerufen, wenn ein RecyclerView einen neuen ViewHolder braucht, um ein ListItem zu repräsentieren
}
\item{
{\bf  Parameters}
  \begin{itemize}
   \item{
\texttt{parent} -- Viewgroup, zu der der neue View hinzugefügt werden soll}
   \item{
\texttt{viewType} -- viewType des neuen Views}
  \end{itemize}
}%end item
\item{{\bf  Returns} -- 
neuer ViewHolder des gewünschten Typs 
}%end item
\end{itemize}
}%end item
\item{ 
\index{setLayout()}
\hypertarget{edu.kit.pse17.go_app.view.recyclerView.adapter.ListAdapter.setLayout()}{{\bf  setLayout}\\}
\begin{lstlisting}[frame=none]
protected abstract int setLayout()\end{lstlisting} %end signature
\begin{itemize}
\item{
{\bf  Description}

Methode wird von Unterklassen implementiert, um einem konkreten Viewholder das richtige Layout zuweisen zu könne
}
\item{{\bf  Returns} -- 
ID des gewünschten XML Layouts aus R.layout 
}%end item
\end{itemize}
}%end item
\end{itemize}
}
}
\subsection{\label{edu.kit.pse17.go_app.view.recyclerView.adapter.UserListAdapter}Class UserListAdapter}{
\hypertarget{edu.kit.pse17.go_app.view.recyclerView.adapter.UserListAdapter}{}\vskip .1in 
* Konkreter ViewHolder, der die ListItems an das user\_list\_item.xml Layout bindet Created by tina on 19.06.17.\vskip .1in 
\subsubsection{Declaration}{
\begin{lstlisting}[frame=none]
public class UserListAdapter
 extends edu.kit.pse17.go_app.view.recyclerView.adapter.ListAdapter\end{lstlisting}
\subsubsection{Constructor summary}{
\begin{verse}
\hyperlink{edu.kit.pse17.go_app.view.recyclerView.adapter.UserListAdapter(java.util.List, edu.kit.pse17.go_app.view.recyclerView.OnListItemClicked)}{{\bf UserListAdapter(List, OnListItemClicked)}} \\
\end{verse}
}
\subsubsection{Method summary}{
\begin{verse}
\hyperlink{edu.kit.pse17.go_app.view.recyclerView.adapter.UserListAdapter.setLayout()}{{\bf setLayout()}} \\
\end{verse}
}
\subsubsection{Constructors}{
\vskip -2em
\begin{itemize}
\item{ 
\index{UserListAdapter(List, OnListItemClicked)}
\hypertarget{edu.kit.pse17.go_app.view.recyclerView.adapter.UserListAdapter(java.util.List, edu.kit.pse17.go_app.view.recyclerView.OnListItemClicked)}{{\bf  UserListAdapter}\\}
\begin{lstlisting}[frame=none]
public UserListAdapter(java.util.List data,edu.kit.pse17.go_app.view.recyclerView.OnListItemClicked onListItemClicked)\end{lstlisting} %end signature
}%end item
\end{itemize}
}
\subsubsection{Methods}{
\vskip -2em
\begin{itemize}
\item{ 
\index{setLayout()}
\hypertarget{edu.kit.pse17.go_app.view.recyclerView.adapter.UserListAdapter.setLayout()}{{\bf  setLayout}\\}
\begin{lstlisting}[frame=none]
protected abstract int setLayout()\end{lstlisting} %end signature
\begin{itemize}
\item{
{\bf  Description copied from \hyperlink{edu.kit.pse17.go_app.view.recyclerView.adapter.ListAdapter}{ListAdapter}{\small \refdefined{edu.kit.pse17.go_app.view.recyclerView.adapter.ListAdapter}} }

Methode wird von Unterklassen implementiert, um einem konkreten Viewholder das richtige Layout zuweisen zu könne
}
\item{{\bf  Returns} -- 
ID des gewünschten XML Layouts aus R.layout 
}%end item
\end{itemize}
}%end item
\end{itemize}
}
\subsubsection{Members inherited from class ListAdapter }{
\texttt{edu.kit.pse17.go_app.view.recyclerView.adapter.ListAdapter} {\small 
\refdefined{edu.kit.pse17.go_app.view.recyclerView.adapter.ListAdapter}}
{\small 

\vskip -2em
\begin{itemize}
\item{\vskip -1.5ex 
\texttt{public void {\bf  addItem}(\texttt{edu.kit.pse17.go\_app.view.recyclerView.listItems.ListItem} {\bf  item})
}%end signature
}%end item
\item{\vskip -1.5ex 
\texttt{protected {\bf  data}}%end signature
}%end item
\item{\vskip -1.5ex 
\texttt{public ListItem {\bf  getItem}(\texttt{int} {\bf  position})
}%end signature
}%end item
\item{\vskip -1.5ex 
\texttt{public int {\bf  getItemCount}()
}%end signature
}%end item
\item{\vskip -1.5ex 
\texttt{public void {\bf  onBindViewHolder}(\texttt{edu.kit.pse17.go\_app.view.recyclerView.ListViewHolder} {\bf  holder},
\texttt{int} {\bf  position})
}%end signature
}%end item
\item{\vskip -1.5ex 
\texttt{public ListViewHolder {\bf  onCreateViewHolder}(\texttt{ViewGroup} {\bf  parent},
\texttt{int} {\bf  viewType})
}%end signature
}%end item
\item{\vskip -1.5ex 
\texttt{protected final {\bf  onListItemClicked}}%end signature
}%end item
\item{\vskip -1.5ex 
\texttt{protected abstract int {\bf  setLayout}()
}%end signature
}%end item
\end{itemize}
}
}
}
\section{Package edu.kit.pse17.go\_app.view.recyclerView}{
\label{edu.kit.pse17.go_app.view.recyclerView}\hypertarget{edu.kit.pse17.go_app.view.recyclerView}{}
\hskip -.05in
\hbox to \hsize{\textit{ Package Contents\hfil Page}}
\vskip .13in
\hbox{{\bf  Interfaces}}
\entityintro{OnListItemClicked}{edu.kit.pse17.go_app.view.recyclerView.OnListItemClicked}{ClickListener für die ListItems eines RecyclerViews Created by tina on 17.06.17.}
\vskip .13in
\hbox{{\bf  Classes}}
\entityintro{ListViewHolder}{edu.kit.pse17.go_app.view.recyclerView.ListViewHolder}{Die Klasse erzeugt ViewHolder-Objekte, die die Datenobjekt für die RecyclerView enthalten Created by tina on 17.06.17.}
\vskip .1in
\vskip .1in
\subsection{\label{edu.kit.pse17.go_app.view.recyclerView.OnListItemClicked}Interface OnListItemClicked}{
\hypertarget{edu.kit.pse17.go_app.view.recyclerView.OnListItemClicked}{}\vskip .1in 
ClickListener für die ListItems eines RecyclerViews Created by tina on 17.06.17.\vskip .1in 
\subsubsection{Declaration}{
\begin{lstlisting}[frame=none]
public interface OnListItemClicked
\end{lstlisting}
\subsubsection{All known subinterfaces}{GroupDetailActivity\small{\refdefined{edu.kit.pse17.go_app.view.GroupDetailActivity}}, GroupDetailActivityAdmin\small{\refdefined{edu.kit.pse17.go_app.view.GroupDetailActivityAdmin}}, GroupListActivity\small{\refdefined{edu.kit.pse17.go_app.view.GroupListActivity}}}
\subsubsection{All classes known to implement interface}{GroupDetailActivity\small{\refdefined{edu.kit.pse17.go_app.view.GroupDetailActivity}}, GroupListActivity\small{\refdefined{edu.kit.pse17.go_app.view.GroupListActivity}}}
\subsubsection{Method summary}{
\begin{verse}
\hyperlink{edu.kit.pse17.go_app.view.recyclerView.OnListItemClicked.onItemClicked(int)}{{\bf onItemClicked(int)}} führt gewünschte Aktion der implemetierenden Klasse aus, falls auf das ListItem an Position position geklickt wird\\
\end{verse}
}
\subsubsection{Methods}{
\vskip -2em
\begin{itemize}
\item{ 
\index{onItemClicked(int)}
\hypertarget{edu.kit.pse17.go_app.view.recyclerView.OnListItemClicked.onItemClicked(int)}{{\bf  onItemClicked}\\}
\begin{lstlisting}[frame=none]
void onItemClicked(int position)\end{lstlisting} %end signature
\begin{itemize}
\item{
{\bf  Description}

führt gewünschte Aktion der implemetierenden Klasse aus, falls auf das ListItem an Position position geklickt wird
}
\item{
{\bf  Parameters}
  \begin{itemize}
   \item{
\texttt{position} -- Position des ListItems, auf das geklickt wurde}
  \end{itemize}
}%end item
\end{itemize}
}%end item
\end{itemize}
}
}
\subsection{\label{edu.kit.pse17.go_app.view.recyclerView.ListViewHolder}Class ListViewHolder}{
\hypertarget{edu.kit.pse17.go_app.view.recyclerView.ListViewHolder}{}\vskip .1in 
Die Klasse erzeugt ViewHolder-Objekte, die die Datenobjekt für die RecyclerView enthalten Created by tina on 17.06.17.\vskip .1in 
\subsubsection{Declaration}{
\begin{lstlisting}[frame=none]
public class ListViewHolder
 extends ViewHolder\end{lstlisting}
\subsubsection{Field summary}{
\begin{verse}
\hyperlink{edu.kit.pse17.go_app.view.recyclerView.ListViewHolder.icon}{{\bf icon}} Icon, das zum Item angezeigt werden soll\\
\hyperlink{edu.kit.pse17.go_app.view.recyclerView.ListViewHolder.subtitle}{{\bf subtitle}} Untertitel des Items\\
\hyperlink{edu.kit.pse17.go_app.view.recyclerView.ListViewHolder.title}{{\bf title}} Titel des Items\\
\end{verse}
}
\subsubsection{Constructor summary}{
\begin{verse}
\hyperlink{edu.kit.pse17.go_app.view.recyclerView.ListViewHolder(View)}{{\bf ListViewHolder(View)}} \\
\hyperlink{edu.kit.pse17.go_app.view.recyclerView.ListViewHolder(View, edu.kit.pse17.go_app.view.recyclerView.OnListItemClicked)}{{\bf ListViewHolder(View, OnListItemClicked)}} Konstruktor\\
\end{verse}
}
\subsubsection{Method summary}{
\begin{verse}
\hyperlink{edu.kit.pse17.go_app.view.recyclerView.ListViewHolder.onClick(View)}{{\bf onClick(View)}} \\
\end{verse}
}
\subsubsection{Fields}{
\begin{itemize}
\item{
\index{title}
\label{edu.kit.pse17.go_app.view.recyclerView.ListViewHolder.title}\hypertarget{edu.kit.pse17.go_app.view.recyclerView.ListViewHolder.title}{\texttt{public TextView\ {\bf  title}}
}
\begin{itemize}
\item{\vskip -.9ex 
Titel des Items}
\end{itemize}
}
\item{
\index{subtitle}
\label{edu.kit.pse17.go_app.view.recyclerView.ListViewHolder.subtitle}\hypertarget{edu.kit.pse17.go_app.view.recyclerView.ListViewHolder.subtitle}{\texttt{public TextView\ {\bf  subtitle}}
}
\begin{itemize}
\item{\vskip -.9ex 
Untertitel des Items}
\end{itemize}
}
\item{
\index{icon}
\label{edu.kit.pse17.go_app.view.recyclerView.ListViewHolder.icon}\hypertarget{edu.kit.pse17.go_app.view.recyclerView.ListViewHolder.icon}{\texttt{public ImageView\ {\bf  icon}}
}
\begin{itemize}
\item{\vskip -.9ex 
Icon, das zum Item angezeigt werden soll}
\end{itemize}
}
\end{itemize}
}
\subsubsection{Constructors}{
\vskip -2em
\begin{itemize}
\item{ 
\index{ListViewHolder(View)}
\hypertarget{edu.kit.pse17.go_app.view.recyclerView.ListViewHolder(View)}{{\bf  ListViewHolder}\\}
\begin{lstlisting}[frame=none]
public ListViewHolder(View itemView)\end{lstlisting} %end signature
}%end item
\item{ 
\index{ListViewHolder(View, OnListItemClicked)}
\hypertarget{edu.kit.pse17.go_app.view.recyclerView.ListViewHolder(View, edu.kit.pse17.go_app.view.recyclerView.OnListItemClicked)}{{\bf  ListViewHolder}\\}
\begin{lstlisting}[frame=none]
public ListViewHolder(View itemView,OnListItemClicked onListItemClicked)\end{lstlisting} %end signature
\begin{itemize}
\item{
{\bf  Description}

Konstruktor
}
\item{
{\bf  Parameters}
  \begin{itemize}
   \item{
\texttt{itemView} -- View, in der die Items angezeigt werden sollen}
   \item{
\texttt{onListItemClicked} -- ClickListener für ListItems}
  \end{itemize}
}%end item
\end{itemize}
}%end item
\end{itemize}
}
\subsubsection{Methods}{
\vskip -2em
\begin{itemize}
\item{ 
\index{onClick(View)}
\hypertarget{edu.kit.pse17.go_app.view.recyclerView.ListViewHolder.onClick(View)}{{\bf  onClick}\\}
\begin{lstlisting}[frame=none]
public void onClick(View v)\end{lstlisting} %end signature
}%end item
\end{itemize}
}
}
}
% ------- textdoclet_include/finish.tex

% add something here

% closing for \chapter{TeXDoclet Java Documentation} {
}

\chapter{Datenbank}

\chapter{Klassendiagramme}

\chapter{Sequenzdiagramme}

\chapter{Finish}{
Lorem ipsum dolor sit amet, consetetur sadipscing elitr, sed diam nonumy eirmod tempor invidunt ut labore et dolore magna aliquyam erat, sed diam voluptua. At vero eos et accusam et justo duo dolores et ea rebum. Stet clita kasd gubergren, no sea takimata sanctus est Lorem ipsum dolor sit amet. Lorem ipsum dolor sit amet, consetetur sadipscing elitr, sed diam nonumy eirmod tempor invidunt ut labore et dolore magna aliquyam erat, sed diam voluptua. At vero eos et accusam et justo duo dolores et ea rebum. Stet clita kasd gubergren, no sea takimata sanctus est Lorem ipsum dolor sit amet.

}
% ------- textdoclet_include/finish.tex end

\end{document}
