\documentclass[11pt,a4paper]{report}
\usepackage{color}
\usepackage{ifthen}
\usepackage{ifpdf}
\usepackage[headings]{fullpage}
\usepackage{listings}
\lstset{language=Java,breaklines=true}
\ifpdf \usepackage[pdftex, pdfpagemode={UseOutlines},bookmarks,colorlinks,linkcolor={blue},plainpages=false,pdfpagelabels,citecolor={red},breaklinks=true]{hyperref}
  \usepackage[pdftex]{graphicx}
  \pdfcompresslevel=9
  \DeclareGraphicsRule{*}{mps}{*}{}
\else
  \usepackage[dvips]{graphicx}
\fi

\newcommand{\entityintro}[3]{%
  \hbox to \hsize{%
    \vbox{%
      \hbox to .2in{}%
    }%
    {\bf  #1}%
    \dotfill\pageref{#2}%
  }
  \makebox[\hsize]{%
    \parbox{.4in}{}%
    \parbox[l]{5in}{%
      \vspace{1mm}%
      #3%
      \vspace{1mm}%
    }%
  }%
}
\newcommand{\refdefined}[1]{
\expandafter\ifx\csname r@#1\endcsname\relax
\relax\else
{$($in \ref{#1}, page \pageref{#1}$)$}\fi}
\date{\today}
\chardef\textbackslash=`\\
\usepackage{pdfpages}
\usepackage[utf8]{inputenc}
\usepackage[T1]{fontenc}
\usepackage[german]{babel}
\usepackage{hyperref}
\hypersetup{
	pdftitle={Pflichtenheft},
	bookmarks=true,
}
\usepackage{csquotes}

\usepackage{fancyhdr}%<-------------to control headers and footers
\usepackage[a4paper,margin=1in,footskip=.25in]{geometry}
\fancyhf{}
\fancyfoot[C]{\thepage} %<----to get page number below text
\pagestyle{fancy} %<-------the page style itself

\usepackage{xcolor}
\usepackage{framed}
\definecolor{shadecolor}{RGB}{220,220,220}
\usepackage{float}


\title{Android GO! App - Pflichtenheft}
\author{Gruppe 3}
\date{11.06.17}

% define custom lists
\usepackage{enumitem}
\usepackage{lipsum}

\begin{document}

\begin{titlepage}
	\begin{center}
	{\scshape\LARGE \bfseries Entwurfsdokument \par}
	\vspace{1cm}
	{\scshape\Large Praktikum der Softwareentwicklung \\ Sommersemester 2017\par}
	\vspace{1.5cm}
	{\huge\bfseries Android GO! App\par}
	\vspace{2cm}
	{\Large\itshape - Gruppe 3 -\par}
	\vfill
	{\bfseries erstellt von:\par}
	Arsenii Dunaev \\
	Florian Kröger \\
	Tina Maria Strößner \\
	Volodymyr Shpylka \\	
	\vfill
	% Bottom of the page
	{\large 09.07.17 \par}	
	\end{center}
\end{titlepage}

\begin{abstract}
Die Android App GO! ist eine mobile Applikation, die speziell zur Organisation von Treffen (z. B. gemeinsames Essen im Café oder in der Mensa) entwickelt wird. Beim erfolgreichen gemeinsamen Losgehen wird der gemittelte GPS-Standort von Mitgliedern der Gruppe angezeigt.\\

Dieses Dokument erläutert den Entwurf des Systems auf der Grundlage des Pflichtenhefts.
\end{abstract}

% ------- textdoclet_include/setup.tex end

\sloppy
\addtocontents{toc}{\protect\markboth{Contents}{Contents}}
\tableofcontents


% add something here

\chapter{Änderungen zum Pflichtenheft} {

Lorem ipsum dolor sit amet, consetetur sadipscing elitr, sed diam nonumy eirmod tempor invidunt ut labore et dolore magna aliquyam erat, sed diam voluptua. At vero eos et accusam et justo duo dolores et ea rebum. Stet clita kasd gubergren, no sea takimata sanctus est Lorem ipsum dolor sit amet. Lorem ipsum dolor sit amet, consetetur sadipscing elitr, sed diam nonumy eirmod tempor invidunt ut labore et dolore magna aliquyam erat, sed diam voluptua. At vero eos et accusam et justo duo dolores et ea rebum. Stet clita kasd gubergren, no sea takimata sanctus est Lorem ipsum dolor sit amet.
}
\chapter{Paketstruktur} {

Lorem ipsum dolor sit amet, consetetur sadipscing elitr, sed diam nonumy eirmod tempor invidunt ut labore et dolore magna aliquyam erat, sed diam voluptua. At vero eos et accusam et justo duo dolores et ea rebum. Stet clita kasd gubergren, no sea takimata sanctus est Lorem ipsum dolor sit amet. Lorem ipsum dolor sit amet, consetetur sadipscing elitr, sed diam nonumy eirmod tempor invidunt ut labore et dolore magna aliquyam erat, sed diam voluptua. At vero eos et accusam et justo duo dolores et ea rebum. Stet clita kasd gubergren, no sea takimata sanctus est Lorem ipsum dolor sit amet.

Lorem ipsum dolor sit amet, consetetur sadipscing elitr, sed diam nonumy eirmod tempor invidunt ut labore et dolore magna aliquyam erat, sed diam voluptua. At vero eos et accusam et justo duo dolores et ea rebum. Stet clita kasd gubergren, no sea takimata sanctus est Lorem ipsum dolor sit amet. Lorem ipsum dolor sit amet, consetetur sadipscing elitr, sed diam nonumy eirmod tempor invidunt ut labore et dolore magna aliquyam erat, sed diam voluptua. At vero eos et accusam et justo duo dolores et ea rebum. Stet clita kasd gubergren, no sea takimata sanctus est Lorem ipsum dolor sit amet.

Lorem ipsum dolor sit amet, consetetur sadipscing elitr, sed diam nonumy eirmod tempor invidunt ut labore et dolore magna aliquyam erat, sed diam voluptua. At vero eos et accusam et justo duo dolores et ea rebum. Stet clita kasd gubergren, no sea takimata sanctus est Lorem ipsum dolor sit amet. Lorem ipsum dolor sit amet, consetetur sadipscing elitr, sed diam nonumy eirmod tempor invidunt ut labore et dolore magna aliquyam erat, sed diam voluptua. At vero eos et accusam et justo duo dolores et ea rebum. Stet clita kasd gubergren, no sea takimata sanctus est Lorem ipsum dolor sit amet.
}

\chapter{Klassenbeschreibungen} {

% ------- textdoclet_include/intro.tex end

\section*{Class Hierarchy}{
\thispagestyle{empty}
\markboth{Class Hierarchy}{Class Hierarchy}
\addcontentsline{toc}{section}{Class Hierarchy}
\subsection*{Classes}
{\raggedright
\hspace{0.0cm} $\bullet$ java.lang.Object {\tiny \refdefined{java.lang.Object}} \\
\hspace{1.0cm} $\bullet$  {\tiny } \\
\hspace{2.0cm} $\bullet$ edu.kit.pse17.go\_app.RecyclerView.ListAdapter {\tiny \refdefined{edu.kit.pse17.go_app.RecyclerView.ListAdapter}} \\
\hspace{1.0cm} $\bullet$ AppCompatActivity {\tiny } \\
\hspace{2.0cm} $\bullet$ edu.kit.pse17.go\_app.GroupListActivity {\tiny \refdefined{edu.kit.pse17.go_app.GroupListActivity}} \\
\hspace{2.0cm} $\bullet$ edu.kit.pse17.go\_app.Login.LogInActivity {\tiny \refdefined{edu.kit.pse17.go_app.Login.LogInActivity}} \\
\hspace{2.0cm} $\bullet$ edu.kit.pse17.go\_app.Login.LoginHelper {\tiny \refdefined{edu.kit.pse17.go_app.Login.LoginHelper}} \\
\hspace{3.0cm} $\bullet$ edu.kit.pse17.go\_app.Login.FirebaseLoginHelper {\tiny \refdefined{edu.kit.pse17.go_app.Login.FirebaseLoginHelper}} \\
\hspace{3.0cm} $\bullet$ edu.kit.pse17.go\_app.Login.GoLoginHelper {\tiny \refdefined{edu.kit.pse17.go_app.Login.GoLoginHelper}} \\
\hspace{2.0cm} $\bullet$ edu.kit.pse17.go\_app.MainActivity {\tiny \refdefined{edu.kit.pse17.go_app.MainActivity}} \\
\hspace{1.0cm} $\bullet$ RecyclerView.ViewHolder {\tiny } \\
\hspace{2.0cm} $\bullet$ edu.kit.pse17.go\_app.RecyclerView.ListViewHolder {\tiny \refdefined{edu.kit.pse17.go_app.RecyclerView.ListViewHolder}} \\
\hspace{1.0cm} $\bullet$ edu.kit.pse17.go\_app.GO {\tiny \refdefined{edu.kit.pse17.go_app.GO}} \\
\hspace{1.0cm} $\bullet$ edu.kit.pse17.go\_app.Group {\tiny \refdefined{edu.kit.pse17.go_app.Group}} \\
\hspace{1.0cm} $\bullet$ edu.kit.pse17.go\_app.RecyclerView.GORecyclerView.GOListItem {\tiny \refdefined{edu.kit.pse17.go_app.RecyclerView.GORecyclerView.GOListItem}} \\
\hspace{1.0cm} $\bullet$ edu.kit.pse17.go\_app.RecyclerView.GroupRecyclerView.GroupListItem {\tiny \refdefined{edu.kit.pse17.go_app.RecyclerView.GroupRecyclerView.GroupListItem}} \\
\hspace{1.0cm} $\bullet$ edu.kit.pse17.go\_app.User {\tiny \refdefined{edu.kit.pse17.go_app.User}} \\
}
\subsection*{Interfaces}
\hspace{0.0cm} $\bullet$ edu.kit.pse17.go\_app.RecyclerView.ListItem {\tiny \refdefined{edu.kit.pse17.go_app.RecyclerView.ListItem}} \\
\hspace{0.0cm} $\bullet$ edu.kit.pse17.go\_app.RecyclerView.OnListItemClicked {\tiny \refdefined{edu.kit.pse17.go_app.RecyclerView.OnListItemClicked}} \\
}
\section{Package edu.kit.pse17.go\_app.Login}{
\label{edu.kit.pse17.go_app.Login}\hypertarget{edu.kit.pse17.go_app.Login}{}
\hskip -.05in
\hbox to \hsize{\textit{ Package Contents\hfil Page}}
\vskip .13in
\hbox{{\bf  Classes}}
\entityintro{FirebaseLoginHelper}{edu.kit.pse17.go_app.Login.FirebaseLoginHelper}{Diese Klasse ist für die Kommunikation mit der Firebase und Goofle API zuständig Created by tina on 17.06.17.}
\entityintro{GoLoginHelper}{edu.kit.pse17.go_app.Login.GoLoginHelper}{Die Klasse ist für die Anmeldung eines Beutzers am GO-Server zuständig Created by tina on 17.06.17.}
\entityintro{LogInActivity}{edu.kit.pse17.go_app.Login.LogInActivity}{Die Klasse zeigt dem User den Login-Screen an und koordiniert den LogIn Prozess Created by tina on 17.06.17.}
\entityintro{LoginHelper}{edu.kit.pse17.go_app.Login.LoginHelper}{Created by tina on 18.06.17.}
\vskip .1in
\vskip .1in
\subsection{\label{edu.kit.pse17.go_app.Login.FirebaseLoginHelper}Class FirebaseLoginHelper}{
\hypertarget{edu.kit.pse17.go_app.Login.FirebaseLoginHelper}{}\vskip .1in 
Diese Klasse ist für die Kommunikation mit der Firebase und Goofle API zuständig Created by tina on 17.06.17.\vskip .1in 
\subsubsection{Declaration}{
\begin{lstlisting}[frame=none]
public class FirebaseLoginHelper
 extends edu.kit.pse17.go_app.Login.LoginHelper\end{lstlisting}
\subsubsection{Field summary}{
\begin{verse}
\hyperlink{edu.kit.pse17.go_app.Login.FirebaseLoginHelper.UID_CODE}{{\bf UID\_CODE}} Name des String-Extras des Intents des Activity Results\\
\end{verse}
}
\subsubsection{Constructor summary}{
\begin{verse}
\hyperlink{edu.kit.pse17.go_app.Login.FirebaseLoginHelper()}{{\bf FirebaseLoginHelper()}} \\
\end{verse}
}
\subsubsection{Method summary}{
\begin{verse}
\hyperlink{edu.kit.pse17.go_app.Login.FirebaseLoginHelper.onActivityResult(int, int, Intent)}{{\bf onActivityResult(int, int, Intent)}} \\
\hyperlink{edu.kit.pse17.go_app.Login.FirebaseLoginHelper.onConnectionFailed(ConnectionResult)}{{\bf onConnectionFailed(ConnectionResult)}} wird aufgerufen, falls Verbindung zu google Play Services fehlschlägt\\
\hyperlink{edu.kit.pse17.go_app.Login.FirebaseLoginHelper.onCreate(Bundle)}{{\bf onCreate(Bundle)}} \\
\hyperlink{edu.kit.pse17.go_app.Login.FirebaseLoginHelper.onStart()}{{\bf onStart()}} startet die Aktivität\\
\end{verse}
}
\subsubsection{Fields}{
\begin{itemize}
\item{
\index{UID\_CODE}
\label{edu.kit.pse17.go_app.Login.FirebaseLoginHelper.UID_CODE}\hypertarget{edu.kit.pse17.go_app.Login.FirebaseLoginHelper.UID_CODE}{\texttt{public static final java.lang.String\ {\bf  UID\_CODE}}
}
\begin{itemize}
\item{\vskip -.9ex 
Name des String-Extras des Intents des Activity Results}
\end{itemize}
}
\end{itemize}
}
\subsubsection{Constructors}{
\vskip -2em
\begin{itemize}
\item{ 
\index{FirebaseLoginHelper()}
\hypertarget{edu.kit.pse17.go_app.Login.FirebaseLoginHelper()}{{\bf  FirebaseLoginHelper}\\}
\begin{lstlisting}[frame=none]
public FirebaseLoginHelper()\end{lstlisting} %end signature
}%end item
\end{itemize}
}
\subsubsection{Methods}{
\vskip -2em
\begin{itemize}
\item{ 
\index{onActivityResult(int, int, Intent)}
\hypertarget{edu.kit.pse17.go_app.Login.FirebaseLoginHelper.onActivityResult(int, int, Intent)}{{\bf  onActivityResult}\\}
\begin{lstlisting}[frame=none]
protected void onActivityResult(int requestCode,int resultCode,Intent data)\end{lstlisting} %end signature
}%end item
\item{ 
\index{onConnectionFailed(ConnectionResult)}
\hypertarget{edu.kit.pse17.go_app.Login.FirebaseLoginHelper.onConnectionFailed(ConnectionResult)}{{\bf  onConnectionFailed}\\}
\begin{lstlisting}[frame=none]
public void onConnectionFailed(ConnectionResult connectionResult)\end{lstlisting} %end signature
\begin{itemize}
\item{
{\bf  Description}

wird aufgerufen, falls Verbindung zu google Play Services fehlschlägt
}
\item{
{\bf  Parameters}
  \begin{itemize}
   \item{
\texttt{connectionResult} -- Ergebnis der fehlgeschlagenen Verbindung}
  \end{itemize}
}%end item
\end{itemize}
}%end item
\item{ 
\index{onCreate(Bundle)}
\hypertarget{edu.kit.pse17.go_app.Login.FirebaseLoginHelper.onCreate(Bundle)}{{\bf  onCreate}\\}
\begin{lstlisting}[frame=none]
protected void onCreate(Bundle savedInstanceState)\end{lstlisting} %end signature
}%end item
\item{ 
\index{onStart()}
\hypertarget{edu.kit.pse17.go_app.Login.FirebaseLoginHelper.onStart()}{{\bf  onStart}\\}
\begin{lstlisting}[frame=none]
protected void onStart()\end{lstlisting} %end signature
\begin{itemize}
\item{
{\bf  Description}

startet die Aktivität
}
\end{itemize}
}%end item
\end{itemize}
}
\subsubsection{Members inherited from class LoginHelper }{
\texttt{edu.kit.pse17.go_app.Login.LoginHelper} {\small 
\refdefined{edu.kit.pse17.go_app.Login.LoginHelper}}
{\small 

\vskip -2em
\begin{itemize}
\item{\vskip -1.5ex 
\texttt{public static final {\bf  ACCOUNT\_DATA\_CODE}}%end signature
}%end item
\item{\vskip -1.5ex 
\texttt{protected void {\bf  returnActivityResult}(\texttt{java.io.Serializable} {\bf  accountData})
}%end signature
}%end item
\item{\vskip -1.5ex 
\texttt{public static final {\bf  SIGN\_IN\_DATA\_CODE}}%end signature
}%end item
\item{\vskip -1.5ex 
\texttt{protected static void {\bf  signIn}(\texttt{Activity} {\bf  activity},
\texttt{int} {\bf  requestCode},
\texttt{java.io.Serializable} {\bf  signInData},
\texttt{java.lang.Class} {\bf  loginHelper})
}%end signature
}%end item
\end{itemize}
}
}
\subsection{\label{edu.kit.pse17.go_app.Login.GoLoginHelper}Class GoLoginHelper}{
\hypertarget{edu.kit.pse17.go_app.Login.GoLoginHelper}{}\vskip .1in 
Die Klasse ist für die Anmeldung eines Beutzers am GO-Server zuständig Created by tina on 17.06.17.\vskip .1in 
\subsubsection{Declaration}{
\begin{lstlisting}[frame=none]
public class GoLoginHelper
 extends edu.kit.pse17.go_app.Login.LoginHelper\end{lstlisting}
\subsubsection{Constructor summary}{
\begin{verse}
\hyperlink{edu.kit.pse17.go_app.Login.GoLoginHelper()}{{\bf GoLoginHelper()}} \\
\end{verse}
}
\subsubsection{Method summary}{
\begin{verse}
\hyperlink{edu.kit.pse17.go_app.Login.GoLoginHelper.onCreate(Bundle)}{{\bf onCreate(Bundle)}} \\
\hyperlink{edu.kit.pse17.go_app.Login.GoLoginHelper.onStart()}{{\bf onStart()}} \\
\end{verse}
}
\subsubsection{Constructors}{
\vskip -2em
\begin{itemize}
\item{ 
\index{GoLoginHelper()}
\hypertarget{edu.kit.pse17.go_app.Login.GoLoginHelper()}{{\bf  GoLoginHelper}\\}
\begin{lstlisting}[frame=none]
public GoLoginHelper()\end{lstlisting} %end signature
}%end item
\end{itemize}
}
\subsubsection{Methods}{
\vskip -2em
\begin{itemize}
\item{ 
\index{onCreate(Bundle)}
\hypertarget{edu.kit.pse17.go_app.Login.GoLoginHelper.onCreate(Bundle)}{{\bf  onCreate}\\}
\begin{lstlisting}[frame=none]
protected void onCreate(Bundle savedInstanceState)\end{lstlisting} %end signature
}%end item
\item{ 
\index{onStart()}
\hypertarget{edu.kit.pse17.go_app.Login.GoLoginHelper.onStart()}{{\bf  onStart}\\}
\begin{lstlisting}[frame=none]
protected void onStart()\end{lstlisting} %end signature
}%end item
\end{itemize}
}
\subsubsection{Members inherited from class LoginHelper }{
\texttt{edu.kit.pse17.go_app.Login.LoginHelper} {\small 
\refdefined{edu.kit.pse17.go_app.Login.LoginHelper}}
{\small 

\vskip -2em
\begin{itemize}
\item{\vskip -1.5ex 
\texttt{public static final {\bf  ACCOUNT\_DATA\_CODE}}%end signature
}%end item
\item{\vskip -1.5ex 
\texttt{protected void {\bf  returnActivityResult}(\texttt{java.io.Serializable} {\bf  accountData})
}%end signature
}%end item
\item{\vskip -1.5ex 
\texttt{public static final {\bf  SIGN\_IN\_DATA\_CODE}}%end signature
}%end item
\item{\vskip -1.5ex 
\texttt{protected static void {\bf  signIn}(\texttt{Activity} {\bf  activity},
\texttt{int} {\bf  requestCode},
\texttt{java.io.Serializable} {\bf  signInData},
\texttt{java.lang.Class} {\bf  loginHelper})
}%end signature
}%end item
\end{itemize}
}
}
\subsection{\label{edu.kit.pse17.go_app.Login.LogInActivity}Class LogInActivity}{
\hypertarget{edu.kit.pse17.go_app.Login.LogInActivity}{}\vskip .1in 
Die Klasse zeigt dem User den Login-Screen an und koordiniert den LogIn Prozess Created by tina on 17.06.17.\vskip .1in 
\subsubsection{Declaration}{
\begin{lstlisting}[frame=none]
public class LogInActivity
 extends AppCompatActivity\end{lstlisting}
\subsubsection{Constructor summary}{
\begin{verse}
\hyperlink{edu.kit.pse17.go_app.Login.LogInActivity()}{{\bf LogInActivity()}} \\
\end{verse}
}
\subsubsection{Method summary}{
\begin{verse}
\hyperlink{edu.kit.pse17.go_app.Login.LogInActivity.onActivityResult(int, int, Intent)}{{\bf onActivityResult(int, int, Intent)}} \\
\hyperlink{edu.kit.pse17.go_app.Login.LogInActivity.onClick(View)}{{\bf onClick(View)}} \\
\hyperlink{edu.kit.pse17.go_app.Login.LogInActivity.onCreate(Bundle)}{{\bf onCreate(Bundle)}} \\
\end{verse}
}
\subsubsection{Constructors}{
\vskip -2em
\begin{itemize}
\item{ 
\index{LogInActivity()}
\hypertarget{edu.kit.pse17.go_app.Login.LogInActivity()}{{\bf  LogInActivity}\\}
\begin{lstlisting}[frame=none]
public LogInActivity()\end{lstlisting} %end signature
}%end item
\end{itemize}
}
\subsubsection{Methods}{
\vskip -2em
\begin{itemize}
\item{ 
\index{onActivityResult(int, int, Intent)}
\hypertarget{edu.kit.pse17.go_app.Login.LogInActivity.onActivityResult(int, int, Intent)}{{\bf  onActivityResult}\\}
\begin{lstlisting}[frame=none]
protected void onActivityResult(int requestCode,int resultCode,Intent data)\end{lstlisting} %end signature
}%end item
\item{ 
\index{onClick(View)}
\hypertarget{edu.kit.pse17.go_app.Login.LogInActivity.onClick(View)}{{\bf  onClick}\\}
\begin{lstlisting}[frame=none]
public void onClick(View v)\end{lstlisting} %end signature
}%end item
\item{ 
\index{onCreate(Bundle)}
\hypertarget{edu.kit.pse17.go_app.Login.LogInActivity.onCreate(Bundle)}{{\bf  onCreate}\\}
\begin{lstlisting}[frame=none]
protected void onCreate(Bundle savedInstanceState)\end{lstlisting} %end signature
}%end item
\end{itemize}
}
}
\subsection{\label{edu.kit.pse17.go_app.Login.LoginHelper}Class LoginHelper}{
\hypertarget{edu.kit.pse17.go_app.Login.LoginHelper}{}\vskip .1in 
Created by tina on 18.06.17.\vskip .1in 
\subsubsection{Declaration}{
\begin{lstlisting}[frame=none]
public class LoginHelper
 extends AppCompatActivity\end{lstlisting}
\subsubsection{All known subclasses}{FirebaseLoginHelper\small{\refdefined{edu.kit.pse17.go_app.Login.FirebaseLoginHelper}}, GoLoginHelper\small{\refdefined{edu.kit.pse17.go_app.Login.GoLoginHelper}}}
\subsubsection{Field summary}{
\begin{verse}
\hyperlink{edu.kit.pse17.go_app.Login.LoginHelper.ACCOUNT_DATA_CODE}{{\bf ACCOUNT\_DATA\_CODE}} \\
\hyperlink{edu.kit.pse17.go_app.Login.LoginHelper.SIGN_IN_DATA_CODE}{{\bf SIGN\_IN\_DATA\_CODE}} \\
\end{verse}
}
\subsubsection{Constructor summary}{
\begin{verse}
\hyperlink{edu.kit.pse17.go_app.Login.LoginHelper()}{{\bf LoginHelper()}} \\
\end{verse}
}
\subsubsection{Method summary}{
\begin{verse}
\hyperlink{edu.kit.pse17.go_app.Login.LoginHelper.returnActivityResult(java.io.Serializable)}{{\bf returnActivityResult(Serializable)}} \\
\hyperlink{edu.kit.pse17.go_app.Login.LoginHelper.signIn(Activity, int, java.io.Serializable, java.lang.Class)}{{\bf signIn(Activity, int, Serializable, Class)}} \\
\end{verse}
}
\subsubsection{Fields}{
\begin{itemize}
\item{
\index{SIGN\_IN\_DATA\_CODE}
\label{edu.kit.pse17.go_app.Login.LoginHelper.SIGN_IN_DATA_CODE}\hypertarget{edu.kit.pse17.go_app.Login.LoginHelper.SIGN_IN_DATA_CODE}{\texttt{public static final java.lang.String\ {\bf  SIGN\_IN\_DATA\_CODE}}
}
}
\item{
\index{ACCOUNT\_DATA\_CODE}
\label{edu.kit.pse17.go_app.Login.LoginHelper.ACCOUNT_DATA_CODE}\hypertarget{edu.kit.pse17.go_app.Login.LoginHelper.ACCOUNT_DATA_CODE}{\texttt{public static final java.lang.String\ {\bf  ACCOUNT\_DATA\_CODE}}
}
}
\end{itemize}
}
\subsubsection{Constructors}{
\vskip -2em
\begin{itemize}
\item{ 
\index{LoginHelper()}
\hypertarget{edu.kit.pse17.go_app.Login.LoginHelper()}{{\bf  LoginHelper}\\}
\begin{lstlisting}[frame=none]
public LoginHelper()\end{lstlisting} %end signature
}%end item
\end{itemize}
}
\subsubsection{Methods}{
\vskip -2em
\begin{itemize}
\item{ 
\index{returnActivityResult(Serializable)}
\hypertarget{edu.kit.pse17.go_app.Login.LoginHelper.returnActivityResult(java.io.Serializable)}{{\bf  returnActivityResult}\\}
\begin{lstlisting}[frame=none]
protected void returnActivityResult(java.io.Serializable accountData)\end{lstlisting} %end signature
}%end item
\item{ 
\index{signIn(Activity, int, Serializable, Class)}
\hypertarget{edu.kit.pse17.go_app.Login.LoginHelper.signIn(Activity, int, java.io.Serializable, java.lang.Class)}{{\bf  signIn}\\}
\begin{lstlisting}[frame=none]
protected static void signIn(Activity activity,int requestCode,java.io.Serializable signInData,java.lang.Class loginHelper)\end{lstlisting} %end signature
}%end item
\end{itemize}
}
}
}
\section{Package edu.kit.pse17.go\_app}{
\label{edu.kit.pse17.go_app}\hypertarget{edu.kit.pse17.go_app}{}
\hskip -.05in
\hbox to \hsize{\textit{ Package Contents\hfil Page}}
\vskip .13in
\hbox{{\bf  Classes}}
\entityintro{GO}{edu.kit.pse17.go_app.GO}{Diese Klasse verwaltet GO Objekte Created by tina on 17.06.17.}
\entityintro{Group}{edu.kit.pse17.go_app.Group}{Diese Klasse verwaltet Gruppen Objekte Created by tina on 17.06.17.}
\entityintro{GroupListActivity}{edu.kit.pse17.go_app.GroupListActivity}{Hauptansicht der App.}
\entityintro{MainActivity}{edu.kit.pse17.go_app.MainActivity}{}
\entityintro{User}{edu.kit.pse17.go_app.User}{Diese Klasse verwaltet User Objekte Created by tina on 17.06.17.}
\vskip .1in
\vskip .1in
\subsection{\label{edu.kit.pse17.go_app.GO}Class GO}{
\hypertarget{edu.kit.pse17.go_app.GO}{}\vskip .1in 
Diese Klasse verwaltet GO Objekte Created by tina on 17.06.17.\vskip .1in 
\subsubsection{Declaration}{
\begin{lstlisting}[frame=none]
public class GO
 extends java.lang.Object\end{lstlisting}
\subsubsection{Constructor summary}{
\begin{verse}
\hyperlink{edu.kit.pse17.go_app.GO(java.lang.String, java.lang.String, java.util.Date, java.util.Date, Location, edu.kit.pse17.go_app.User)}{{\bf GO(String, String, Date, Date, Location, User)}} Konstruktor\\
\end{verse}
}
\subsubsection{Method summary}{
\begin{verse}
\hyperlink{edu.kit.pse17.go_app.GO.getDescription()}{{\bf getDescription()}} \\
\hyperlink{edu.kit.pse17.go_app.GO.getEnd()}{{\bf getEnd()}} \\
\hyperlink{edu.kit.pse17.go_app.GO.getLocation()}{{\bf getLocation()}} \\
\hyperlink{edu.kit.pse17.go_app.GO.getName()}{{\bf getName()}} \\
\hyperlink{edu.kit.pse17.go_app.GO.getOwner()}{{\bf getOwner()}} \\
\hyperlink{edu.kit.pse17.go_app.GO.getStart()}{{\bf getStart()}} \\
\hyperlink{edu.kit.pse17.go_app.GO.setDescription(java.lang.String)}{{\bf setDescription(String)}} \\
\hyperlink{edu.kit.pse17.go_app.GO.setEnd(java.util.Date)}{{\bf setEnd(Date)}} \\
\hyperlink{edu.kit.pse17.go_app.GO.setLocation(Location)}{{\bf setLocation(Location)}} \\
\hyperlink{edu.kit.pse17.go_app.GO.setName(java.lang.String)}{{\bf setName(String)}} \\
\hyperlink{edu.kit.pse17.go_app.GO.setOwner(edu.kit.pse17.go_app.User)}{{\bf setOwner(User)}} \\
\hyperlink{edu.kit.pse17.go_app.GO.setStart(java.util.Date)}{{\bf setStart(Date)}} \\
\end{verse}
}
\subsubsection{Constructors}{
\vskip -2em
\begin{itemize}
\item{ 
\index{GO(String, String, Date, Date, Location, User)}
\hypertarget{edu.kit.pse17.go_app.GO(java.lang.String, java.lang.String, java.util.Date, java.util.Date, Location, edu.kit.pse17.go_app.User)}{{\bf  GO}\\}
\begin{lstlisting}[frame=none]
public GO(java.lang.String name,java.lang.String description,java.util.Date start,java.util.Date end,Location location,User owner)\end{lstlisting} %end signature
\begin{itemize}
\item{
{\bf  Description}

Konstruktor
}
\item{
{\bf  Parameters}
  \begin{itemize}
   \item{
\texttt{name} -- GO Bezeichnung}
   \item{
\texttt{description} -- Go Beschreibung}
   \item{
\texttt{start} -- Startzeitpunkt}
   \item{
\texttt{end} -- Endzeitpunkt}
   \item{
\texttt{location} -- Treffpunkt (kann null sein)}
   \item{
\texttt{owner} -- GO-Verantwortlicher}
  \end{itemize}
}%end item
\end{itemize}
}%end item
\end{itemize}
}
\subsubsection{Methods}{
\vskip -2em
\begin{itemize}
\item{ 
\index{getDescription()}
\hypertarget{edu.kit.pse17.go_app.GO.getDescription()}{{\bf  getDescription}\\}
\begin{lstlisting}[frame=none]
public java.lang.String getDescription()\end{lstlisting} %end signature
}%end item
\item{ 
\index{getEnd()}
\hypertarget{edu.kit.pse17.go_app.GO.getEnd()}{{\bf  getEnd}\\}
\begin{lstlisting}[frame=none]
public java.util.Date getEnd()\end{lstlisting} %end signature
}%end item
\item{ 
\index{getLocation()}
\hypertarget{edu.kit.pse17.go_app.GO.getLocation()}{{\bf  getLocation}\\}
\begin{lstlisting}[frame=none]
public Location getLocation()\end{lstlisting} %end signature
}%end item
\item{ 
\index{getName()}
\hypertarget{edu.kit.pse17.go_app.GO.getName()}{{\bf  getName}\\}
\begin{lstlisting}[frame=none]
public java.lang.String getName()\end{lstlisting} %end signature
}%end item
\item{ 
\index{getOwner()}
\hypertarget{edu.kit.pse17.go_app.GO.getOwner()}{{\bf  getOwner}\\}
\begin{lstlisting}[frame=none]
public User getOwner()\end{lstlisting} %end signature
}%end item
\item{ 
\index{getStart()}
\hypertarget{edu.kit.pse17.go_app.GO.getStart()}{{\bf  getStart}\\}
\begin{lstlisting}[frame=none]
public java.util.Date getStart()\end{lstlisting} %end signature
}%end item
\item{ 
\index{setDescription(String)}
\hypertarget{edu.kit.pse17.go_app.GO.setDescription(java.lang.String)}{{\bf  setDescription}\\}
\begin{lstlisting}[frame=none]
public void setDescription(java.lang.String description)\end{lstlisting} %end signature
}%end item
\item{ 
\index{setEnd(Date)}
\hypertarget{edu.kit.pse17.go_app.GO.setEnd(java.util.Date)}{{\bf  setEnd}\\}
\begin{lstlisting}[frame=none]
public void setEnd(java.util.Date end)\end{lstlisting} %end signature
}%end item
\item{ 
\index{setLocation(Location)}
\hypertarget{edu.kit.pse17.go_app.GO.setLocation(Location)}{{\bf  setLocation}\\}
\begin{lstlisting}[frame=none]
public void setLocation(Location location)\end{lstlisting} %end signature
}%end item
\item{ 
\index{setName(String)}
\hypertarget{edu.kit.pse17.go_app.GO.setName(java.lang.String)}{{\bf  setName}\\}
\begin{lstlisting}[frame=none]
public void setName(java.lang.String name)\end{lstlisting} %end signature
}%end item
\item{ 
\index{setOwner(User)}
\hypertarget{edu.kit.pse17.go_app.GO.setOwner(edu.kit.pse17.go_app.User)}{{\bf  setOwner}\\}
\begin{lstlisting}[frame=none]
public void setOwner(User owner)\end{lstlisting} %end signature
}%end item
\item{ 
\index{setStart(Date)}
\hypertarget{edu.kit.pse17.go_app.GO.setStart(java.util.Date)}{{\bf  setStart}\\}
\begin{lstlisting}[frame=none]
public void setStart(java.util.Date start)\end{lstlisting} %end signature
}%end item
\end{itemize}
}
}
\subsection{\label{edu.kit.pse17.go_app.Group}Class Group}{
\hypertarget{edu.kit.pse17.go_app.Group}{}\vskip .1in 
Diese Klasse verwaltet Gruppen Objekte Created by tina on 17.06.17.\vskip .1in 
\subsubsection{Declaration}{
\begin{lstlisting}[frame=none]
public class Group
 extends java.lang.Object\end{lstlisting}
\subsubsection{Constructor summary}{
\begin{verse}
\hyperlink{edu.kit.pse17.go_app.Group(java.lang.String, java.lang.String, Icon, java.util.ArrayList, int)}{{\bf Group(String, String, Icon, ArrayList, int)}} Konstruktor\\
\end{verse}
}
\subsubsection{Method summary}{
\begin{verse}
\hyperlink{edu.kit.pse17.go_app.Group.getDescription()}{{\bf getDescription()}} \\
\hyperlink{edu.kit.pse17.go_app.Group.getIcon()}{{\bf getIcon()}} \\
\hyperlink{edu.kit.pse17.go_app.Group.getMemberCount()}{{\bf getMemberCount()}} \\
\hyperlink{edu.kit.pse17.go_app.Group.getMembers()}{{\bf getMembers()}} \\
\hyperlink{edu.kit.pse17.go_app.Group.getName()}{{\bf getName()}} \\
\hyperlink{edu.kit.pse17.go_app.Group.setDescription(java.lang.String)}{{\bf setDescription(String)}} \\
\hyperlink{edu.kit.pse17.go_app.Group.setIcon(Icon)}{{\bf setIcon(Icon)}} \\
\hyperlink{edu.kit.pse17.go_app.Group.setMemberCount(int)}{{\bf setMemberCount(int)}} \\
\hyperlink{edu.kit.pse17.go_app.Group.setMembers(java.util.ArrayList)}{{\bf setMembers(ArrayList)}} \\
\hyperlink{edu.kit.pse17.go_app.Group.setName(java.lang.String)}{{\bf setName(String)}} \\
\end{verse}
}
\subsubsection{Constructors}{
\vskip -2em
\begin{itemize}
\item{ 
\index{Group(String, String, Icon, ArrayList, int)}
\hypertarget{edu.kit.pse17.go_app.Group(java.lang.String, java.lang.String, Icon, java.util.ArrayList, int)}{{\bf  Group}\\}
\begin{lstlisting}[frame=none]
public Group(java.lang.String name,java.lang.String description,Icon icon,java.util.ArrayList members,int memberCount)\end{lstlisting} %end signature
\begin{itemize}
\item{
{\bf  Description}

Konstruktor
}
\item{
{\bf  Parameters}
  \begin{itemize}
   \item{
\texttt{name} -- Gruppenname}
   \item{
\texttt{description} -- Gruppenbeschreibung}
   \item{
\texttt{icon} -- Gruppenicon}
   \item{
\texttt{members} -- Liste aller Gruppenmitglieder}
   \item{
\texttt{memberCount} -- Anzahl der Gruppenmitglieder}
  \end{itemize}
}%end item
\end{itemize}
}%end item
\end{itemize}
}
\subsubsection{Methods}{
\vskip -2em
\begin{itemize}
\item{ 
\index{getDescription()}
\hypertarget{edu.kit.pse17.go_app.Group.getDescription()}{{\bf  getDescription}\\}
\begin{lstlisting}[frame=none]
public java.lang.String getDescription()\end{lstlisting} %end signature
}%end item
\item{ 
\index{getIcon()}
\hypertarget{edu.kit.pse17.go_app.Group.getIcon()}{{\bf  getIcon}\\}
\begin{lstlisting}[frame=none]
public Icon getIcon()\end{lstlisting} %end signature
}%end item
\item{ 
\index{getMemberCount()}
\hypertarget{edu.kit.pse17.go_app.Group.getMemberCount()}{{\bf  getMemberCount}\\}
\begin{lstlisting}[frame=none]
public int getMemberCount()\end{lstlisting} %end signature
}%end item
\item{ 
\index{getMembers()}
\hypertarget{edu.kit.pse17.go_app.Group.getMembers()}{{\bf  getMembers}\\}
\begin{lstlisting}[frame=none]
public java.util.ArrayList getMembers()\end{lstlisting} %end signature
}%end item
\item{ 
\index{getName()}
\hypertarget{edu.kit.pse17.go_app.Group.getName()}{{\bf  getName}\\}
\begin{lstlisting}[frame=none]
public java.lang.String getName()\end{lstlisting} %end signature
}%end item
\item{ 
\index{setDescription(String)}
\hypertarget{edu.kit.pse17.go_app.Group.setDescription(java.lang.String)}{{\bf  setDescription}\\}
\begin{lstlisting}[frame=none]
public void setDescription(java.lang.String description)\end{lstlisting} %end signature
}%end item
\item{ 
\index{setIcon(Icon)}
\hypertarget{edu.kit.pse17.go_app.Group.setIcon(Icon)}{{\bf  setIcon}\\}
\begin{lstlisting}[frame=none]
public void setIcon(Icon icon)\end{lstlisting} %end signature
}%end item
\item{ 
\index{setMemberCount(int)}
\hypertarget{edu.kit.pse17.go_app.Group.setMemberCount(int)}{{\bf  setMemberCount}\\}
\begin{lstlisting}[frame=none]
public void setMemberCount(int memberCount)\end{lstlisting} %end signature
}%end item
\item{ 
\index{setMembers(ArrayList)}
\hypertarget{edu.kit.pse17.go_app.Group.setMembers(java.util.ArrayList)}{{\bf  setMembers}\\}
\begin{lstlisting}[frame=none]
public void setMembers(java.util.ArrayList members)\end{lstlisting} %end signature
}%end item
\item{ 
\index{setName(String)}
\hypertarget{edu.kit.pse17.go_app.Group.setName(java.lang.String)}{{\bf  setName}\\}
\begin{lstlisting}[frame=none]
public void setName(java.lang.String name)\end{lstlisting} %end signature
}%end item
\end{itemize}
}
}
\subsection{\label{edu.kit.pse17.go_app.GroupListActivity}Class GroupListActivity}{
\hypertarget{edu.kit.pse17.go_app.GroupListActivity}{}\vskip .1in 
Hauptansicht der App. Zeigt alle Gruppen eines Benutzers\vskip .1in 
\subsubsection{Declaration}{
\begin{lstlisting}[frame=none]
public class GroupListActivity
 extends AppCompatActivity implements edu.kit.pse17.go_app.RecyclerView.OnListItemClicked\end{lstlisting}
\subsubsection{Constructor summary}{
\begin{verse}
\hyperlink{edu.kit.pse17.go_app.GroupListActivity()}{{\bf GroupListActivity()}} \\
\end{verse}
}
\subsubsection{Method summary}{
\begin{verse}
\hyperlink{edu.kit.pse17.go_app.GroupListActivity.onClick(View)}{{\bf onClick(View)}} \\
\hyperlink{edu.kit.pse17.go_app.GroupListActivity.onCreate(Bundle)}{{\bf onCreate(Bundle)}} \\
\hyperlink{edu.kit.pse17.go_app.GroupListActivity.onItemClicked(int)}{{\bf onItemClicked(int)}} \\
\hyperlink{edu.kit.pse17.go_app.GroupListActivity.start(Activity)}{{\bf start(Activity)}} starts the Activity\\
\end{verse}
}
\subsubsection{Constructors}{
\vskip -2em
\begin{itemize}
\item{ 
\index{GroupListActivity()}
\hypertarget{edu.kit.pse17.go_app.GroupListActivity()}{{\bf  GroupListActivity}\\}
\begin{lstlisting}[frame=none]
public GroupListActivity()\end{lstlisting} %end signature
}%end item
\end{itemize}
}
\subsubsection{Methods}{
\vskip -2em
\begin{itemize}
\item{ 
\index{onClick(View)}
\hypertarget{edu.kit.pse17.go_app.GroupListActivity.onClick(View)}{{\bf  onClick}\\}
\begin{lstlisting}[frame=none]
public void onClick(View v)\end{lstlisting} %end signature
}%end item
\item{ 
\index{onCreate(Bundle)}
\hypertarget{edu.kit.pse17.go_app.GroupListActivity.onCreate(Bundle)}{{\bf  onCreate}\\}
\begin{lstlisting}[frame=none]
protected void onCreate(Bundle savedInstanceState)\end{lstlisting} %end signature
}%end item
\item{ 
\index{onItemClicked(int)}
\hypertarget{edu.kit.pse17.go_app.GroupListActivity.onItemClicked(int)}{{\bf  onItemClicked}\\}
\begin{lstlisting}[frame=none]
void onItemClicked(int position)\end{lstlisting} %end signature
}%end item
\item{ 
\index{start(Activity)}
\hypertarget{edu.kit.pse17.go_app.GroupListActivity.start(Activity)}{{\bf  start}\\}
\begin{lstlisting}[frame=none]
public void start(Activity activity)\end{lstlisting} %end signature
\begin{itemize}
\item{
{\bf  Description}

starts the Activity
}
\item{
{\bf  Parameters}
  \begin{itemize}
   \item{
\texttt{activity} -- Activity from which the groupListActivity is started}
  \end{itemize}
}%end item
\end{itemize}
}%end item
\end{itemize}
}
}
\subsection{\label{edu.kit.pse17.go_app.MainActivity}Class MainActivity}{
\hypertarget{edu.kit.pse17.go_app.MainActivity}{}\vskip .1in 
\subsubsection{Declaration}{
\begin{lstlisting}[frame=none]
public class MainActivity
 extends AppCompatActivity\end{lstlisting}
\subsubsection{Constructor summary}{
\begin{verse}
\hyperlink{edu.kit.pse17.go_app.MainActivity()}{{\bf MainActivity()}} \\
\end{verse}
}
\subsubsection{Method summary}{
\begin{verse}
\hyperlink{edu.kit.pse17.go_app.MainActivity.onCreate(Bundle)}{{\bf onCreate(Bundle)}} \\
\end{verse}
}
\subsubsection{Constructors}{
\vskip -2em
\begin{itemize}
\item{ 
\index{MainActivity()}
\hypertarget{edu.kit.pse17.go_app.MainActivity()}{{\bf  MainActivity}\\}
\begin{lstlisting}[frame=none]
public MainActivity()\end{lstlisting} %end signature
}%end item
\end{itemize}
}
\subsubsection{Methods}{
\vskip -2em
\begin{itemize}
\item{ 
\index{onCreate(Bundle)}
\hypertarget{edu.kit.pse17.go_app.MainActivity.onCreate(Bundle)}{{\bf  onCreate}\\}
\begin{lstlisting}[frame=none]
protected void onCreate(Bundle savedInstanceState)\end{lstlisting} %end signature
}%end item
\end{itemize}
}
}
\subsection{\label{edu.kit.pse17.go_app.User}Class User}{
\hypertarget{edu.kit.pse17.go_app.User}{}\vskip .1in 
Diese Klasse verwaltet User Objekte Created by tina on 17.06.17.\vskip .1in 
\subsubsection{Declaration}{
\begin{lstlisting}[frame=none]
public class User
 extends java.lang.Object implements java.io.Serializable\end{lstlisting}
\subsubsection{Constructor summary}{
\begin{verse}
\hyperlink{edu.kit.pse17.go_app.User(java.lang.String, java.lang.String, java.lang.String, Icon)}{{\bf User(String, String, String, Icon)}} Konstruktor\\
\end{verse}
}
\subsubsection{Constructors}{
\vskip -2em
\begin{itemize}
\item{ 
\index{User(String, String, String, Icon)}
\hypertarget{edu.kit.pse17.go_app.User(java.lang.String, java.lang.String, java.lang.String, Icon)}{{\bf  User}\\}
\begin{lstlisting}[frame=none]
public User(java.lang.String uid,java.lang.String name,java.lang.String email,Icon icon)\end{lstlisting} %end signature
\begin{itemize}
\item{
{\bf  Description}

Konstruktor
}
\item{
{\bf  Parameters}
  \begin{itemize}
   \item{
\texttt{uid} -- User-ID (--\textgreater  übernommen von FirebaseUser-Objekt aus der FirebaseAPI (eindeutig)}
   \item{
\texttt{name} -- Benutzername}
   \item{
\texttt{email} -- E-Mailadresse, die bei der Anmeldung verwendet wurde. Wird verwendet, um User nach anderen Usern suchen zu lassen}
   \item{
\texttt{icon} -- Profilbild}
  \end{itemize}
}%end item
\end{itemize}
}%end item
\end{itemize}
}
}
}
\section{Package edu.kit.pse17.go\_app.RecyclerView}{
\label{edu.kit.pse17.go_app.RecyclerView}\hypertarget{edu.kit.pse17.go_app.RecyclerView}{}
\hskip -.05in
\hbox to \hsize{\textit{ Package Contents\hfil Page}}
\vskip .13in
\hbox{{\bf  Interfaces}}
\entityintro{ListItem}{edu.kit.pse17.go_app.RecyclerView.ListItem}{Created by tina on 18.06.17.}
\entityintro{OnListItemClicked}{edu.kit.pse17.go_app.RecyclerView.OnListItemClicked}{Created by tina on 17.06.17.}
\vskip .13in
\hbox{{\bf  Classes}}
\entityintro{ListAdapter}{edu.kit.pse17.go_app.RecyclerView.ListAdapter}{Created by tina on 17.06.17.}
\entityintro{ListViewHolder}{edu.kit.pse17.go_app.RecyclerView.ListViewHolder}{Die Klasse erzeugt ViewHolder-Objekte, die die Datenobjekt für die RecyclerView enthalten Created by tina on 17.06.17.}
\vskip .1in
\vskip .1in
\subsection{\label{edu.kit.pse17.go_app.RecyclerView.ListItem}Interface ListItem}{
\hypertarget{edu.kit.pse17.go_app.RecyclerView.ListItem}{}\vskip .1in 
Created by tina on 18.06.17.\vskip .1in 
\subsubsection{Declaration}{
\begin{lstlisting}[frame=none]
public interface ListItem
\end{lstlisting}
\subsubsection{All known subinterfaces}{GroupListItem\small{\refdefined{edu.kit.pse17.go_app.RecyclerView.GroupRecyclerView.GroupListItem}}, GOListItem\small{\refdefined{edu.kit.pse17.go_app.RecyclerView.GORecyclerView.GOListItem}}}
\subsubsection{All classes known to implement interface}{GroupListItem\small{\refdefined{edu.kit.pse17.go_app.RecyclerView.GroupRecyclerView.GroupListItem}}, GOListItem\small{\refdefined{edu.kit.pse17.go_app.RecyclerView.GORecyclerView.GOListItem}}}
\subsubsection{Method summary}{
\begin{verse}
\hyperlink{edu.kit.pse17.go_app.RecyclerView.ListItem.getIcon()}{{\bf getIcon()}} \\
\hyperlink{edu.kit.pse17.go_app.RecyclerView.ListItem.getSubtitle()}{{\bf getSubtitle()}} \\
\hyperlink{edu.kit.pse17.go_app.RecyclerView.ListItem.getTitle()}{{\bf getTitle()}} \\
\hyperlink{edu.kit.pse17.go_app.RecyclerView.ListItem.setIcon(Icon)}{{\bf setIcon(Icon)}} \\
\hyperlink{edu.kit.pse17.go_app.RecyclerView.ListItem.setSubtitle(T)}{{\bf setSubtitle(T)}} \\
\hyperlink{edu.kit.pse17.go_app.RecyclerView.ListItem.setTitle(java.lang.String)}{{\bf setTitle(String)}} \\
\end{verse}
}
\subsubsection{Methods}{
\vskip -2em
\begin{itemize}
\item{ 
\index{getIcon()}
\hypertarget{edu.kit.pse17.go_app.RecyclerView.ListItem.getIcon()}{{\bf  getIcon}\\}
\begin{lstlisting}[frame=none]
Icon getIcon()\end{lstlisting} %end signature
}%end item
\item{ 
\index{getSubtitle()}
\hypertarget{edu.kit.pse17.go_app.RecyclerView.ListItem.getSubtitle()}{{\bf  getSubtitle}\\}
\begin{lstlisting}[frame=none]
java.lang.String getSubtitle()\end{lstlisting} %end signature
}%end item
\item{ 
\index{getTitle()}
\hypertarget{edu.kit.pse17.go_app.RecyclerView.ListItem.getTitle()}{{\bf  getTitle}\\}
\begin{lstlisting}[frame=none]
java.lang.String getTitle()\end{lstlisting} %end signature
}%end item
\item{ 
\index{setIcon(Icon)}
\hypertarget{edu.kit.pse17.go_app.RecyclerView.ListItem.setIcon(Icon)}{{\bf  setIcon}\\}
\begin{lstlisting}[frame=none]
void setIcon(Icon icon)\end{lstlisting} %end signature
}%end item
\item{ 
\index{setSubtitle(T)}
\hypertarget{edu.kit.pse17.go_app.RecyclerView.ListItem.setSubtitle(T)}{{\bf  setSubtitle}\\}
\begin{lstlisting}[frame=none]
void setSubtitle(java.lang.Object t)\end{lstlisting} %end signature
}%end item
\item{ 
\index{setTitle(String)}
\hypertarget{edu.kit.pse17.go_app.RecyclerView.ListItem.setTitle(java.lang.String)}{{\bf  setTitle}\\}
\begin{lstlisting}[frame=none]
void setTitle(java.lang.String title)\end{lstlisting} %end signature
}%end item
\end{itemize}
}
}
\subsection{\label{edu.kit.pse17.go_app.RecyclerView.OnListItemClicked}Interface OnListItemClicked}{
\hypertarget{edu.kit.pse17.go_app.RecyclerView.OnListItemClicked}{}\vskip .1in 
Created by tina on 17.06.17.\vskip .1in 
\subsubsection{Declaration}{
\begin{lstlisting}[frame=none]
public interface OnListItemClicked
\end{lstlisting}
\subsubsection{All known subinterfaces}{GroupListActivity\small{\refdefined{edu.kit.pse17.go_app.GroupListActivity}}}
\subsubsection{All classes known to implement interface}{GroupListActivity\small{\refdefined{edu.kit.pse17.go_app.GroupListActivity}}}
\subsubsection{Method summary}{
\begin{verse}
\hyperlink{edu.kit.pse17.go_app.RecyclerView.OnListItemClicked.onItemClicked(int)}{{\bf onItemClicked(int)}} \\
\end{verse}
}
\subsubsection{Methods}{
\vskip -2em
\begin{itemize}
\item{ 
\index{onItemClicked(int)}
\hypertarget{edu.kit.pse17.go_app.RecyclerView.OnListItemClicked.onItemClicked(int)}{{\bf  onItemClicked}\\}
\begin{lstlisting}[frame=none]
void onItemClicked(int position)\end{lstlisting} %end signature
}%end item
\end{itemize}
}
}
\subsection{\label{edu.kit.pse17.go_app.RecyclerView.ListAdapter}Class ListAdapter}{
\hypertarget{edu.kit.pse17.go_app.RecyclerView.ListAdapter}{}\vskip .1in 
Created by tina on 17.06.17.\vskip .1in 
\subsubsection{Declaration}{
\begin{lstlisting}[frame=none]
public class ListAdapter
 extends <any>\end{lstlisting}
\subsubsection{Constructor summary}{
\begin{verse}
\hyperlink{edu.kit.pse17.go_app.RecyclerView.ListAdapter(java.util.List, edu.kit.pse17.go_app.RecyclerView.OnListItemClicked)}{{\bf ListAdapter(List, OnListItemClicked)}} \\
\end{verse}
}
\subsubsection{Method summary}{
\begin{verse}
\hyperlink{edu.kit.pse17.go_app.RecyclerView.ListAdapter.getItem(int)}{{\bf getItem(int)}} \\
\hyperlink{edu.kit.pse17.go_app.RecyclerView.ListAdapter.getItemCount()}{{\bf getItemCount()}} \\
\hyperlink{edu.kit.pse17.go_app.RecyclerView.ListAdapter.onBindViewHolder(edu.kit.pse17.go_app.RecyclerView.ListViewHolder, int)}{{\bf onBindViewHolder(ListViewHolder, int)}} \\
\hyperlink{edu.kit.pse17.go_app.RecyclerView.ListAdapter.onCreateViewHolder(ViewGroup, int)}{{\bf onCreateViewHolder(ViewGroup, int)}} \\
\end{verse}
}
\subsubsection{Constructors}{
\vskip -2em
\begin{itemize}
\item{ 
\index{ListAdapter(List, OnListItemClicked)}
\hypertarget{edu.kit.pse17.go_app.RecyclerView.ListAdapter(java.util.List, edu.kit.pse17.go_app.RecyclerView.OnListItemClicked)}{{\bf  ListAdapter}\\}
\begin{lstlisting}[frame=none]
public ListAdapter(java.util.List data,OnListItemClicked onListItemClicked)\end{lstlisting} %end signature
}%end item
\end{itemize}
}
\subsubsection{Methods}{
\vskip -2em
\begin{itemize}
\item{ 
\index{getItem(int)}
\hypertarget{edu.kit.pse17.go_app.RecyclerView.ListAdapter.getItem(int)}{{\bf  getItem}\\}
\begin{lstlisting}[frame=none]
public ListItem getItem(int position)\end{lstlisting} %end signature
}%end item
\item{ 
\index{getItemCount()}
\hypertarget{edu.kit.pse17.go_app.RecyclerView.ListAdapter.getItemCount()}{{\bf  getItemCount}\\}
\begin{lstlisting}[frame=none]
public int getItemCount()\end{lstlisting} %end signature
}%end item
\item{ 
\index{onBindViewHolder(ListViewHolder, int)}
\hypertarget{edu.kit.pse17.go_app.RecyclerView.ListAdapter.onBindViewHolder(edu.kit.pse17.go_app.RecyclerView.ListViewHolder, int)}{{\bf  onBindViewHolder}\\}
\begin{lstlisting}[frame=none]
public void onBindViewHolder(ListViewHolder holder,int position)\end{lstlisting} %end signature
}%end item
\item{ 
\index{onCreateViewHolder(ViewGroup, int)}
\hypertarget{edu.kit.pse17.go_app.RecyclerView.ListAdapter.onCreateViewHolder(ViewGroup, int)}{{\bf  onCreateViewHolder}\\}
\begin{lstlisting}[frame=none]
public ListViewHolder onCreateViewHolder(ViewGroup parent,int viewType)\end{lstlisting} %end signature
}%end item
\end{itemize}
}
}
\subsection{\label{edu.kit.pse17.go_app.RecyclerView.ListViewHolder}Class ListViewHolder}{
\hypertarget{edu.kit.pse17.go_app.RecyclerView.ListViewHolder}{}\vskip .1in 
Die Klasse erzeugt ViewHolder-Objekte, die die Datenobjekt für die RecyclerView enthalten Created by tina on 17.06.17.\vskip .1in 
\subsubsection{Declaration}{
\begin{lstlisting}[frame=none]
public class ListViewHolder
 extends RecyclerView.ViewHolder\end{lstlisting}
\subsubsection{Field summary}{
\begin{verse}
\hyperlink{edu.kit.pse17.go_app.RecyclerView.ListViewHolder.icon}{{\bf icon}} Icon, das zum Item angezeigt werden soll\\
\hyperlink{edu.kit.pse17.go_app.RecyclerView.ListViewHolder.subtitle}{{\bf subtitle}} Untertitel des Items\\
\hyperlink{edu.kit.pse17.go_app.RecyclerView.ListViewHolder.title}{{\bf title}} Titel des Items\\
\end{verse}
}
\subsubsection{Constructor summary}{
\begin{verse}
\hyperlink{edu.kit.pse17.go_app.RecyclerView.ListViewHolder(View, edu.kit.pse17.go_app.RecyclerView.OnListItemClicked)}{{\bf ListViewHolder(View, OnListItemClicked)}} \\
\end{verse}
}
\subsubsection{Method summary}{
\begin{verse}
\hyperlink{edu.kit.pse17.go_app.RecyclerView.ListViewHolder.onClick(View)}{{\bf onClick(View)}} \\
\end{verse}
}
\subsubsection{Fields}{
\begin{itemize}
\item{
\index{title}
\label{edu.kit.pse17.go_app.RecyclerView.ListViewHolder.title}\hypertarget{edu.kit.pse17.go_app.RecyclerView.ListViewHolder.title}{\texttt{public TextView\ {\bf  title}}
}
\begin{itemize}
\item{\vskip -.9ex 
Titel des Items}
\end{itemize}
}
\item{
\index{subtitle}
\label{edu.kit.pse17.go_app.RecyclerView.ListViewHolder.subtitle}\hypertarget{edu.kit.pse17.go_app.RecyclerView.ListViewHolder.subtitle}{\texttt{public TextView\ {\bf  subtitle}}
}
\begin{itemize}
\item{\vskip -.9ex 
Untertitel des Items}
\end{itemize}
}
\item{
\index{icon}
\label{edu.kit.pse17.go_app.RecyclerView.ListViewHolder.icon}\hypertarget{edu.kit.pse17.go_app.RecyclerView.ListViewHolder.icon}{\texttt{public ImageView\ {\bf  icon}}
}
\begin{itemize}
\item{\vskip -.9ex 
Icon, das zum Item angezeigt werden soll}
\end{itemize}
}
\end{itemize}
}
\subsubsection{Constructors}{
\vskip -2em
\begin{itemize}
\item{ 
\index{ListViewHolder(View, OnListItemClicked)}
\hypertarget{edu.kit.pse17.go_app.RecyclerView.ListViewHolder(View, edu.kit.pse17.go_app.RecyclerView.OnListItemClicked)}{{\bf  ListViewHolder}\\}
\begin{lstlisting}[frame=none]
public ListViewHolder(View itemView,OnListItemClicked onListItemClicked)\end{lstlisting} %end signature
\begin{itemize}
\item{
{\bf  Parameters}
  \begin{itemize}
   \item{
\texttt{itemView} -- View, in der die Items angezeigt werden sollen}
   \item{
\texttt{onListItemClicked} -- ClickListener für ListItems}
  \end{itemize}
}%end item
\end{itemize}
}%end item
\end{itemize}
}
\subsubsection{Methods}{
\vskip -2em
\begin{itemize}
\item{ 
\index{onClick(View)}
\hypertarget{edu.kit.pse17.go_app.RecyclerView.ListViewHolder.onClick(View)}{{\bf  onClick}\\}
\begin{lstlisting}[frame=none]
public void onClick(View v)\end{lstlisting} %end signature
}%end item
\end{itemize}
}
}
}
\section{Package edu.kit.pse17.go\_app.RecyclerView.GroupRecyclerView}{
\label{edu.kit.pse17.go_app.RecyclerView.GroupRecyclerView}\hypertarget{edu.kit.pse17.go_app.RecyclerView.GroupRecyclerView}{}
\hskip -.05in
\hbox to \hsize{\textit{ Package Contents\hfil Page}}
\vskip .13in
\hbox{{\bf  Classes}}
\entityintro{GroupListItem}{edu.kit.pse17.go_app.RecyclerView.GroupRecyclerView.GroupListItem}{This class represents ListItems that display information about a group to be displayed in a RecyclerView Created by tina on 17.06.17.}
\vskip .1in
\vskip .1in
\subsection{\label{edu.kit.pse17.go_app.RecyclerView.GroupRecyclerView.GroupListItem}Class GroupListItem}{
\hypertarget{edu.kit.pse17.go_app.RecyclerView.GroupRecyclerView.GroupListItem}{}\vskip .1in 
This class represents ListItems that display information about a group to be displayed in a RecyclerView Created by tina on 17.06.17.\vskip .1in 
\subsubsection{Declaration}{
\begin{lstlisting}[frame=none]
public class GroupListItem
 extends java.lang.Object implements edu.kit.pse17.go_app.RecyclerView.ListItem\end{lstlisting}
\subsubsection{Constructor summary}{
\begin{verse}
\hyperlink{edu.kit.pse17.go_app.RecyclerView.GroupRecyclerView.GroupListItem()}{{\bf GroupListItem()}} \\
\end{verse}
}
\subsubsection{Method summary}{
\begin{verse}
\hyperlink{edu.kit.pse17.go_app.RecyclerView.GroupRecyclerView.GroupListItem.getIcon()}{{\bf getIcon()}} \\
\hyperlink{edu.kit.pse17.go_app.RecyclerView.GroupRecyclerView.GroupListItem.getSubtitle()}{{\bf getSubtitle()}} \\
\hyperlink{edu.kit.pse17.go_app.RecyclerView.GroupRecyclerView.GroupListItem.getTitle()}{{\bf getTitle()}} \\
\hyperlink{edu.kit.pse17.go_app.RecyclerView.GroupRecyclerView.GroupListItem.setIcon(Icon)}{{\bf setIcon(Icon)}} \\
\hyperlink{edu.kit.pse17.go_app.RecyclerView.GroupRecyclerView.GroupListItem.setSubtitle(java.lang.Integer)}{{\bf setSubtitle(Integer)}} \\
\hyperlink{edu.kit.pse17.go_app.RecyclerView.GroupRecyclerView.GroupListItem.setTitle(java.lang.String)}{{\bf setTitle(String)}} \\
\end{verse}
}
\subsubsection{Constructors}{
\vskip -2em
\begin{itemize}
\item{ 
\index{GroupListItem()}
\hypertarget{edu.kit.pse17.go_app.RecyclerView.GroupRecyclerView.GroupListItem()}{{\bf  GroupListItem}\\}
\begin{lstlisting}[frame=none]
public GroupListItem()\end{lstlisting} %end signature
}%end item
\end{itemize}
}
\subsubsection{Methods}{
\vskip -2em
\begin{itemize}
\item{ 
\index{getIcon()}
\hypertarget{edu.kit.pse17.go_app.RecyclerView.GroupRecyclerView.GroupListItem.getIcon()}{{\bf  getIcon}\\}
\begin{lstlisting}[frame=none]
Icon getIcon()\end{lstlisting} %end signature
}%end item
\item{ 
\index{getSubtitle()}
\hypertarget{edu.kit.pse17.go_app.RecyclerView.GroupRecyclerView.GroupListItem.getSubtitle()}{{\bf  getSubtitle}\\}
\begin{lstlisting}[frame=none]
java.lang.String getSubtitle()\end{lstlisting} %end signature
}%end item
\item{ 
\index{getTitle()}
\hypertarget{edu.kit.pse17.go_app.RecyclerView.GroupRecyclerView.GroupListItem.getTitle()}{{\bf  getTitle}\\}
\begin{lstlisting}[frame=none]
java.lang.String getTitle()\end{lstlisting} %end signature
}%end item
\item{ 
\index{setIcon(Icon)}
\hypertarget{edu.kit.pse17.go_app.RecyclerView.GroupRecyclerView.GroupListItem.setIcon(Icon)}{{\bf  setIcon}\\}
\begin{lstlisting}[frame=none]
void setIcon(Icon icon)\end{lstlisting} %end signature
}%end item
\item{ 
\index{setSubtitle(Integer)}
\hypertarget{edu.kit.pse17.go_app.RecyclerView.GroupRecyclerView.GroupListItem.setSubtitle(java.lang.Integer)}{{\bf  setSubtitle}\\}
\begin{lstlisting}[frame=none]
public void setSubtitle(java.lang.Integer memberCount)\end{lstlisting} %end signature
}%end item
\item{ 
\index{setTitle(String)}
\hypertarget{edu.kit.pse17.go_app.RecyclerView.GroupRecyclerView.GroupListItem.setTitle(java.lang.String)}{{\bf  setTitle}\\}
\begin{lstlisting}[frame=none]
void setTitle(java.lang.String title)\end{lstlisting} %end signature
}%end item
\end{itemize}
}
}
}
\section{Package edu.kit.pse17.go\_app.RecyclerView.GORecyclerView}{
\label{edu.kit.pse17.go_app.RecyclerView.GORecyclerView}\hypertarget{edu.kit.pse17.go_app.RecyclerView.GORecyclerView}{}
\hskip -.05in
\hbox to \hsize{\textit{ Package Contents\hfil Page}}
\vskip .13in
\hbox{{\bf  Classes}}
\entityintro{GOListItem}{edu.kit.pse17.go_app.RecyclerView.GORecyclerView.GOListItem}{This class represents ListItems that display information about a GO to be displayed in a RecyclerView Created by tina on 17.06.17.}
\vskip .1in
\vskip .1in
\subsection{\label{edu.kit.pse17.go_app.RecyclerView.GORecyclerView.GOListItem}Class GOListItem}{
\hypertarget{edu.kit.pse17.go_app.RecyclerView.GORecyclerView.GOListItem}{}\vskip .1in 
This class represents ListItems that display information about a GO to be displayed in a RecyclerView Created by tina on 17.06.17.\vskip .1in 
\subsubsection{Declaration}{
\begin{lstlisting}[frame=none]
public class GOListItem
 extends java.lang.Object implements edu.kit.pse17.go_app.RecyclerView.ListItem\end{lstlisting}
\subsubsection{Constructor summary}{
\begin{verse}
\hyperlink{edu.kit.pse17.go_app.RecyclerView.GORecyclerView.GOListItem()}{{\bf GOListItem()}} \\
\end{verse}
}
\subsubsection{Method summary}{
\begin{verse}
\hyperlink{edu.kit.pse17.go_app.RecyclerView.GORecyclerView.GOListItem.getIcon()}{{\bf getIcon()}} \\
\hyperlink{edu.kit.pse17.go_app.RecyclerView.GORecyclerView.GOListItem.getSubtitle()}{{\bf getSubtitle()}} \\
\hyperlink{edu.kit.pse17.go_app.RecyclerView.GORecyclerView.GOListItem.getTitle()}{{\bf getTitle()}} \\
\hyperlink{edu.kit.pse17.go_app.RecyclerView.GORecyclerView.GOListItem.setIcon(Icon)}{{\bf setIcon(Icon)}} \\
\hyperlink{edu.kit.pse17.go_app.RecyclerView.GORecyclerView.GOListItem.setSubtitle(java.util.Date)}{{\bf setSubtitle(Date)}} \\
\hyperlink{edu.kit.pse17.go_app.RecyclerView.GORecyclerView.GOListItem.setTitle(java.lang.String)}{{\bf setTitle(String)}} \\
\end{verse}
}
\subsubsection{Constructors}{
\vskip -2em
\begin{itemize}
\item{ 
\index{GOListItem()}
\hypertarget{edu.kit.pse17.go_app.RecyclerView.GORecyclerView.GOListItem()}{{\bf  GOListItem}\\}
\begin{lstlisting}[frame=none]
public GOListItem()\end{lstlisting} %end signature
}%end item
\end{itemize}
}
\subsubsection{Methods}{
\vskip -2em
\begin{itemize}
\item{ 
\index{getIcon()}
\hypertarget{edu.kit.pse17.go_app.RecyclerView.GORecyclerView.GOListItem.getIcon()}{{\bf  getIcon}\\}
\begin{lstlisting}[frame=none]
Icon getIcon()\end{lstlisting} %end signature
}%end item
\item{ 
\index{getSubtitle()}
\hypertarget{edu.kit.pse17.go_app.RecyclerView.GORecyclerView.GOListItem.getSubtitle()}{{\bf  getSubtitle}\\}
\begin{lstlisting}[frame=none]
java.lang.String getSubtitle()\end{lstlisting} %end signature
}%end item
\item{ 
\index{getTitle()}
\hypertarget{edu.kit.pse17.go_app.RecyclerView.GORecyclerView.GOListItem.getTitle()}{{\bf  getTitle}\\}
\begin{lstlisting}[frame=none]
java.lang.String getTitle()\end{lstlisting} %end signature
}%end item
\item{ 
\index{setIcon(Icon)}
\hypertarget{edu.kit.pse17.go_app.RecyclerView.GORecyclerView.GOListItem.setIcon(Icon)}{{\bf  setIcon}\\}
\begin{lstlisting}[frame=none]
void setIcon(Icon icon)\end{lstlisting} %end signature
}%end item
\item{ 
\index{setSubtitle(Date)}
\hypertarget{edu.kit.pse17.go_app.RecyclerView.GORecyclerView.GOListItem.setSubtitle(java.util.Date)}{{\bf  setSubtitle}\\}
\begin{lstlisting}[frame=none]
public void setSubtitle(java.util.Date date)\end{lstlisting} %end signature
}%end item
\item{ 
\index{setTitle(String)}
\hypertarget{edu.kit.pse17.go_app.RecyclerView.GORecyclerView.GOListItem.setTitle(java.lang.String)}{{\bf  setTitle}\\}
\begin{lstlisting}[frame=none]
void setTitle(java.lang.String title)\end{lstlisting} %end signature
}%end item
\end{itemize}
}
}
}
% ------- textdoclet_include/finish.tex

% add something here

% closing for \chapter{TeXDoclet Java Documentation} {
}

\chapter{Datenbank}

\chapter{Klassendiagramme}

\chapter{Sequenzdiagramme}

\chapter{Finish}{
Lorem ipsum dolor sit amet, consetetur sadipscing elitr, sed diam nonumy eirmod tempor invidunt ut labore et dolore magna aliquyam erat, sed diam voluptua. At vero eos et accusam et justo duo dolores et ea rebum. Stet clita kasd gubergren, no sea takimata sanctus est Lorem ipsum dolor sit amet. Lorem ipsum dolor sit amet, consetetur sadipscing elitr, sed diam nonumy eirmod tempor invidunt ut labore et dolore magna aliquyam erat, sed diam voluptua. At vero eos et accusam et justo duo dolores et ea rebum. Stet clita kasd gubergren, no sea takimata sanctus est Lorem ipsum dolor sit amet.

}
% ------- textdoclet_include/finish.tex end

\end{document}
