\documentclass[11pt,a4paper]{report}
\usepackage{color}
\usepackage{ifthen}
\usepackage{ifpdf}
\usepackage[headings]{fullpage}
\usepackage{listings}
\lstset{language=Java,breaklines=true}
\ifpdf \usepackage[pdftex, pdfpagemode={UseOutlines},bookmarks,colorlinks,linkcolor={blue},plainpages=false,pdfpagelabels,citecolor={red},breaklinks=true]{hyperref}
  \usepackage[pdftex]{graphicx}
  \pdfcompresslevel=9
  \DeclareGraphicsRule{*}{mps}{*}{}
\else
  \usepackage[dvips]{graphicx}
\fi

\newcommand{\entityintro}[3]{%
  \hbox to \hsize{%
    \vbox{%
      \hbox to .2in{}%
    }%
    {\bf  #1}%
    \dotfill\pageref{#2}%
  }
  \makebox[\hsize]{%
    \parbox{.4in}{}%
    \parbox[l]{5in}{%
      \vspace{1mm}%
      #3%
      \vspace{1mm}%
    }%
  }%
}
\newcommand{\refdefined}[1]{
\expandafter\ifx\csname r@#1\endcsname\relax
\relax\else
{$($in \ref{#1}, page \pageref{#1}$)$}\fi}
\date{\today}
\chardef\textbackslash=`\\
\usepackage{pdfpages}
\usepackage[utf8]{inputenc}
\usepackage[T1]{fontenc}
\usepackage[german]{babel}
\usepackage{hyperref}
\hypersetup{
	pdftitle={Pflichtenheft},
	bookmarks=true,
}
\usepackage{csquotes}

\usepackage{fancyhdr}%<-------------to control headers and footers
\usepackage[a4paper,margin=1in,footskip=.25in]{geometry}
\fancyhf{}
\fancyfoot[C]{\thepage} %<----to get page number below text
\pagestyle{fancy} %<-------the page style itself

\usepackage{xcolor}
\usepackage{framed}
\definecolor{shadecolor}{RGB}{220,220,220}
\usepackage{float}


\title{Android GO! App - Pflichtenheft}
\author{Gruppe 3}
\date{11.06.17}

% define custom lists
\usepackage{enumitem}
\usepackage{lipsum}

\begin{document}

\begin{titlepage}
	\begin{center}
	{\scshape\LARGE \bfseries Entwurfsdokument \par}
	\vspace{1cm}
	{\scshape\Large Praktikum der Softwareentwicklung \\ Sommersemester 2017\par}
	\vspace{1.5cm}
	{\huge\bfseries Android GO! App\par}
	\vspace{2cm}
	{\Large\itshape - Gruppe 3 -\par}
	\vfill
	{\bfseries erstellt von:\par}
	Arsenii Dunaev \\
	Florian Kröger \\
	Tina Maria Strößner \\
	Volodymyr Shpylka \\	
	\vfill
	% Bottom of the page
	{\large 09.07.17 \par}	
	\end{center}
\end{titlepage}

\begin{abstract}
Die Android App GO! ist eine mobile Applikation, die speziell zur Organisation von Treffen (z. B. gemeinsames Essen im Café oder in der Mensa) entwickelt wird. Beim erfolgreichen gemeinsamen Losgehen wird der gemittelte GPS-Standort von Mitgliedern der Gruppe angezeigt.\\

Dieses Dokument erläutert den Entwurf des Systems auf der Grundlage des Pflichtenhefts.
\end{abstract}

% ------- textdoclet_include/setup.tex end

\sloppy
\addtocontents{toc}{\protect\markboth{Contents}{Contents}}
\tableofcontents


% add something here

\chapter{Änderungen zum Pflichtenheft} {

Lorem ipsum dolor sit amet, consetetur sadipscing elitr, sed diam nonumy eirmod tempor invidunt ut labore et dolore magna aliquyam erat, sed diam voluptua. At vero eos et accusam et justo duo dolores et ea rebum. Stet clita kasd gubergren, no sea takimata sanctus est Lorem ipsum dolor sit amet. Lorem ipsum dolor sit amet, consetetur sadipscing elitr, sed diam nonumy eirmod tempor invidunt ut labore et dolore magna aliquyam erat, sed diam voluptua. At vero eos et accusam et justo duo dolores et ea rebum. Stet clita kasd gubergren, no sea takimata sanctus est Lorem ipsum dolor sit amet.
}
\chapter{Paketstruktur} {

Lorem ipsum dolor sit amet, consetetur sadipscing elitr, sed diam nonumy eirmod tempor invidunt ut labore et dolore magna aliquyam erat, sed diam voluptua. At vero eos et accusam et justo duo dolores et ea rebum. Stet clita kasd gubergren, no sea takimata sanctus est Lorem ipsum dolor sit amet. Lorem ipsum dolor sit amet, consetetur sadipscing elitr, sed diam nonumy eirmod tempor invidunt ut labore et dolore magna aliquyam erat, sed diam voluptua. At vero eos et accusam et justo duo dolores et ea rebum. Stet clita kasd gubergren, no sea takimata sanctus est Lorem ipsum dolor sit amet.

Lorem ipsum dolor sit amet, consetetur sadipscing elitr, sed diam nonumy eirmod tempor invidunt ut labore et dolore magna aliquyam erat, sed diam voluptua. At vero eos et accusam et justo duo dolores et ea rebum. Stet clita kasd gubergren, no sea takimata sanctus est Lorem ipsum dolor sit amet. Lorem ipsum dolor sit amet, consetetur sadipscing elitr, sed diam nonumy eirmod tempor invidunt ut labore et dolore magna aliquyam erat, sed diam voluptua. At vero eos et accusam et justo duo dolores et ea rebum. Stet clita kasd gubergren, no sea takimata sanctus est Lorem ipsum dolor sit amet.

Lorem ipsum dolor sit amet, consetetur sadipscing elitr, sed diam nonumy eirmod tempor invidunt ut labore et dolore magna aliquyam erat, sed diam voluptua. At vero eos et accusam et justo duo dolores et ea rebum. Stet clita kasd gubergren, no sea takimata sanctus est Lorem ipsum dolor sit amet. Lorem ipsum dolor sit amet, consetetur sadipscing elitr, sed diam nonumy eirmod tempor invidunt ut labore et dolore magna aliquyam erat, sed diam voluptua. At vero eos et accusam et justo duo dolores et ea rebum. Stet clita kasd gubergren, no sea takimata sanctus est Lorem ipsum dolor sit amet.
}

\chapter{Klassenbeschreibungen} {

% ------- textdoclet_include/intro.tex end

\section*{Class Hierarchy}{
\thispagestyle{empty}
\markboth{Class Hierarchy}{Class Hierarchy}
\addcontentsline{toc}{section}{Class Hierarchy}
\subsection*{Classes}
{\raggedright
\hspace{0.0cm} $\bullet$ java.lang.Object {\tiny \refdefined{java.lang.Object}} \\
\hspace{1.0cm} $\bullet$ edu.kit.informatik.sdq.pse.go.Test {\tiny \refdefined{edu.kit.informatik.sdq.pse.go.Test}} \\
}
}
\section{Package edu.kit.informatik.sdq.pse.go}{
\label{edu.kit.informatik.sdq.pse.go}\hypertarget{edu.kit.informatik.sdq.pse.go}{}
\hskip -.05in
\hbox to \hsize{\textit{ Package Contents\hfil Page}}
\vskip .13in
\hbox{{\bf  Classes}}
\entityintro{Test}{edu.kit.informatik.sdq.pse.go.Test}{Created by tina on 17.06.17.}
\vskip .1in
\vskip .1in
\subsection{\label{edu.kit.informatik.sdq.pse.go.Test}Class Test}{
\hypertarget{edu.kit.informatik.sdq.pse.go.Test}{}\vskip .1in 
Created by tina on 17.06.17. Test class for TexDoclet\vskip .1in 
\subsubsection{Declaration}{
\begin{lstlisting}[frame=none]
public class Test
 extends java.lang.Object\end{lstlisting}
\subsubsection{Constructor summary}{
\begin{verse}
\hyperlink{edu.kit.informatik.sdq.pse.go.Test()}{{\bf Test()}} \\
\end{verse}
}
\subsubsection{Method summary}{
\begin{verse}
\hyperlink{edu.kit.informatik.sdq.pse.go.Test.testMethod(int)}{{\bf testMethod(int)}} Diese Methode testet die Umsetzung von JAvadoc-Kommentaren in.tex-Files via TeXDoclet\\
\end{verse}
}
\subsubsection{Constructors}{
\vskip -2em
\begin{itemize}
\item{ 
\index{Test()}
\hypertarget{edu.kit.informatik.sdq.pse.go.Test()}{{\bf  Test}\\}
\begin{lstlisting}[frame=none]
public Test()\end{lstlisting} %end signature
}%end item
\end{itemize}
}
\subsubsection{Methods}{
\vskip -2em
\begin{itemize}
\item{ 
\index{testMethod(int)}
\hypertarget{edu.kit.informatik.sdq.pse.go.Test.testMethod(int)}{{\bf  testMethod}\\}
\begin{lstlisting}[frame=none]
public void testMethod(int testParam)\end{lstlisting} %end signature
\begin{itemize}
\item{
{\bf  Description}

Diese Methode testet die Umsetzung von JAvadoc-Kommentaren in.tex-Files via TeXDoclet
}
\item{
{\bf  Parameters}
  \begin{itemize}
   \item{
\texttt{testParam} -- ein Test-Parameter}
  \end{itemize}
}%end item
\end{itemize}
}%end item
\end{itemize}
}
}
}
% ------- textdoclet_include/finish.tex

% add something here

% closing for \chapter{TeXDoclet Java Documentation} {
}

\chapter{Finish}{
Lorem ipsum dolor sit amet, consetetur sadipscing elitr, sed diam nonumy eirmod tempor invidunt ut labore et dolore magna aliquyam erat, sed diam voluptua. At vero eos et accusam et justo duo dolores et ea rebum. Stet clita kasd gubergren, no sea takimata sanctus est Lorem ipsum dolor sit amet. Lorem ipsum dolor sit amet, consetetur sadipscing elitr, sed diam nonumy eirmod tempor invidunt ut labore et dolore magna aliquyam erat, sed diam voluptua. At vero eos et accusam et justo duo dolores et ea rebum. Stet clita kasd gubergren, no sea takimata sanctus est Lorem ipsum dolor sit amet.

}
% ------- textdoclet_include/finish.tex end

\end{document}
