\documentclass[parskip=full]{scrartcl}
\usepackage{pdfpages}
\usepackage[utf8]{inputenc}
\usepackage[T1]{fontenc}
\usepackage[german]{babel}
\usepackage{hyperref}
\hypersetup{
	pdftitle={Pflichtenheft},
	bookmarks=true,
}
\usepackage{csquotes}

\usepackage{fancyhdr}%<-------------to control headers and footers
\usepackage[a4paper,margin=1in,footskip=.25in]{geometry}
\fancyhf{}
\fancyfoot[C]{\thepage} %<----to get page number below text
\pagestyle{fancy} %<-------the page style itself

\usepackage{xcolor}
\usepackage{framed}
\definecolor{shadecolor}{RGB}{220,220,220}
\usepackage{float}


\title{Klassenbeschreibungen}
\author{Gruppe 3}
\date{09.07.17}

% define custom lists
\usepackage{enumitem}
\usepackage{lipsum}

% add glossary
\usepackage{glossaries}
\makeglossaries


\begin{document}

\begin{titlepage}
	\begin{center}
	{\scshape\LARGE \bfseries Entwurfsdokument \par}
	\vspace{1cm}
	{\scshape\Large Praktikum der Softwareentwicklung \\ Sommersemester 2017\par}
	\vspace{1.5cm}
	{\huge\bfseries Android GO! App\par}
	\vspace{2cm}
	{\Large\itshape - Gruppe 3 -\par}
	\vfill
	{\bfseries erstellt von:\par}
	Arsenii Dunaev \\
	Florian Kröger \\
	Tina Maria Strößner \\
	Volodymyr Shpylka \\	
	\vfill
	% Bottom of the page
	{\large 09.07.17 \par}	
	\end{center}
\end{titlepage}

\tableofcontents

\newpage


\section{Einleitung}
% Überblick und Paketstruktur erklären
% Änderungen zum Pflichtenheft

\section{Klassenbeschreibungen}
% Javadoc Beschreibungen aller Klassen, Methoden, Konstruktoren, Packages --> nicht private Methoden

\section{Sequenzdiagramme}
%anhand der Testszenarien aus dem Pflichtenheft


%Klassendiagramm
%Erläuterung von benutzten Entwurfsmustern

\printglossary	
\end{document}



\grid
