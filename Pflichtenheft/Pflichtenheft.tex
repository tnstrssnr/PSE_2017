\documentclass[parskip=full]{scrartcl}
\usepackage{pdfpages}
\usepackage[utf8]{inputenc}
\usepackage[T1]{fontenc}
\usepackage[german]{babel}
\usepackage{hyperref}
\hypersetup{
	pdftitle={Pflichtenheft},
	bookmarks=true,
}
\usepackage{csquotes}

\usepackage{fancyhdr}%<-------------to control headers and footers
\usepackage[a4paper,margin=1in,footskip=.25in]{geometry}
\fancyhf{}
\fancyfoot[C]{\thepage} %<----to get page number below text
\pagestyle{fancy} %<-------the page style itself


\title{Android GO! App - Pflichtenheft}
\author{Gruppe 3}
\date{11.06.17}

% define custom lists
\usepackage{enumitem}
\usepackage{lipsum}

\def\threedigits#1{%
  \ifnum#1<100 0\fi
  \ifnum#1<10 0\fi
  \number#1}

\begin{document}

\begin{titlepage}
	\begin{center}
	%TODO evtl App-Logo ergänzen
	%\includegraphics[width=0.15\textwidth]{example-image-1x1}\par\vspace{1cm}
	{\scshape\LARGE \bfseries Pflichtenheft \par}
	\vspace{1cm}
	{\scshape\Large Praktikum der Softwareentwicklung \\ Sommersemester 2017\par}
	\vspace{1.5cm}
	{\huge\bfseries Android GO! App\par}
	\vspace{2cm}
	{\Large\itshape - Gruppe 3 -\par}
	\vfill
	{\bfseries erstellt von:\par}
	Arsenii Dunaev \\
	Florian Kröger \\
	Houra Mortazavi \\ %TODO Nachname überprüfen
	Tina Maria Strößner \\
	Volodymyr Shpylka \\	
	\vfill
	% Bottom of the page
	{\large 11.06.17 \par}	
	\end{center}
\end{titlepage}

\tableofcontents

%TODO Gliederung ggfs nochmal überarbeiten
\newpage
\section{Zielbestimmung}
Die App dient zur Erleichterung und Strukturierung von Treffen. 
Man trifft sich jeden Tag und die Vereinbarungen im Text-Format in WhatsApp sind unübersichtlich. 
Dises Problem wird durch die GO-App gelöst, indem wir Struktur in den Vereinbarungsprozess reinbringen.  
\subsection{Musskriterien}
% Anzahl Gruppen, Gruppenmitglieder, GOs, etc. beschränken
% grobe Zusammenfassung d. fktl. Kriterien (+ referenz auf die jeweiligen Funktionen)

\begin{itemize}
	\item Benutzer  
	\begin{itemize}
	 	\item Registrierung
	 	\item Anmeldung
	 	\item Unterscheidung zwischen Administrator der Gruppe und ordentlichen Teilnehmer
	\end{itemize} 
	\item Gruppen
		\begin{itemize}
	 		%\item Jeder kann eine Gruppe erstellen und Nutzer in die Gruppe einladen
	 		%\item Jeder Teilnehmer der Gruppe kann ein Event in der Gruppe erstellen
	 		%\item Jeder Teilnehmer der Gruppe kann sich zum Event in der Gruppe anschließen
	 		%Das oben kommentiert sind eher schon Funktionalen Anforderungen
	 		\item Gruppen erstellen
	 		\item Events mit Eventdetails in der Gruppe erstellen 
	 		\item Zu den Gruppen eingeladen werden können
	 		\item Gemittelten Standort der Eventsteilnehmer anzeigen 
		\end{itemize}
\end{itemize}
	
\subsection{Wunschkriterien}
% Profilbilder
% Gruppenbilder
% Benachrichtigungen ??
\begin{itemize}
	\item Profilbilder
	\item Gruppenbild
	\item Benachrichtigungen (wie z.B. ''Event in 30 Minuten'', oder ''Jemand ist in der nähe'')
\end{itemize}

\subsection{Abgrenzungskriterien}
% kein IM
% kein Social Media
% kein Live-Navigationssystem mit Routenplanungsfunkton, etc.
Diese App ist kein Instant Messaging Platform: Textnachrichten, Filesharing wie z.B. in WhatsApp werden nicht unterstützt.


\section{Produkteinsatz}
%TODO was muss hier hin?
%\subsection{Anwendungsbereiche} Ich glaube diese Section ist unnötig.
\subsection{Zielgruppen}
% Studenten
Als Zielgruppe werden die junge Leute 13-35 Jahre alt berücksichtigt, die zusammen mit den Freunden rausgehen und eine einfache App brauchen, um leichter und strukturierter die Treffen zu organisieren.

\subsection{Betriebsbedingungen}
% mind. benötigte Android-Version
% funktionierende Netzwerkverbindung
% Zugriff auf GPS-Daten erlaubt
\begin{itemize}
\item Die App braucht funktionierende Netzwerkverbindung.\\
\item Der Zugriff auf GPS-Daten soll erlaubt werden.\\
\item Zugriff auf Kontakten ist wünschenswert für bessere User-Experience.
\end{itemize}
\section{Produktumgebung}

\subsection{Software}
\begin{itemize}
	\item Client Seite:\\
	Android 4 oder höher wird auf dem Smartphone verlangt.
	\item Server Seite:\\
	Tomcat 8 
\end{itemize}


\subsection{Hardware}
% \subsection{Produktschnittstellen} --> inwiefern ist das im Pflichtenheft relevant?
\begin{itemize}
\item Client Seite: \\ Android Gerät.\\
\item Server Seite: \\ Tomcat 8 Server.
\end{itemize}


\section{Funktionen}

\subsection{Benutzerkontofunktionen}

\begin{enumerate}[label={\textbf{/F\protect\threedigits{\theenumi}0/}}, leftmargin=*]
	
	
	\item \textit{Registrieren} Ein beliebiger Benutzer, der zuvor die App auf seinem Andoid-Ger"at installiert hat, kann sich in dem System registrieren. Zum erfolgreichen Registrieren m"ussen mindestens folgende Angaben gemacht werden:
	\begin{itemize}
		\item Telefonnummer %Email ?? --> wie soll die Verifikation des Benutzerkontos erfolgen
	\end{itemize}
	 Das System "uberpr"uft, ob die Telefonnummer noch nicht im System registriert ist. Das System sendet dem Benutzer eine SMS %Email ??
	 mit einem Best"atigungscode und fordert ihn auf, diesen in der App einzugeben %Referenz auf GUI-Ansicht
	 Wird der Code korrekt eingegeben ist die Registrierung erfolgreich abgeschlossen. Das System loggt den Benutzer automatisch ein und es wird *GUI-Ansicht* angezeigt %Referenz auf GUI-Ansicht
	
	
	\item \textit{Anmelden} Ein Benutzer, der zuvor bereits ein Benutzerkonto angelegt hat, kann sich in dem System einloggen durch Angabe seines
	\begin{itemize}
		\item Telefonnummer
	\end{itemize}
	Passt die Telefonnummer zu einem existierenden Benutzerkonto, wird dem Benutzer eine SMS mit einem Bestätigungscode geschickt. Wird der Code korrekt eingegeben ist die Registrierung erfolgreich abgeschlossen. Danach wird dem Benutzer seine *GUI-Ansicht* angezeigt und der Anmeldevorgang ist erfolgreich abgeschlossen.\\
Bemerkung: Anmelden soll nur dann aufgeruft werden, wenn der Benutzer vom neuen Gerät sich anmeldet. Nach dem Anmelden muss der Benutzer im System bleiben und wird vom System nicht ausgeloggt.
	
	%Benutzerkonto mit Telefon-# verbinden und immer angemeldet bleiben?
	\item \textit{Benutzerkennung anfordern}
	\item \textit{Kontoinformationen anzeigen}
	\item \textit{Anzeigenamen "andern}
	\item \textit{Passwort "andern}
	%\item \textit{Profilbild "andern}
	\item \textit{Benutzerkonto l"oschen}
	\item \textit{Abmelden}
\end{enumerate}

\subsection{Gruppen}

\subsubsection{von allen Benutzern ausführbare Funktionen} %TODO bessere Überschrift überlegen
Im Folgenden wird stets angenommen, dass der Benutzer ein Benutzerkonto besitzt und bereits im System eingeloggt ist.

\begin{enumerate}[label={\textbf{/F\protect\threedigits{\theenumi}0/}}, leftmargin=*]
%keie Kontakte, sondern Leute direkt über den Benutzernamen in eine Gruppe hinzufügen
	\item \textit{Gruppe erstellen}
	\item \textit{Aus Gruppe austreten}
	\item \textit{Gruppenanfrage beantworten} %bestätigen und ablehnen getrennt?
	\item \textit{Gruppeninformationen abrufen} %Mitglieder, Admin, aktuelle GOs
	\item \textit{Gruppen anzeigen} %zählt hier "alle gruppen in denen man ist ansehen" als Funktion?
\end{enumerate}

\subsubsection{Administratorfunktionen}
Im Folgenden wird angenommen, dass der Benutzer in seinem Benutzerkonto angemeldet ist und der Administrator einer Gruppe ist.

\begin{enumerate}[label={\textbf{/F\protect\threedigits{\theenumi}0/}}, leftmargin=*, resume]
	\item \textit{Gruppenname "andern}
	%\item \textit{Gruppenbild "andern}
	\item \textit{Gruppenmitglied hinzuf"ugen}
	\item \textit{Gruppenmitglied entfernen}	
	\item \textit{Admin hinzuf"ugen}
	\item \textit{Gruppe l"oschen}
	\item \textit{GO erstellen}
\end{enumerate}
	
\subsection{GO-Funktionen}

\subsubsection{von allen Benutzern ausführbare Funktionen}
Im Folgenden wird angenommen, dass der Benutzer in seinem Benutzerkonto angemeldet und Mitglied einer Gruppe ist.

\begin{enumerate}[label={\textbf{/F\protect\threedigits{\theenumi}0/}}, leftmargin=*, resume]	
	\item \textit{Dem Event beitreten}
	\item \textit{Losgehen}
	\item \textit{Aus Event austreten}
	\item \textit{Standorte abfragen} % alle ?? Sekunden aktualisieren
	\item \textit{GO-Informationen abrufen}
	%\item \textit{Zeit und Ort vorschlagen} --> evtl. Wunschkriterium?
\end{enumerate}

\subsubsection{Administratorfunktionen}
Im Folgenden wird angenommen, dass der Benutzer in seinem Benutzerkonto angemeldet ist und der Administrator einer Gruppe ist.

\begin{enumerate}[label={\textbf{/F\protect\threedigits{\theenumi}0/}}, leftmargin=*, resume]	
	\item \textit{GO erstellen}
	\item \textit{GO-Daten ändern} %Anfangszeitpunkt + Dauer festlegen --> GO beendet sich von selbst; Maximaldauer vorgeben (z.B. 10h); GO-Beschreibung, Name, Zielort (sollte auch unbestimmt sein dürfen)
	\item \textit{GO starten}
	\item \textit{GO beenden}
\end{enumerate}

\subsection{Sonstiges}
Im Folgenden wird angenommen, dass der Benutzer in seinem Benutzerkonto angemeldet ist.

\begin{enumerate}[label={\textbf{/F\protect\threedigits{\theenumi}0/}}, leftmargin=*, resume]	
	\item \textit{Benachrichtigungseinstellungen ändern}
	\item \textit{About-Seite anzeigen}
	\item \textit{Lizenzinformationen anzeigen}
\end{enumerate}

\section{Produktdaten}

\begin{enumerate}[label={\textbf{/D\protect\threedigits{\theenumi}0/}}, leftmargin=*]
	\item \textit{Benutzerdaten} Für jeden im System registrierten Benutzer m"ussen folgende Informationen gespeichert werden:
	\begin{itemize}
		\item Benutzer-ID (eindeutig)
		\item Benutzername (eindeutig)
		\item Name
		%\item Profilbild
		\item Telefonnummer
		%\Passwort (verschlüsselt)
	\end{itemize}
	\item \textit{Gruppendaten} Für jede erstellte Gruppe müssen folgende Informationen gespeichert werden:
	\begin{itemize}
	\item Gruppen-ID (eindeutig)
		\item Gruppenname
		\item Mitglieder
		\item Administratoren
		%\item offene Gruppenanfragen --> wie sollte man länger ausstehende Anfragen verwalten?
		%\item Ersteller --> evtl. unnötig
	\end{itemize}
	\item \textit{Go-Daten} Für jedes erstellte Event müssen folgende Informationen gespeichert werden:
	\begin{itemize}
		\item GO-ID (eindeutig)
		\item Uhrzeit
		\item Dauer
		\item Zielort (optional)
		\item beigetretene Benutzer
		\item losgegangene Benutzer
		\item Standorte der losgegangenen Benutzer
	\end{itemize}
\end{enumerate}

\section{Nichtfunktionale Anforderungen}

\section{Benutzeroberfläche}
%TODO Übergangsdiagramm erstellen

\section{Qualitäts-Zielbestimmungen}
% Funktionalität, Zuverlässigkeit, Benutzbarkeit, Effizienz, Änderbarkeit, Übertragbarkeit

\section{globale Testfälle und Testszenarien}

\begin{enumerate}[label={\textbf{/T\protect\threedigits{\theenumi}0/}}, leftmargin=*]
	\item bla
	\item bla bla
\end{enumerate}

\section{Entwicklungsumbgebung}

\section{Anhang}	
\end{document}
