\documentclass[parskip=full]{scrartcl}
\usepackage{pdfpages}
\usepackage[utf8]{inputenc}
\usepackage[T1]{fontenc}
\usepackage[german]{babel}
\usepackage{hyperref}
\hypersetup{
	pdftitle={Pflichtenheft},
	bookmarks=true,
}
\usepackage{csquotes}

\usepackage{fancyhdr}%<-------------to control headers and footers
\usepackage[a4paper,margin=1in,footskip=.25in]{geometry}
\fancyhf{}
\fancyfoot[C]{\thepage} %<----to get page number below text
\pagestyle{fancy} %<-------the page style itself


\title{Android GO! App - Pflichtenheft}
\author{Gruppe 3}
\date{11.06.17}

% define custom lists
\usepackage{enumitem}
\usepackage{lipsum}

\def\threedigits#1{%
  \ifnum#1<100 0\fi
  \ifnum#1<10 0\fi
  \number#1}

\begin{document}

\begin{titlepage}
	\begin{center}
	%TODO evtl App-Logo ergänzen
	%\includegraphics[width=0.15\textwidth]{example-image-1x1}\par\vspace{1cm}
	{\scshape\LARGE \bfseries Pflichtenheft \par}
	\vspace{1cm}
	{\scshape\Large Praktikum der Softwareentwicklung \\ Sommersemester 2017\par}
	\vspace{1.5cm}
	{\huge\bfseries Android GO! App\par}
	\vspace{2cm}
	{\Large\itshape - Gruppe 3 -\par}
	\vfill
	{\bfseries erstellt von:\par}
	Arsenii Dunaev \\ %TODO Nachname überprüfen
	Florian Kröger \\
	Houra Mortazavi \\ %TODO Nachname überprüfen
	Tina Maria Strößner \\
	Volodymyr Shpylka \\	
	\vfill
	% Bottom of the page
	{\large 11.06.17 \par}	
	\end{center}
\end{titlepage}

\tableofcontents

%TODO Gliederung ggfs nochmal überarbeiten

\section{Zielbestimmung}

\subsection{Musskriterien}
\subsection{Wunschkriterien}
\subsection{Abgrenzungskriterien}


\section{Produkteinsatz}

\subsection{Anwendungsbereiche}
\subsection{Zielgruppen}
\subsection{Betriebsbedingungen}


\section{Produktumgebung}

\subsection{Software}
\subsection{Hardware}
% \subsection{Produktschnittstellen}


\section{Funktionen}

\subsection{Benutzerkontofunktionen}

\begin{enumerate}[label={\textbf{/F\protect\threedigits{\theenumi}0/}}, leftmargin=*]
	\item \textit{Anmelden}
	\item \textit{Registrieren}
	\item \textit{Benutzerkennung anfordern}
	\item \textit{Anzeigenamen "andern}
	\item \textit{Passwort "andern}
	\item \textit{Profilbild "andern}
	\item \textit{Benutzerkonto l"oschen}
\end{enumerate}

\subsection{Gruppen und Kontakte}

\begin{enumerate}[label={\textbf{/F\protect\threedigits{\theenumi}0/}}, leftmargin=*]
	\item \textit{Kontakt hinzufügen}
	\item \textit{Kontakt anzeigen}
	\item \textit{Kontakt entfernen}
	\item \textit{Gruppe erstellen}
	\item \textit{Aus Gruppe austreten}
\end{enumerate}

\subsection{Adminfunktionen}

\begin{enumerate}[label={\textbf{/F\protect\threedigits{\theenumi}0/}}, leftmargin=*, resume]
	\item \textit{Gruppenname "andern}
	\item \textit{Gruppenbild "andern}
	\item \textit{Gruppenmitglied hinzuf"ugen}
	\item \textit{Gruppenmitglied entfernen}	
	\item \textit{Admin hinzuf"ugen}
	\item \textit{Gruppe l"oschen}
	\item \textit{GO erstellen}
\end{enumerate}

\section{Produktdaten}

\begin{enumerate}[label={\textbf{/D\protect\threedigits{\theenumi}0/}}, leftmargin=*]
	\item bla
	\item bla bla
\end{enumerate}

\section{Produktleistungen}

\section{Benutzeroberfläche}

\section{Qualitäts-Zielbestimmungen}
% Funktionalität, Zuverlässigkeit, Benutzbarkeit, Effizienz, Änderbarkeit, Übertragbarkeit

\section{globale Testfälle und Testszenarien}

\begin{enumerate}[label={\textbf{/T\protect\threedigits{\theenumi}0/}}, leftmargin=*]
	\item bla
	\item bla bla
\end{enumerate}

\section{Entwicklungsumbgebung}

\section{Anhang}	
\end{document}
\grid
\grid
