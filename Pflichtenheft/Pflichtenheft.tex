\documentclass[parskip=full]{scrartcl}
\usepackage{pdfpages}
\usepackage[utf8]{inputenc}
\usepackage[T1]{fontenc}
\usepackage[german]{babel}
\usepackage{hyperref}
\hypersetup{
	pdftitle={Pflichtenheft},
	bookmarks=true,
}
\usepackage{csquotes}

\usepackage{fancyhdr}%<-------------to control headers and footers
\usepackage[a4paper,margin=1in,footskip=.25in]{geometry}
\fancyhf{}
\fancyfoot[C]{\thepage} %<----to get page number below text
\pagestyle{fancy} %<-------the page style itself


\title{Android GO! App - Pflichtenheft}
\author{Gruppe 3}
\date{11.06.17}

% define custom lists
\usepackage{enumitem}
\usepackage{lipsum}

% add glossary
\usepackage{glossaries}
\makeglossaries
\newglossaryentry{System}
{
	name={System},
	description={Die Kombination aus Mobile App und Server},
}

\newglossaryentry{GO} %TODO anderen Namen finden
{
	name={GO},
	description={Ein Event in einer Gruppe, dem Mitglieder beitreten können, um ihren Standort mit anderen GO-Teilnehmern zu teilen},
}
\newglossaryentry{Benutzer}
{
	name={Benutzer},
	description={Eine Person, die die App installiert und sich mit einem Benutzeraccount registriert hat bzw. plant dies zu tun},
}
\newglossaryentry{Benutzername}
{
	name={Benutzername},
	description={Ein im System eindeutiger Name, der es ermöglicht einen Benutzer mit Benutzeraccount zu identifizieren. Er kann nach Erstellung eines Accounts nicht mehr geändert werden},
}
\newglossaryentry{Anzeigename}
{
	name={Anzeigename},
	description={Ein im System nicht-eindeutiger Name, den der Benutzer jederzeit ändern kann},
}
\newglossaryentry{Hauptansicht}
{
	name={Hauptansicht},
	description={Ansicht, die geöffnet wird, sobald sich ein Benutzer angemeldet hat. Dies entspricht der Ansicht *GUI-Ansicht*},
}
\newglossaryentry{App}
{
	name={App},
	description={mobile Applikation, die auf dem Smartphone/Tablet des Benutzers installiert ist. Der Benutzer interagiert ausschließlich mit der Applikation und nicht mit dem Webserver direkt.},
}

\def\threedigits#1{%
  \ifnum#1<100 0\fi
  \ifnum#1<10 0\fi
  \number#1}

\begin{document}

\begin{titlepage}
	\begin{center}
	%TODO evtl App-Logo ergänzen
	%\includegraphics[width=0.15\textwidth]{example-image-1x1}\par\vspace{1cm}
	{\scshape\LARGE \bfseries Pflichtenheft \par}
	\vspace{1cm}
	{\scshape\Large Praktikum der Softwareentwicklung \\ Sommersemester 2017\par}
	\vspace{1.5cm}
	{\huge\bfseries Android GO! App\par}
	\vspace{2cm}
	{\Large\itshape - Gruppe 3 -\par}
	\vfill
	{\bfseries erstellt von:\par}
	Arsenii Dunaev \\
	Florian Kröger \\
	Houra Mortazavi \\ %TODO Nachname überprüfen
	Tina Maria Strößner \\
	Volodymyr Shpylka \\	
	\vfill
	% Bottom of the page
	{\large 11.06.17 \par}	
	\end{center}
\end{titlepage}

\tableofcontents

%TODO Gliederung ggfs nochmal überarbeiten
\newpage
\section{Zielbestimmung}
Die \gls{App} dient zur Erleichterung und Strukturierung von Treffen. 
Man trifft sich jeden Tag und die Vereinbarungen im Text-Format in WhatsApp sind unübersichtlich. 
Dises Problem wird durch die GO-App gelöst, indem wir Struktur in den Vereinbarungsprozess reinbringen.  
\subsection{Musskriterien}
% Anzahl Gruppen, Gruppenmitglieder, GOs, etc. beschränken
% grobe Zusammenfassung d. fktl. Kriterien (+ referenz auf die jeweiligen Funktionen)

\begin{itemize}[itemsep=0pt]
	\item Benutzer  
	\begin{itemize}
	 	\item Registrierung
	 	\item Anmeldung
	 	\item Unterscheidung zwischen Administrator der Gruppe und ordentlichen Teilnehmer
	\end{itemize} 
	\item Gruppen
		\begin{itemize}
	 		%\item Jeder kann eine Gruppe erstellen und Nutzer in die Gruppe einladen
	 		%\item Jeder Teilnehmer der Gruppe kann ein Event in der Gruppe erstellen
	 		%\item Jeder Teilnehmer der Gruppe kann sich zum Event in der Gruppe anschließen
	 		%Das oben kommentiert sind eher schon Funktionalen Anforderungen
	 		\item Gruppen erstellen
	 		\item Events mit Eventdetails in der Gruppe erstellen 
	 		\item Zu den Gruppen eingeladen werden können
	 		\item Gemittelten Standort der Eventsteilnehmer anzeigen 
		\end{itemize}
\end{itemize}
	
\subsection{Wunschkriterien}
% Profilbilder
% Gruppenbilder
% Benachrichtigungen ??
\begin{itemize}
	\item Profilbilder
	\item Gruppenbild
	\item Benachrichtigungen (wie z.B. ''Event in 30 Minuten'', oder ''Jemand ist in der nähe'')
\end{itemize}

\subsection{Abgrenzungskriterien}
% kein IM
% kein Social Media
% kein Live-Navigationssystem mit Routenplanungsfunkton, etc.
Diese App ist kein Instant Messaging Platform: Textnachrichten, Filesharing wie z.B. in WhatsApp werden nicht unterstützt.

\newpage
\section{Produkteinsatz}
%TODO was muss hier hin?
%\subsection{Anwendungsbereiche} Ich glaube diese Section ist unnötig.
\subsection{Zielgruppen}
% Studenten
Als Zielgruppe werden die junge Leute 13-35 Jahre alt berücksichtigt, die zusammen mit den Freunden rausgehen und eine einfache App brauchen, um leichter und strukturierter die Treffen zu organisieren.

\subsection{Betriebsbedingungen}
% mind. benötigte Android-Version
% funktionierende Netzwerkverbindung
% Zugriff auf GPS-Daten erlaubt
\begin{itemize}
\item Die App braucht funktionierende Netzwerkverbindung.\\
\item Der Zugriff auf GPS-Daten soll erlaubt werden.\\
\item Zugriff auf Kontakten ist wünschenswert für bessere User-Experience.
\end{itemize}

\newpage
\section{Produktumgebung}

\subsection{Software}
\begin{itemize}
	\item Client Seite:\\
	Android 4 oder höher wird auf dem Smartphone verlangt.
	\item Server Seite:\\
	Tomcat 8 
\end{itemize}


\subsection{Hardware}
% \subsection{Produktschnittstellen} --> inwiefern ist das im Pflichtenheft relevant?
\begin{itemize}
\item Client Seite: \\ Android Gerät.\\
\item Server Seite: \\ Tomcat 8 Server.
\end{itemize}

\newpage
\section{Funktionen}

\subsection{Benutzerkontofunktionen}

\begin{enumerate}[label={\textbf{/F\protect\threedigits{\theenumi}0/}}, leftmargin=*]
	
	\item \textit{Registrieren} Ein beliebiger Benutzer, der zuvor die App auf seinem Andoid-Ger"at installiert hat, kann sich in dem \gls{System} registrieren. Zum erfolgreichen Registrieren m"ussen mindestens folgende Angaben gemacht werden:
	\begin{itemize}
		\item Telefonnummer %Email ?? --> wie soll die Verifikation des Benutzerkontos erfolgen
		\item Passwort (muss zweimal identisch eingegeben werden)
	\end{itemize}
	 Das System "uberpr"uft, ob die Telefonnummer noch nicht im System registriert ist. Das System sendet dem Benutzer eine SMS %Email ??
	 mit einem Best"atigungscode und fordert ihn auf, diesen in der App einzugeben %Referenz auf GUI-Ansicht
	 Wird der Code korrekt eingegeben ist die Registrierung erfolgreich abgeschlossen. Das System loggt den Benutzer automatisch ein und es wird *GUI-Ansicht* angezeigt %Referenz auf GUI-Ansicht
	 	
	\item \textit{Anmelden} Ein Benutzer, der zuvor bereits ein Benutzerkonto angelegt hat, kann sich in dem System einloggen durch Angabe von
	\begin{itemize}
		\item Telefonnummer
		\item Passwort
	\end{itemize}
	Passt die Telefonnummer zu einem existierenden Benutzerkonto, wird dem Benutzer eine SMS mit einem Bestätigungscode geschickt. Wird der Code korrekt eingegeben ist die Registrierung erfolgreich abgeschlossen. Danach wird dem Benutzer seine *GUI-Ansicht* angezeigt und der Anmeldevorgang ist erfolgreich abgeschlossen.\\
Bemerkung: Anmelden soll nur dann aufgeruft werden, wenn der Benutzer vom neuen Gerät sich anmeldet. Nach dem Anmelden muss der Benutzer im System bleiben und wird vom System nicht ausgeloggt.
	
	%Benutzerkonto mit Telefon-# verbinden und immer angemeldet bleiben?
	\item \textit{Benutzerkennung anfordern} Ein Benutzer kann, sollte er sein Passwort vergessen haben, seine Benutzerkennung anfordern und eine neues Passwort setzen. Wird auf dem Anmeldescreen (vgl. *GUI-Ansicht*) die 'Passwort vergessen'-Option gewählt, schickt der Server dem Benutzer eine SMS mit einem Zugangscode. Dieser Zugangscode ist 10min lang gültig. Der Benutzer gibt den Zugangscode ein und kann anschließend eine neues Passwort setzen. Danach loggt das System den Benutzer ein und lädt die Hauptansicht der Applikation. Beim nächsten Anmelden kann sich der Benutzer mit dem neuen Passwort anmelden.
	
	\item \textit{Persönliche Daten ändern} Das System speichert persönliche Daten der Benutzer (siehe  \ref{persönliche Daten}). Der Benutzer kann diese Daten ändern. Dazu ruft der Benutzer sein Profil auf (vgl. *GUI-Ansicht*) und betätigt die Schaltfläche 'Profil bearbeiten'. In der sich öffnenden Aktivität kann der Benutzer die gewünschten Änderungen vornehmen und mit dem Button 'Profil speichern' sichern. Die geänderten Daten werden dann vom System übernommen.\\
	Um das Benutzerpasswort ändern zu können, muss zuerst das alte Passwort korrekt eingegeben werden. Danach gibt der Nutzer zweimal das neue Passwort ein (beide Eingaben müssen identisch sein). Danach wird die Änderung ebenfalls mit 'Profil speichern' bestätigt. Ab sofort kann sich der Benutzer mit seinem neuen Passwort einloggen.
	
	\item \textit{Abmelden} Ein Benutzer kann sich aus seinem Account abmelden, indem er in der Einstellungs-Ansicht (vgl. *GUI-Ansicht*) die Option 'Abmelden' auswählt. Das System meldet daraufhin den Benutzer von seinem Benutzerkonto ab und lädt die Anmelde-Ansicht (vgl. *GUI-Ansicht*).
	
	\item \textit{Benutzerkonto l"oschen}
	Ein Benutzer hat die Möglichkeit seinen Benutzeraccount zu löschen. Dazu ruft der Benutzer sein Profil auf (vgl. *GUI-Ansicht*) und betätigt den Button 'Profil löschen'. Das System fragt denBenutzer ob das Konto wirklich gelöscht werden soll. Wird dies bestätigt so wird der Benutzer von dem System aus seinem Account ausgeloggt. Anschließend löscht das System alle Daten des Benutzerkontos (vgl. \ref{Profildaten}) und entfernt den Benutzer aus sämtlichen Gruppen und GO's denen er zuvor beigetreten war.
\end{enumerate}

\subsection{Gruppen}

\subsubsection{von allen Benutzern ausführbare Funktionen} %TODO bessere Überschrift überlegen
Im Folgenden wird stets angenommen, dass der Benutzer ein Benutzerkonto besitzt und bereits im System eingeloggt ist.

\begin{enumerate}[label={\textbf{/F\protect\threedigits{\theenumi}0/}}, leftmargin=*, resume]
%keie Kontakte, sondern Leute direkt über den Benutzernamen in eine Gruppe hinzufügen
	\item \textit{Gruppen anzeigen} %zählt hier "alle gruppen in denen man ist ansehen" als Funktion?
	Ein Benutzer kann sich alle Gruppen, in denen er zu diesem Zeitpunkt Mitglied ist, anzeigen lassen, indem er den 'Gruppen'-Reiter der \gls{Hauptansicht} auswählt (vgl. *GUI-Ansicht*)
	
	\item \textit{Gruppeninformationen abrufen} \label{Gruppeninfo anzeigen} %Mitglieder, Admin, aktuelle GOs
	Ein Benutzer, der Mitglied einer Gruppe ist, kann sich von der App bestimmte Systeminformationen zu dieser Gruppe anzeigen lassen. Dazu wählt der Benutzer das Gruppenicon der gewünschten Gruppe in der *GUI-Ansicht* aus. Darauffolgend werden ihm vom System folgende Informationen über die Gruppe angezeigt:
	\begin{itemize}
		\item Gruppenname
		%\item Gruppenbild
		%\item Gruppenbeschreibung
		\item aktuelle Mitglieder der Gruppe (offene Gruppenmitgliedsanfragen werden nur den Administratoren der Gruppe angezeigt, ehemalige Mitglieder der Gruppe werden nicht angezeigt)
		\item aktuelle Administratoren der Gruppe
		\item aktuelle \gls{GO}'s der Gruppe %kann geklickt werden, um in die Ansicht des jeweiligen GOs zu gelangen --> Querverweis einfügen
	\end{itemize}
	
	\item \textit{Gruppe erstellen}
	Ein User kann eine neue Gruppe erstellen. Dazu muss in der Ansicht *GUI-Ansicht* die Auswahl "Gruppe erstellen" gewählt werden und anschließend ein Gruppenname angegeben werden. Die Funktion bietet die Option bei der Erstellung der Gruppe Gruppenmitglieder hinzuzufügen (vgl. *GUI-Ansicht*). Für genauere Erläuterungen hierzu siehe \ref{Mitglieder hinzufügen}.
	Der Ersteller der Gruppe wird automatisch zum (zu diesem Zeitpunkt) einzigen Administrator derselben.
	
	\item \textit{Aus Gruppe austreten}
	Ein \gls{Benutzer}, der Mitglied einer Gruppe ist, kann aus derselben austreten. Sollte der Benutzer einziger Administrator der Gruppe sein, muss zuerst ein weiteres aktuelles Mitglied der Gruppe zum Administrator ernannt werden (siehe Funktion \ref{Admin hinzufügen}, bevor diese Funktion ausgeführt werden kann. \\
	Um aus einer Gruppe auszutreten wird vom Benutzer in der Gruppensansicht (vgl. *GUI-Ansicht*) die Gruppeninformation aufgerufen (vgl. *GUI-Ansicht*) und anschlie"send auf den Button "Gruppe verlassen" gedrückt. Das System entfernt den Benutzer aus der Gruppe und die Funktion ist erfolgreich abgeschlossen. Die Gruppe bleibt im System weiterhin bestehen.\\
	In dem Fall, dass der austretende Benutzer das einzige Mitglied der Gruppe ist (und somit auch Administrator), entfällt das Ernennen eines neuen Administrators vor dem Austritt und die Gruppe wird nach dem Ausführen dieser Funktion automatisch von dem System gelöscht.
	
	\item \textit{Gruppenanfrage beantworten} \label{Gruppenanfrage beantworten}
	Diese Funktion setzt voraus, dass ein Benutzer eine Anfrage auf Gruppenmitgliedschaft von einem anderen Benutzer bekommen hat (vgl. Funktion \ref{Mitglieder hinzufügen}). Dem Benutzer werden offene Gruppenanfragen durch ein grün hinterlegtes Gruppenicon in der Gruppenansicht angezeigt (vgl *GUI-Ansicht*). Durch Klicken auf das Icon wird die Funktion "Gruppenanfrage beantworten" gestartet (vgl. *GUI-Ansicht*). Der Benutzer hat dann zwei Möglichkeiten auf die Anfrage zu reagieren:
	\begin{itemize}
		\item \textit{Bestätigen} bestätigt der Benutzer die Anfrage, so wird er vom System der Gruppe hinzugefügt.
		\item \textit{Ablehnen} lehnt der Benutzer die Anfrage ab, wird keine weitere Aktion des Systems ausgeführt.
	\end{itemize}
Nach Beantwortung der Anfrage wird diese vom System gelöscht.
\end{enumerate}

\subsubsection{Administratorfunktionen}
Im Folgenden wird angenommen, dass der Benutzer in seinem Benutzerkonto angemeldet und der Administrator einer Gruppe ist.\\
Um eine der folgenden Funktionen ausführen zu können muss der Administrator zunächst in die Gruppeninformationsansicht wechseln (vgl. *GUI-Ansicht* und \ref{Gruppeninfo anzeigen}). Unter der Voraussetzung, das es sich bei dem Benutzer um einen Administrator dieser Gruppe handelt, wird ihm eine Schaltfläche 'Gruppe bearbeiten' angezeigt. Die Betätigung dieser Schaltfläche startet eine Aktivität, die es dem Benutzer erlaubt Änderungen an der aktuellen Konfiguration der Gruppe vorzunehmen. Es wird im Folgenden Abschnitt angenommen, dass der Benutzer diese Aktivität bereits gestartet hat.

\begin{enumerate}[label={\textbf{/F\protect\threedigits{\theenumi}0/}}, leftmargin=*, resume]

	\item \textit{Gruppenname "andern}
	Der Gruppenname kann von einem Administrator geändert werden, indem er den aktuellen Gruppennamen, der in einem editierbaren Textfeld angezeigt wird, durch den neuen erstetzt und im Anschluss auf den Button "Änderungen speichern" klickt (vgl. *GUI-Ansicht*).
	%\item \textit{Gruppenbeschreibung ändern}
	%\item \textit{Gruppenbild "andern}
	
	\item \textit{Gruppenmitglied hinzuf"ugen} \label{Mitglieder hinzufügen}
	Ein Administrator kann einer Gruppe neue Mitglieder hinzufügen. Dazu wählt er den Button 'Mitglied hinzufügen' aus. Anschließend kann anhand des \gls{Benutzername}ns nach dem gewünschten Benutzer gesucht werden. Das System zeigt dem Administrator Benutzer an, deren Benutzernamen dem gesuchten entsprechen. Der Administrator wählt den gewünschten Benutzer aus, woraufhin das System diesem Nutzer eine Gruppenanfrage sendet (siehe \ref{Gruppenanfrage beantworten}). Solange die Gruppenanfrage offen ist, wird das potentielle Gruppenmitglied für die Administratoren der Gruppe angezeigt (Status der Mitgliedschaft ist farblich gekennzeichnet), für normale Gruppenmitglieder sind offene Gruppenanfragen nicht sichtbar.
	
	\item \textit{Gruppenmitglied entfernen}
	Ein Administrator kann beliebige Mitglieder der Gruppe entfernen, indem er das rote Minuszeichen neben dem zu entfernenden Benutzer anklickt (vgl. *GUI-Ansicht*) und anschließend die Durchführung der Aktion bestätigt. Der ausgewählte Benutzer wird vom System aus der Gruppe entfernt.
	
	\item \textit{Admin hinzuf"ugen} \label{Admin hinzufügen}
	Ein Administrator kann ein weiteres Mitglied der Gruppe zum Administrator ernennen. Dazu wählt er den 'Als Admin hinzufügen'-Option neben dem gewünschten Gruppenmitglied aus (vgl. *GUI-Ansicht*). Das System ändert die Rolle des Gruppenmitglieds zu 'Adminsitrator'. Das Gruppenmitglied erhält eine Benachrichtigung (???) über den neuen Status und kann ab sofort alle Funktionen eines Administrators der Gruppe ausführen.
	
	\item \textit{Gruppe l"oschen}
	Ein Adminsitrator kann eine Gruppe löschen indem er den "Gruppe löschen"-Button auswählt. Das System löscht die Gruppe. Sie wird ab diesem Zeitpunkt nicht mehr in der Gruppenansicht der Mitglieder angezeigt. Mitglieder erhalten keine Benachrichtigung über die Löschung der Gruppe (???).
	
\end{enumerate}
	
\subsection{GO-Funktionen}

\subsubsection{von allen Benutzern ausführbare Funktionen}
Im Folgenden wird angenommen, dass der Benutzer in seinem Benutzerkonto angemeldet und Mitglied einer Gruppe ist.

\begin{enumerate}[label={\textbf{/F\protect\threedigits{\theenumi}0/}}, leftmargin=*, resume]	
	\item \textit{GO erstellen}
	\item \textit{Dem Event beitreten}
	\item \textit{Losgehen}
	\item \textit{Aus Event austreten}
	\item \textit{Standorte abfragen} % alle ?? Sekunden aktualisieren
	\item \textit{GO-Informationen abrufen}
	%\item \textit{Zeit und Ort vorschlagen} --> evtl. Wunschkriterium?
\end{enumerate}

\subsubsection{Administratorfunktionen}
Im Folgenden wird angenommen, dass der Benutzer in seinem Benutzerkonto angemeldet ist und der Administrator einer Gruppe ist.

\begin{enumerate}[label={\textbf{/F\protect\threedigits{\theenumi}0/}}, leftmargin=*, resume]	
	\item \textit{GO erstellen}
	\item \textit{GO-Daten ändern} %Anfangszeitpunkt + Dauer festlegen --> GO beendet sich von selbst; Maximaldauer vorgeben (z.B. 10h); GO-Beschreibung, Name, Zielort (sollte auch unbestimmt sein dürfen)
	\item \textit{GO starten}
	\item \textit{GO beenden}
\end{enumerate}

\subsection{Sonstiges}
Im Folgenden wird angenommen, dass der Benutzer in seinem Benutzerkonto angemeldet ist.

\begin{enumerate}[label={\textbf{/F\protect\threedigits{\theenumi}0/}}, leftmargin=*, resume]	
	\item \textit{Benachrichtigungseinstellungen ändern}
	\item \textit{About-Seite anzeigen}
	\item \textit{Lizenzinformationen anzeigen}
\end{enumerate}

\newpage
\section{Produktdaten}

\begin{enumerate}[label={\textbf{/D\protect\threedigits{\theenumi}0/}}, leftmargin=*]
	\item \textit{Profildaten} \label{Profildaten}
	\item \textit{Persönliche Daten} \label{persönliche Daten} Für jeden im System registrierten Benutzer m"ussen folgende Informationen gespeichert werden:
	\begin{itemize}
		\item Benutzer-ID (eindeutig)
		\item Benutzername (eindeutig)
		\item Name
		%\item Profilbild
		\item Telefonnummer
		\item Passwort (verschlüsselt)
	\end{itemize}
	\item \textit{Gruppendaten} Für jede erstellte Gruppe müssen folgende Informationen gespeichert werden:
	\begin{itemize}
	\item Gruppen-ID (eindeutig)
		\item Gruppenname
		\item Mitglieder
		\item Administratoren
		\item offene Gruppenanfragen --> wie sollte man länger ausstehende Anfragen verwalten?
	\end{itemize}
	\item \textit{Go-Daten} Für jedes erstellte Event müssen folgende Informationen gespeichert werden:
	\begin{itemize}
		\item GO-ID (eindeutig)
		\item Uhrzeit
		\item Dauer
		\item Zielort (optional)
		\item beigetretene Benutzer
		\item losgegangene Benutzer
		\item Standorte der losgegangenen Benutzer
	\end{itemize}
\end{enumerate}

\newpage
\section{Nichtfunktionale Anforderungen}

\newpage
\section{Benutzeroberfläche}
%TODO Übergangsdiagramm erstellen

\newpage
\section{Qualitäts-Zielbestimmungen}
% Funktionalität, Zuverlässigkeit, Benutzbarkeit, Effizienz, Änderbarkeit, Übertragbarkeit

\newpage
\section{globale Testfälle und Testszenarien}

\begin{enumerate}[label={\textbf{/T\protect\threedigits{\theenumi}0/}}, leftmargin=*]
	\item bla
	\item bla bla
\end{enumerate}

\newpage
\section{Entwicklungsumbgebung}

\newpage
\section{Anhang}

\newpage
\printglossary	
\end{document}



